\documentclass[a4paper,10pt]{article}




\usepackage[utf8]{inputenc}
\usepackage[T1]{fontenc}
\usepackage{lastpage}
\usepackage{fancyhdr} % pour personnaliser les en-têtes
\usepackage[frenchb]{babel}
\usepackage{amsfonts,amssymb}
\usepackage{amsmath,amsthm}
\usepackage{paralist}
\usepackage{xspace}
\usepackage{xcolor}
\usepackage{variations}
\usepackage{xypic}
\usepackage{eurosym}
\usepackage{graphicx}
\usepackage[np]{numprint}
\usepackage{hyperref} 
\usepackage{listings} % pour écrire des codes avec coloration syntaxique  

\usepackage{tikz}
\usetikzlibrary{calc, arrows, plotmarks,decorations.pathreplacing}
\usepackage{colortbl}
\usepackage{multirow}
\usepackage[top=1.5cm,bottom=1.5cm,right=1.5cm,left=1.5cm]{geometry}



\theoremstyle{definition}
\newtheorem*{remarque}{Remarque}
\newtheorem{exo}{Exercice}
\newtheorem{definition}{Définition}


\newcommand{\vtab}{\rule[-0.4em]{0pt}{1.2em}}
\newcommand{\V}{\overrightarrow}
\renewcommand{\thesection}{\Roman{section} }
\renewcommand{\thesubsection}{\arabic{subsection} }
\renewcommand{\thesubsubsection}{\alph{subsubsection} }

\newcommand{\C}{\mathbb{C}}
\newcommand{\R}{\mathbb{R}}
\newcommand{\Q}{\mathbb{Q}}
\newcommand{\Z}{\mathbb{Z}}
\newcommand{\N}{\mathbb{N}}


\definecolor{vert}{RGB}{11,160,78}
\definecolor{rouge}{RGB}{255,120,120}

\pagestyle{fancy}
\lhead{}\chead{}\rhead{}\lfoot{}\cfoot{\thepage}\rfoot{\textbf{}}\renewcommand{\headrulewidth}{0pt}\renewcommand{\footrulewidth}{0.4pt}%Tourner la page S.V.P.



\begin{document}
	
	\leftline{\bfseries Lycée Maurice Genevoix \hfill Année~2021-2022}
	\leftline{\bfseries Nom: }
	\leftline{\bfseries Prénom:\hfill $Seconde_{.....}$}
	\rule[0.5ex]{\textwidth}{0.1mm}	
	
	\begin{center}
		\large \sc Test 2 : Soldes d'hiver, inéquations et calculs avec les puissances
	\end{center}

\begin{exo}	\textit{\textbf{Soldes d'hiver}}\\[0.25cm]
Soit une enceinte Bluetooth qui coûtent 139 euros, donner son prix arrondi au centime après une remise de 20\% suivie d'une remise de 12\%.\hfill\textbf{/1}\\[0.5cm]
.\dotfill \\\\
.\dotfill \\

\end{exo}

\begin{exo} \textit{\textbf{Une équation et une inéquation:}}\\\\
\begin{enumerate}
	\item Résoudre l'équation suivante:\hfill\textbf{/1.5}
	$$2y+\dfrac{3}{4} \ = \  \dfrac{3}{7}$$\hfill\\[4cm]
	\item 	Donner l'ensemble solution de l'inéquation:\hfill\textbf{/1.5}
	$$-4x+3 \leq -1$$ \hfill\\[4cm]
\end{enumerate}

\end{exo}

\begin{exo} \textit{\textbf{}}\\
	\item Exprimer sous la forme d'une seule puissance les nombres suivants:\hfill\textbf{/1}\\\\
	\begin{enumerate}[$\square$]
		\item $ 7^{-5}\times \left(\dfrac{7^3}{7^{-4}}\right)^2 \ = \  $ .\dotfill \\\\
		
		\item $ (5^4)^{-2}\times \dfrac{(5^{3})^2}{(5^{-2})^2}  \ = \ $.\dotfill \\\\
		
	\end{enumerate}



\end{exo}
\end{document}





