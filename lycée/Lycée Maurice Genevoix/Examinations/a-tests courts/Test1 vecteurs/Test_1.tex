\documentclass[a4paper,10pt]{article}




\usepackage[utf8]{inputenc}
\usepackage[T1]{fontenc}
\usepackage{lastpage}
\usepackage{fancyhdr} % pour personnaliser les en-têtes
\usepackage[frenchb]{babel}
\usepackage{amsfonts,amssymb}
\usepackage{amsmath,amsthm}
\usepackage{paralist}
\usepackage{xspace}
\usepackage{xcolor}
\usepackage{variations}
\usepackage{xypic}
\usepackage{eurosym}
\usepackage{graphicx}
\usepackage[np]{numprint}
\usepackage{hyperref} 
\usepackage{listings} % pour écrire des codes avec coloration syntaxique  

\usepackage{tikz}
\usetikzlibrary{calc, arrows, plotmarks,decorations.pathreplacing}
\usepackage{colortbl}
\usepackage{multirow}
\usepackage[top=2cm,bottom=1.5cm,right=2cm,left=1.5cm]{geometry}



\theoremstyle{definition}
\newtheorem*{remarque}{Remarque}
\newtheorem{exo}{Exercice}
\newtheorem{definition}{Définition}


\newcommand{\vtab}{\rule[-0.4em]{0pt}{1.2em}}
\newcommand{\V}{\overrightarrow}
\renewcommand{\thesection}{\Roman{section} }
\renewcommand{\thesubsection}{\arabic{subsection} }
\renewcommand{\thesubsubsection}{\alph{subsubsection} }

\newcommand{\C}{\mathbb{C}}
\newcommand{\R}{\mathbb{R}}
\newcommand{\Q}{\mathbb{Q}}
\newcommand{\Z}{\mathbb{Z}}
\newcommand{\N}{\mathbb{N}}


\definecolor{vert}{RGB}{11,160,78}
\definecolor{rouge}{RGB}{255,120,120}

\pagestyle{fancy}
\lhead{}\chead{}\rhead{}\lfoot{}\cfoot{\thepage/1}\rfoot{\textbf{}}\renewcommand{\headrulewidth}{0pt}\renewcommand{\footrulewidth}{0.4pt}%Tourner la page S.V.P.



\begin{document}
	
	\leftline{\bfseries Lycée Maurice Genevoix \hfill Année~2020-2021}
	\leftline{\bfseries Nom: }
	\leftline{\bfseries Prénom:\hfill $Seconde_{.....}$}
	\rule[0.5ex]{\textwidth}{0.1mm}	
	
	\begin{center}
		\large \sc Test 1 : Premières notions avec les vecteurs
	\end{center}

\begin{exo}	\textit{\textbf{Comparer deux vecteurs}}\\
Avec la figure ci-dessous remplir les cases du tableau par (OUI / NON).\hfill\textbf{/4}
	\begin{figure}[h!]
				
				\begin{minipage}[c]{0.35\linewidth}
					\raggedright
					\definecolor{uuuuuu}{rgb}{0.2,0.2,0.2}
					\definecolor{cqcqcq}{rgb}{0.7,0.7,0.7}
					\definecolor{qqwuqq}{rgb}{0.,0.3,0.}
					\begin{tikzpicture}[line cap=round,line join=round,>=triangle 45,x=1.5cm,y=1.5cm]
					\draw [color=cqcqcq,, xstep=1.5cm,ystep=1.5cm] (0.4,0.3) grid (3.8,4.4);
					\clip(0.4,0.3) rectangle (3.8,4.4);
					\draw (1.,1.)-- (3.,1.);
					\draw (3.,1.)-- (3.,3.);
					\draw (1.,1.)-- (1.,3.);
					\draw (1.,3.)-- (2.,4.);
					\draw (3.,3.)-- (2.,4.);
					\draw (1.,3.)-- (2.,2.);
					\draw (2.,2.)-- (3.,3.);
					\draw (1.,2.)-- (3.,2.);
					\begin{scriptsize}
					\draw [fill=black] (1.,1.) circle (1.5pt);
					\draw[color=black] (0.7268549455211993,1.2399289349479812) node {$A$};
					\draw [fill=black] (3.,1.) circle (1.5pt);
					\draw[color=black] (3.1591466211181767,1.2139151202357141) node {$B$};
					\draw [fill=black] (3.,3.) circle (1.5pt);
					\draw[color=black] (3.1331328064059094,3.177958131011882) node {$C$};
					\draw [fill=black] (1.,3.) circle (1.5pt);
					\draw[color=black] (0.8699309264386685,3.1909650383680157) node {$D$};
					\draw [fill=black] (2.,4.) circle (1.5pt);
					\draw[color=black] (2.092580217915224,4.270538348927101) node {$E$};
					\draw [fill=black] (2.,3.) circle (1.5pt);
					\draw[color=black] (2.1576147546958917,3.203971945724149) node {$G$};
					\draw [fill=uuuuuu] (2.,2.) circle (1.5pt);
					\draw[color=uuuuuu] (1.819435163436419,1.7902743027546594) node {$H$};
					\draw [fill=black] (1.,2.) circle (1.5pt);
					\draw[color=black] (0.8179032970141343,2.189433171945731) node {$I$};
					\draw [fill=black] (3.,2.) circle (1.5pt);
					\draw[color=black] (3.1851604358304435,2.202440079301865) node {$J$};
					\draw [fill=black] (2.,1.) circle (1.5pt);
					\draw[color=black] (1.8714627928609533,0.8017493436885088) node {$K$};
					\end{scriptsize}
					\end{tikzpicture}
				\end{minipage}
				\hfill\vrule\hfill
				\begin{minipage}[c]{0.52\linewidth}
					{\renewcommand{\arraystretch}{2} 
						\begin{tabular}{|c|c|c|c|} 
							\hline
							deux vecteurs & même directions & même normes & même sens\\
							\hline
							$\V{AD}$ et $\V{BC}$ &  & & \\
							\hline 
							$\V{HG}$ et $\V{JB}$ & & &\\
							\hline 
							$\V{DE}$ et $\V{AC}$ & & &\\
							\hline 
							$\V{GC}$ et $\V{GD}$ & & &\\
							\hline 
					\end{tabular}}
				\end{minipage}
			\end{figure}
\end{exo}

\begin{exo} \textit{\textbf{Construction d'un triangle}}\\\\
\begin{minipage}[t]{1\linewidth}
\begin{minipage}[c]{0.45\linewidth}
\begin{enumerate}
\item Placer trois points non alignés K,L et M dans le plan quadrillé à droite et tracer le triangle KLM . \hfill\textbf{/1} \\
\item Construire le point N qui soit l'image du point M par la translation de vecteur $\V{KL}$.\hfill\textbf{/1} \\
\item Tracer en rouge le vecteur $\V{KN}$ puis tracer en vert un autre représentant du vecteur $\V{KN}$.\hfill\textbf{/1}\\
\item[4.] Donner la nature du quadrilatère KLNM en justifiant votre raisonnement.\hfill\textbf{/1}\\
\end{enumerate} 
\end{minipage} 
\hfill
\begin{minipage}[c]{0.45\linewidth}
 $$\begin{tikzpicture}[scale=0.5]
\draw[dotted] (0,0) grid (17,11);
\end{tikzpicture}$$	  
\end{minipage}

    \hfill\\\hfill\\
    .\dotfill \\\\
    .\dotfill

\end{minipage}

\end{exo}
\begin{exo} \textit{\textbf{Le milieu d'un segment}}\\
\begin{enumerate}
    \item Soient $A \left(x_A;y_A\right)$ et $C \left(x_C;y_C\right)$ deux points du plan. Donner la formule en fonction des coordonnées de $A$ et de $C$ pour trouver les coordonnées du milieu du segment $[AC]$.\hfill\textbf{/1}\\\\
 .\dotfill \\\\
  .\dotfill \\
 \item $A \left(3;5\right)$ et $C \left(-2;4\right)$ deux points du plan. Donner les coordonnées du milieu du segment $[AC]$.\hfill\textbf{/1}\\\\
 .\dotfill \\\\
 .\dotfill 
\end{enumerate}

\end{exo}
\end{document}





