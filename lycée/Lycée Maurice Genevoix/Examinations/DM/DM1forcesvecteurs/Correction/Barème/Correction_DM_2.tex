\documentclass[a4paper,10.9pt]{article}
\usepackage[utf8]{inputenc}
\usepackage[T1]{fontenc}
\usepackage{fancyhdr} % pour personnaliser les en-têtes
\usepackage{lastpage}
\usepackage[frenchb]{babel}
\usepackage{amsfonts,amssymb}
\usepackage{amsmath,amsthm}
\usepackage{paralist}
\usepackage{xspace}
\usepackage{xcolor}
\usepackage{variations}
\usepackage{xypic}
\usepackage{eurosym,multicol}
\usepackage{graphicx}
\usepackage[np]{numprint}
\usepackage{hyperref} 
\usepackage{listings} % pour écrire des codes avec coloration syntaxique  

\usepackage{tikz}
\usetikzlibrary{calc, arrows, plotmarks,decorations.pathreplacing}
\usepackage{colortbl}
\usepackage{multirow}
\usepackage[top=1.5cm,bottom=1.5cm,right=1.5cm,left=1.5cm]{geometry}

\newtheorem{defi}{Définition}
\newtheorem{thm}{Théorème}
\newtheorem{thm-def}{Théorème/Définition}
\newtheorem{rmq}{Remarque}
\newtheorem{prop}{Propriété}
\newtheorem{cor}{Corollaire}
\newtheorem{lem}{Lemme}
\newtheorem{ex}{Exemple}
\newtheorem{cex}{Contre-exemple}
\newtheorem{prop-def}{Propriété-définition}
\newtheorem{exer}{Exercice}
\newtheorem{nota}{Notation}
\newtheorem{ax}{Axiome}
\newtheorem{appl}{Application}
\newtheorem{csq}{Conséquence}
\theoremstyle{definition}
\newtheorem{exo}{Exercice}


\newcommand{\vtab}{\rule[-0.4em]{0pt}{1.2em}}
\newcommand{\V}{\overrightarrow}
\renewcommand{\thesection}{\Roman{section} }
\renewcommand{\thesubsection}{\arabic{subsection} }
\renewcommand{\thesubsubsection}{\alph{subsubsection} }
\newcommand{\C}{\mathbb{C}}
\newcommand{\R}{\mathbb{R}}
\newcommand{\Q}{\mathbb{Q}}
\newcommand{\Z}{\mathbb{Z}}
\newcommand{\N}{\mathbb{N}}


\definecolor{vert}{RGB}{11,160,78}
\definecolor{rouge}{RGB}{255,120,120}
% Set the beginning of a LaTeX document
\pagestyle{fancy}
\begin{document}
	
\lhead{Lycée Le Maurice Genevoix}\chead{}\rhead{Année~2021-2022}\lfoot{M. Botcazou}\cfoot{\thepage/3}\rfoot{\textbf{Tourner la page S.V.P.}}\renewcommand{\headrulewidth}{0.4pt}\renewcommand{\footrulewidth}{0.4pt}

\hfill\\[-1cm]
$$	\fbox{\text{\Large{ \sc Barème du Devoir Maison }}}$$	


\begin{exo}\textit{\textbf{Les vecteurs, des bons outils pour faire de la physique:}}  
\section*{Une pomme accrochée dans un arbre :}

\textbf{QUESTIONS:}\\
\begin{enumerate}
	\item 
	 \begin{enumerate}
	 	\item \textbf{1 points} \\
	 	\item \textbf{1.5 points} \\
	 \end{enumerate} 
	 
	\item \begin{enumerate}
		\item \textbf{1 points} \\
		\item \textbf{1.5 points} \\
	\end{enumerate}
	\item 
	\begin{enumerate}
		\item \textbf{1 points} \\
		\item \textbf{1.5 points} \\
	\end{enumerate}
	\item
	 \begin{enumerate}
		\item \textbf{1 points} \\
		\item \textbf{1.5 points} \\
	\end{enumerate}
	\item  \textbf{2 points} \\
\end{enumerate}
\section*{Une pomme qui chute vers le sol :}

\textbf{QUESTIONS:}\\
\begin{enumerate}
	\item[6.] 
	\begin{enumerate}
		\item \textbf{1 points} \\
		\item \textbf{1 points} \\
	\end{enumerate} 
	\item[7.]  
	\begin{enumerate}
		\item \textbf{1.5 points} \\
	
		\item  \textbf{1.5 points} \\
	\end{enumerate}
	\item[8.] \textbf{3 points} \\
	
	\textbf{BONUS:}\\
	\item[9.]  \textbf{1 points} \\
	
	
\end{enumerate}


\end{exo}
\end{document}