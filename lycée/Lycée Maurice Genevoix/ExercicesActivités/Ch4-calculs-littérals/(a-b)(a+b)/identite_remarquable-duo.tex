\documentclass[t,12pt]{beamer}
\usepackage[T1]{fontenc}
\usepackage[francais]{babel}
\usepackage{amssymb,amsmath,graphicx,amsthm}
\usepackage{tikz}
\usetikzlibrary{calc, arrows, plotmarks,decorations.pathreplacing}
\usepackage{colortbl}
\usepackage{multirow}



\usetheme{Warsaw}

\definecolor{zzt}{rgb}{0.6,0.2,0.}
\definecolor{cqc}{rgb}{0.7529411764705882,0.7529411764705882,0.7529411764705882}

\begin{document}

\begin{frame}
\frametitle{Visualiser graphiquement une Identit� Remarquable}
Soient $0< a < b$ deux r�els. Reproduire le sch�ma ci-dessous et  r�pondre aux questions suivantes: \\
\begin{minipage}[T]{1\linewidth}
\begin{minipage}[T]{0.5\linewidth}
$$\begin{tikzpicture}[scale=0.5]
\draw[dotted] (0,-3) grid (8,7);

\draw [-, line width = 0.6pt, >=latex',color=black](3.5,6)--(3.5,3.5)--(1,3.5)--(1,6)--cycle;
\fill[line width=2.pt,color=zzt,fill=zzt,fill opacity=0.7](3.5,6)--(3.5,3.5)--(1,3.5)--(1,6)--cycle;

\draw [-, line width = 0.6pt, >=latex',color=black](3.5,6)--(7,6)--(7,0)--(3.5,0)--cycle;
\fill[line width=2.pt,color=blue,fill=blue,fill opacity=0.4](3.5,6)--(7,6)--(7,0)--(3.5,0)--cycle;

\draw [-, line width = 0.6pt, >=latex',color=black](3.5,-2.5)--(3.5,0)--(7,0)--(7,-2.5)--cycle;
\fill[line width=2.pt,color=blue,fill=blue,fill opacity=0.4](3.5,-2.5)--(3.5,0)--(7,0)--(7,-2.5)--cycle;

\draw [-, line width = 0.6pt, >=latex',color=black](1,3.5)--(3.5,3.5)--(3.5,-2.5)--(1,-2.5)--cycle;
\fill[line width=2.pt,color=magenta,fill=magenta,fill opacity=0.4](1,3.5)--(3.5,3.5)--(3.5,-2.5)--(1,-2.5)--cycle;


\draw (2.2,6) node {\tiny$|$} (1,4.7) node {\tiny$-$} (7,-1.4)node {\tiny$-$}  ;
\draw (1,0.55)node {\tiny$-$} (1,0.45)node {\tiny$-$} (7,3.05)node {\tiny$-$} (7,2.95)node {\tiny$-$};
\begin{scriptsize}
\draw[color=black] (2.2,6.2) node[above] {$a$} ;
\draw[color=black] (5.1,6.1) node[above] {$b-a$} (7.2,3) node[right] {$b$};

\end{scriptsize}

\end{tikzpicture}$$
\end{minipage}
\hfill\begin{minipage}[T]{0.45\linewidth}
\begin{enumerate}
\item Donner l'aire des carr�s et des rectangles de diff�rentes couleurs sur ce sch�ma.
\item Donner l'aire du grand rectangle de deux mani�res diff�rentes. 
\item En d�duire la troisi�me identit� remarquable : $(b-a)(b+a)=b^2-a^2$
\end{enumerate}
\end{minipage}
\end{minipage}



\end{frame}

\end{document}

