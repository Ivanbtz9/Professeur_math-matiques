\documentclass[t,12pt]{beamer}
\usepackage[T1]{fontenc}
\usepackage[francais]{babel}
\usepackage{amssymb,amsmath,graphicx,amsthm}
\usepackage{tikz}
\usetikzlibrary{calc, arrows, plotmarks,decorations.pathreplacing}
\usepackage{colortbl}
\usepackage{multirow}



\usetheme{Warsaw}

\definecolor{zzt}{rgb}{0.6,0.2,0.}
\definecolor{cqc}{rgb}{0.7529411764705882,0.7529411764705882,0.7529411764705882}

\begin{document}

\begin{frame}
\frametitle{Visualiser graphiquement une Identit� Remarquable}
Soient $0< a < \dfrac{b}{2}$ deux r�els. Reproduire le sch�ma ci-dessous et  r�pondre aux questions suivantes: \\
\begin{minipage}[T]{1\linewidth}
\begin{minipage}[T]{0.5\linewidth}
$$\begin{tikzpicture}[scale=0.5]
\draw[dotted] (0,-3) grid (10,7);

\draw [-, line width = 0.6pt, >=latex',color=black](4,6)--(4,3)--(1,3)--(1,6)--cycle;
\fill[line width=2.pt,color=zzt,fill=zzt,fill opacity=0.7](4,6)--(4,3)--(1,3)--(1,6)--cycle;

\draw [-, line width = 0.6pt, >=latex',color=black](4,6)--(9,6)--(9,1)--(4,1)--cycle;
\fill[line width=2.pt,color=blue,fill=blue,fill opacity=0.4](4,6)--(9,6)--(9,1)--(4,1)--cycle;

\draw [-, line width = 0.6pt, >=latex',color=black](1,3)--(4,3)--(4,1)--(1,1)--cycle;
\fill[line width=2.pt,color=magenta,fill=magenta,fill opacity=0.4](1,3)--(4,3)--(4,1)--(1,1)--cycle;

\draw [-, line width = 0.6pt, >=latex',color=black](1,-2)--(1,1)--(9,1)--(9,-2)--cycle;
\fill[line width=2.pt,color=teal,fill=teal,fill opacity=0.4](1,-2)--(1,1)--(9,1)--(9,-2)--cycle;

\draw (2.5,6) node {\tiny$|$} (1,4.5) node {\tiny$-$}  ;
\draw (1,2.05)node {\tiny$-$} (1,1.95)node {\tiny$-$} (4,2.05)node {\tiny$-$} (4,1.95)node {\tiny$-$};
\begin{scriptsize}
\draw[color=black] (2.5,6.2) node[above] {$a$} ;
\draw[color=black] (6.5,6.2) node[above] {$b-a$} (1,-0.5) node[left] {$a$};

\end{scriptsize}

\end{tikzpicture}$$
\end{minipage}
\hfill\begin{minipage}[T]{0.45\linewidth}
\begin{enumerate}
\item Donner l'aire des carr�s et des rectangles de diff�rentes couleurs sur ce sch�ma.
\item Donner l'aire du grand carr� de deux mani�res diff�rentes. 
\item En d�duire la deuxi�me identit� remarquable : $(a-b)^2=a^2-2ab +b^2$
\end{enumerate}
\end{minipage}
\end{minipage}



\end{frame}

\end{document}

