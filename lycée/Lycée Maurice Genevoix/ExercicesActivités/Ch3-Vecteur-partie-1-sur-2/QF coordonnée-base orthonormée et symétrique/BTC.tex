\documentclass[t,12pt]{beamer}
\usepackage[T1]{fontenc}
\usepackage[francais]{babel}
\usepackage{amssymb,amsmath,graphicx,amsthm}


\newcommand{\vtab}{\rule[-0.4em]{0pt}{1.2em}}
\newcommand{\V}{\overrightarrow}
\renewcommand{\thesection}{\Roman{section} }
\renewcommand{\thesubsection}{\arabic{subsection} }
\renewcommand{\thesubsubsection}{\alph{subsubsection} }
\newcommand{\C}{\mathbb{C}}
\newcommand{\R}{\mathbb{R}}
\newcommand{\Q}{\mathbb{Q}}
\newcommand{\Z}{\mathbb{Z}}
\newcommand{\N}{\mathbb{N}}


\usetheme{Warsaw}


\begin{document}

\begin{frame}
\frametitle{Exercice pr�paratoire au contr�le du 15/12/21}

Soient $(O,\V{i}, \V{j})$ un rep�re orthonorm� du plan et $B(-6;-5)$, $C(-2;7)$ deux points du plan.\\\hfill\\ 

\begin{enumerate}
\item Donner les coordonn�es du vecteur $\V{BC}$ et d�composer ce vecteur dans la base orthonorm�e  $(\V{i}, \V{j})$.\\\hfill\\ 

\item On note $F$ le sym�trique du point $B$ par rapport au point $C$. Donner les coordonn�es du point $F$.\\\hfill\\

\item Donner les coordonn�es du vecteur $4\V{CF}$ \\\hfill\\ 

 
\end{enumerate}
\end{frame}

\begin{frame}

\end{frame}

\begin{frame}
\frametitle{�valuation sur 5 points:}

Soient $(O,\V{i}, \V{j})$ un rep�re orthonorm� du plan et $K(-8;-1)$, $L(2;-7)$ deux points du plan.\\\hfill\\ 

\begin{enumerate}
\item Donner les coordonn�es du vecteur $\V{LK}$ et d�composer ce vecteur dans la base orthonorm�e  $(\V{i}, \V{j})$.\hfill\textbf{/1}\\\hfill\\ 

\item On note $M$ le sym�trique du point $L$ par rapport au point $K$. Donner les coordonn�es du point $M$.\hfill\textbf{/2.5}\\\hfill\\

\item Donner les coordonn�es du vecteur $6\V{KM}$\hfill\textbf{/1.5} \\\hfill\\ 

 
\end{enumerate}
\end{frame}

\end{document}