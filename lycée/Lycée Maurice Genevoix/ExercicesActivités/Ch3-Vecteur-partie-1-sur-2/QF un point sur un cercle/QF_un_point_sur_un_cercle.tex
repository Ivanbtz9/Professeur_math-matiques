\documentclass[t,12pt]{beamer}
\usepackage[utf8]{inputenc}
\usepackage[T1]{fontenc}
\usepackage[frenchb]{babel}
\usepackage{amssymb,amsmath,graphicx,amsthm}


\newcommand{\vtab}{\rule[-0.4em]{0pt}{1.2em}}
\newcommand{\V}{\overrightarrow}
\renewcommand{\thesection}{\Roman{section} }
\renewcommand{\thesubsection}{\arabic{subsection} }
\renewcommand{\thesubsubsection}{\alph{subsubsection} }
\newcommand{\C}{\mathbb{C}}
\newcommand{\R}{\mathbb{R}}
\newcommand{\Q}{\mathbb{Q}}
\newcommand{\Z}{\mathbb{Z}}
\newcommand{\N}{\mathbb{N}}


\usetheme{Warsaw}


\begin{document}

\begin{frame}
\frametitle{Norme d'un vecteur et ses utilités }
Soit (O,I,J) un repère orthonormé du plan. \\[0.25cm]
\begin{enumerate}
	\item Écrire avec vos mots la définition de ce qu'est le cercle $\mathcal{C}$ de centre le point $A(2;1)$ et de rayon $\sqrt{5}$.\\[0.25cm]
	\item Soit $B(4,0)$, donner les coordonnées du vecteur $\V{AB}$.\\[0.25cm]
	\item Calculer $||\V{AB}||$. \\[0.25cm]
	\item Le point $B$ est-il un point du cercle $\mathcal{C}$?\\[0.25cm]
	\item Trouver un point qui n'appartient pas au cercle $\mathcal{C}$ en donnant une justification. 
\end{enumerate}

\end{frame}

\begin{frame}
	
\end{frame}

\begin{frame}
	\frametitle{Test de 5 points}
	Soit (O,I,J) un repère orthonormé du plan. \\[0.25cm]
	\begin{enumerate}
		\item Écrire avec vos mots la définition de ce qu'est le cercle $\mathcal{C}$ de centre le point $A(1;-1)$ et de rayon $\sqrt{8}$.\\[0.25cm]
		\item Soit $B(3,1)$, donner les coordonnées du vecteur $\V{AB}$.\\[0.25cm]
		\item Calculer $||\V{AB}||$. \\[0.25cm]
		\item Le point $B$ est-il un point du cercle $\mathcal{C}$?\\[0.25cm]
	\end{enumerate}
	
\end{frame}



\end{document}