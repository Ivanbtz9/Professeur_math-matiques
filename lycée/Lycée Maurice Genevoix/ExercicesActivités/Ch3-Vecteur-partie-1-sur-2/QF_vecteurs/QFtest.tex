\documentclass[t,12pt]{beamer}
\usepackage[utf8]{inputenc}
\usepackage[T1]{fontenc}
\usepackage[frenchb]{babel}
\usepackage{amssymb,amsmath,graphicx,amsthm}


\newcommand{\vtab}{\rule[-0.4em]{0pt}{1.2em}}
\newcommand{\V}{\overrightarrow}
\renewcommand{\thesection}{\Roman{section} }
\renewcommand{\thesubsection}{\arabic{subsection} }
\renewcommand{\thesubsubsection}{\alph{subsubsection} }
\newcommand{\C}{\mathbb{C}}
\newcommand{\R}{\mathbb{R}}
\newcommand{\Q}{\mathbb{Q}}
\newcommand{\Z}{\mathbb{Z}}
\newcommand{\N}{\mathbb{N}}


\usetheme{Warsaw}


\begin{document}

\begin{frame}
\frametitle{Question 1}

Soient $K(-2;-7)$, $C(-2;6)$ deux points du plan.\\\hfill\\ 


Donner les coordonnées du vecteur $\V{KC}$ et décomposer ce vecteur dans la base orthonormée  $(\V{i}, \V{j})$.\\\hfill\\ 

\end{frame}

\begin{frame}
\frametitle{Question 2}
Résoudre  le système d'équations suivant:

$$\left \{
\begin{array}{rcl}
9x_C+ 2&= & 12 \\
8y_C+1&= & 17
\end{array}
\right.$$
\end{frame}

\begin{frame}
\frametitle{Question 3}

Soient $K(\dfrac{1}{3};\dfrac{6}{12})$, $M(\dfrac{2}{3};\dfrac{6}{12})$ deux points du plan.\\\hfill\\ 


Donner les coordonnées du point $V$ qui est au milieu du segment $[KM]$ .\\\hfill\\ 

\end{frame}

\begin{frame}
\frametitle{Question 4}


Soient $
\V{UV}\begin{pmatrix}
 12 \\
16
\end{pmatrix} \  \text{et}  \ \V{MN}\begin{pmatrix}
 1 \\
-7
\end{pmatrix}
$\\\hfill\\[0.5cm]

Donner les coordonnées de  $ \dfrac{1}{4}\V{UV} + \V{MN} $
\end{frame}

\end{document}