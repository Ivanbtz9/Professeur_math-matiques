\documentclass[a4paper,10pt]{article}
%\usepackage{calc}
%\usepackage{esvect}
%\setlength{\parindent}{0mm}
%\usepackage [alwaysadjust]{paralist}
%\renewcommand*{\tabularxcolumn}[1]{m{#1}} %centrage vertical des cellules d'un tableau tabularx
\setlength{\oddsidemargin}{0.25in}

% Set width of the text - What is left will be the right margin.
% In this case, right margin is 8.5in - 1.25in - 6in = 1.25in.
\setlength{\textwidth}{6in}

% Set top margin - The default is 1 inch, so the following 
% command sets a 0.75-inch top margin.
\setlength{\topmargin}{-0.25in}
% Set height of the text - What is left will be the bottom margin.
% In this case, bottom margin is 11in - 0.75in - 9.5in = 0.75in
\setlength{\textheight}{8in}
\usepackage[latin1]{inputenc}
%\usepackage{fancyhdr} % pour personnaliser les en-t�tes
\usepackage{lastpage}
%\pagestyle{fancy}
\usepackage[frenchb]{babel}
\usepackage{amsfonts,amssymb}
\usepackage{amsmath,amsthm}
\usepackage{paralist}
\usepackage{xspace}
\usepackage{xcolor}
\usepackage{variations}
\usepackage{xypic}
\usepackage{eurosym}
\usepackage{graphicx}
\usepackage{hyperref} 
\usepackage{listings} % pour �crire des codes avec coloration syntaxique  
\lstdefinestyle{customc}{
	belowcaptionskip=1\baselineskip,
	breaklines=true,
	frame=none,
	backgroundcolor=\color{gray!20!},
	xleftmargin={1em},
	language=html,
	showstringspaces=false,
	basicstyle=\small\ttfamily,
	keywordstyle=\bfseries\color{red!60!black},
	commentstyle=\color{blue},
	identifierstyle=\color{black},
	stringstyle=\color{black},
	extendedchars=true,
	literate={�}{{\`a}}1 {�}{{\'e}}1 {�}{{\`e}}1 {�}{{\`u}}1 {�}{{\^e}}1,
}
\lstset{escapechar=@,style=customc,numbers=left, stepnumber=1}

\usepackage{xlop} % pour poser les op�rations � la main
\usepackage{pstricks,pst-plot} 
\usepackage{tikz}
\usetikzlibrary{calc, arrows, plotmarks, babel,decorations.pathreplacing}
\usepackage{colortbl}
\usepackage{multirow}
\usepackage[top=1.5cm,bottom=1.5cm,right=1.25cm,left=1.25cm]{geometry}
\newtheorem{defi}{D\'{e}finition}
\newtheorem{thm}{Th\'{e}or\`{e}me}
\newtheorem{thm-def}{Th�or�me-d�finition}
\newtheorem{rmq}{Remarque}
\newtheorem{prop}{Propri\'{e}t\'{e}}
\newtheorem{cor}{Corollaire}
\newtheorem{lem}{Lemme}
\newtheorem{ex}{Exemple}
\newtheorem{cex}{Contre-exemple}
\newtheorem{prop-def}{Propri�t�-d�finition}
\newtheorem{exer}{Exercice}
\newtheorem{nota}{Notation}
\newtheorem{ax}{Axiome}
\newtheorem{appl}{Application}
\newtheorem{csq}{Cons�quence}
\def\di{\displaystyle}

\newcommand{\vtab}{\rule[-0.4em]{0pt}{1.2em}}
%\newcommand{\rectvert}{\fcolorbox{green}{green}{\begin{minipage}{1\linewidth}\end{minipage}}}
%\newenvironment{rectvert}{\fcolorbox{green}{green}\begin{minipage}{1\linewidth}}{\end{minipage}}

\newcommand{\V}{\overrightarrow}
\definecolor{vert}{RGB}{11,160,78}
\definecolor{rouge}{RGB}{255,120,120}
\renewcommand{\thesection}{\Roman{section} }
\renewcommand{\thesubsection}{\arabic{subsection} }
\renewcommand{\thesubsubsection}{\alph{subsubsection} }

\newcommand\opidivzarbi[3][nil]{%
	\begingroup
	\opset{#1}%
	\opidiv*{#2}{#3}{q}{r}%
	\begin{tabular}{r|l}
		\opdisplay{operandstyle.1}{#2} &
		\opdisplay{operandstyle.2}{#3} \\\cline{2-2}
		\opdisplay{remainderstyle}{r} &
		\opdisplay{resultstyle}{q}
	\end{tabular}%
	\endgroup
}

% Set the beginning of a LaTeX document
\begin{document}
	
	%\fancyhead{}
	%\fancyfoot[L]{Chapitre 3 - Activit� d'introduction}
	%\fancyfoot[C]{Page \thepage/2}
	%\fancyfoot[R]{Mme Beudez}
	%\renewcommand{\footrulewidth}{0.4pt}
	
	
	
	\title{}         % Enter your title between curly braces
	\author{}        % Enter your name between curly braces
	\date{}          % Enter your date or \today between curly braces
	\maketitle
	
	\noindent Dans le rep�re orthonorm� $(O;I,J)$ ci-dessous, lire les coordonn�es des points $A$, $B$ et $C$.\\
	\\
%	
	$$\begin{tikzpicture}[scale=1]
	\draw[dotted] (-6,-6) grid (6,6);
	\draw [->, line width = 1pt, >=latex'](-6,0) -- (6,0);
	\draw [->, line width = 1pt, >=latex'](0,-6) -- (0,6);
	\draw[line width = 1 pt] (0,0) node{$+$} node[below left]{$O$};
	\draw[line width = 1 pt] (1,0) node{$+$} node[below]{$I$};
	\draw[line width = 1 pt] (0,1) node{$+$} node[left]{$J$};
	\draw[line width = 1 pt] (2,5) node{$+$} node[below left]{$A$};
	\draw[line width = 1 pt] (-3,4) node{$+$} node[below left]{$B$};
	\draw[line width = 1 pt] (3,0) node{$+$} node[below]{$C$};	
	\end{tikzpicture}$$   
	
	
	\newpage
	
	
	\noindent Dans le rep�re orthonorm� $(O;I,J)$ ci-dessous, lire les coordonn�es des points $A$, $B$ et $C$.\\
	\\
	%	
	$$\begin{tikzpicture}[scale=1]
	\draw[dotted] (-6,-6) grid (6,6);
	\draw [->, line width = 1pt, >=latex'](-6,0) -- (6,0);
	\draw [->, line width = 1pt, >=latex'](0,-6) -- (0,6);
	\draw[line width = 1 pt] (0,0) node{$+$} node[below left]{$O$};
	\draw[line width = 1 pt] (1,0) node{$+$} node[below]{$I$};
	\draw[line width = 1 pt] (0,1) node{$+$} node[left]{$J$};
	\draw[line width = 1 pt] (-3,2) node{$+$} node[below left]{$A$};
	\draw[line width = 1 pt] (0,4) node{$+$} node[below left]{$B$};
	\draw[line width = 1 pt] (2,-1) node{$+$} node[below]{$C$};	
	\end{tikzpicture}$$   
	
	
	
	
% Set the ending of a LaTeX document
\end{document}
 
