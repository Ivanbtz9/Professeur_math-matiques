\documentclass[a4paper,10pt]{article}

\usepackage[utf8]{inputenc}
\usepackage[T1]{fontenc}
\usepackage{fancyhdr} % pour personnaliser les en-têtes
\usepackage{lastpage}
\usepackage[frenchb]{babel}
\usepackage{amsfonts,amssymb}
\usepackage{amsmath,amsthm}
\usepackage{paralist}
\usepackage{xspace}
\usepackage{xcolor}
\usepackage{variations}
\usepackage{xypic}
\usepackage{eurosym}
\usepackage{graphicx}
\usepackage[np]{numprint}
\usepackage{hyperref} 
\usepackage{listings} % pour écrire des codes avec coloration syntaxique  

\usepackage{tikz}
\usetikzlibrary{calc, arrows, plotmarks, babel,decorations.pathreplacing}
\usepackage{colortbl}
\usepackage{multirow}
\usepackage[top=1.5cm,bottom=1.5cm,right=1.5cm,left=1.5cm]{geometry}
\newtheorem{defi}{Définition}
\newtheorem{thm}{Théorème}
\newtheorem{rmq}{Remarque}
\newtheorem{prop}{Propriété}
\newtheorem{cor}{Corollaire}
\newtheorem{lem}{Lemme}
\newtheorem{ex}{Exemple}
\newtheorem{cex}{Contre-exemple}
\newtheorem{prop-def}{Propriété-définition}
\newtheorem{exer}{Exercice}
\newtheorem{nota}{Notation}
\newtheorem{ax}{Axiome}
\newtheorem{appl}{Application}
\newtheorem{csq}{Conséquence}
\def\di{\displaystyle}


\newcommand{\vtab}{\rule[-0.4em]{0pt}{1.2em}}
\newcommand{\V}{\overrightarrow}
\newcommand{\C}{\mathbb{C}}
\newcommand{\R}{\mathbb{R}}
\newcommand{\Q}{\mathbb{Q}}
\newcommand{\Z}{\mathbb{Z}}
\newcommand{\N}{\mathbb{N}}

\renewcommand{\thesection}{\Roman{section} }
\renewcommand{\thesubsection}{\arabic{subsection} }
\renewcommand{\thesubsubsection}{\alph{subsubsection} }
\renewcommand{\thesection}{\Roman{section} }
\renewcommand{\thesubsection}{\arabic{subsection} }
\renewcommand{\thesubsubsection}{\alph{subsubsection} }

\definecolor{vert}{RGB}{11,160,78}
\definecolor{rouge}{RGB}{255,120,120}


\pagestyle{fancy}

\begin{document}
	\lhead{}\chead{}\rhead{}\lfoot{Chapitre 8 - Exercice de groupe}\cfoot{\thepage/5}\rfoot{M. Botcazou}\renewcommand{\headrulewidth}{0pt}\renewcommand{\footrulewidth}{0.4pt}
	$$\fbox{\text{\Large{ La fonction carré}}}$$
	\hfil\\
	
	\noindent La fonction carré est donnée par la relation algébrique suivante \quad
	$f:~x \longmapsto x^2$\\
	\begin{enumerate}
		\item Pour tout nombre réel $x\in\R$ pouvons-nous mettre ce nombre au carré ? En déduire le domaine de définition de la fonction carré. \\
		\item Grâce à un tableau de valeurs, tracer dans un repère orthogonal la courbe associée à la fonction carré sur l'intervalle $[-3~;~3]$ .\\
		\textit{(Indication: faites des pas de 0.25 pour choisir vos nombres dans l'intervalle $[-3~;~3]$ )}\\ 
		
		\noindent\fcolorbox{vert}{white}{
			\begin{minipage}{1\linewidth}
				\begin{defi}
					Dans un repère orthogonal, la courbe représentative de la fonction carrée est appelée \textbf{parabole}.
				\end{defi}
			\end{minipage}}
		\hfill\\[-0.5cm]
		
		\item Par lecture graphique donner le tableau de variations de la fonction carré.\\   
		\item Par lecture graphique donner le tableau de signes de la fonction carré.\\
		\item Remarquez-vous une symétrie sur la courbe représentative de la fonction carré ? Si oui laquelle ?\\
		\item
		\begin{enumerate}
			\item En utilisant les données calculées pour tracer la courbe de la fonction carrée, comparer les nombres suivants:\\\\ 
			$\begin{array}{cccc}
			
			\bullet\quad f(-3)~~...~~f(3) 
			&\bullet\quad f(-2)~~...~~f(2)&
			\bullet\quad f(-1.25)~~...~~f(1.25) &\bullet\quad f(-0.75)~~...~~f(0.75) \\[3mm]		
			\end{array}$\\
			\item Soit $a\geq0$, comparer les deux nombres suivants: \quad $f(-a)~~...~~f(a)$\\
			\textit{(Si vous trouvez cela trop abstrait repenser au point précédent)}\\\hfill\\  
		 \noindent\fcolorbox{rouge}{white}{\begin{minipage}{1\linewidth}\begin{prop}
		 		Soit $f$ une fonction définie sur $\R$. La fonction $f$ est dite \textbf{paire} sur $\R$ si pour tout $x \geq 0$ 
		 		$$f(-x) \ = \ f(x)$$
		 		La courbe représentative de la fonction $f$ présentera une symétrie par rapport à l'axe des ordonnées.
		 		\end{prop}\end{minipage}}\hfill\\
	 \item Que pouvez-vous dire sur la parité de la fonction carré?\\
	   
 \end{enumerate}
\item Résoudre graphiquement et  algébriquement les inéquations suivantes:\\\\
$\begin{array}{lclcl}
a)\quad f(x) \ \leq \ 4
&\quad\quad\quad&b)\quad f(x) \ \geq \ 3&\quad\quad\quad&  
c)\quad f(x) \ < \ 7

\end{array}$\\
	

	
	\item  L'objectif est de démontrer pourquoi la fonction carré est décroissante sur l'intervalle $]-\infty~;~0]$ \\et croissante sur l'intervalle $[0~;~+\infty[$\\
	\begin{enumerate}
	\item  Soit $a$ et $b$ deux réels. À l'aide d'une identité remarquable donner une forme factorisée de $$f(a)-f(b) \ = \ a^2 - b^2$$ 

	\item Si $a \leq b \leq 0$ que pouvez-vous dire du signe de $f(a)-f(b)$. La fonction carré respecte-t-elle l'ordre entre les abscisses et les ordonnées sur $]-\infty~;~0]$ ? Qu'en déduisez-vous?\\
	\item Si $0\leq a \leq b$ que pouvez-vous dire du signe de $f(a)-f(b)$. La fonction carré respecte-t-elle l'ordre entre les abscisses et les ordonnées sur $[0~;~+\infty[$ ? Qu'en déduisez-vous?\\
	\end{enumerate} 
	\end{enumerate}
	\newpage
		$$\fbox{\text{\Large{ La fonction cube}}}$$
	\hfil\\

		\noindent La fonction cube est donnée par la relation algébrique suivante \quad
	$f:~x \longmapsto x^3$\\
	\begin{enumerate}
		\item Pour tout nombre réel $x\in\R$ pouvons-nous mettre ce nombre au cube ? En déduire le domaine de définition de la fonction cube. \\
		\item Grâce à un tableau de valeurs, tracer dans un repère orthogonal la courbe associée à la fonction cube sur l'intervalle $[-3~;~3]$ .\\
		\textit{(Indication: faites des pas de 0.5 pour choisir vos nombres dans l'intervalle $[-3~;~3]$ )}\\ 
		\item Par lecture graphique donner le tableau de variations de la fonction cube.\\   
		\item Par lecture graphique donner le tableau de signes de la fonction cube.\\
		\item Remarquez-vous une symétrie sur la courbe représentative de la fonction cube ? Si oui laquelle ?\\
		\item
		\begin{enumerate}
			\item En utilisant les données calculées pour tracer la courbe de la fonction cube, écrire le nombre de gauche en fonction du nombre de droite pour chacun des cas suivants\\\\ 
			$\begin{array}{cccc}
			
			\bullet\quad f(-3)~~et~~f(3) 
			&\bullet\quad f(-2)~~et~~f(2)&
			\bullet\quad f(-1.5)~~et~~f(1.5) &\bullet\quad f(-0.5)~~et~~f(0.5) \\[3mm]		
			\end{array}$\\
			\item Soit $a\geq0$, écrire le nombre $f(-a)$ en fonction du nombre $f(a)$\\
			\textit{(Si vous trouvez cela trop abstrait repenser au point précédent)}\\  
		
		\noindent\fcolorbox{rouge}{white}{\begin{minipage}{1\linewidth}\begin{prop}
				Soit $f$ une fonction définie sur $\R$. La fonction $f$ est dite \textbf{impaire} sur $\R$ si pour tout $x \geq 0$ 
				$$f(-x) \ = \ - f(x)$$
				La courbe représentative de la fonction $f$ présentera une symétrie par rapport à l'origine $O$ du repère.	
		\end{prop}\end{minipage}}\hfill\\
			\item Que pouvez-vous dire sur la parité de la fonction cube?\\
			
		\end{enumerate}
		\item Résoudre graphiquement et  algébriquement les inéquations suivantes:\\\\
		$\begin{array}{lclcl}
		a)\quad f(x) \ \leq \ 1
		&\quad\quad\quad&b)\quad f(x) \ \geq \ -8&\quad\quad\quad&  
		c)\quad f(x) \ < \ \dfrac{1}{27}
		
		\end{array}$\\
		
	
	 
		\item Soit $a\in\R$, l'objectif est d'étudier la fonction donnée par la relation algébrique suivante \quad
		$$g_a:~x \longmapsto (x+a)^3$$
		\begin{enumerate}
			\item À l'aide d'un logiciel de géométrie dynamique (Geogebra ou une calculatrice graphique), représenter la courbe de la fonction ~$g_a$~ pour différentes valeurs de $a$.\\
			\textit{(Vous pouvez prendre $a=-1$, $a=-0.5$, $a=0$, $a=0.5$ et $a=1$. Créer un curseur pourra vous aider.)}\\
			\item Pour chacune des valeurs de $a$, étudier l'image de $-a$ par la fonction ~$g_a$. Que remarquez-vous?\\			
			\item Donner l'expression développée de $(x+a)^3$. C'est une identité remarquable du troisième degré que je vous invite à apprendre par-cœur. 
		\end{enumerate}
	\end{enumerate}	
	\newpage
	
	$$\fbox{\text{\Large{ La fonction inverse}}}$$
	\hfil\\\hfil\\
	\noindent La fonction inverse est donnée par la relation algébrique suivante \quad
	$f:~x \longmapsto \dfrac{1}{x}$\\
	\begin{enumerate}
		\item Pour tout nombre réel $x\in\R$ pouvons-nous diviser le nombre $1$ par le nombre $x$ ? En déduire le domaine de définition de la fonction inverse. \\
		\item Grâce à un tableau de valeurs, tracer dans un repère orthogonal la courbe associée à la fonction inverse sur $[-4~;~0[~\cup~]0~;~4]$ \\
		\textit{(Indication: choisissez (2cm = 1 unité) pour l'échelle de votre repère et faites des pas de 0.25 pour choisir vos nombres)}\\ \hfill\\
		\noindent\fcolorbox{vert}{white}{
			\begin{minipage}{1\linewidth}
				\begin{defi}
					Dans un repère orthogonal, la courbe représentative de la fonction inverse est appelée \textbf{hyperbole}.
				\end{defi}
		\end{minipage}}\hfill\\
		\item Par lecture graphique donner le tableau de variations de la fonction inverse. La fonction inverse est-elle décroissante sur $\R^*$?\\
	  
		\item Par lecture graphique donner le tableau de signes de la fonction inverse.\\
		\item Remarquez-vous une symétrie sur la courbe représentative de la fonction inverse ? Si oui laquelle ?\\
		\item
		\begin{enumerate}
			\item En utilisant les données calculées pour tracer la courbe de la fonction inverse, écrire le nombre de gauche en fonction du nombre de droite pour chacun des cas suivants:\\\\ 
			$\begin{array}{cccc}
			
			\bullet\quad f(-3)~~et~~f(3) 
			&\bullet\quad f(-2)~~et~~f(2)&
			\bullet\quad f(-1.5)~~et~~f(1.5) &\bullet\quad f(-0.5)~~et~~f(0.5) \\[3mm]		
			\end{array}$\\
			\item Soit $a\geq0$, écrire le nombre $f(-a)$ en fonction du nombre $f(a)$\\
			\textit{(Si vous trouvez cela trop abstrait repenser au point précédent)}\\  
			%\noindent\fcolorbox{rouge}{white}{\begin{minipage}{1\linewidth}\begin{prop}
			%Soit $f$ une fonction définie sur $\R^*$. La fonction $f$ est dite \textbf{impaire} sur $\R^*$ si pour tout $ x > 0$ 
			%$$f(-x) \ = \ - f(x)$$
			%La courbe représentative de la fonction $f$ présentera une symétrie par rapport à l'origine $O$ du repère. 
			%\end{prop}\end{minipage}}
			\item Que pouvez-vous dire sur la parité de la fonction inverse?\\
			
		\end{enumerate}
		\item Résoudre graphiquement et  algébriquement les inéquations suivantes:\\\\
		$\begin{array}{lclcl}
		a)\quad f(x) \ \leq \ 0
		&\quad\quad\quad&b)\quad f(x) \ \geq \ 2&\quad\quad\quad&  
		c)\quad f(x) \ \geq \ -4
		
		\end{array}$\\
	
		\item L'objectif est de démontrer pourquoi la fonction inverse est décroissante sur l'intervalle $]0~;~+\infty[$\\
		\begin{enumerate}
			\item Soit $a$ et $b$ deux réels strictement positifs. Montrer que $$f(a)-f(b) \ = \ \dfrac{b-a}{ab}$$ 
			\item Si $0 < a \leq b $ que pouvez-vous dire du signe de $f(a)-f(b)$. La fonction inverse respecte-t-elle l'ordre entre les abscisses et les ordonnées sur $[0~;~+\infty[$ ? Qu'en déduisez-vous?\\
		\end{enumerate}
		
			 
	\end{enumerate}
	\newpage
	 $$\fbox{\text{\Large{ La fonction valeur absolue}}}$$
	 \hfil\\\hfil\\
	\noindent La fonction valeur absolue donne la partie positive d'un nombre, elle est définie par la relation algébrique \\suivante \quad
	$f:~x \longmapsto |x|$\\
	\begin{rmq}
		On rappelle que: $ |x| \ = \ \Big\{\normalsize\begin{array}{cl}
			-x &\ \text{si} \ x \leq 0\\
			x &\ \text{si} \ x > 0
		\end{array} $
	\end{rmq}
	\begin{enumerate}
		\item Pour tout nombre réel $x\in\R$ pouvons-nous prendre la partie positive de ce nombre? En déduire le domaine de définition de la fonction valeur absolue. \\
		\item Grâce à un tableau de valeurs, tracer dans un repère orthonormé la courbe associée à la fonction valeur absolue sur l'intervalle $[-5~;~5]$ .\\ \hfil\\
		\noindent\fcolorbox{vert}{white}{
			\begin{minipage}{1\linewidth}
				\begin{defi}
					La fonction valeur absolue est une fonction \textbf{ affine par morceaux}.
				\end{defi}
		\end{minipage}}\hfil\\
		\item Par lecture graphique donner le tableau de variations de la fonction valeur absolue.\\   
		\item Par lecture graphique donner le tableau de signes de la fonction valeur absolue.\\
		\item Remarquez-vous une symétrie sur la courbe représentative de la fonction valeur absolue ? Si oui laquelle ?\\
		\item
		\begin{enumerate}
			\item En utilisant les données calculées pour tracer la courbe de la fonction valeur absolue, comparer les nombres suivants:\\\\ 
			$\begin{array}{cccc}
			
			\bullet\quad f(-3)~~...~~f(3) 
			&\bullet\quad f(-2)~~...~~f(2)&
			\bullet\quad f(-1.25)~~...~~f(1.25) &\bullet\quad f(-0.75)~~...~~f(0.75) \\[3mm]		
			\end{array}$\\
			\item Soit $a\geq0$, comparer les deux nombres suivants: \quad $f(-a)~~...~~f(a)$\\
			\textit{(Si vous trouvez cela trop abstrait repenser au point précédent)}\\  
			
			%\noindent\fcolorbox{rouge}{white}{\begin{minipage}{1\linewidth}\begin{prop}Soit $f$ une fonction définie sur $\R$. La fonction $f$ est dite \textbf{paire} sur $\R$ si pour tout $x \geq 0$ $$f(-x) \ = \ f(x)$$ La courbe représentative de la fonction $f$ présentera une symétrie par rapport à l'axe des ordonnées.\end{prop}\end{minipage}}\hfil\\
			\item Que pouvez-vous dire sur la parité de la fonction valeur absolue?\\
			
		\end{enumerate}
		\item Résoudre graphiquement et  algébriquement les inéquations suivantes:\\\\
		$\begin{array}{lclcl}
		a)\quad f(x) \ \leq \ 4
		&\quad\quad\quad&b)\quad f(x) \ \geq \ 3&\quad\quad\quad&  
		c)\quad f(x) \ > \ -2
		
		\end{array}$\\
		
		\item Soit $a\in\R$, l'objectif est d'étudier la fonction donnée par la relation algébrique suivante \quad
		$$g_a:~x \longmapsto |x-a|$$ 
		  \begin{enumerate}
			\item À l'aide d'un logiciel de géométrie dynamique (Geogebra ou une calculatrice graphique), représenter la courbe de la fonction ~$g_a$~ pour différentes valeurs de $a$.\\
			\textit{(Vous pouvez prendre $a=-2$, $a=-1$, $a=0$, $a=1$ et $a=2$. Créer un curseur pourra vous aider.)}\\
			\item Pour chacune des valeurs de $a$, étudier l'image de $a$ par la fonction ~$g_a$. Que remarquez-vous?\\			
		\end{enumerate}
		
	\end{enumerate}
\newpage$$\fbox{\text{\Large{ La fonction racine carrée}}}$$
\hfill\\\hfil\\
\noindent La fonction racine carrée est donnée par la relation algébrique suivante \quad
$f:~x \longmapsto \sqrt{x}$\\
\begin{enumerate}
	\item Pour tout nombre réel $x\in\R$ existe-t-il une racine carrée du nombre $x$ ? En déduire le domaine de définition de la fonction racine carrée. \\
	\item Grâce à un tableau de valeurs, tracer dans un repère orthogonal la courbe associée à la fonction racine carré sur l'intervalle $[0~;~16]$ \\
	\textit{(Indication: regardez l'image des nombres $\{0;1;4;9;16\}$ et arrondissez les autres images à $10^{-1}$ près. )}\\ 
	\item Par lecture graphique donner le tableau de variations de la fonction racine carrée.\\   
	\item Par lecture graphique donner le tableau de signes de la fonction racine carrée.\\
	\item Résoudre graphiquement et  algébriquement les inéquations suivantes:\\\\
	$\begin{array}{lclcl}
	a)\quad f(x) \ \leq \ 2
	&\quad\quad\quad&b)\quad f(x) \ \geq \ -2&\quad\quad\quad&  
	c)\quad f(x) \ < \ 8
	
\end{array}$\\
\item Choisissez deux nombres $a$ et $b$ positifs.
\begin{enumerate}
\item Calculer les valeurs suivantes:~
$f(a+b)$, $f(a)$ et $f(b)$\\
\item Comparer les deux valeurs ci-dessous grâce à l'un des symboles $(\leq~,~= ~,~\geq)$
$$f(a+b)~~....~~f(a)+f(b)$$
\end{enumerate}


\item  L'objectif est de démontrer pourquoi la fonction racine carrée est croissante sur l'intervalle $[0~;~+\infty[$\\
\begin{enumerate} 
	\item Soit $a$ et $b$ deux réels positifs. En utilisant une astuce de calcul pour faire apparaitre une identité remarquable montrer que: $$f(a)-f(b) \ = \ \dfrac{a - b}{\sqrt{a}+\sqrt{b}}$$ 
	
	\item Si $ a \leq b  $ que pouvez-vous dire du signe de $f(a)-f(b)$. La fonction racine carrée respecte-t-elle l'ordre entre les abscisses et les ordonnées sur $[0~;~+\infty[$ ~? \\
	\item Que venez-vous de démontrer sur le sens de variation de la fonction racine carré? \\
\end{enumerate} 
\item Dans cette question on considère un point $A$ qui se déplace sur la courbe représentative de la fonction racine carrée dans un repère orthonormé. Nous noterons $x$ l'abscisse du point $A$ et $O$ l'origine du repère orthonormé.\\
\begin{enumerate}
	\item À l'aide d'un logiciel de géométrie dynamique (Geogebra), représenter cette situation. \\
	\item Donner l'ordonnée du point $A$ et rappeler les coordonnées du point $O$.\\ 
	\item Donner les coordonnées du vecteur $\vec{OA}$\\
	\item Calculer  $\|\vec{OA}\|$\\ 
	\item Montrer que: ~ $OA \ = \ f(x)\times f(x+1)$
\end{enumerate}  	 

\end{enumerate}


	
	
\end{document}