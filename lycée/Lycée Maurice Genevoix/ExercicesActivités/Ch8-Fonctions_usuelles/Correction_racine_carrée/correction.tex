\documentclass[a4paper,10pt]{article}

\usepackage[utf8]{inputenc}
\usepackage[T1]{fontenc}
\usepackage{fancyhdr} % pour personnaliser les en-têtes
\usepackage{lastpage}
\usepackage[frenchb]{babel}
\usepackage{amsfonts,amssymb}
\usepackage{amsmath,amsthm}
\usepackage{paralist}
\usepackage{xspace}
\usepackage{xcolor}
\usepackage{variations}
\usepackage{xypic}
\usepackage{eurosym}
\usepackage{graphicx}
\usepackage[np]{numprint}
\usepackage{hyperref} 
\usepackage{listings} % pour écrire des codes avec coloration syntaxique  

\usepackage{tikz,tkz-tab}
\usetikzlibrary{calc, arrows, plotmarks, babel,decorations.pathreplacing}
\usepackage{colortbl}
\usepackage{multirow}
\usepackage[top=1.5cm,bottom=1.5cm,right=1.5cm,left=1.5cm]{geometry}
\newtheorem{defi}{Définition}
\newtheorem{thm}{Théorème}
\newtheorem{rmq}{Remarque}
\newtheorem{prop}{Propriété}
\newtheorem{cor}{Corollaire}
\newtheorem{lem}{Lemme}
\newtheorem{ex}{Exemple}
\newtheorem{cex}{Contre-exemple}
\newtheorem{prop-def}{Propriété-définition}
\newtheorem{exer}{Exercice}
\newtheorem{nota}{Notation}
\newtheorem{ax}{Axiome}
\newtheorem{appl}{Application}
\newtheorem{csq}{Conséquence}
\def\di{\displaystyle}


\newcommand{\vtab}{\rule[-0.4em]{0pt}{1.2em}}
\newcommand{\V}{\overrightarrow}
\newcommand{\C}{\mathbb{C}}
\newcommand{\R}{\mathbb{R}}
\newcommand{\Q}{\mathbb{Q}}
\newcommand{\Z}{\mathbb{Z}}
\newcommand{\N}{\mathbb{N}}

\renewcommand{\thesection}{\Roman{section} }
\renewcommand{\thesubsection}{\arabic{subsection} }
\renewcommand{\thesubsubsection}{\alph{subsubsection} }
\renewcommand{\thesection}{\Roman{section} }
\renewcommand{\thesubsection}{\arabic{subsection} }
\renewcommand{\thesubsubsection}{\alph{subsubsection} }

\definecolor{vert}{RGB}{11,160,78}
\definecolor{rouge}{RGB}{255,120,120}


\pagestyle{fancy}

\begin{document}
	\lhead{}\chead{}\rhead{}\lfoot{Chapitre 8 - Exercice de groupe}\cfoot{\thepage/5}\rfoot{M. Botcazou}\renewcommand{\headrulewidth}{0pt}\renewcommand{\footrulewidth}{0.4pt}
	
\newpage$$\fbox{\text{\Large{ La fonction racine carrée}}}$$
\hfill\\\hfil\\
\noindent La fonction racine carrée est donnée par la relation algébrique suivante \quad
$f:~x \longmapsto \sqrt{x}$\\
\begin{enumerate}
	\item Pour tout nombre réel positif $x\in [0;+\infty[$ il existe une racine carrée du nombre $x$ que l'on note $\sqrt{x}$. Le domaine de définition de la fonction racine carrée est $\R^{+} = [0;+\infty[$. \hfill\textbf{1/} \\
	\item Ci-dessous un tableau de valeurs pour la fonction racine carré sur l'intervalle $[0~;~16]$, tel que les images sont arrondies à $10^{-1}$ près.  \hfill\textbf{3/}\\
	
		$$\begin{tabular}{|c|c|c|c|c|c|c|c|c|c|c|c|c|c|c|c|c|c|}
		\hline
		x&0&1&2&3&4&5&6&7&8&9&10&11&12&13&14&15&16\\
		\hline
		$\sqrt{x}$ & 0 &1 & 1,4&1,7 &2 &2,2 &2,4 & 2,6 &2,8 &3 &3,1 & 3,3&3,4 &3,6 &3,7 &3,9 &4\\ 
		\hline
	\end{tabular}$$	
	\hfill\\
	$$\shorthandoff{:}\begin{tikzpicture}[xscale=0.9, yscale = 0.7]
	\clip (-1.5,-1.5) rectangle (17,5);
	\draw[step=0.5,dotted] (-1.5,-1.5) grid (17,5);
	\draw [->, line width = 1pt, >=latex'](-1.5,0) -- (17,0);
	\draw (16.7,0) node[below]{\footnotesize$x$};
	\draw (0,4.7) node[right]{\footnotesize$y$};
	\foreach \x in {1,...,16}
	\draw[shift={(\x,0)}] (0pt,2pt) -- (0pt,-2pt) node[below] {\tiny $\x$};
	\draw [->, line width = 1pt, >=latex'](0,-3.5) -- (0,5);
	\foreach \y in {1,...,4}
	\draw[shift={(0,\y)}] (2pt,0pt) -- (-2pt,0pt) node[left] {\tiny $\y$};
	\draw (0,0) node[below right]{\tiny $0$};
	\draw[domain=0:16,samples=500,color=blue] plot ({\x},{\x^0.5});
	\draw[color=blue] (16,3.5) node[left]{\footnotesize$\mathcal{C}_f$};
	\end{tikzpicture}\shorthandon{:}$$
	
	\item Le tableau de signes de la fonction racine carrée sur l'intervalle $[0~;~16]$.\hfill\textbf{1/}\\
	$$\begin{tikzpicture}
	\tkzTabInit{$x$ / 1 , $f(x)$ / 1}{$0$, $16$}
	\tkzTabLine{z, +  }
	\end{tikzpicture}
	$$
	
	\item Le tableau de variations de la fonction racine carrée sur l'intervalle $[0~;~16]$.\hfill\textbf{1/}\\
	$$\begin{tikzpicture}
	\tikzset{h style/.style = {pattern=north west lines}}
	\tkzTabInit[lgt=1,espcl=2]{$x$ /1,  $f(x)$ /1}{$0$,$16$}%
	\tkzTabVar{-/ $0$  / , +/ $4$ / }
	\end{tikzpicture}$$


\item 
\begin{enumerate} 
	\item Soit $a$ et $b$ deux réels positifs. \hfill\textbf{1/} $$f(a)-f(b) \ = \  \sqrt{a}- \sqrt{b}   \ = \ \dfrac{(\sqrt{a}- \sqrt{b})(\sqrt{a}+ \sqrt{b})}{\sqrt{a}+ \sqrt{b}} \ = \ \dfrac{\sqrt{a}^2- \sqrt{b}^2}{\sqrt{a}+ \sqrt{b}} \ = \ \dfrac{a - b}{\sqrt{a}+\sqrt{b}}$$ 
	
	\item Si $ a \leq b  $  alors $ a -b \leq 0$. De plus $\sqrt{a}+ \sqrt{b}$ est une somme de nombre positifs donc c'est un nombre positif. Donc $f(a)-f(b)$ est négatif.\hfill\textbf{0.5/}
	$$f(a)-f(b) \leq  0 \  \Longleftrightarrow \ f(a) \leq f(b)$$ 
	La fonction racine carrée respecte l'ordre entre les abscisses et les ordonnées sur $[0~;~+\infty[$. \\
	\item La fonction racine carrée est croissante sur $[0~;~+\infty[$. $[0~;~+\infty[$. \hfill\textbf{0.5/}
	\newpage 
\end{enumerate} 
\item Dans cette question on considère un point $A$ qui se déplace sur la courbe représentative de la fonction racine carrée dans un repère orthonormé. Nous noterons $x$ l'abscisse du point $A$ et $O$ l'origine du repère orthonormé.\\
\begin{enumerate}
	\item \hfill\textbf{2/} 
	$$\shorthandoff{:}\begin{tikzpicture}[xscale=0.9, yscale = 0.7]
	\clip (-1.5,-1.5) rectangle (17,5);
	\draw[step=0.5,dotted] (-1.5,-1.5) grid (17,5);
	\draw [->, line width = 1pt, >=latex'](-1.5,0) -- (17,0);
	\draw (16.7,0) node[below]{\footnotesize$x$};
	\draw (0,4.7) node[right]{\footnotesize$y$};
	\foreach \x in {1,...,16}
	\draw[shift={(\x,0)}] (0pt,2pt) -- (0pt,-2pt) node[below] {\tiny $\x$};
	\draw [->, line width = 1pt, >=latex'](0,-3.5) -- (0,5);
	\foreach \y in {1,...,4}
	\draw[shift={(0,\y)}] (2pt,0pt) -- (-2pt,0pt) node[left] {\tiny $\y$};
	\draw (0,0) node[below right]{\tiny $0$};
	\draw[domain=0:16,samples=500,color=blue] plot ({\x},{\x^0.5});
	\draw[color=blue] (16,3.5) node[left]{\footnotesize$\mathcal{C}_f$};
	\draw[->, line width = 1pt, >=latex'] (0,0)--(4,2); % l'axe des abscisses
	\draw (4,2) node {$\times$};
	\draw (4,2) node[above] {$A$};
	\end{tikzpicture}\shorthandon{:}$$
	
	
	\item L'ordonnée du point $A$ est $\sqrt{x}$ et les coordonnées du point $O$ sont $\left(0;0\right)$.\\ 
	\item Les coordonnées du vecteur: $$\V{OA} \left(\begin{array}{c}x\\\sqrt{x}	\end{array}\right)$$
	\item  $\|\V{OA}\| \ = \  \sqrt{x^2 + \sqrt{x}^2} \ = \ \sqrt{x^2 + x}$\\ 
	\item   $OA \ = \ \|\V{OA}\| \ = \ \sqrt{x^2 + x} \ = \ \ = \ \sqrt{x(x+1)} \ = \ \sqrt{x}\times\sqrt{x+1} \ = \ f(x)\times f(x+1)$
\end{enumerate}  	 

\end{enumerate}


	
	
\end{document}