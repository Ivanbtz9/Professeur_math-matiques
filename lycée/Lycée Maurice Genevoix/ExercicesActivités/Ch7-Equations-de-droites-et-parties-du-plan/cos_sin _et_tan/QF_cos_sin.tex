\documentclass[t,12pt]{beamer}
\usepackage[utf8]{inputenc}
\usepackage[T1]{fontenc}
\usepackage{fancyhdr} % pour personnaliser les en-têtes
\usepackage{lastpage}
\usepackage[frenchb]{babel}
\usepackage{amsfonts,amssymb}
\usepackage{amsmath,amsthm}
\usepackage{paralist}
\usepackage{enumerate}
\usepackage{xspace}
\usepackage{xcolor}
\usepackage{variations}
\usepackage{xypic}
\usepackage{eurosym,multicol}
\usepackage{graphicx}
\usepackage[np]{numprint}
\usepackage{hyperref} 
\usepackage{setspace}
\usepackage{listings} % pour écrire des codes avec coloration syntaxique  

\usepackage{tikz}
\usetikzlibrary{calc, arrows, plotmarks,decorations.pathreplacing}
\usepackage{colortbl}
\usepackage{multirow}


\newtheorem{defi}{Définition}
\newtheorem{thm}{Théorème}
\newtheorem{thm-def}{Théorème/Définition}
\newtheorem{rmq}{Remarque}
\newtheorem{prop}{Propriété}
\newtheorem{cor}{Corollaire}
\newtheorem{lem}{Lemme}
\newtheorem{ex}{Exemple}
\newtheorem{cex}{Contre-exemple}
\newtheorem{prop-def}{Propriété-définition}
\newtheorem{exer}{Exercice}
\newtheorem{nota}{Notation}
\newtheorem{ax}{Axiome}
\newtheorem{appl}{Application}
\newtheorem{csq}{Conséquence}
%\def\di{\displaystyle}


\newcommand{\vtab}{\rule[-0.4em]{0pt}{1.2em}}
\newcommand{\V}{\overrightarrow}
\renewcommand{\thesection}{\Roman{section} }
\renewcommand{\thesubsection}{\arabic{subsection} }
\renewcommand{\thesubsubsection}{\alph{subsubsection} }
\newcommand{\C}{\mathbb{C}}
\newcommand{\R}{\mathbb{R}}
\newcommand{\Q}{\mathbb{Q}}
\newcommand{\Z}{\mathbb{Z}}
\newcommand{\N}{\mathbb{N}}



\usetheme{Warsaw}

\title{Cours: sur le cosinus et le sinus}
\author{}
\date{}
\begin{document}
\maketitle	

\begin{frame}
	\frametitle{ }

\begin{minipage}[t]{1\linewidth}
	\begin{minipage}[c]{0.45\linewidth}
	$$\begin{tikzpicture}[scale=1.6]
		\draw[->, line width = 1pt, >=latex'] (-1.5,0) -- (1.5,0);
		\draw[->, line width = 1pt, >=latex'] (0,-1.5) -- (0,1.5);
		\draw (0,0) node[below left]{\footnotesize$O$};
		\draw (1.5,0) node[below]{\footnotesize$x$};
		\draw (0,1.5) node[left]{\footnotesize$y$};
		\draw (0.8,-0.8) node[right]{\footnotesize$\mathcal{C}$};
		\draw (0,1) node{$+$} node[above left]{\footnotesize$1$};
		\draw (1,0) node{$+$} node[above right]{\footnotesize$1$};
		\draw (-1,0) node{$+$} node[above left]{\footnotesize$-1$};
		\draw (0,-1) node{$+$} node[below left]{\footnotesize$-1$};
		\draw(0,0) circle (1cm);
		\end{tikzpicture}$$
	\end{minipage}\hfill
	\begin{minipage}[c]{0.5\linewidth}
		Reproduire ce schéma sur votre cahier d'exercice et placer dans le quart en haut à droite du plan un point $A$ sur le cercle $\mathcal{C}$. On notera $\theta$ l'angle à droite formé entre la droite $(OI)$ et la droite $(OA)$. 
	\end{minipage}
\end{minipage}
		



	

\end{frame}

\begin{frame}
	\frametitle{}

	$$\begin{tikzpicture}[scale=2.5]
\draw[->, line width = 1pt, >=latex'] (-1.5,0) -- (1.5,0);
\draw[->, line width = 1pt, >=latex'] (0,-1.5) -- (0,1.5);
\draw (0,0) node[below left]{\footnotesize$O$};
\draw (1.5,0) node[below]{\footnotesize$x$};
\draw (0,1.5) node[left]{\footnotesize$y$};
\draw (0.8,-0.8) node[right]{\footnotesize$\mathcal{C}$};
\draw (0,1) node{$+$} node[above left]{\footnotesize$1$};
\draw (1,0) node{$+$} node[above right]{\footnotesize$1$};
\draw (-1,0) node{$+$} node[above left]{\footnotesize$-1$};
\draw (0,-1) node{$+$} node[below left]{\footnotesize$-1$};
\draw(0,0) circle (1cm);
\end{tikzpicture}$$
	
\end{frame}





\end{document}