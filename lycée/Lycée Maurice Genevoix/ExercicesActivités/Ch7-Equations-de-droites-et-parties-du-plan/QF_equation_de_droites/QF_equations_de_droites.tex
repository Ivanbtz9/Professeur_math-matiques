\documentclass[t,12pt]{beamer}
\usepackage[utf8]{inputenc}
\usepackage[T1]{fontenc}
\usepackage{fancyhdr} % pour personnaliser les en-têtes
\usepackage{lastpage}
\usepackage[frenchb]{babel}
\usepackage{amsfonts,amssymb}
\usepackage{amsmath,amsthm}
\usepackage{paralist}
\usepackage{enumerate}
\usepackage{xspace}
\usepackage{xcolor}
\usepackage{variations}
\usepackage{xypic}
\usepackage{eurosym,multicol}
\usepackage{graphicx}
\usepackage[np]{numprint}
\usepackage{hyperref} 
\usepackage{setspace}
\usepackage{listings} % pour écrire des codes avec coloration syntaxique  

\usepackage{tikz}
\usetikzlibrary{calc, arrows, plotmarks,decorations.pathreplacing}
\usepackage{colortbl}
\usepackage{multirow}


\newtheorem{defi}{Définition}
\newtheorem{thm}{Théorème}
\newtheorem{thm-def}{Théorème/Définition}
\newtheorem{rmq}{Remarque}
\newtheorem{prop}{Propriété}
\newtheorem{cor}{Corollaire}
\newtheorem{lem}{Lemme}
\newtheorem{ex}{Exemple}
\newtheorem{cex}{Contre-exemple}
\newtheorem{prop-def}{Propriété-définition}
\newtheorem{exer}{Exercice}
\newtheorem{nota}{Notation}
\newtheorem{ax}{Axiome}
\newtheorem{appl}{Application}
\newtheorem{csq}{Conséquence}
%\def\di{\displaystyle}


\newcommand{\vtab}{\rule[-0.4em]{0pt}{1.2em}}
\newcommand{\V}{\overrightarrow}
\renewcommand{\thesection}{\Roman{section} }
\renewcommand{\thesubsection}{\arabic{subsection} }
\renewcommand{\thesubsubsection}{\alph{subsubsection} }
\newcommand{\C}{\mathbb{C}}
\newcommand{\R}{\mathbb{R}}
\newcommand{\Q}{\mathbb{Q}}
\newcommand{\Z}{\mathbb{Z}}
\newcommand{\N}{\mathbb{N}}



\usetheme{Warsaw}

\title{Questions: trouver les équations de droites}
\author{}
\date{}
\begin{document}
\maketitle	

\begin{frame}
	\frametitle{Question 1: }
On considère ci-dessous le plan $\mathcal{P}$ muni d'un repère orthonormé. Donner l'équation de la droite $(d_1)$ en donnant l'ensemble des points associés à $(d_1)$.
\hfill\\[-1.5cm]
		$$\shorthandoff{:}\begin{tikzpicture}[scale=0.7]
	\clip (-5.1,-5.1) rectangle (5.1,5.1);
	\draw[step=0.5,dotted] (-4,-3) grid (4,3);
	\draw [->, line width = 1pt, >=latex'](-4,0) -- (4,0);
	\draw [->, line width = 1pt, >=latex'](0,-3) -- (0,3);
	\draw [-, color=blue](-4,2) -- (4,2);
	\draw (0,1) node{$+$} node[above left]{\footnotesize$J$};
	\draw (1,0) node{$+$} node[below left]{\footnotesize$I$};
	\draw (0,-0.2) node[below left]{\footnotesize$O$};
	\draw (4,0) node[below left]{\footnotesize$x$};
	\draw (0,3) node[below left]{\footnotesize$y$};
	\draw (3.5,2.81) node[below left]{\footnotesize$(d_1)$};
	\end{tikzpicture}\shorthandon{:}$$



	

\end{frame}

\begin{frame}
	\frametitle{Question 2: }
On considère ci-dessous le plan $\mathcal{P}$ muni d'un repère orthonormé. Donner l'équation de la droite $(d_2)$ en donnant l'ensemble des points associés à $(d_2)$.
\hfill\\[-1.5cm]
$$\shorthandoff{:}\begin{tikzpicture}[scale=0.7]
\clip (-5.1,-5.1) rectangle (5.1,5.1);
\draw[step=0.5,dotted] (-4,-3) grid (4,3);
\draw [->, line width = 1pt, >=latex'](-4,0) -- (4,0);
\draw [->, line width = 1pt, >=latex'](0,-3) -- (0,3);
\draw (0,1) node{$+$} node[above left]{\footnotesize$J$};
\draw (1,0) node{$+$} node[below left]{\footnotesize$I$};
\draw (0,-0.2) node[below left]{\footnotesize$O$};
\draw (4,0) node[below left]{\footnotesize$x$};
\draw (0,3) node[below left]{\footnotesize$y$};
\draw (3.5,3) node[below left]{\footnotesize$(d_2)$};
\draw[domain=-4:4,samples=200,color=blue] plot ({\x},{2/3*\x});
\end{tikzpicture}\shorthandon{:}$$
		
		

	
\end{frame}

\begin{frame}
	\frametitle{Question 3: }
On considère ci-dessous le plan $\mathcal{P}$ muni d'un repère orthonormé. Donner l'équation de la droite $(d_3)$ en donnant l'ensemble des points associés à $(d_3)$.
\hfill\\[-1.5cm]
$$\shorthandoff{:}\begin{tikzpicture}[scale=0.7]
\clip (-5.1,-5.1) rectangle (5.1,5.1);
\draw[step=0.5,dotted] (-4,-3) grid (4,3);
\draw [->, line width = 1pt, >=latex'](-4,0) -- (4,0);
\draw [->, line width = 1pt, >=latex'](0,-3) -- (0,3);
\draw [-, color=blue](-2.5,-3) -- (-2.5,3);
\draw (0,1) node{$+$} node[above left]{\footnotesize$J$};
\draw (1,0) node{$+$} node[below left]{\footnotesize$I$};
\draw (0,-0.2) node[below left]{\footnotesize$O$};
\draw (4,0) node[below left]{\footnotesize$x$};
\draw (0,3) node[below left]{\footnotesize$y$};
\draw (-2.5,2.81) node[below left]{\footnotesize$(d_3)$};
\end{tikzpicture}\shorthandon{:}$$
	
	
\end{frame}

\begin{frame}
	\frametitle{Question 4: }
On considère ci-dessous le plan $\mathcal{P}$ muni d'un repère orthonormé. Donner l'équation de la droite $(d_4)$ en donnant l'ensemble des points associés à $(d_4)$.
\hfill\\[-1.5cm]
$$\shorthandoff{:}\begin{tikzpicture}[scale=0.7]
\clip (-5.1,-5.1) rectangle (5.1,5.1);
\draw[step=0.5,dotted] (-2,-4) grid (4,3);
\draw [->, line width = 1pt, >=latex'](-2,0) -- (4,0);
\draw [->, line width = 1pt, >=latex'](0,-4) -- (0,3);
\draw (0,1) node{$+$} node[above left]{\footnotesize$J$};
\draw (1,0) node{$+$} node[below left]{\footnotesize$I$};
\draw (0,-0.2) node[below left]{\footnotesize$O$};
\draw (4,0) node[below left]{\footnotesize$x$};
\draw (0,3) node[below left]{\footnotesize$y$};
\draw (3.9,3) node[below left]{\footnotesize$(d_4)$};
\draw[domain=-2:4,samples=200,color=blue] plot ({\x},{\x-1.5});
\end{tikzpicture}\shorthandon{:}$$
		
	
	
	
\end{frame}

\begin{frame}
	\frametitle{Question 5: }
On considère ci-dessous le plan $\mathcal{P}$ muni d'un repère orthonormé. Donner l'équation de la droite $(d_5)$ en donnant l'ensemble des points associés à $(d_5)$.
\hfill\\[-1.5cm]
$$\shorthandoff{:}\begin{tikzpicture}[scale=0.7]
\clip (-5.1,-5.1) rectangle (5.1,5.1);
\draw[step=0.5,dotted] (-2,-4) grid (4,3);
\draw [->, line width = 1pt, >=latex'](-2,0) -- (4,0);
\draw [->, line width = 1pt, >=latex'](0,-4) -- (0,3);
\draw (0,1) node{$+$} node[above left]{\footnotesize$J$};
\draw (1,0) node{$+$} node[below right]{\footnotesize$I$};
\draw (0,-0.2) node[below left]{\footnotesize$O$};
\draw (4,0) node[below left]{\footnotesize$x$};
\draw (0,3) node[below left]{\footnotesize$y$};
\draw (3.9,-3) node[below left]{\footnotesize$(d_5)$};
\draw[domain=-2:4,samples=200,color=blue] plot ({\x},{-1*\x+0.5});
\end{tikzpicture}\shorthandon{:}$$
	
	
\end{frame}



\end{document}