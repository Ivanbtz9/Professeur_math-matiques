\documentclass[t,12pt]{beamer}
\usepackage[utf8]{inputenc}
\usepackage[T1]{fontenc}
\usepackage{fancyhdr} % pour personnaliser les en-têtes
\usepackage{lastpage}
\usepackage[frenchb]{babel}
\usepackage{amsfonts,amssymb}
\usepackage{amsmath,amsthm}
\usepackage{paralist}
\usepackage{enumerate}
\usepackage{xspace}
\usepackage{xcolor}
\usepackage{variations}
\usepackage{xypic}
\usepackage{eurosym,multicol}
\usepackage{graphicx}
\usepackage[np]{numprint}
\usepackage{hyperref} 
\usepackage{setspace}
\usepackage{listings} % pour écrire des codes avec coloration syntaxique  

\usepackage{tikz}
\usetikzlibrary{calc, arrows, plotmarks,decorations.pathreplacing}
\usepackage{colortbl}
\usepackage{multirow}


\newtheorem{defi}{Définition}
\newtheorem{thm}{Théorème}
\newtheorem{thm-def}{Théorème/Définition}
\newtheorem{rmq}{Remarque}
\newtheorem{prop}{Propriété}
\newtheorem{cor}{Corollaire}
\newtheorem{lem}{Lemme}
\newtheorem{ex}{Exemple}
\newtheorem{cex}{Contre-exemple}
\newtheorem{prop-def}{Propriété-définition}
\newtheorem{exer}{Exercice}
\newtheorem{nota}{Notation}
\newtheorem{ax}{Axiome}
\newtheorem{appl}{Application}
\newtheorem{csq}{Conséquence}
%\def\di{\displaystyle}


\newcommand{\vtab}{\rule[-0.4em]{0pt}{1.2em}}
\newcommand{\V}{\overrightarrow}
\renewcommand{\thesection}{\Roman{section} }
\renewcommand{\thesubsection}{\arabic{subsection} }
\renewcommand{\thesubsubsection}{\alph{subsubsection} }
\newcommand{\C}{\mathbb{C}}
\newcommand{\R}{\mathbb{R}}
\newcommand{\Q}{\mathbb{Q}}
\newcommand{\Z}{\mathbb{Z}}
\newcommand{\N}{\mathbb{N}}



\usetheme{Warsaw}

\title{Questions sur les parties du plan et équation de droite}
\author{}
\date{}
\begin{document}
\maketitle	

\begin{frame}
	\frametitle{Question 1: }
On considère ci-dessous le plan $\mathcal{P}$ muni d'un repère orthonormé $(O,I,J)$ 
\hfill\\[-0.2cm]
		$$\shorthandoff{:}\begin{tikzpicture}[scale=0.8]
	\clip (-3.1,-3.1) rectangle (3.1,3.1);
	\draw[step=0.5,dotted] (-3,-3) grid (3,3);
	\draw [->, line width = 1pt, >=latex'](-3,0) -- (3,0);
	\draw [->, line width = 1pt, >=latex'](0,-3) -- (0,3);
	\draw (0,1) node{$+$} node[above left]{$J$};
	\draw (1,0) node{$+$} node[below left]{$I$};
	\draw (0,0) node{$+$} node[below left]{$O$};
	\draw (3,0) node[below left]{$x$};
	\draw (0,3) node[below left]{$y$};
	\end{tikzpicture}\shorthandon{:}$$
Reproduire ce repère et colorier en rouge l'ensemble des points suivant:
$\{(x,y) \ \text{tel que:} \  x\geq0 \ \text{et} \  y\geq0  \}$	

	

\end{frame}

\begin{frame}
	\frametitle{Question 2: }
On considère ci-dessous le plan $\mathcal{P}$ muni d'un repère orthonormé $(O,I,J)$ 
\hfill\\[-0.2cm]
$$\shorthandoff{:}\begin{tikzpicture}[scale=0.8]
\clip (-3.1,-3.1) rectangle (3.1,3.1);
\draw[step=0.5,dotted] (-3,-3) grid (3,3);
\draw [->, line width = 1pt, >=latex'](-3,0) -- (3,0);
\draw [->, line width = 1pt, >=latex'](0,-3) -- (0,3);
\draw (0,1) node{$+$} node[above left]{$J$};
\draw (1,0) node{$+$} node[below left]{$I$};
\draw (0,0) node{$+$} node[below left]{$O$};
\draw (3,0) node[below left]{$x$};
\draw (0,3) node[below left]{$y$};
\end{tikzpicture}\shorthandon{:}$$
Colorier en rouge l'ensemble des points suivant:
$\{(x,y) \ | \  x\geq0 \ \text{et} \  y\leq0  \} \cup \{(x,y) \ | \  x\leq0 \ \text{et} \  y\geq0  \} $		
		

	
\end{frame}

\begin{frame}
	\frametitle{Question 3: }
	On considère ci-dessous le plan $\mathcal{P}$ muni d'un repère orthonormé $(O,I,J)$ 
	\hfill\\[-0.2cm]
	$$\shorthandoff{:}\begin{tikzpicture}[scale=0.8]
	\clip (-3.1,-3.1) rectangle (3.1,3.1);
	\draw[step=0.5,dotted] (-3,-3) grid (3,3);
	\draw [->, line width = 1pt, >=latex'](-3,0) -- (3,0);
	\draw [->, line width = 1pt, >=latex'](0,-3) -- (0,3);
	\draw (0,1) node{$+$} node[above left]{$J$};
	\draw (1,0) node{$+$} node[below left]{$I$};
	\draw (0,0) node{$+$} node[below left]{$O$};
	\draw (3,0) node[below left]{$x$};
	\draw (0,3) node[below left]{$y$};
	\end{tikzpicture}\shorthandon{:}$$
	Colorier en rouge l'ensemble des points suivant:
	$$\{(x,y) \ | \  x =2  \} $$		
	
	
	
\end{frame}

\begin{frame}
	\frametitle{Question 4: }
	On considère ci-dessous le plan $\mathcal{P}$ muni d'un repère orthonormé $(O,I,J)$ 
	\hfill\\[-0.2cm]
	$$\shorthandoff{:}\begin{tikzpicture}[scale=0.8]
	\clip (-3.1,-3.1) rectangle (3.1,3.1);
	\draw[step=0.5,dotted] (-3,-3) grid (3,3);
	\draw [->, line width = 1pt, >=latex'](-3,0) -- (3,0);
	\draw [->, line width = 1pt, >=latex'](0,-3) -- (0,3);
	\draw (0,1) node{$+$} node[above left]{$J$};
	\draw (1,0) node{$+$} node[below left]{$I$};
	\draw (0,0) node{$+$} node[below left]{$O$};
	\draw (3,0) node[below left]{$x$};
	\draw (0,3) node[below left]{$y$};
	\end{tikzpicture}\shorthandon{:}$$
	Colorier en rouge l'ensemble des points suivant:
	$$\{(x,y) \ | \  y =1  \} $$		
	
	
	
\end{frame}

\begin{frame}
	\frametitle{Question 5: }
	On considère ci-dessous le plan $\mathcal{P}$ muni d'un repère orthonormé $(O,I,J)$ 
	\hfill\\[-0.2cm]
	$$\shorthandoff{:}\begin{tikzpicture}[scale=0.8]
	\clip (-3.1,-3.1) rectangle (3.1,3.1);
	\draw[step=0.5,dotted] (-3,-3) grid (3,3);
	\draw [->, line width = 1pt, >=latex'](-3,0) -- (3,0);
	\draw [->, line width = 1pt, >=latex'](0,-3) -- (0,3);
	\draw (0,1) node{$+$} node[above left]{$J$};
	\draw (1,0) node{$+$} node[below left]{$I$};
	\draw (0,0) node{$+$} node[below left]{$O$};
	\draw (3,0) node[below left]{$x$};
	\draw (0,3) node[below left]{$y$};
	\end{tikzpicture}\shorthandon{:}$$
	Colorier en rouge l'ensemble des points suivant:
	$$\{(x,y) \ | \  x =-3  \} $$		
	
	
	
\end{frame}

\begin{frame}
	\frametitle{Question 6: }
	On considère ci-dessous le plan $\mathcal{P}$ muni d'un repère orthonormé $(O,I,J)$ 
	\hfill\\[-0.2cm]
	$$\shorthandoff{:}\begin{tikzpicture}[scale=0.8]
	\clip (-3.1,-3.1) rectangle (3.1,3.1);
	\draw[step=0.5,dotted] (-3,-3) grid (3,3);
	\draw [->, line width = 1pt, >=latex'](-3,0) -- (3,0);
	\draw [->, line width = 1pt, >=latex'](0,-3) -- (0,3);
	\draw (0,1) node{$+$} node[above left]{$J$};
	\draw (1,0) node{$+$} node[below left]{$I$};
	\draw (0,0) node{$+$} node[below left]{$O$};
	\draw (3,0) node[below left]{$x$};
	\draw (0,3) node[below left]{$y$};
	\end{tikzpicture}\shorthandon{:}$$
	Colorier en rouge l'ensemble des points suivant:
	$$\{(x,y) \ | \  y =-2  \} $$		
	
	
	
\end{frame}

\begin{frame}
	\frametitle{Question 7: }
	On considère ci-dessous le plan $\mathcal{P}$ muni d'un repère orthonormé $(O,I,J)$ 
	\hfill\\[-0.2cm]
	$$\shorthandoff{:}\begin{tikzpicture}[scale=0.8]
	\clip (-3.1,-3.1) rectangle (3.1,3.1);
	\draw[step=0.5,dotted] (-3,-3) grid (3,3);
	\draw [->, line width = 1pt, >=latex'](-3,0) -- (3,0);
	\draw [->, line width = 1pt, >=latex'](0,-3) -- (0,3);
	\draw (0,1) node{$+$} node[above left]{$J$};
	\draw (1,0) node{$+$} node[below left]{$I$};
	\draw (0,0) node{$+$} node[below left]{$O$};
	\draw (3,0) node[below left]{$x$};
	\draw (0,3) node[below left]{$y$};
	\end{tikzpicture}\shorthandon{:}$$
	Colorier en rouge l'ensemble des points suivant:
	$$\{(x,y) \ | \  y = x  \} $$			
	
\end{frame}

\begin{frame}
	\frametitle{Question 8: }
	On considère ci-dessous le plan $\mathcal{P}$ muni d'un repère orthonormé $(O,I,J)$ 
	\hfill\\[-0.2cm]
	$$\shorthandoff{:}\begin{tikzpicture}[scale=0.8]
	\clip (-3.1,-3.1) rectangle (3.1,3.1);
	\draw[step=0.5,dotted] (-3,-3) grid (3,3);
	\draw [->, line width = 1pt, >=latex'](-3,0) -- (3,0);
	\draw [->, line width = 1pt, >=latex'](0,-3) -- (0,3);
	\draw (0,1) node{$+$} node[above left]{$J$};
	\draw (1,0) node{$+$} node[below left]{$I$};
	\draw (0,0) node{$+$} node[below left]{$O$};
	\draw (3,0) node[below left]{$x$};
	\draw (0,3) node[below left]{$y$};
	\end{tikzpicture}\shorthandon{:}$$
	Colorier en rouge l'ensemble des points suivant:
	$$\{(x,y) \ | \  y = 3x  \} $$			
	
\end{frame}

\begin{frame}
	\frametitle{Question 9: }
	On considère ci-dessous le plan $\mathcal{P}$ muni d'un repère orthonormé $(O,I,J)$ 
	\hfill\\[-0.2cm]
	$$\shorthandoff{:}\begin{tikzpicture}[scale=0.8]
	\clip (-3.1,-3.1) rectangle (3.1,3.1);
	\draw[step=0.5,dotted] (-3,-3) grid (3,3);
	\draw [->, line width = 1pt, >=latex'](-3,0) -- (3,0);
	\draw [->, line width = 1pt, >=latex'](0,-3) -- (0,3);
	\draw (0,1) node{$+$} node[above left]{$J$};
	\draw (1,0) node{$+$} node[below left]{$I$};
	\draw (0,0) node{$+$} node[below left]{$O$};
	\draw (3,0) node[below left]{$x$};
	\draw (0,3) node[below left]{$y$};
	\end{tikzpicture}\shorthandon{:}$$
	Colorier en rouge l'ensemble des points suivant:
	$$\{(x,y) \ | \  y = -4x  \} $$			
	
\end{frame}


\begin{frame}
	\frametitle{Question 10: }
	On considère ci-dessous le plan $\mathcal{P}$ muni d'un repère orthonormé $(O,I,J)$ 
	\hfill\\[-0.2cm]
	$$\shorthandoff{:}\begin{tikzpicture}[scale=0.8]
	\clip (-3.1,-3.1) rectangle (3.1,3.1);
	\draw[step=0.5,dotted] (-3,-3) grid (3,3);
	\draw [->, line width = 1pt, >=latex'](-3,0) -- (3,0);
	\draw [->, line width = 1pt, >=latex'](0,-3) -- (0,3);
	\draw (0,1) node{$+$} node[above left]{$J$};
	\draw (1,0) node{$+$} node[below left]{$I$};
	\draw (0,0) node{$+$} node[below left]{$O$};
	\draw (3,0) node[below left]{$x$};
	\draw (0,3) node[below left]{$y$};
	\end{tikzpicture}\shorthandon{:}$$
	Colorier en rouge l'ensemble des points suivant:
	$$\{(x,y) \ | \  y = 2x +1  \} $$			
	
\end{frame}

\begin{frame}
	\frametitle{Question 11: }
	On considère ci-dessous le plan $\mathcal{P}$ muni d'un repère orthonormé $(O,I,J)$ 
	\hfill\\[-0.2cm]
	$$\shorthandoff{:}\begin{tikzpicture}[scale=0.8]
	\clip (-3.1,-3.1) rectangle (3.1,3.1);
	\draw[step=0.5,dotted] (-3,-3) grid (3,3);
	\draw [->, line width = 1pt, >=latex'](-3,0) -- (3,0);
	\draw [->, line width = 1pt, >=latex'](0,-3) -- (0,3);
	\draw (0,1) node{$+$} node[above left]{$J$};
	\draw (1,0) node{$+$} node[below left]{$I$};
	\draw (0,0) node{$+$} node[below left]{$O$};
	\draw (3,0) node[below left]{$x$};
	\draw (0,3) node[below left]{$y$};
	\end{tikzpicture}\shorthandon{:}$$
	Colorier en rouge l'ensemble des points suivant:
	$$\{(x,y) \ | \  y = -x -1  \} $$			
	
\end{frame}


\end{document}