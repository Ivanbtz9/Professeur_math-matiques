\documentclass[10pt,a4paper]{article}


\usepackage[T1]{fontenc}
\usepackage[francais]{babel}
\usepackage{times}
\usepackage[utf8]{inputenc}
\usepackage{enumitem,fancyhdr,tcolorbox,tablists,color}
\usepackage{amsmath,amssymb,amsthm, mathrsfs,pifont}
\usepackage{pgf,tikz,pgfplots,tkz-tab}
\usepackage{hyperref,cancel,array, xcolor}
\usepackage{amsthm,amssymb,graphicx}
\usepackage{geometry}
\geometry{top=1.5cm, bottom=2cm, left=2cm, right=1.5cm}


\newtheorem{thm}{Théorème}
\newtheorem*{pro}{Propriété}
\newtheorem*{exemple}{Exemple}

\theoremstyle{definition}
\newtheorem*{remarque}{Remarque}
\theoremstyle{definition}
\newtheorem{exo}{Exercice}
\newtheorem{definition}{Définition}


\newcommand{\notni}{\not\owns}
\newcommand{\fleche}{$\rightsquigarrow$\ }
\newcommand{\R}{\mathbb{R}}
\newcommand{\Z}{\mathbb{Z}}
\newcommand{\N}{\mathbb{N}}
\renewcommand{\thesection}{\Roman{section} }
\renewcommand{\thesubsection}{\arabic{subsection} }
\renewcommand{\thesubsubsection}{\alph{subsubsection} }


\begin{document}
		
\noindent{\bfseries Lycée Maurice Genevoix \hfill Année~2021-2022}\\
{\bfseries Niveau: Seconde \hfill M.Botcazou   }\\
{\bfseries mail: ibotca52@gmail.com}\\
\rule[0.5ex]{\textwidth}{0.1mm}	

\begin{center}
	\Large \sc Progression 	
\end{center}


\section{Nombres réels}
\subsection{Ensembles de nombres }
\subsection{Intervalles réels}
\subsection{Valeur absolue et distance à un nombre}

\section{Information chiffrée et statistique}
\subsection{Proportions et pourcentages}
\subsection{Évolution d'une quantité positive}
\subsection{Évolutions successives}

\section{Vecteurs du plan - Partie 1/2}
\subsection{Translation et vecteur associé}
\subsection{Coordonnées d'un vecteur dans un repère orthonormé}
\subsection{Calcul de la norme d'un vecteur dans un repère orthonormé}
\subsection{Premières opérations sur les vecteurs}
\subsection{Traduction de situations géométriques à l'aide de vecteurs}

\section{Calcul littéral et manipulation algébrique}
\subsection{Puissances entières et racines carrées}
\subsection{Transformations d’expressions algébriques (développer-factoriser)}
\subsection{Identités remarquables}
\subsection{Transformations d'expressions fractionnaires}

\section{Premières notions avec les fonctions}
\subsection{Vocabulaire et applications}
\subsection{Courbe représentative et lecture graphique}
\subsection{Résolution d'équations et d'inéquations }
\subsection{Variations et extremums}

\section{Probabilités}
\subsection{Langage probabiliste}
\subsection{Calculs de probabilités}
\subsection{Échantillonnage}

\newpage

\section{Équations de droites}
\subsection{Droites et vecteurs directeurs}
\subsection{Équations cartésiennes}
\subsection{Systèmes d'équations linéaires}

\section{Fonctions usuelles}
\subsection{Fonctions affines et produits de fonctions affines}
\subsection{Fonctions carré et cube}
\subsection{Fonctions racine caré et inverse}

\section{Vecteurs du plan - Partie 2/2}
\subsection{Colinéarité de deux vecteurs}
\subsection{Orthogonalité de deux vecteurs}
\subsection{Calcul de la norme d'un vecteur dans un repère orthonormé}

\section{Arithmétique}
\subsection{Multiples et diviseurs}
\subsection{Division euclidienne et parité}
\subsection{Nombres premiers}





\end{document}