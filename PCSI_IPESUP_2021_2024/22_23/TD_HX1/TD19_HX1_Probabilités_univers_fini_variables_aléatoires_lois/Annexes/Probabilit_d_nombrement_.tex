\documentclass[11pt]{article}

 %Configuration de la feuille 
 
\usepackage{amsmath,amssymb,enumerate,graphicx,pgf,tikz,fancyhdr}
\usepackage[utf8]{inputenc}
\usetikzlibrary{arrows}
\usepackage{geometry}
\usepackage{tabvar}
\geometry{hmargin=2.2cm,vmargin=1.5cm}\pagestyle{fancy}
\lfoot{\bfseries http://www.bibmath.net}
\rfoot{\bfseries\thepage}
\cfoot{}
\renewcommand{\footrulewidth}{0.5pt} %Filet en bas de page

 %Macros utilisées dans la base de données d'exercices 

\newcommand{\mtn}{\mathbb{N}}
\newcommand{\mtns}{\mathbb{N}^*}
\newcommand{\mtz}{\mathbb{Z}}
\newcommand{\mtr}{\mathbb{R}}
\newcommand{\mtk}{\mathbb{K}}
\newcommand{\mtq}{\mathbb{Q}}
\newcommand{\mtc}{\mathbb{C}}
\newcommand{\mch}{\mathcal{H}}
\newcommand{\mcp}{\mathcal{P}}
\newcommand{\mcb}{\mathcal{B}}
\newcommand{\mcl}{\mathcal{L}}
\newcommand{\mcm}{\mathcal{M}}
\newcommand{\mcc}{\mathcal{C}}
\newcommand{\mcmn}{\mathcal{M}}
\newcommand{\mcmnr}{\mathcal{M}_n(\mtr)}
\newcommand{\mcmnk}{\mathcal{M}_n(\mtk)}
\newcommand{\mcsn}{\mathcal{S}_n}
\newcommand{\mcs}{\mathcal{S}}
\newcommand{\mcd}{\mathcal{D}}
\newcommand{\mcsns}{\mathcal{S}_n^{++}}
\newcommand{\glnk}{GL_n(\mtk)}
\newcommand{\mnr}{\mathcal{M}_n(\mtr)}
\DeclareMathOperator{\ch}{ch}
\DeclareMathOperator{\sh}{sh}
\DeclareMathOperator{\vect}{vect}
\DeclareMathOperator{\card}{card}
\DeclareMathOperator{\comat}{comat}
\DeclareMathOperator{\imv}{Im}
\DeclareMathOperator{\rang}{rg}
\DeclareMathOperator{\Fr}{Fr}
\DeclareMathOperator{\diam}{diam}
\DeclareMathOperator{\supp}{supp}
\newcommand{\veps}{\varepsilon}
\newcommand{\mcu}{\mathcal{U}}
\newcommand{\mcun}{\mcu_n}
\newcommand{\dis}{\displaystyle}
\newcommand{\croouv}{[\![}
\newcommand{\crofer}{]\!]}
\newcommand{\rab}{\mathcal{R}(a,b)}
\newcommand{\pss}[2]{\langle #1,#2\rangle}
 %Document 

\begin{document} 

\begin{center}\textsc{{\huge }}\end{center}

% Exercice 1225


\vskip0.3cm\noindent\textsc{Exercice 1} - Au moins un six!
\vskip0.2cm
Soit $n\geq 1$. On lance $n$ fois un dé parfaitement équilibré. Quelle est la probabilité d'obtenir
\begin{enumerate}
\item au moins une fois le chiffre 6?
\item au moins deux fois le chiffre 6?
\item au moins $k$ fois le chiffre 6?
\end{enumerate}


% Exercice 1222


\vskip0.3cm\noindent\textsc{Exercice 2} - Dés pipés
\vskip0.2cm
On dispose d'un dé pipé tel que la probabilité d'obtenir une face soit proportionnelle au chiffre porté par cette face. On lance le dé pipé. 
\begin{enumerate}
\item Donner un espace probabilisé modélisant l'expérience aléatoire.
\item Quelle est la probabilité d'obtenir un chiffre pair?
\item Reprendre les questions si cette fois le dé est pipé de sorte que la probabilité d'une face paire soit le double de la probabilité d'une face impaire.
\end{enumerate}




\vskip0.5cm
\noindent{\small Cette feuille d'exercices a été conçue à l'aide du site \textsf{https://www.bibmath.net}}

%Vous avez accès aux corrigés de cette feuille par l'url : https://www.bibmath.net/ressources/justeunefeuille.php?id=28394
\end{document}