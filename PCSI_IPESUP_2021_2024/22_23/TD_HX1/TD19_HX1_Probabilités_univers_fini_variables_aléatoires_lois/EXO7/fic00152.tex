
%%%%%%%%%%%%%%%%%% PREAMBULE %%%%%%%%%%%%%%%%%%

\documentclass[11pt,a4paper]{article}

\usepackage{amsfonts,amsmath,amssymb,amsthm}
\usepackage[utf8]{inputenc}
\usepackage[T1]{fontenc}
\usepackage[francais]{babel}
\usepackage{mathptmx}
\usepackage{fancybox}
\usepackage{graphicx}
\usepackage{ifthen}

\usepackage{tikz}   

\usepackage{hyperref}
\hypersetup{colorlinks=true, linkcolor=blue, urlcolor=blue,
pdftitle={Exo7 - Exercices de mathématiques}, pdfauthor={Exo7}}

\usepackage{geometry}
\geometry{top=2cm, bottom=2cm, left=2cm, right=2cm}

%----- Ensembles : entiers, reels, complexes -----
\newcommand{\Nn}{\mathbb{N}} \newcommand{\N}{\mathbb{N}}
\newcommand{\Zz}{\mathbb{Z}} \newcommand{\Z}{\mathbb{Z}}
\newcommand{\Qq}{\mathbb{Q}} \newcommand{\Q}{\mathbb{Q}}
\newcommand{\Rr}{\mathbb{R}} \newcommand{\R}{\mathbb{R}}
\newcommand{\Cc}{\mathbb{C}} \newcommand{\C}{\mathbb{C}}
\newcommand{\Kk}{\mathbb{K}} \newcommand{\K}{\mathbb{K}}

%----- Modifications de symboles -----
\renewcommand{\epsilon}{\varepsilon}
\renewcommand{\Re}{\mathop{\mathrm{Re}}\nolimits}
\renewcommand{\Im}{\mathop{\mathrm{Im}}\nolimits}
\newcommand{\llbracket}{\left[\kern-0.15em\left[}
\newcommand{\rrbracket}{\right]\kern-0.15em\right]}
\renewcommand{\ge}{\geqslant} \renewcommand{\geq}{\geqslant}
\renewcommand{\le}{\leqslant} \renewcommand{\leq}{\leqslant}

%----- Fonctions usuelles -----
\newcommand{\ch}{\mathop{\mathrm{ch}}\nolimits}
\newcommand{\sh}{\mathop{\mathrm{sh}}\nolimits}
\renewcommand{\tanh}{\mathop{\mathrm{th}}\nolimits}
\newcommand{\cotan}{\mathop{\mathrm{cotan}}\nolimits}
\newcommand{\Arcsin}{\mathop{\mathrm{arcsin}}\nolimits}
\newcommand{\Arccos}{\mathop{\mathrm{arccos}}\nolimits}
\newcommand{\Arctan}{\mathop{\mathrm{arctan}}\nolimits}
\newcommand{\Argsh}{\mathop{\mathrm{argsh}}\nolimits}
\newcommand{\Argch}{\mathop{\mathrm{argch}}\nolimits}
\newcommand{\Argth}{\mathop{\mathrm{argth}}\nolimits}
\newcommand{\pgcd}{\mathop{\mathrm{pgcd}}\nolimits} 

%----- Structure des exercices ------

\newcommand{\exercice}[1]{\video{0}}
\newcommand{\finexercice}{}
\newcommand{\noindication}{}
\newcommand{\nocorrection}{}

\newcounter{exo}
\newcommand{\enonce}[2]{\refstepcounter{exo}\hypertarget{exo7:#1}{}\label{exo7:#1}{\bf Exercice \arabic{exo}}\ \  #2\vspace{1mm}\hrule\vspace{1mm}}

\newcommand{\finenonce}[1]{
\ifthenelse{\equal{\ref{ind7:#1}}{\ref{bidon}}\and\equal{\ref{cor7:#1}}{\ref{bidon}}}{}{\par{\footnotesize
\ifthenelse{\equal{\ref{ind7:#1}}{\ref{bidon}}}{}{\hyperlink{ind7:#1}{\texttt{Indication} $\blacktriangledown$}\qquad}
\ifthenelse{\equal{\ref{cor7:#1}}{\ref{bidon}}}{}{\hyperlink{cor7:#1}{\texttt{Correction} $\blacktriangledown$}}}}
\ifthenelse{\equal{\myvideo}{0}}{}{{\footnotesize\qquad\texttt{\href{http://www.youtube.com/watch?v=\myvideo}{Vidéo $\blacksquare$}}}}
\hfill{\scriptsize\texttt{[#1]}}\vspace{1mm}\hrule\vspace*{7mm}}

\newcommand{\indication}[1]{\hypertarget{ind7:#1}{}\label{ind7:#1}{\bf Indication pour \hyperlink{exo7:#1}{l'exercice \ref{exo7:#1} $\blacktriangle$}}\vspace{1mm}\hrule\vspace{1mm}}
\newcommand{\finindication}{\vspace{1mm}\hrule\vspace*{7mm}}
\newcommand{\correction}[1]{\hypertarget{cor7:#1}{}\label{cor7:#1}{\bf Correction de \hyperlink{exo7:#1}{l'exercice \ref{exo7:#1} $\blacktriangle$}}\vspace{1mm}\hrule\vspace{1mm}}
\newcommand{\fincorrection}{\vspace{1mm}\hrule\vspace*{7mm}}

\newcommand{\finenonces}{\newpage}
\newcommand{\finindications}{\newpage}


\newcommand{\fiche}[1]{} \newcommand{\finfiche}{}
%\newcommand{\titre}[1]{\centerline{\large \bf #1}}
\newcommand{\addcommand}[1]{}

% variable myvideo : 0 no video, otherwise youtube reference
\newcommand{\video}[1]{\def\myvideo{#1}}

%----- Presentation ------

\setlength{\parindent}{0cm}

\definecolor{myred}{rgb}{0.93,0.26,0}
\definecolor{myorange}{rgb}{0.97,0.58,0}
\definecolor{myyellow}{rgb}{1,0.86,0}

\newcommand{\LogoExoSept}[1]{  % input : echelle       %% NEW
{\usefont{U}{cmss}{bx}{n}
\begin{tikzpicture}[scale=0.1*#1,transform shape]
  \fill[color=myorange] (0,0)--(4,0)--(4,-4)--(0,-4)--cycle;
  \fill[color=myred] (0,0)--(0,3)--(-3,3)--(-3,0)--cycle;
  \fill[color=myyellow] (4,0)--(7,4)--(3,7)--(0,3)--cycle;
  \node[scale=5] at (3.5,3.5) {Exo7};
\end{tikzpicture}}
}


% titre
\newcommand{\titre}[1]{%
\vspace*{-4ex} \hfill \hspace*{1.5cm} \hypersetup{linkcolor=black, urlcolor=black} 
\href{http://exo7.emath.fr}{\LogoExoSept{3}} 
 \vspace*{-5.7ex}\newline 
\hypersetup{linkcolor=blue, urlcolor=blue}  {\Large \bf #1} \newline 
 \rule{12cm}{1mm} \vspace*{3ex}}

%----- Commandes supplementaires ------



\begin{document}

%%%%%%%%%%%%%%%%%% EXERCICES %%%%%%%%%%%%%%%%%%
\fiche{f00152, quinio, 2011/05/20}

Exercices : Martine Quinio

\titre{Variables aléatoires discrètes}

\exercice{6005, quinio, 2011/05/20}

\enonce{006005}{}
Une entreprise pharmaceutique décide de faire des économies sur les
tarifs d'affranchissements des courriers publicitaires à envoyer aux
clients.
Pour cela, elle décide d'affranchir, au hasard, une proportion de 3
lettres sur 5 au tarif urgent, les autres au tarif normal.
\begin{enumerate}
\item Quatre lettres sont envoyées dans un cabinet médical de quatre médecins:
quelle est la probabilité des événements:

A : <<Au moins l'un d'entre eux reçoit une lettre au tarif urgent>>.

B : <<Exactement 2 médecins sur les quatre reçoivent une lettre au tarif urgent>>.

\item Soit $X$ la variable aléatoire: <<nombre de lettres affranchies au tarif
urgent parmi 10 lettres>>:
Quelle est la loi de probabilité de $X$, quelle est son espérance,
quelle est sa variance?
\end{enumerate}
\finenonce{006005}


\finexercice
\exercice{6006, quinio, 2011/05/20}

\enonce{006006}{}
On prend au hasard, en même temps, trois ampoules dans un lot de 15 dont
5 sont défectueuses. Calculer la probabilité des événements:

$A$ : au moins une ampoule est défectueuse;

$B$ : les 3 ampoules sont défectueuses;

$C$ : exactement une ampoule est défectueuse.
\finenonce{006006}


\finexercice
\exercice{6007, quinio, 2011/05/20}

\enonce{006007}{}
Un avion peut accueillir 20 personnes; des statistiques montrent 
que 25\% clients ayant réservé ne viennent pas.
Soit $X$ la variable aléatoire: <<nombre de clients qui viennent après
réservation parmi 20>>.
Quelle est la loi de $X$ ? (on ne donnera que la forme générale)
quelle est son espérance, son écart-type ?
Quelle est la probabilité pour que $X$ soit égal à 15 ?
\finenonce{006007}


\finexercice
\exercice{6008, quinio, 2011/05/20}

\enonce{006008}{}
L'oral d'un concours comporte au total 100 sujets; les candidats
tirent au sort trois sujets et choisissent alors le sujet traité parmi
ces trois sujets. Un candidat se présente en ayant révisé 60
sujets sur les 100.
\begin{enumerate}
\item Quelle est la probabilité pour que le candidat ait révisé:
\begin{enumerate}
\item les trois sujets tirés;
\item exactement deux sujets sur les trois sujets;
\item aucun des trois sujets.
\end{enumerate}
\item Définir une variable aléatoire associée à ce problème 
et donner sa loi de probabilité, son espérance.
\end{enumerate}
\finenonce{006008}


\finexercice
\exercice{6009, quinio, 2011/05/20}
\enonce{006009}{}
Un candidat se présente à un concours où, cette fois, les
20 questions sont données sous forme de QCM. A chaque question, sont
proposées 4 réponses, une seule étant exacte. L'examinateur fait
le compte des réponses exactes données par les candidats.
Certains candidats répondent au hasard à chaque question; pour
ceux-la, définir une variable aléatoire associée à ce problème 
et donner sa loi de probabilité, son espérance.
\finenonce{006009}


\finexercice
\exercice{6010, quinio, 2011/05/20}

\enonce{006010}{}
Dans une poste d'un petit village, on remarque qu'entre 10 heures
et 11 heures, la probabilité pour que deux personnes entrent durant 
la même minute est considérée comme nulle et que l'arrivée des
personnes est indépendante de la minute considérée. 
On a observé que la probabilité pour qu'une personne se présente entre la
minute $n$ et la minute $n+1$ est: $p = 0.1$. On veut calculer la probabilité 
pour que : 3,4,5,6,7,8... personnes se présentent au guichet entre 10h et 11h.
\begin{enumerate}
\item Définir une variable aléatoire adaptée, puis répondre au problème considéré.

\item Quelle est la probabilité pour que au moins 10 personnes
se présentent au guichet entre 10h et 11h?
\end{enumerate}
\finenonce{006010}


\finexercice
\exercice{6011, quinio, 2011/05/20}

\enonce{006011}{}
Si dans une population une personne sur cent est un centenaire,
quelle est la probabilité de trouver au moins un centenaire parmi $100$
personnes choisies au hasard ? Et parmi $200$ personnes ?
\finenonce{006011}


\finexercice
\exercice{6012, quinio, 2011/05/20}

\enonce{006012}{}
Un industriel doit vérifier l'état de marche de ses machines et en
remplacer certaines le cas échéant. D'après des statistiques précédentes, 
il évalue à 30\% la probabilité pour une
machine de tomber en panne en 5 ans; parmi ces dernières, 
la probabilité de devenir hors d'usage suite à une panne plus grave est 
évaluée à 75\%; cette probabilité est de 40\% pour une machine
n'ayant jamais eu de panne.
\begin{enumerate}
\item Quelle est la probabilité pour une machine donnée de plus de cinq
ans d'être hors d'usage ?

\item Quelle est la probabilité pour une machine hors d'usage de n'avoir
jamais eu de panne auparavant ?

\item Soit $X$ la variable aléatoire <<nombre de machines
qui tombent en panne au bout de 5 ans, parmi 10 machines choisies au
hasard>>. Quelle est la loi de probabilité de $X$, (on
donnera le type de loi et les formules de calcul), son espérance, sa
variance et son écart-type ?

\item Calculer $P[X=5]$.
\end{enumerate}
\finenonce{006012}


\finexercice
\exercice{6013, quinio, 2011/05/20}

\enonce{006013}{}
Une population comporte en moyenne une personne mesurant plus de 1m90 sur
80 personnes.
Sur 100 personnes, calculer la probabilité qu'il y ait au moins une
personne mesurant plus de 1.90m (utiliser une loi de Poisson).
Sur 300 personnes, calculer la probabilité qu'il y ait au moins une
personne mesurant plus de 1.90m.
\finenonce{006013}


\finexercice

\finfiche

 \finenonces 



 \finindications 

\noindication
\noindication
\noindication
\noindication
\noindication
\noindication
\noindication
\noindication
\noindication


\newpage

\correction{006005}
\begin{enumerate}
\item On utilise une loi binomiale, loi de la variable aléatoire:
<<nombre de lettres affranchies au tarif urgent parmi 4 lettres>>
$n=5$, $p=\frac{3}{5}$. 
On obtient $P(A)=1-(\frac{2}{5})^{4}=0.9744$, 
$P(B)=\binom{4}{2}(\frac{2}{5})^{2}(\frac{3}{5})^{2}=0.3456$.

\item La loi de probabilité de $X$ est une loi binomiale, loi de la variable
aléatoire: <<nombre de lettres affranchies au tarif
urgent parmi 10 lettres>>.
$n=10$, $p=\frac{3}{5}$, son espérance est $np=6$, sa variance est $np(1-p)=\frac{12}{5}$.
\end{enumerate}
\fincorrection
\correction{006006}
On utilise une loi hypergéométrique

$P(A)=1-\frac{\binom{10}{3}}{\binom{15}{3}}=
0.736\,26$

$P(B)=\frac{\binom{5}{3}}{\binom{15}{3}}=2.\,
197\,8\times 10^{-2}$

$P(C)=\frac{\binom{5}{1}\binom{10}{2}}{\binom{15}{3}}=0.494\,51$
\fincorrection
\correction{006007}
Soit $X$ la variable aléatoire nombre de clients
qui viennent après réservation parmi 20.
La loi de $X$ est une loi binomiale de paramètres $n=20$, $p=0.75$.
Son espérance est $np=15$, son écart-type est $\sqrt{np(1-p)}=\sqrt{15\cdot 0.25}$.
La probabilité pour que $X$ soit égal à 15 est $\binom{20}{15}0.75^{15}0.25^{5}=0.202\,33$.
\fincorrection
\correction{006008}
La variable aléatoire associée à ce problème est $X$ 
<<nombre de sujets révisés parmi les $3$>> ; son support est l'ensemble
$\{0,1,2,3\}$. La loi de $X$ est une loi hypergéométrique puisque
l'événement $[X=k]$, pour $k$ compris entre $0$ et $3$, se produit si
le candidat tire $k$ sujet(s) parmi les $60$ révisés, et $3-k$ sujets
parmi les $40$ non révisés.

Alors:
\begin{enumerate}
\item Les trois sujets tirés ont été révisés : $P[X=3]=\frac{\binom{60}{3}}{\binom{100}{3}}$.
\item Deux des trois sujets tirés ont été révisés: $P[X=2]=\frac{\binom{60}{2}.\binom{40}{1}}{\binom{100}{3}}$.
\item Aucun des trois sujets: $P[X=0]=\frac{\binom{40}{3}}{\binom{100}{3}}$.
\end{enumerate}
La loi de probabilité de $X$ est donnée sur le support $\{0,1,2,3\}$ par:
\begin{equation*}
P[X=k]=\frac{\binom{60}{k}.\binom{40}{3-k}}{\binom{100}{3}}
\end{equation*}

Résultats numériques:

$k=0:P[X=0]$ $\simeq 6.\,110\times 10^{-2}$

$k=1:P[X=1] \simeq 0.289\,$

$k=2:P[X=2]\simeq 0.438$

$k=3:P[X=3] \simeq 0.212$

L'espérance est $E(X)=1.8$ (selon la formule $E(X)=np$).
\fincorrection
\correction{006009}
Puisque les réponses sont données au hasard, chaque grille-réponses est en fait la 
répétition indépendante de $20$ épreuves aléatoires (il y a $4^{20}$ grilles-réponses). 
Pour chaque question la probabilité de succès est de $\frac{1}{4}$ et
l'examinateur fait le compte des succès: la variable aléatoire $X$,
nombre de bonnes réponses, obéit à une loi binomiale donc on a
directement les résultats.
Pour toute valeur de $k$ comprise entre $0$ et $20$:
$P[X=k]=C_{20}^{k}(\frac{1}{4})^{k}(1-\frac{1}{4})^{20-k}$, ce qui
donne la loi de cette variable aléatoire.

Quelle est l'espérance d'un candidat fumiste? C'est $E(X)=np=5$
\fincorrection
\correction{006010}
Une variable aléatoire adaptée à ce problème est le nombre $X$ de personnes 
se présentant au guichet entre 10h et 11h. Compte tenu
des hypothèses, on partage l'heure en $60$ minutes. Alors $X$ suit une
loi binomiale de paramètres $n=60$ et $p=0.1$. On est dans le cas de
processus poissonnien : on peut approcher la loi de $X$ par la loi de
Poisson de paramètre $\lambda =60\times 0.1=6$.
L'espérance de $X$ est donc $E(X)=6$;

On peut alors calculer les probabilités demandées:
$P[X=k]=\frac{6^{k}e^{-6}}{k!}$. Valeurs lues dans une table ou calculées :
$P[X=3]\simeq 0.9\%;$ $P[X=4]\simeq 13.4\%;$ $P[X=5]=P[X=6]\simeq
\,16.1\%;$
$P[X=7]\,\simeq 13.8\%;P[X=8]\simeq 10.3\,\%.$

Remarque : de façon générale si le paramètre 
$\lambda$ d'une loi de Poisson est un entier $K$, on a:
$P[X=K-1]=\frac{K^{K-1}e^{-K}}{(K-1)!}=\frac{K^{K}e^{-K}}{K!}=P[X=K]\,.$

Calculons maintenant la probabilité pour que au moins $10$
personnes se présentent au guichet entre $10$h et $11$h: C'est $P[X\geq
10]=1-\sum_{k=0}^{9}\frac{6^{k}e^{-6}}{k!}\simeq 8.392\times
10^{-2}.$

\fincorrection
\correction{006011}
La probabilité $p=\frac{1}{100}$ étant faible, on peut appliquer la
loi de Poisson d'espérance $100p=1$ au nombre $X$ de centenaires pris
parmi cent personnes. On cherche donc: $P[X\geq 1]=1-P[X=0]=1-e^{-1}\simeq
63\%$.

Sur un groupe de $200$ personnes: l'espérance est 2 donc: $P[X'\geq 1]=1-e^{-2}\simeq 86\%.$
La probabilité des événements : $[X' =1]$ et $[X' =2]$
sont les mêmes et valent: $0.14$.
Ainsi, sur 200 personnes, la probabilité de trouver exactement un
centenaire vaut $0.14$, égale à la probabilité de trouver
exactement deux centenaires. Cette valeur correspond au maximum de probabilité 
pour une loi de Poisson d'espérance $2$ et se généralise.
Si $X$ obeit à une loi de Poisson d'espérance $K$, alors le maximum de
probabilité est obtenu pour les événements $[X=K-1]$ et $[X=K].$
\fincorrection
\correction{006012}
\begin{enumerate}
\item 30\% est la probabilité de l'événement Panne, noté $Pa$;
la probabilité pour une machine donnée de plus de cinq ans, d'être hors d'usage est
$P(HU)=P(HU/Pa) P(Pa)+P(HU/nonPa) P(nonPa)=0.3\cdot 0.75+0.4\cdot
0.7=0.505$.

\item La probabilité pour une machine hors d'usage de n'avoir jamais eu de
panne auparavant est 
$P(\text{non}\,Pa/HU)=P(HU/\text{non}\,Pa) P(\text{non}\,Pa)/P(HU)=0.4\cdot 0.7/
0.505\,=0.554\,46$.

\item La loi de probabilité de $X$ est une loi binomiale, $n= 10$, $p=0.3$, espérance $3$.

\item $P[X=5]=\binom{10}{5}(0.3)^{5}(0.7)^{5}=0.102\,92$
\end{enumerate}
\fincorrection
\correction{006013}
Le nombre $X$ de personnes mesurant plus de 1.90m parmi 100 obéit à
une loi de Poisson de paramètre $\frac{100}{80}$.

La probabilité qu'il y ait au moins une personne mesurant plus de 1.90m
est donc $1-P[X=0]=1-e^{-\frac{100}{80}}=1-e^{-\frac{5}{4}}=0.713\,50$.

Sur 300 personnes:
la probabilité qu'il y ait au moins une personne mesurant plus de 1.90m
est donc $1-P[Y=0]=1-e^{-\frac{300}{80}}=0.976\,48$.

\fincorrection


\end{document}

