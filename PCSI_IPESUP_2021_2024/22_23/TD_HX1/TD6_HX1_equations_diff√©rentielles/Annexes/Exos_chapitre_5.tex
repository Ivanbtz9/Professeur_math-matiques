\documentclass[a4paper,10pt]{article}

 %Configuration de la feuille 
\usepackage[landscape]{geometry}
\usepackage{multicol}

\usepackage{amsmath,amssymb,enumerate,graphicx,pgf,tikz,fancyhdr}
\usepackage[utf8]{inputenc}
\usetikzlibrary{arrows}
\usepackage{geometry}
\usepackage{tabvar}
\geometry{hmargin=2.2cm,vmargin=1.5cm}\pagestyle{fancy}
\rfoot{\bfseries\thepage}
\cfoot{}
\renewcommand{\footrulewidth}{0.5pt} %Filet en bas de page

 %Macros utilisées dans la base de données d'exercices 

\newcommand{\mtn}{\mathbb{N}}
\newcommand{\mtns}{\mathbb{N}^*}
\newcommand{\mtz}{\mathbb{Z}}
\newcommand{\mtr}{\mathbb{R}}
\newcommand{\mtk}{\mathbb{K}}
\newcommand{\mtq}{\mathbb{Q}}
\newcommand{\mtc}{\mathbb{C}}
\newcommand{\mch}{\mathcal{H}}
\newcommand{\mcp}{\mathcal{P}}
\newcommand{\mcb}{\mathcal{B}}
\newcommand{\mcl}{\mathcal{L}}
\newcommand{\mcm}{\mathcal{M}}
\newcommand{\mcc}{\mathcal{C}}
\newcommand{\mcmn}{\mathcal{M}}
\newcommand{\mcmnr}{\mathcal{M}_n(\mtr)}
\newcommand{\mcmnk}{\mathcal{M}_n(\mtk)}
\newcommand{\mcsn}{\mathcal{S}_n}
\newcommand{\mcs}{\mathcal{S}}
\newcommand{\mcd}{\mathcal{D}}
\newcommand{\mcsns}{\mathcal{S}_n^{++}}
\newcommand{\glnk}{GL_n(\mtk)}
\newcommand{\mnr}{\mathcal{M}_n(\mtr)}
\DeclareMathOperator{\ch}{ch}
\DeclareMathOperator{\sh}{sh}
\DeclareMathOperator{\vect}{vect}
\DeclareMathOperator{\card}{card}
\DeclareMathOperator{\comat}{comat}
\DeclareMathOperator{\imv}{Im}
\DeclareMathOperator{\rang}{rg}
\DeclareMathOperator{\Fr}{Fr}
\DeclareMathOperator{\diam}{diam}
\DeclareMathOperator{\supp}{supp}
\newcommand{\veps}{\varepsilon}
\newcommand{\mcu}{\mathcal{U}}
\newcommand{\mcun}{\mcu_n}
\newcommand{\dis}{\displaystyle}
\newcommand{\croouv}{[\![}
\newcommand{\crofer}{]\!]}
\newcommand{\rab}{\mathcal{R}(a,b)}
\newcommand{\pss}[2]{\langle #1,#2\rangle}
 %Document 

\begin{document} 
\begin{multicols}{4}

\begin{center}\textsc{{\huge Chapitre 3 : somme et produits}}\end{center}

\lhead{Groupe IPESUP}\chead{}\rhead{Année~2022-2023}\lfoot{M.Botcazou \& M.Dupré}\cfoot{\thepage/3}\rfoot{\textbf{Tournez la page S.V.P.}}\renewcommand{\headrulewidth}{0.4pt}\renewcommand{\footrulewidth}{0.4pt}


%TODO : rajouter du binôme de Newton, des sommes simples, l'exo avec l'égalité de la moyenne (développement), retrouver une nouvelle esthétique

\subsection*{Variables muettes}

\textcolor{blue}{\large{\bf Exercice 3.1}} (*)
%TODO : injectif, surjectif
Calculer les primitives des fonctions suivantes
 
\begin{enumerate}
\item $x \mapsto e^{x} \cos(x)$
\item $x \mapsto \sqrt{e^x-1}$
\item $x \mapsto x \sqrt[3]{1+x}$
\end{enumerate}

\textcolor{blue}{\large{\bf Exercice 3.2}} (*)

Trouver les couples $(a,b) \in \mathbb{R}^2$ tels que l'équation
$$ y" + a y' + by = 0 $$ admet une unique solution sur $\mathbb{R}$ de la forme $x \mapsto e^{\alpha x}$ avec $\alpha \in \mathbb{R}$

\textcolor{blue}{\large{\bf Exercice 3.2}} (*)

Trouver les couples $(a,b) \in \mathbb{R}^2$ tels que l'équation
$$ y" + a y' + by = 0 $$ admet une unique solution sur $\mathbb{R}$ de la forme $x \mapsto e^{\alpha x}$ avec $\alpha \in \mathbb{R}$

\end{multicols}
\end{document}