
%%%%%%%%%%%%%%%%%% PREAMBULE %%%%%%%%%%%%%%%%%%

\documentclass[11pt,a4paper]{article}

\usepackage{amsfonts,amsmath,amssymb,amsthm}
\usepackage[utf8]{inputenc}
\usepackage[T1]{fontenc}
\usepackage[francais]{babel}
\usepackage{mathptmx}
\usepackage{fancybox}
\usepackage{graphicx}
\usepackage{ifthen}

\usepackage{tikz}   

\usepackage{hyperref}
\hypersetup{colorlinks=true, linkcolor=blue, urlcolor=blue,
pdftitle={Exo7 - Exercices de mathématiques}, pdfauthor={Exo7}}

\usepackage{geometry}
\geometry{top=2cm, bottom=2cm, left=2cm, right=2cm}

%----- Ensembles : entiers, reels, complexes -----
\newcommand{\Nn}{\mathbb{N}} \newcommand{\N}{\mathbb{N}}
\newcommand{\Zz}{\mathbb{Z}} \newcommand{\Z}{\mathbb{Z}}
\newcommand{\Qq}{\mathbb{Q}} \newcommand{\Q}{\mathbb{Q}}
\newcommand{\Rr}{\mathbb{R}} \newcommand{\R}{\mathbb{R}}
\newcommand{\Cc}{\mathbb{C}} \newcommand{\C}{\mathbb{C}}
\newcommand{\Kk}{\mathbb{K}} \newcommand{\K}{\mathbb{K}}

%----- Modifications de symboles -----
\renewcommand{\epsilon}{\varepsilon}
\renewcommand{\Re}{\mathop{\mathrm{Re}}\nolimits}
\renewcommand{\Im}{\mathop{\mathrm{Im}}\nolimits}
\newcommand{\llbracket}{\left[\kern-0.15em\left[}
\newcommand{\rrbracket}{\right]\kern-0.15em\right]}
\renewcommand{\ge}{\geqslant} \renewcommand{\geq}{\geqslant}
\renewcommand{\le}{\leqslant} \renewcommand{\leq}{\leqslant}

%----- Fonctions usuelles -----
\newcommand{\ch}{\mathop{\mathrm{ch}}\nolimits}
\newcommand{\sh}{\mathop{\mathrm{sh}}\nolimits}
\renewcommand{\tanh}{\mathop{\mathrm{th}}\nolimits}
\newcommand{\cotan}{\mathop{\mathrm{cotan}}\nolimits}
\newcommand{\Arcsin}{\mathop{\mathrm{arcsin}}\nolimits}
\newcommand{\Arccos}{\mathop{\mathrm{arccos}}\nolimits}
\newcommand{\Arctan}{\mathop{\mathrm{arctan}}\nolimits}
\newcommand{\Argsh}{\mathop{\mathrm{argsh}}\nolimits}
\newcommand{\Argch}{\mathop{\mathrm{argch}}\nolimits}
\newcommand{\Argth}{\mathop{\mathrm{argth}}\nolimits}
\newcommand{\pgcd}{\mathop{\mathrm{pgcd}}\nolimits} 

%----- Structure des exercices ------

\newcommand{\exercice}[1]{\video{0}}
\newcommand{\finexercice}{}
\newcommand{\noindication}{}
\newcommand{\nocorrection}{}

\newcounter{exo}
\newcommand{\enonce}[2]{\refstepcounter{exo}\hypertarget{exo7:#1}{}\label{exo7:#1}{\bf Exercice \arabic{exo}}\ \  #2\vspace{1mm}\hrule\vspace{1mm}}

\newcommand{\finenonce}[1]{
\ifthenelse{\equal{\ref{ind7:#1}}{\ref{bidon}}\and\equal{\ref{cor7:#1}}{\ref{bidon}}}{}{\par{\footnotesize
\ifthenelse{\equal{\ref{ind7:#1}}{\ref{bidon}}}{}{\hyperlink{ind7:#1}{\texttt{Indication} $\blacktriangledown$}\qquad}
\ifthenelse{\equal{\ref{cor7:#1}}{\ref{bidon}}}{}{\hyperlink{cor7:#1}{\texttt{Correction} $\blacktriangledown$}}}}
\ifthenelse{\equal{\myvideo}{0}}{}{{\footnotesize\qquad\texttt{\href{http://www.youtube.com/watch?v=\myvideo}{Vidéo $\blacksquare$}}}}
\hfill{\scriptsize\texttt{[#1]}}\vspace{1mm}\hrule\vspace*{7mm}}

\newcommand{\indication}[1]{\hypertarget{ind7:#1}{}\label{ind7:#1}{\bf Indication pour \hyperlink{exo7:#1}{l'exercice \ref{exo7:#1} $\blacktriangle$}}\vspace{1mm}\hrule\vspace{1mm}}
\newcommand{\finindication}{\vspace{1mm}\hrule\vspace*{7mm}}
\newcommand{\correction}[1]{\hypertarget{cor7:#1}{}\label{cor7:#1}{\bf Correction de \hyperlink{exo7:#1}{l'exercice \ref{exo7:#1} $\blacktriangle$}}\vspace{1mm}\hrule\vspace{1mm}}
\newcommand{\fincorrection}{\vspace{1mm}\hrule\vspace*{7mm}}

\newcommand{\finenonces}{\newpage}
\newcommand{\finindications}{\newpage}


\newcommand{\fiche}[1]{} \newcommand{\finfiche}{}
%\newcommand{\titre}[1]{\centerline{\large \bf #1}}
\newcommand{\addcommand}[1]{}

% variable myvideo : 0 no video, otherwise youtube reference
\newcommand{\video}[1]{\def\myvideo{#1}}

%----- Presentation ------

\setlength{\parindent}{0cm}

\definecolor{myred}{rgb}{0.93,0.26,0}
\definecolor{myorange}{rgb}{0.97,0.58,0}
\definecolor{myyellow}{rgb}{1,0.86,0}

\newcommand{\LogoExoSept}[1]{  % input : echelle       %% NEW
{\usefont{U}{cmss}{bx}{n}
\begin{tikzpicture}[scale=0.1*#1,transform shape]
  \fill[color=myorange] (0,0)--(4,0)--(4,-4)--(0,-4)--cycle;
  \fill[color=myred] (0,0)--(0,3)--(-3,3)--(-3,0)--cycle;
  \fill[color=myyellow] (4,0)--(7,4)--(3,7)--(0,3)--cycle;
  \node[scale=5] at (3.5,3.5) {Exo7};
\end{tikzpicture}}
}


% titre
\newcommand{\titre}[1]{%
\vspace*{-4ex} \hfill \hspace*{1.5cm} \hypersetup{linkcolor=black, urlcolor=black} 
\href{http://exo7.emath.fr}{\LogoExoSept{3}} 
 \vspace*{-5.7ex}\newline 
\hypersetup{linkcolor=blue, urlcolor=blue}  {\Large \bf #1} \newline 
 \rule{12cm}{1mm} \vspace*{3ex}}

%----- Commandes supplementaires ------



\begin{document}

%%%%%%%%%%%%%%%%%% EXERCICES %%%%%%%%%%%%%%%%%%

\fiche{f00105, rouget, 2010/07/11}

\titre{Calculs de limites, développements limités, développements asymptotiques} 

Exercices de Jean-Louis Rouget.
Retrouver aussi cette fiche sur \texttt{\href{http://www.maths-france.fr}{www.maths-france.fr}}

\begin{center}
* très facile\quad** facile\quad*** difficulté moyenne\quad**** difficile\quad***** très difficile\\
I~:~Incontournable\quad T~:~pour travailler et mémoriser le cours
\end{center}


\exercice{5426, exo7, 2010/07/06}
\enonce{005426}{IT}
Etudier l'existence et la valeur éventuelle des limites suivantes
\begin{enumerate}
 \item $\lim_{x\rightarrow \pi/2}(\sin x)^{1/(2x-\pi)}$
 \item $\lim_{x\rightarrow \pi/2}|\tan x|^{\cos x}$
 \item $\lim_{n\rightarrow +\infty}\left(\cos(\frac{n\pi}{3n+1})+\sin(\frac{n\pi}{6n+1})\right)^n$
 \item $\lim_{x\rightarrow 0}(\cos x)^{\ln|x|}$
 \item $\lim_{x\rightarrow \pi/2}\cos x.e^{1/(1-\sin x)}$
 \item $\lim_{x\rightarrow \pi/3}\frac{2\cos^2x+\cos x-1}{2\cos^2x-3\cos x+1}$
 \item $\lim_{x\rightarrow 0}\left(\frac{1+\tan x}{1+\tanh x}\right)^{1/\sin x}$
 \item $\lim_{x\rightarrow e,\;x<e}(\ln x)^{\ln(e-x)}$
 \item $\lim_{x\rightarrow 1,\;x>1}\frac{x^x-1}{\ln(1-\sqrt{x^2-1})}$
 \item $\lim_{x\rightarrow +\infty}\frac{x\ln(\ch x-1)}{x^2+1}$
 \item $\lim_{x\rightarrow 0,\;x>0}\frac{(\sin x)^x-x^{\sin x}}{\ln(x-x^2)+x-\ln x}$
 \item $\lim_{x\rightarrow +\infty}\left(\frac{\ln(x+1)}{\ln x}\right)^x$
 \item $\lim_{x \rightarrow 1/\sqrt{2}}\frac{(\Arcsin x)^2-\frac{\pi^2}{16}}{2x^2-1}$
 \item $\lim_{x\rightarrow +\infty}\left(\frac{\cos(a+\frac{1}{x})}{\cos a}\right)^x\;(\mbox{où}\;\cos a\neq0)$
\end{enumerate}
\finenonce{005426}


\finexercice
\exercice{5427, exo7, 2010/07/06}
\enonce{005427}{IT}
Déterminer les développements limités à l'ordre demandé au voisinage des points indiqués~:
\begin{enumerate}
 \item $\frac{1}{1-x^2-x^3}\;(\mbox{ordre}\;7\;\mbox{en}\;0)$
 \item $\frac{1}{\cos x}\;(\mbox{ordre}\;7\;\mbox{en}\;0)$
 \item $\Arccos\sqrt{\frac{x}{\tan x}}\;(\mbox{ordre}\;3\;\mbox{en}\;0)$
 \item $\tan x\;(\mbox{ordre}\;3\;\mbox{en}\;\frac{\pi}{4})$
 \item $(\ch x)^{1/x^2}\;(\mbox{ordre}\;2\;\mbox{en}\;0)$
 \item $\tan^3x(\cos(x^2)-1)\;(\mbox{ordre}\;8\;\mbox{en}\;0)$
 \item $\frac{\ln(1+x)}{x^2}\;(\mbox{ordre}\;3\;\mbox{en}\;1)$ 
 \item $\Arctan(\cos x)\;(\mbox{ordre}\;5\;\mbox{en}\;0)$
 \item $\Arctan\sqrt{\frac{x+1}{x+2}}\;(\mbox{ordre}\;2\;\mbox{en}\;0)$
 \item $\frac{1}{x^2}-\frac{1}{\Arcsin^2x}\;(\mbox{ordre}\;5\;\mbox{en}\;0)$
 \item $\int_{x}^{x^2}\frac{1}{\sqrt{1+t^4}}\;dt\;(\mbox{ordre}\;10\;\mbox{en}\;0)$
 \item $\ln\left(\sum_{k=0}^{99}\frac{x^k}{k!}\right)\;(\mbox{ordre}\;100\;\mbox{en}\;0)$
 \item $\tan\sqrt[3]{4(\pi^3+x^3)}\;(\mbox{ordre}\;3\;\mbox{en}\;\pi)$
\end{enumerate}
\finenonce{005427}


\finexercice
\exercice{5428, exo7, 2010/07/06}
\enonce{005428}{***}
Soit $0<a<b$. Etude complète de la fonction $f(x)=\left(\frac{a^x+b^x}{2}\right)^{1/x}$. 
\finenonce{005428}


\finexercice
\exercice{5429, exo7, 2010/07/06}
\enonce{005429}{**}
Etude au voisinage de $+\infty$ de $\sqrt{x^2-3}-\sqrt[3]{8x^3+7x^2+1}$.
\finenonce{005429}


\finexercice
\exercice{5430, exo7, 2010/07/06}
\enonce{005430}{**}
Soit $f(x)=\frac{x}{1-x^2}$. Calculer $f^{(n)}(0)$ en moins de $10$ secondes puis $f^{(n)}(x)$ pour $|x|\neq1$ en à peine plus de temps).
\finenonce{005430}


\finexercice
\exercice{5431, exo7, 2010/07/06}
\enonce{005431}{IT}
\begin{enumerate}
\item  Equivalent simple en $+\infty$ et $-\infty$ de $\sqrt{x^2+3x+5}-x+1$.
\item  Equivalent simple en $0$, $1$, $2$ et $+\infty$ de $3x^2-6x$
\item Equivalent simple en $0$ de $(\sin x)^{x-x^2}-(x-x^2)^{\sin x}$.
\item  Equivalent simple en $+\infty$ de $x^{\tanh x}$.
\item  Equivalent simple en $0$ de $\tan(\sin x)-\sin(\tan x)$.
\end{enumerate}
\finenonce{005431}


\finexercice
\exercice{5432, exo7, 2010/07/06}
\enonce{005432}{**IT}
Développement asymptotique à la précision $\frac{1}{n^3}$ de $u_n=\frac{1}{n!}\sum_{k=0}^{n}k!$.
\finenonce{005432}


\finexercice
\exercice{5433, exo7, 2010/07/06}
\enonce{005433}{**IT}
\begin{enumerate}
\item  Développement asymptotique à la précision $x^2$ en $0$ de $\frac{1}{x(e^x-1)}-\frac{1}{x^2}$.
\item  Développement asymptotique à la précision $\frac{1}{x^3}$ en $+\infty$ de $x\ln(x+1)-(x+1)\ln x$.
\end{enumerate}
\finenonce{005433}


\finexercice
\exercice{5434, exo7, 2010/07/06}
\enonce{005434}{**}
Soient $a>0$ et $b>0$. Pour $n\in\Nn^*$ et $x\in\Rr$, on pose $f_n(x)=\left(1+\frac{x}{n}\right)^n$.
\begin{enumerate}
\item  Equivalent simple quand $n$ tend vers $+\infty$ de $f_n(a+b)-f_n(a)f_n(b)$.
\item  Même question pour $e^{-a}f_n(a)-1+\frac{a^2}{2n}$.
\end{enumerate}
\finenonce{005434}


\finexercice
\exercice{5435, rouget, 2010/07/06}
\enonce{005435}{***I}
Soit $u_0\in]0,\frac{\pi}{2}]$. Pour $n\in\Nn$, on pose $u_{n+1}=\sin(u_n)$.
\begin{enumerate}
\item  Montrer brièvement que la suite $u$ est strictement positive et converge vers $0$.
\item 
\begin{enumerate}
\item Déterminer un réel $\alpha$ tel que la suite $u_{n+1}^\alpha-u_n^\alpha$ ait une limite finie non nulle.
\item En utilisant le lemme de \textsc{Cesaro}, déterminer un équivalent simple de $u_n$.
\end{enumerate}
\end{enumerate}

\finenonce{005435}


\finexercice
\exercice{5436, rouget, 2010/07/06}
\enonce{005436}{**I}
Soit $u$ la suite définie par la donnée de son premier terme $u_0>0$ et la relation $\forall n\in\Nn,\;u_{n+1}=u_ne^{-u_n}$. Equivalent simple de $u_n$ quand $n$ tend vers $+\infty$.
\finenonce{005436}


\finexercice
\exercice{5437, rouget, 2010/07/06}
\enonce{005437}{***I}
\begin{enumerate}
\item  Montrer que l'équation $\tan x=x$ a une unique solution dans l'intervalle $[n\pi,(n+1)\pi]$ pour $n$ entier naturel donné. On note $x_n$ cette solution.
\item  Trouver un développement asymptotique de $x_n$ à la précision $\frac{1}{n^2}$.
\end{enumerate}
\finenonce{005437}


\finexercice
\exercice{5438, rouget, 2010/07/06}
\enonce{005438}{}
\begin{enumerate}
\item  Montrer que l'équation $x+\ln x=k$ admet, pour $k$ réel donné, une unique solution dans $]0,+\infty[$, notée $x_k$.
\item  Montrer que, quand $k$ tend vers $+\infty$, on a~:~$x_k=ak+b\ln k+c\frac{\ln k}{k}+o\left(\frac{\ln k}{k}\right)$ où $a$, $b$ et $c$ sont des constantes à déterminer.
\end{enumerate}
\finenonce{005438}


\finexercice\exercice{5439, rouget, 2010/07/10}
\enonce{005439}{**}
Soit $f(x)=1+x+x^2+x^3\sin\frac{1}{x^2}$ si $x\neq0$ et $1$ si $x=0$.
\begin{enumerate}
\item  Montrer que $f$ admet en $0$ un développement limité d'ordre $2$.
\item  Montrer que $f$ est dérivable sur $\Rr$.
\item  Montrer que $f'$ n'admet en $0$ aucun développement limité d'aucun ordre que ce soit.
\end{enumerate}
\finenonce{005439}


\finexercice
\exercice{5440, rouget, 2010/07/10}
\enonce{005440}{**IT}
Etude au voisinage de $0$ de $f(x)=\frac{1}{x}-\frac{1}{\Arcsin x}$ (existence d'une tangente~?)
\finenonce{005440}


\finexercice
\exercice{5441, rouget, 2010/07/10}
\enonce{005441}{**I}
\begin{enumerate}
\item  La fonction $x\mapsto\Arccos x$ admet-elle en $1$ (à gauche) un développement limité d'ordre $0$~?~d'ordre $1$~?
\item  Equivalent simple de $\Arccos x$ en $1$.
\end{enumerate}
\finenonce{005441}


\finexercice
\exercice{5442, rouget, 2010/07/10}
\enonce{005442}{***}
\begin{enumerate}
\item  Développement limité à l'ordre $n$ en $0$ de $f(x)=\frac{1}{(1-x)^2(1+x)}$.
\item  Soit $a_k$ le $k$-ème coefficient. Montrer que $a_k$ est le nombre de solutions dans $\Nn^2$ de l'équation $p+2q=k$.
\end{enumerate}
\finenonce{005442}


\finexercice

\finfiche


 \finenonces 



 \finindications 

\noindication
\noindication
\noindication
\noindication
\noindication
\noindication
\noindication
\noindication
\noindication
\noindication
\noindication
\noindication
\noindication
\noindication
\noindication
\noindication
\noindication


\newpage

\correction{005426}
\begin{enumerate}
 \item  Si $x\in]0,\pi[$, $\sin x>0$, de sorte que la fonction proposée est bien définie sur un voisinage pointé de $\frac{\pi}{2}$ (c'est-à-dire un voisinage de $\frac{\pi}{2}$ auquel on a enlevé le point $\frac{\pi}{2}$) et de plus $(\sin x)^{1/(2x-\pi)}=e^{\ln(\sin x)/(2x-\pi)}$.
Quand $x$ tend vers $\frac{\pi}{2}$, $\sin x$ tend vers $1$ et donc

$$\ln(\sin x)\sim\sin x-1=-\left(1-\cos\left(\frac{\pi}{2}-x\right)\right)\sim-\frac{1}{2}\left(\frac{\pi}{2}-x\right)^2=-\frac{(2x-\pi)^2}{8}.$$
Donc, $\frac{\ln(\sin x)}{2x-\pi}\sim-\frac{2x-\pi}{8}\rightarrow0$ et enfin $(\sin x)^{1/(2x-\pi)}=e^{\ln(\sin x)/(2x-\pi)}\rightarrow e^0=1$.

\begin{center}
\shadowbox{
$\lim_{x \rightarrow \frac{\pi}{2}}(\sin x)^{1/(2x-\pi)}=1$.
}
\end{center}
 \item  Si $x\in]0,\pi[\setminus\left\{\frac{\pi}{2}\right\}$, $|\tan x|>0$, de sorte que la fonction proposée est bien définie sur un voisinage pointé de $\frac{\pi}{2}$ et de plus $|\tan x|^{\cos x}=e^{\cos x\ln(|\tan x|)}$. Quand $x$ tend vers $\frac{\pi}{2}$, 

$$\ln|\tan x|=\ln|\sin x|-\ln|\cos x|\sim-\ln|\cos x|,$$
puis $\cos x\ln|\tan x|\sim-\cos x\ln|\cos x|\rightarrow0$ (car, quand $u$ tend vers $0$, $u\ln u\rightarrow 0$).
Donc, $|\tan x|^{\cos x}=e^{\cos x\ln|\tan x|}\rightarrow e^0=1$.

\begin{center}
\shadowbox{
$\lim_{x \rightarrow \frac{\pi}{2}}|\tan x|^{\cos x}=1$.
}
\end{center}
 \item  Quand $n$ tend vers $+\infty$, $\cos\frac{n\pi}{3n+1}+\sin\frac{n\pi}{6n+1}\rightarrow\cos\frac{\pi}{3}+\sin\frac{\pi}{6}=1$ (et on est en présence d'une indétermination du type $1^{+\infty}$). Quand $n$ tend vers $+\infty$,

\begin{align*}\ensuremath
\cos\frac{n\pi}{3n+1}&=\cos\left(\frac{\pi}{3}\left(1+\frac{1}{3n}\right)^{-1}\right)
=\cos\left(\frac{\pi}{3}-\frac{\pi}{9n}+o\left(\frac{1}{n}\right)\right)\\
 &=\frac{1}{2}\cos\left(\frac{\pi}{9n}+o\left(\frac{1}{n}\right)\right)+\frac{\sqrt{3}}{2}\sin\left(\frac{\pi}{9n}+o\left(\frac{1}{n}\right)\right)=\frac{1}{2}\left(1+o\left(\frac{1}{n}\right)\right)+\frac{\sqrt{3}}{2}\left(\frac{\pi}{9n}+o\left(\frac{1}{n}\right)\right)\\
  &=\frac{1}{2}+\frac{\sqrt{3}\pi}{18n}+o\left(\frac{1}{n}\right)
\end{align*}
De même, 

\begin{align*}\ensuremath
\sin\frac{n\pi}{6n+1}&=\sin\left(\frac{\pi}{6}\left(1+\frac{1}{6n}\right)^{-1}\right)=\sin\left(\frac{\pi}{6}-\frac{\pi}{36n}
+o\left(\frac{1}{n}\right)\right)\\
 &=\frac{1}{2}\cos\left(\frac{\pi}{36n}+o\left(\frac{1}{n}\right)\right)-\frac{\sqrt{3}}{2}
 \sin\left(\frac{\pi}{36n}+o\left(\frac{1}{n}\right)\right)
 =\frac{1}{2}-\frac{\sqrt{3}\pi}{72n}+o\left(\frac{1}{n}\right).
\end{align*}
Puis, 

$$n\ln\left(\cos\frac{n\pi}{3n+1}+\sin\frac{n\pi}{6n+1}\right)=n\ln\left(1+\frac{\sqrt{3}\pi}{24n}+ o\left(\frac{1}{n}\right)\right)= n\left(\frac{\sqrt{3}\pi}{24n}+o\left(\frac{1}{n}\right)\right)=\frac{\sqrt{3}\pi}{24}+o(1),$$
et donc

\begin{center}
\shadowbox{
$\lim_{n\rightarrow +\infty}\left(\cos\frac{n\pi}{3n+1}+\sin\frac{n\pi}{6n+1}\right)^n=e^{\sqrt{3}\pi/24}$.
}
\end{center}
 \item  Quand $x$ tend vers $0$, $\ln(\cos x)\sim\cos x-1\sim-\frac{x^2}{2}$. Puis, $\ln|x|\ln(\cos x)\sim-\frac{x^2}{2}\ln|x|\rightarrow0$.
Donc, $(\cos x)^{\ln|x|}\rightarrow e^0=1$.

\begin{center}
\shadowbox{
$\lim_{x\rightarrow 0}(\cos x)^{\ln|x|}=1$.
}
\end{center}
 \item  Quand $x$ tend vers $\frac{\pi}{2}$, $\frac{1}{1-\sin x}$ tend vers $+\infty$. Posons $h=x-\frac{\pi}{2}$ puis $\varepsilon=\mbox{sgn}(h)$, de sorte que 

$$(\cos x)e^{1/(1-\sin x)}=-\varepsilon|\sin h|e^{1/(1-\cos h)}=-\varepsilon e^{\ln|\sin h|+\frac{1}{1-\cos h}}.$$
Or, quand $h$ tend vers $0$, 

$$\ln|\sin h|+\frac{1}{1-\cos h}=\frac{(1-\cos h)\ln|\sin h|+1}{1-\cos h}=\frac{(-\frac{h^2}{2}+o(h^2))(\ln|h|+o(\ln|h|))+1}{\frac{h^2}{2}+o(h^2)}=\frac{1+o(1)}{\frac{h^2}{2}+o(h^2)}
\sim\frac{2}{h^2},$$ et donc, quand $h$ tend vers $0$, $\ln|\sin h|+\frac{1}{1-\cos h}\sim\frac{2}{h^2}\rightarrow+\infty$. Par suite,

\begin{center}
\shadowbox{
$\lim_{x\rightarrow \pi/2,\;x<\pi/2}\cos(x)e^{1/(1-\sin x)}=+\infty$ et $\lim_{x\rightarrow \pi/2,\;x>\pi/2}\cos(x)e^{1/(1-\sin x)}=-\infty$.
}
\end{center}

 \item  Pour $x\in\Rr$, $2\cos^2x-3\cos x+1=(2\cos x-1)(\cos x-1)$ et donc 

$$\forall x\in\Rr,\;2\cos^2x-3\cos x+1=0\Leftrightarrow x\in\left(\pm\frac{\pi}{3}+2\pi\Zz\right)\cup2\pi\Zz.$$
Pour $x\notin\left(\pm\frac{\pi}{3}+2\pi\Zz\right)\cup2\pi\Zz$,
  
$$\frac{2\cos^2x+\cos x-1}{2\cos^2x-3\cos x+1}=\frac{(2\cos x-1)(\cos x+1)}{(2\cos x-1)(\cos x-1)}
=\frac{\cos x+1}{\cos x-1},$$
et donc, $\lim_{x\rightarrow \pi/3}\frac{2\cos^2x+\cos x-1}{2\cos^2x-3\cos x+1}=\frac{\frac{1}{2}+1}{\frac{1}{2}-1}=-3$.

\begin{center}
\shadowbox{
$\lim_{x\rightarrow \pi/3}\frac{2\cos^2x+\cos x-1}{2\cos^2x-3\cos x+1}=-3$.
}
\end{center}
 \item  Quand $x$ tend vers $0$, 
 
$$\frac{1+\tan x}{1+\tanh x}=\frac{1+x+o(x)}{1+x+o(x)}=(1+x+o(x)(1-x+o(x))=1+o(x).$$
Puis, quand $x$ tend vers $0$,

$$\frac{1}{\sin x}\ln\left(\frac{1+\tan x}{1+\tanh x}\right)=\frac{\ln(1+o(x))}{x+o(x)}=\frac{o(x)}{x+o(x)}=\frac{o(1)}{1+o(1)}\rightarrow0.$$
Donc, 

\begin{center}
\shadowbox{
$\lim_{x\rightarrow 0}\left(\frac{1+\tan x}{1+\tanh x}\right)^{1/\sin x}=1$.
}
\end{center}

 \item  Quand $x$ tend vers $e$ par valeurs inférieures, $\ln(x)$ tend vers $1$ et donc

$$\ln(\ln x)\sim\ln x-1=\ln\left(\frac{x}{e}\right)\sim\frac{x}{e}-1=-\frac{1}{e}(e-x),$$
puis,

$$\ln(e-x)\ln(\ln x)\sim-\frac{1}{e}(e-x)\ln(e-x)\rightarrow 0,$$
et donc $(\ln x)^{\ln(e-x)}=e^{\ln(e-x)\ln(\ln x)}\rightarrow 1$.

\begin{center}
\shadowbox{
$\displaystyle\lim_{\substack{x\rightarrow e\\ x<e}}(\ln x)^{\ln(e-x)}=1$.
}
\end{center}

 \item  Quand $x$ tend vers $1$ par valeurs supérieures, $x\ln x\rightarrow0$, et donc

$$x^x-1=e^{x\ln x}-1\sim x\ln x\sim1\times(x-1)=x-1.$$
Ensuite, $\sqrt{x^2-1}$ tend vers $0$ et donc

$$\ln(1-\sqrt{x^2-1})\sim-\sqrt{x^2-1}=-\sqrt{(x-1)(x+1)}\sim-\sqrt{2(x-1)}.$$
Finalement, quand $x$ tend vers $1$ par valeurs supérieures, 

$$\frac{x^x-1}{\ln(1-\sqrt{x^2-1})}\sim\frac{x-1}{-\sqrt{2(x-1)}}=-\frac{1}{\sqrt{2}}\sqrt{x-1}\rightarrow0.$$

\begin{center}
\shadowbox{
$\displaystyle\lim_{\substack{x\rightarrow 1\\ x>e}}\frac{x^x-1}{\ln(1-\sqrt{x^2-1})}=0$.
}
\end{center}
 \item  Quand $x$ tend vers $+\infty$, 

$$\ln(\ch x-1)\sim\ln(\ch x)\sim\ln\left(\frac{e^x}{2}\right)=x-\ln2\sim x,$$ et donc 

$$\frac{x\ln(\ch x-1)}{x^2+1}\sim\frac{x\times x}{x^2}=1.$$

\begin{center}
\shadowbox{
$\displaystyle\lim_{x\rightarrow+\infty}\frac{x\ln(\ch x-1)}{x^2+1}=1$.
}
\end{center}
\item  Quand $x$ tend vers $0$ par valeurs supérieures,

$$\ln(x-x^2)+x-\ln x=x+\ln(1-x)=-\frac{x^2}{2}+o(x^2)\sim-\frac{x^2}{2}.$$
Ensuite,

$$(\sin x)^x=e^{x\ln(\sin x)}=e^{x\ln(x-\frac{x^3}{6}+o(x^3))}=e^{x\ln x}e^{x\ln(1-\frac{x^2}{6}+o(x^2))}=x^xe^{-\frac{x^3}{6}+o(x^3)}=x^x\left(1-\frac{x^3}{6}+o(x^3)\right),$$
et,

$$x^{\sin x}=e^{(x-\frac{x^3}{6}+o(x^3))\ln x}=e^{x\ln x}e^{-\frac{x^3\ln x}{6}+o(x^3\ln x)}=x^x\left(1-\frac{x^3\ln x}{6}+o(x^3\ln x)\right).$$
Donc,

$$(\sin x)^x-x^{\sin x}=x^x\left(1-\frac{x^3}{6}+o(x^3)\right)-x^x\left(1-\frac{x^3\ln x}{6}+o(x^3\ln x)\right)=x^x\left(\frac{x^3\ln x}{6}+o(x^3\ln x)\right)\sim\frac{x^3\ln x}{6},$$
et enfin

$$\frac{(\sin x)^x-x^{\sin x}}{\ln(x-x2^)+x-\ln x}\sim\frac{x^3\ln x/6}{-x^2/2}=-\frac{x\ln x}{3}\rightarrow0.$$

\begin{center}
\shadowbox{
$\displaystyle\lim_{\substack{x\rightarrow 0\\ x>0}}\frac{(\sin x)^x-x^{\sin x}}{\ln(x-x2^)+x-\ln x}=0$.
}
\end{center}
\item Quand $x$ tend vers $+\infty$,

$$\ln(x+1)=\ln x+\ln\left(1+\frac{1}{x}\right)=\ln x+\frac{1}{x}+o\left(\frac{1}{x}\right),$$
puis
 
$$\frac{\ln(x+1)}{\ln x}=1+\frac{1}{x\ln x}+o\left(\frac{1}{x\ln x}\right).$$
 

Ensuite,

$$x\ln\left(\frac{\ln(x+1)}{\ln x}\right)=x\ln\left(1+\frac{1}{x\ln x}+o\left(\frac{1}{x\ln x}\right)\right)=\frac{1}{\ln x}+o\left(\frac{1}{\ln x}\right)\rightarrow0.$$
Donc, $\left(\frac{\ln(x+1)}{\ln x}\right)^x=\text{exp}\left(x\ln\left(\frac{\ln(x+1)}{\ln x}\right)\right)\rightarrow e^0=1$.

\begin{center}
\shadowbox{
$\displaystyle\lim_{x\rightarrow+\infty}\left(\frac{\ln(x+1)}{\ln x}\right)^x=1$.
}
\end{center}
 \item Quand $x$ tend vers $\frac{1}{\sqrt{2}}$,

\begin{align*}\ensuremath
\frac{(\Arcsin x)^2-\frac{\pi^2}{16}}{2x^2-1}&=\frac{1}{2}\times\frac{\Arcsin x+\frac{\pi}{4}}{x+\frac{1}{\sqrt{2}}}
\times\frac{\Arcsin x-\frac{\pi}{4}}{x-\frac{1}{\sqrt{2}}}\sim\frac{1}{2}\times\frac{\frac{\pi}{4}+\frac{\pi}{4}}{\frac{1}{\sqrt{2}}+\frac{1}{\sqrt{2}}}\times\frac{\Arcsin x-\frac{\pi}{4}}{x-\frac{1}{\sqrt{2}}}=\frac{\pi}{4\sqrt{2}}\frac{\Arcsin x-\frac{\pi}{4}}{x-\frac{1}{\sqrt{2}}}\\
 &\rightarrow\frac{\pi}{4\sqrt{2}}(\Arcsin)'(\frac{1}{\sqrt{2}})=\frac{\pi}{4\sqrt{2}}\frac{1}{\sqrt{1-\frac{1}{2}}}=\frac{\pi}{4}.
\end{align*}

\begin{center}
\shadowbox{
$\lim_{x \rightarrow 1/\sqrt{2}}\frac{(\Arcsin x)^2-\frac{\pi^2}{16}}{2x^2-1}=\frac{\pi}{4}
$.
}
\end{center}
 \item Quand $x$ tend vers $+\infty$,

\begin{align*}\ensuremath
x\ln\left(\frac{\cos\left(a+\frac{1}{x}\right)}{\cos a}\right)&=x\ln\left(\cos\frac{1}{x}-\tan a\sin\frac{1}{x}\right)=x\ln\left(1-\frac{\tan a}{x}+o\left(\frac{1}{x}\right)\right)=x\left(-\frac{\tan a}{x}+o\left(\frac{1}{x}\right)\right)\\
 &=-\tan a+o(1),
\end{align*}
et donc $\lim_{x\rightarrow +\infty}\left(\frac{\cos\left(a+\frac{1}{x}\right)}{\cos a}\right)^x=e^{-\tan a}$.

\begin{center}
\shadowbox{
$\lim_{x\rightarrow +\infty}\left(\frac{\cos\left(a+\frac{1}{x}\right)}{\cos a}\right)^x=e^{-\tan a}
$.
}
\end{center}
\end{enumerate}
\fincorrection
\correction{005427}
\begin{enumerate}
 \item 
\begin{align*}\ensuremath 
\frac{1}{1-x^2-x^3}\underset{x\rightarrow0}{=}1+(x^2+x^3)+(x^2+x^3)^2+(x^2+x^3)^3+o(x^7)= 1+x^2+x^3+x^4+2x^5+2x^6+3x^7+o(x^7).
\end{align*}

\begin{center}
\shadowbox{
$\frac{1}{1-x^2-x^3}\underset{x\rightarrow0}{=}1+x^2+x^3+x^4+2x^5+2x^6+3x^7+o(x^7)$.
}
\end{center}
 \item 

\begin{align*}\ensuremath
\frac{1}{\cos x}&\underset{x\rightarrow0}{=}\left(1-\frac{x^2}{2}+\frac{x^4}{24}-\frac{x^6}{720}+o(x^7)\right)^{-1}
=1+\left(\frac{x^2}{2}-\frac{x^4}{24}+\frac{x^6}{720}\right)+\left(\frac{x^2}{2}-\frac{x^4}{24}\right)^2+\left(\frac{x^2}{2}\right)^3+o(x^7)\\
 &=1+\frac{x^2}{2}+x^4\left(-\frac{1}{24}+\frac{1}{4}\right)+x^6\left(\frac{1}{720}-\frac{1}{24}+\frac{1}{8}\right)+o(x^7)
=1+\frac{1}{2}x^2+\frac{5}{24}x^4+\frac{61}{720}x^6+o(x^7).
\end{align*}

\begin{center}
\shadowbox{
$\frac{1}{\cos x}\underset{x\rightarrow0}{=}1+\frac{1}{2}x^2+\frac{5}{24}x^4+\frac{61}{720}x^6+o(x^7)$.
}
\end{center}
 \item  \textbf{Remarques.}

  \begin{enumerate}
  \item Pour $x\in\left]-\frac{\pi}{2},\frac{\pi}{2}\right[\setminus\{0\}$, on a $0<\frac{x}{\tan x}<1$ et donc la fonction $x\mapsto\Arccos\left(\frac{x}{\tan x}\right)$ est définie sur $\left]-\frac{\pi}{2},\frac{\pi}{2}\right[\setminus\{0\}$ (qui est un voisinage pointé de $0$).\rule[-2mm]{0mm}{6mm}
  \item Quand $x$ tend vers $0$, $\frac{x}{\tan x}\rightarrow1$ et donc $\Arccos\left(\frac{x}{\tan x}\right)=o(1)$ (développement limité à l'ordre $0$).
  \item La fonction $x\mapsto\Arccos x$ n'est pas dérivable en $1$ et n'admet donc pas en $1$ de développement limité d'ordre supérieur ou égal à $1$ (donc à priori, c'est mal parti).
  \item La fonction proposée est paire et, si elle admet en $0$ un développement limité d'ordre $3$, sa partie régulière ne contient que des exposants pairs.

  \end{enumerate}

\textbullet~Recherche d'un équivalent simple de $\Arccos x$ en $1$ à gauche.
Quand $x$ tend vers $1$ par valeurs inférieures, $\Arccos x\rightarrow 0$ et donc,

$$\Arccos x\sim\sin(\Arccos x)=\sqrt{1-x^2}=\sqrt{(1+x)(1-x)}\sim\sqrt{2}\sqrt{1-x}.$$ 
\textbullet~Déterminons un équivalent simple de $\Arccos\left(\sqrt{\frac{x}{\tan x}}\right)$ en $0$. D'après ce qui précède,

$$\Arccos\left(\sqrt{\frac{x}{\tan x}}\right)\sim\sin\left(\Arccos\left(\sqrt{\frac{x}{\tan x}}\right)\right)=\sqrt{1-\left(\sqrt{\frac{x}{\tan x}}\right)^2}=\sqrt{\frac{\tan x-x}{\tan x}}\sim\sqrt{\frac{x^3/3}{x}}=\frac{|x|}{\sqrt{3}}.$$
Ainsi, la fonction $x\mapsto\Arccos\left(\sqrt{\frac{x}{\tan x}}\right)$ n'est pas dérivable en $0$ (mais est dérivable à droite et à gauche) et n'admet donc pas de développement limité d'ordre supérieur ou égal à $1$ (mais admet éventuellement des développements limités à gauche et à droite pour lesquels la remarque initiale sur la parité des exposants ne tient plus).
\textbullet~Déterminons un équivalent simple de $f(x)=\Arccos\left(\sqrt{\frac{x}{\tan x}}\right)-\frac{x}{\sqrt{3}}$ quand $x$ tend vers $0$ par valeurs supérieures.

\begin{align*}\ensuremath
\Arccos\left(\sqrt{\frac{x}{\tan x}}\right)-\frac{x}{\sqrt{3}}&\sim\sin\left(\Arccos\left(\sqrt{\frac{x}{\tan x}}\right)-\frac{x}{\sqrt{3}}\right)\\
 &=\sin\left(\Arccos\left(\sqrt{\frac{x}{\tan x}}\right)\right)\cos\left(\frac{x}{\sqrt{3}}\right)-\sin\left(\frac{x}{\sqrt{3}}\right)\cos\left(\Arccos\left(\sqrt{\frac{x}{\tan x}}\right)\right)\\
 &=\sqrt{\frac{\tan x-x}{\tan x}}\cos\left(\frac{x}{\sqrt{3}}\right)-\sqrt{\frac{x}{\tan x}}\sin\left(\frac{x}{\sqrt{3}}\right)=g(x)
\end{align*}
Maintenant,

\begin{align*}\ensuremath
\sqrt{\frac{\tan x-x}{\tan x}}&=\left(\left(\frac{x^3}{3}+\frac{2x^5}{15}+o(x^5)\right)\left(x+\frac{x^3}{3}+o(x^3)\right)^{-1}\right)^{1/2}=
\frac{1}{\sqrt{x}}\left(\left(\frac{x^3}{3}+\frac{2x^5}{15}+o(x^5)\right)\left(1-\frac{x^2}{3}+o(x^2)\right)\right)^{1/2}\\
 &=\frac{1}{\sqrt{x}}\left(\frac{x^3}{3}+\frac{x^5}{45}+o(x^5)\right)^{1/2}=\frac{x}{\sqrt{3}}\left(1+\frac{x^2}{15}+o(x^2)\right)^{1/2}\\
 &=\frac{x}{\sqrt{3}}+\frac{x^3}{30\sqrt{3}}+o(x^3),
\end{align*}
et donc,

$$
\sqrt{\frac{\tan x-x}{\tan x}}\cos\left(\frac{x}{\sqrt{3}}\right)=\left(\frac{x}{\sqrt{3}}+\frac{x^3}{30\sqrt{3}}+o(x^3)\right)\left(1-\frac{x^2}{6}+o(x^2)\right)=\frac{x}{\sqrt{3}}-\frac{2x^3}{15\sqrt{3}}+o(x^3).
$$
Ensuite,

\begin{align*}\ensuremath
\sqrt{\frac{x}{\tan x}}\sin\left(\frac{x}{\sqrt{3}}\right)&=\left(1+\frac{x^2}{3}+o(x^2)\right)^{-1/2}\left(\frac{x}{\sqrt{3}}-\frac{x^3}{18\sqrt{3}}+o(x^3)\right)=\left(1-\frac{x^2}{6}+o(x^2)\right)\left(\frac{x}{\sqrt{3}}-\frac{x^3}{18\sqrt{3}}+o(x^3)\right)\\
 &=\frac{x}{\sqrt{3}}-\frac{2x^3}{9\sqrt{3}}+o(x^3),
\end{align*}
et finalement,

$$g(x)=\left(\frac{x}{\sqrt{3}}-\frac{2x^3}{15\sqrt{3}}+o(x^3)\right)-\left(\frac{x}{\sqrt{3}}-\frac{2x^3}{9\sqrt{3}}+o(x^3)\right)=\frac{4x^3}{45\sqrt{3}}+o(x^3)\sim\frac{4x^3}{45\sqrt{3}}.$$

Ainsi, quand $x$ tend vers $0$ par valeurs supérieures,

$$\Arccos\left(\sqrt{\frac{x}{\tan x}}\right)-\frac{x}{\sqrt{3}}=\frac{4x^3}{45\sqrt{3}}+o(x^3).$$
$f$ étant paire, on en déduit que

\begin{center}
\shadowbox{$\Arccos\left(\sqrt{\frac{x}{\tan x}}\right)\underset{x\rightarrow0}{=}\frac{|x|}{\sqrt{3}}+\frac{4|x|^3}{45\sqrt{3}}+o(x^3).$}
\end{center}
(Ce n'est pas un développement limité).
 \item  La fonction $x\mapsto\tan x$ est trois fois dérivable en $\frac{\pi}{4}$ et admet donc en $\frac{\pi}{4}$ un développement limité d'ordre $3$ à savoir son développement de \textsc{Taylor}-\textsc{Young}.
$\tan\frac{\pi}{4}=1$ puis $(\tan)'\left(\frac{\pi}{4}\right)=1+\tan^2\frac{\pi}{4}=2$. Ensuite, $(\tan)''(x)=2\tan x(1+\tan^2x)$ et $(\tan)''(\frac{\pi}{4})=4$. Enfin,

$$(\tan)^{(3)}(x)=2(1+\tan^2x)^2+4\tan^2x(1+\tan^2x),$$
et $(\tan)^{(3)}(\frac{\pi}{4})=16$. Finalement,

\begin{center}
\shadowbox{$\tan x\underset{x\rightarrow\pi/4}=1+2\left(x-\frac{\pi}{4}\right)+2\left(x-\frac{\pi}{4}\right)^2+\frac{8}{3}\left(x-\frac{\pi}{4}\right)^3+o\left(\left(x-\frac{\pi}{4}\right)^3\right).$
}
\end{center}
 \item 

$$\frac{1}{x^2}\ln(\ch x)\underset{x\rightarrow0}{=}\frac{1}{x^2}\ln\left(1+\frac{x^2}{2}+\frac{x^4}{24}+o(x^4)\right)=\frac{1}{x^2}\left(\frac{x^2}{2}+\frac{x^4}{24}-\frac{1}{2}\left(\frac{x^2}{2}\right)^2+o(x^4)\right)=\frac{1}{2}-\frac{x^2}{12}+o(x^2),$$
et donc 

$$(\ch x)^{1/x^2}=e^{\frac{1}{2}-\frac{x^2}{12}+o(x^2)}=e^{1/2}e^{-\frac{x^2}{12}+o(x^2)}=\sqrt{e}-\frac{\sqrt{e}}{12}x^2+o(x^2).$$

\begin{center}
\shadowbox{$(\ch x)^{1/x^2}=\underset{x\rightarrow0}=\sqrt{e}-\frac{\sqrt{e}}{12}x^2+o(x^2).$
}
\end{center}
 \item  $\tan^3x(\cos(x^2)-1)=\tan x\times\tan^2x(\cos(x^2)-1)$ et un équivalent de $\tan^2x(\cos(x^2)-1)$ en $0$ est $-\frac{x^6}{2}$. On écrit donc $\tan x$ à l'ordre $2$. De même, un équivalent de $\tan^3x$ est $x^3$ et on écrit donc $\cos(x^2)-1$ à l'ordre $5$.
$$\tan^3x(\cos(x^2)-1)\underset{x\rightarrow0}{=}(x+o(x^2))^3\left(-\frac{x^4}{2}+o(x^5)\right)=(x^3+o(x^4))\left(-\frac{x^4}{2}
+o(x^5)\right)=-\frac{x^7}{2}+o(x^8).$$

\begin{center}
\shadowbox{$\tan^3x(\cos(x^2)-1)\underset{x\rightarrow0}{=}-\frac{x^7}{2}+o(x^8).$
}
\end{center}

 \item  On pose $h=x-1$ ou encore $x=1+h$, de sorte que $x$ tend vers $1$ si et seulement si $h$ tend vers $0$.

\begin{align*}\ensuremath
\frac{\ln(1+x)}{x^2}&=\ln(2+h)(1+h)^{-2}\\
 &\underset{h\rightarrow0}{=}\left(\ln2+\ln\left(1+\frac{h}{2}\right)\right)\left(1-2h+\frac{(-2)(-3)}{2}h^2+\frac{(-2)(-3)(-4)}{6}h^3
 +o(h^3)\right)\\
  &=\left(\ln2+\frac{h}{2}-\frac{h^2}{8}+\frac{h^3}{24}+o(h^3)\right)(1-2h+3h^2-4h^3+o(h^3))\\
  &=\ln2+\left(\frac{1}{2}-2\ln2\right)h+\left(3\ln2-\frac{9}{8}\right)h^2+\left(-4\ln2+\frac{43}{24}\right)h^3 + o(h^3).
\end{align*}
Donc,

\begin{center}
\shadowbox{
$\frac{\ln(1+x)}{x^2}\underset{x\rightarrow1}{=}\ln2+\left(\frac{1}{2}-2\ln2\right)(x-1)+\left(3\ln2-\frac{9}{8}\right)(x-1)^2+\left(-4\ln2+\frac{43}{24}\right)(x-1)^3+o((x-1)^3)$.
}
\end{center}
 \item  Pour $x$ réel, posons $f(x)=\Arctan(\cos x)$.
$f$ est dérivable sur $\Rr$, et pour $x$ réel, $f'(x)=-\frac{\sin x}{1+\cos^2x}$. Puis,

\begin{align*}\ensuremath
f'(x)&\underset{x\rightarrow0}{=}-\left(x-\frac{x^3}{6}+o(x^4)\right)\left(1+\left(1-\frac{x^2}{2}+o(x^3)\right)^2\right)^{-1}\\
 &=-\left(x-\frac{x^3}{6}+o(x^4)\right)\left(2-x^2+o(x^3)\right)^{-1}=-\frac{1}{2}\left(x-\frac{x^3}{6}+o(x^4)\right)
\left(1-\frac{x^2}{2}+o(x^3)\right)^{-1}\\
 &=-\frac{1}{2}\left(x-\frac{x^3}{6}+o(x^4)\right)\left(1+\frac{x^2}{2}+o(x^3)\right)=-\frac{x}{2}-\frac{x^3}{6}+o(x^4).
\end{align*}
Donc, $f'$ admet un développement limité d'ordre $4$ en $0$ et on sait que $f$ admet en $0$ un développement limité d'ordre $5$ obtenu par intégration.

$$\Arctan(\cos x)\underset{x\rightarrow0}{=}\Arctan(\cos0)-\frac{x^2}{4}-\frac{x^4}{24}+o(x^5)=\frac{\pi}{4}-\frac{x^2}{4}-\frac{x^4}{24}+o(x^5).$$

\begin{center}
\shadowbox{
$\Arctan(\cos x)\underset{x\rightarrow0}{=}\frac{\pi}{4}-\frac{x^2}{4}-\frac{x^4}{24}+o(x^5).$
}
\end{center}
 \item  Pour $x>-1$, posons $f(x)=\Arctan\sqrt{\frac{x+1}{x+2}}$. $f$ est dérivable sur $]-1,+\infty[$ et pour $x>-1$,
 
\begin{align*}\ensuremath
f'(x)&=\frac{1}{(x+2)^2}\frac{1}{2\sqrt{\frac{x+1}{x+2}}}\frac{1}{1+\frac{x+1}{x+2}}=\frac{1}{2}\times\frac{1}{(2x+3)\sqrt{(1+x)(2+x)}}\\
 &=\frac{1}{2\times3\times\sqrt{2}}\left(1+\frac{2x}{3}\right)^{-1}(1+x)^{-1/2}\left(1+\frac{x}{2}\right)^{-1/2}=\frac{1}{6\sqrt{2}}\left(1-\frac{2x}{3}+o(x)\right)\left(1-\frac{x}{2}+o(x)\right)\left(1-\frac{x}{4}+o(x)\right)\\
 &=\frac{1}{6\sqrt{2}}\left(1-\left(\frac{2}{3}+\frac{1}{2}+\frac{1}{4}\right)x+o(x)\right)=\frac{1}{6\sqrt{2}}\left(1-\frac{17x}{12}+o(x)\right).
\end{align*}
Ainsi, $f'$ admet donc en $0$ un développement limité d'ordre $1$ et on sait alors que $f$ admet en $0$ un développement limité d'ordre $2$ obtenu par intégration.

\begin{center}
\shadowbox{
$\Arctan\sqrt{\frac{x+1}{x+2}}\underset{x\rightarrow0}{=}\Arctan\frac{1}{\sqrt{2}}+\frac{1}{6\sqrt{2}}x-\frac{17}{144\sqrt{2}}x^2+o(x^2).$
}
\end{center}
 \item  

\begin{align*}\ensuremath 
\frac{1}{\sqrt{1-x^2}}&=(1-x^2)^{-1/2}\underset{x\rightarrow0}{=}1+\left(-\frac{1}{2}\right)(-x^2)+\frac{\left(-\frac{1}{2}\right)\left(-\frac{3}{2}\right)}{2}(-x^2)^2+\frac{\left(-\frac{1}{2}\right)\left(-\frac{3}{2}\right)\left(-\frac{5}{2}\right)}{6}(-x^2)^3+o(x^7)\\
 &\underset{x\rightarrow0}{=}1+\frac{1}{2}x^2+\frac{3}{8}x^4+\frac{5}{16}x^6+o(x^7).
\end{align*}

Donc, $\Arcsin x\underset{x\rightarrow0}{=}x+\frac{x^3}{6}+\frac{3x^5}{40}+\frac{5x^7}{112}+o(x^8)$. Ensuite,

\begin{align*}\ensuremath
\frac{1}{\Arcsin^2x}&=(\Arcsin x)^{-2}\underset{x\rightarrow0}{=}\frac{1}{x^2}\left(1+\frac{x^2}{6}+\frac{3x^4}{40}+\frac{5x^6}{112}+o(x^7)\right)^{-2}\\
 &=\frac{1}{x^2}\left(1-2\left(\frac{x^2}{6}+\frac{3x^4}{40}+\frac{5x^6}{112}\right)+3\left(\frac{x^2}{6}+\frac{3x^4}{40}\right)^2-4\left(\frac{x^2}{6}\right)^3+o(x^7)\right)\\
 &=\frac{1}{x^2}-\frac{1}{3}+\left(-\frac{3}{20}+\frac{1}{12}\right)x^2+\left(-\frac{5}{56}+\frac{3}{40}-\frac{1}{54}\right)x^4+o(x^5)\\
 &=\frac{1}{x^2}-\frac{1}{3}-\frac{x^2}{15}-\frac{31x^4}{945}+o(x^5).
\end{align*}
Finalement,

\begin{center}
\shadowbox{
$\frac{1}{x^2}-\frac{1}{\Arcsin^2x}\underset{x\rightarrow0}{=}\frac{1}{3}+\frac{x^2}{15}+\frac{31x^4}{945}+o(x^5).$
}
\end{center}
\item  Pour $x$ réel, posons $f(x)=\frac{1}{\sqrt{1+x^4}}$. $f$ est continue sur $\Rr$ et admet donc des primitives sur $\Rr$. Soit $F$ la primitive de $f$ sur $\Rr$ qui s'annule en $0$ puis, pour $x$ réel, soit $g(x)=\int_{x}^{x^2}\frac{1}{\sqrt{1+t^4}}\;dt$.
$g$ est définie sur $\Rr$ et, pour $x$ réel $g(x)=F(x^2)-F(x)$. $g$ est dérivable sur $\Rr$ et, pour tout réel $x$,

$$g'(x)=2xF'(x^2)-F'(x)=2xf(x^2)-f(x)=\frac{2x}{\sqrt{1+x^8}}-\frac{1}{\sqrt{1+x^4}}.$$
Puis, 

$$g'(x)\underset{x\rightarrow0}{=}2x\left(1-\frac{1}{2}x^8+o(x^8)\right)-\left(1-\frac{1}{2}x^4+\frac{3}{8}x^8+o(x^9)\right)=-1+2x+\frac{1}{2}x^4-\frac{3}{8}x^8-x^9+o(x^9).$$
Ainsi $g'$ admet un développement limité d'ordre $9$ en $0$ et on sait que $g$ admet un développement limité d'ordre $10$ en $0$ obtenu par intégration. En tenant compte de $g(0)=0$, on obtient

\begin{center}
\shadowbox{
$g(x)\underset{x\rightarrow0}{=}-x+x^2+\frac{x^5}{10}-\frac{x^9}{24}-\frac{x^{10}}{10}+o(x^{10}).$
}\end{center}
\item 
\begin{align*}\ensuremath
\ln\left(\sum_{k=0}^{99}\frac{x^k}{k!}\right)&\underset{x\rightarrow0}{=}\ln\left(e^x-\frac{x^{100}}{100!}+o(x^{100})\right)=\ln(e^x)+\ln\left(1-e^{-x}\left(\frac{x^{100}}{100!}+o(x^{100})\right)\right)\\
 &=x+\ln\left(1-(1+o(1))\left(\frac{x^{100}}{(100)!}+o(x^{100})\right)\right)=x+\ln\left(1-\frac{x^{100}}{(100)!}+o(x^{100})\right)=x-\frac{x^{100}}{(100)!}+o(x^{100})
\end{align*}

\begin{center}
\shadowbox{
$\ln\left(\sum_{k=0}^{99}\frac{x^k}{k!}\right)\underset{x\rightarrow0}{=}x-\frac{x^{100}}{(100)!}+o(x^{100}).$
}\end{center}
\item Posons $h=x-\pi$ ou encore $x=\pi+h$ de sorte que $x$ tend vers $\pi$ si et seulement si $h$ tend vers $0$.

\begin{align*}\ensuremath
\sqrt[3]{4(\pi^3+x^3)}&=\sqrt[3]{4(\pi^3+(\pi+h)^3)}=\sqrt[3]{8\pi^3+12\pi^2h+12\pi h^2+4h^3}\\
 &\underset{h\rightarrow0}{=}2\pi\left(1+\frac{3h}{2\pi}+\frac{3h^2}{2\pi^2}+\frac{h^3}{2\pi^3}\right)^{1/3}\\
 &=2\pi\left(1+\frac{1}{3}\left(\frac{3h}{2\pi}+\frac{3h^2}{2\pi^2}+\frac{h^3}{2\pi^3}\right)-\frac{1}{9}\left(\frac{3h}{2\pi}+\frac{3h^2}{2\pi^2}\right)^2+\frac{5}{81}\left(\frac{3h}{2\pi}\right)^3+o(h^3)\right)\\
 &=2\pi+h+h^2\left(\frac{1}{\pi}-\frac{1}{2\pi}\right   )+h^3\left(\frac{1}{3\pi^2}-\frac{1}{\pi^2}+\frac{5}{12\pi^2}\right)+o(h^3)\\
 &=2\pi+h+\frac{h^2}{2\pi}-\frac{h^3}{4\pi^2}+o(h^3).
\end{align*}
Puis,

\begin{align*}\ensuremath
\tan(\sqrt[3]{4(\pi^3+x^3)})&=\tan\left(h+\frac{h^2}{2\pi}-\frac{h^3}{4\pi^2}+o(h^3)\right)\\
 &=\left(h+\frac{h^2}{2\pi}-\frac{h^3}{4\pi^2}\right)+\frac{1}{3}h^3+o(h^3)=h+\frac{h^2}{2\pi}+\left(\frac{1}{3}-\frac{1}{4\pi^2}\right)h^3+o(h^3).
\end{align*}
Finalement,

\begin{center}
\shadowbox{
$\tan(\sqrt[3]{4(\pi^3+x^3)})\underset{x\rightarrow\pi}{=}(x-1)+\frac{1}{2\pi}(x-1)^2+\left(\frac{1}{3}-\frac{1}{4\pi^2}\right)(x-1)^3+o((x-1)^3)$.
}
\end{center}
\end{enumerate}
\fincorrection
\correction{005428}
Puisque $a>0$, $b>0$ et que pour tout réel $x$, $\frac{a^x+b^x}{2}>0$, $f$ est définie sur $\Rr^*$, et pour

\begin{center}
$\forall x\in\Rr^*$, $f(x)=\text{exp}\left(\frac{1}{x}\ln\left(\frac{a^x+b^x}{2}\right)\right)$.
\end{center}
\textbf{Etude en 0}.

\begin{align*}\ensuremath
\ln\left(\frac{a^x+b^x}{2}\right)&=\ln\left(\frac{e^{x\ln a}+e^{x\ln b}}{2}\right)\underset{x\rightarrow0}{=}\ln\left(1+x\left(\frac{1}{2}\ln a+\frac{1}{2}\ln b\right)+x^2\left(\frac{1}{4}\ln^2a+\frac{1}{4}\ln^2b\right)+o(x^2)\right)\\
 &=\ln\left(1+x\ln\left(\sqrt{ab}\right)+x^2\frac{\ln^2a+\ln^2b}{4}+o(x^2)\right)=x\ln\left(\sqrt{ab}\right)+x^2\frac{\ln^2a+\ln^2b}{4}-\frac{1}{2}(x\ln\sqrt{ab})^2+o(x^2)\\
 &=x\ln\left(\sqrt{ab}\right)+\frac{1}{8}(\ln^2a-2\ln a\ln b+\ln^2b)x^2+o(x^2)=x\ln\left(\sqrt{ab}\right)+x^2\frac{1}{8}\ln^2\left(\frac{a}{b}\right)+o(x^2).
\end{align*}
Enfin,

$$f(x)=\left(\frac{a^x+b^x}{2}\right)^{1/x}=\mbox{exp}(\ln(\sqrt{ab})+\frac{1}{8}\ln^2\frac{a}{b}x+o(x))=\sqrt{ab}(1+\frac{1}{8}x\ln^2\frac{a}{b}+o(x)).$$
Ainsi, $f$ se prolonge par continuité en $0$ en posant $f(0)=\sqrt{ab}$. Le prolongement obtenu est dérivable en $0$ et
$f'(0)=\frac{\sqrt{ab}}{8}\ln^2\frac{a}{b}(>0)$.
\textbf{Etude en} $\bf{+\infty}$.
\begin{align*}\ensuremath
\frac{1}{x}\ln\left(\frac{1}{2}(a^x+b^x)\right)&=\frac{1}{x}\left(\ln(b^x)-\ln2+\ln\left(1+\left(\frac{a}{b}\right)^x\right)\right)
=\frac{1}{x}(x\ln b+o(x))\quad(\mbox{car}\;0<\frac{a}{b}<1)\\
 &=\ln b+o(1).
\end{align*}
et $\lim_{x\rightarrow +\infty}f(x)=b(=\mbox{Max}\{a,b\})$.
\textbf{Etude en} $\bf{-\infty}$. Pour tout réel $x$,

$$f(-x)=\left(\frac{a^{-x}+b^{-x}}{2}\right)^{-1/x}= \left(\frac{a^{x}+b^{x}}{2a^xb^x}\right)^{-1/x}=\frac{ab}{f(x)},$$
et donc,

$$\lim_{x\rightarrow -\infty}f(x)=\lim_{X\rightarrow +\infty}f(-X)=\lim_{X\rightarrow +\infty}\frac{ab}{f(X)}=\frac{ab}{b}=a\quad(=\mbox{Min}\{a,b\}).$$
\textbf{Dérivée et variations}.
$f$ est dérivable sur $]-\infty,0\cup]0,+\infty[$ en vertu de théorèmes généraux (et aussi en $0$ d'après l'étude faite plus haut), et pour $x\neq0$ (puisque $f>0$ sur $\Rr^*$),

$$\frac{f'(x)}{f(x)}=(\ln f)'(x)=\left(\frac{1}{x}\ln\left(\frac{a^x+b^x}{2}\right)\right)'(x)=-\frac{1}{x^2}\ln\left(\frac{a^x+b^x}{2}\right)+\frac{1}{x}
\frac{a^x\ln a+b^x\ln b}{a^x+b^x}.$$
$f'$ a le même signe que $(\ln f)'$ qui, elle-même, a le même signe que la fonction $g$ définie sur $\Rr$ par

$$\forall x\in\Rr,\;g(x)=-\ln\left(\frac{a^x+b^x}{2}\right)+x\frac{a^x\ln a+b^x\ln b}{a^x+b^x}.$$
$g$ est dérivable sur $\Rr$ et, pour $x$ réel,

\begin{align*}\ensuremath
g'(x)&=-\frac{a^x\ln a+b^x\ln b}{a^x+b^x}+\frac{a^x\ln a+b^x\ln b}{a^x+b^x}+x\frac{(a^x\ln^2a+b^x\ln^2b)(a^x+b^x)-(a^x\ln a+b^x\ln b)^2}{(a^x+b^x)^2}\\
 &=x\frac{(ab)^x(\ln a-\ln b)^2}{(a^x+b^x)^2}.
\end{align*}
$g'$ est donc strictement négative sur $]-\infty,0[$ et strictement positive sur $]0,+\infty[$. Par suite, $g$ est strictement décroissante sur $]-\infty,0]$ et strictement croissante sur $[0,+\infty[$. $g'$ admet donc un minimum global strict en $0$ et puisque $g(0)=0$, on en déduit que $g$ est strictement positive sur $\Rr^*$. De même, $f'$ est strictement positive sur $\Rr^*$. En tant compte de l'étude en $0$, on a montré que $f$ est dérivable sur $\Rr$ et que $f'$ est strictement positive sur $\Rr$. $f$ est donc strictement croissante sur $\Rr$.
Le \textbf{graphe de $f$} a l'allure suivante~:

%$$\includegraphics{../images/img005428-1}$$

On peut noter que les inégalités $\lim_{x\rightarrow -\infty}f<f(-1)<f(0)<f(1)<\lim_{x\rightarrow +\infty}f$ fournissent~:

$$a<\frac{1}{\frac{1}{2}(\frac{1}{a}+\frac{1}{b})}<\sqrt{ab}<\frac{a+b}{2}<b.$$
\fincorrection
\correction{005429}
Quand $x$ tend vers $+\infty$,

$$\sqrt{x^2-3}=x\left(1-\frac{3}{x^2}\right)^{1/2}=x\left(1-\frac{3}{2x^2}+o(\frac{1}{x^2})\right)=x-\frac{3}{2x}+o\left(\frac{1}{x}\right),$$
et,

$$\sqrt[3]{8x^3+7x^2+1}=2x\left(1+\frac{7}{8x}+o\left(\frac{1}{x^2}\right)\right)^{1/3}=2x\left(1+\frac{7}{24x}-\frac{49}{576x^2}+o\left(\frac{1}{x^2}\right)\right)=2x+\frac{7}{12}-\frac{49}{288x}+o\left(\frac{1}{x}\right).$$
Donc,

$$f(x)\underset{x\rightarrow+\infty}{=}-x-\frac{7}{12}-\frac{383}{288x}+o\left(\frac{1}{x}\right).$$
La courbe représentative de $f$ admet donc en $+\infty$ une droite asymptote d'équation $y=-x-\frac{7}{12}$. De plus, le signe de $f(x)-\left(-x-\frac{7}{12}\right)$ est, au voisinage de $+\infty$, le signe de $-\frac{383}{288x}$. Donc la courbe représentative de $f$ est au-dessous de la droite d'équation $y=-x-\frac{7}{12}$ au voisinage de $+\infty$.
\fincorrection
\correction{005430}
$f$ est de classe $C^\infty$ sur son domaine $\Rr\setminus\{-1,1\}$ en tant que fraction rationelle et en particulier admet un développement limité à tout ordre en $0$. Pour tout entier naturel $n$, on a

$$f(x)\underset{x\rightarrow0}{=}x+x^3+...+x^{2n+1}+o(x^{2n+1}),$$
Par unicité des coefficients d'un développement limité et d'après la formule de \textsc{Taylor}-\textsc{Young}, on obtient

\begin{center}
\shadowbox{
$\forall n\in\Nn,\;f^{(2n)}(0)=0\;\mbox{et}\;f^{(2n+1)}(0)=(2n+1)!.$
}
\end{center}
Ensuite, pour $x\notin\{-1,1\}$, et $n$ entier naturel donné, 

$$f^{(n)}(x)=\frac{1}{2}\left(\frac{1}{1-x}-\frac{1}{1+x}\right)^{(n)}(x)=\frac{n!}{2}\left(\frac{1}{(1-x)^{n+1}}-
\frac{(-1)^{n}}{(1+x)^{n+1}}\right).$$
\fincorrection
\correction{005431}

\begin{enumerate}
 \item  $$\sqrt{x^2+3x+5}-x+1\underset{x\rightarrow-\infty}{\sim}-x-x=-2x,$$ 
et,

$$\sqrt{x^2+3x+5}-x+1=\frac{(x^2+3x+5)-(x-1)^2}{\sqrt{x^2+3x+5}+x-1}
\underset{x\rightarrow+\infty}{\sim}\frac{3x+2x}{x+x}=\frac{5}{2}.$$
 \item  $3x^2-6x\underset{x\rightarrow0}{\sim}-6x$ et $3x^2-6x\underset{x\rightarrow+\infty}{\sim}3x^2$. Ensuite, quand $x$ tend vers $1$, $3x^2-6x$ tend vers $-3\neq0$ et donc, $3x^2-6x\underset{x\rightarrow1}{\sim}-3$. Enfin, 
$3x^2-6x=3x(x-2)\underset{x\rightarrow2}{\sim}6(x-2)$.

\begin{center}
\shadowbox{
$3x^2-6x\underset{x\rightarrow0}{\sim}-6x$\qquad$3x^2-6x\underset{x\rightarrow+\infty}{\sim}3x^2$\qquad$3x^2-6x\underset{x\rightarrow0}{\sim}-3$\qquad$3x^2-6x\underset{x\rightarrow2}{\sim}6(x-2)$.
}
\end{center}
 \item  $$(x-x^2)\ln(\sin x)\underset{x\rightarrow0}{=}(x-x^2)\ln x+(x-x^2)\ln\left(1-\frac{x^2}{6}+o(x^2)\right)=x\ln x-x^2\ln x+o(x^2\ln x).$$ 
Ensuite,

$$\sin x\ln(x-x^2)\underset{x\rightarrow0}{=}\left(x-\frac{x^3}{6}+o(x^3)\right)(\ln x+\ln(1-x))=(x-\frac{x^3}{6}+o(x^3))(\ln x-x+o(x))=x\ln x+o(x^2\ln x).$$
Donc,

\begin{align*}\ensuremath
(\sin x)^{x-x^2}-(x-x^2)^{\sin x}&=e^{x\ln x}(e^{-x^2\ln x+o(x^2\ln x)}-e^{o(x^2\ln x)})=e^{x\ln x}(1-x^2\ln x-1+o(x^2\ln x))\\
 &=(1+o(1))(-x^2\ln x+o(x^2\ln x))\underset{x\rightarrow0}{\sim}-x^2\ln x.
\end{align*}

\begin{center}
\shadowbox{
$(\sin x)^{x-x^2}-(x-x^2)^{\sin x}\underset{x\rightarrow0}{\sim}-x^2\ln x$.
}
\end{center}
 \item  $\tanh x=\frac{1-e^{-2x}}{1+e^{-2x}}\underset{x\rightarrow+\infty}{=}=(1-e^{-2x})(1-e^{-2x}+o(e^{-2x}))=1-2e^{-2x}+o(e^{-2x})$, et donc $\tanh x\ln x=(1-2e^{-2x}+o(e^{-2x}))\ln x=\ln x+o(1)$. Par suite,

$$x^{\mbox{\scriptsize{th}}x}\underset{x\rightarrow+\infty}{\sim}e^{\ln x}=x.$$
 \item  \textbf{Tentative à l'ordre 3.}

$\tan(\sin x)\underset{x\rightarrow0}{=}\tan\left(x-\frac{x^3}{6}+o(x^3)\right)=\left(x-\frac{x^3}{6}\right)+\frac{1}{3}(x)^3+o(x^3)=x+\frac{x^3}{6}+o(x^3)$, et,

$\sin(\tan x)\underset{x\rightarrow0}{=}\sin\left(x+\frac{x^3}{3}+o(x^3)\right)=\left(x+\frac{x^3}{3}\right)-\frac{1}{6}(x)^3+o(x^3)=x+\frac{x^3}{6}+o(x^3)$.
Donc, $\tan(\sin x)-\sin(\tan x)\underset{x\rightarrow0}{=}o(x^3)$. L'ordre $3$ est insuffisant pour obtenir un équivalent.
\textbf{Tentative à l'ordre 5.}

\begin{align*}\ensuremath
\tan(\sin x)&\underset{x\rightarrow0}{=}\tan\left(x-\frac{x^3}{6}+\frac{x^5}{120}+o(x^5)\right)
=\left(x-\frac{x^3}{6}+\frac{x^5}{120}\right)+\frac{1}{3}\left(x-\frac{x^3}{6}\right)^3+\frac{2}{15}(x)^5+o(x^5)\\
 &=x+\frac{x^3}{6}+x^5\left(\frac{1}{120}-\frac{1}{6}+\frac{2}{15}\right)+o(x^5)=x+\frac{x^3}{6}-\frac{x^5}{40}+o(x^5),
\end{align*}
et,

\begin{align*}\ensuremath
\sin(\tan x)&\underset{x\rightarrow0}{=}\sin\left(x+\frac{x^3}{3}+\frac{2x^5}{15}+o(x^5)\right)
=\left(x+\frac{x^3}{3}+\frac{2x^5}{15}\right)-\frac{1}{6}\left(x+\frac{x^3}{3}\right)^3+\frac{1}{120}(x)^5+o(x^5)\\
 &=x+\frac{x^3}{6}+\left(\frac{2}{15}-\frac{1}{6}+\frac{1}{120}\right)x^5+o(x^5)=x+\frac{x^3}{6}-\frac{x^5}{40}+o(x^5).
\end{align*}
Donc, $\tan(\sin x)-\sin(\tan x)\underset{x\rightarrow0}{=}o(x^5)$. L'ordre $5$ est insuffisant pour obtenir un équivalent. Le contact entre les courbes représentatives des fonctions $x\mapsto\sin(\tan x)$ et $x\mapsto\tan(\sin x)$ est très fort.
\textbf{Tentative à l'ordre 7.}

\begin{align*}\ensuremath
\tan(\sin x)&\underset{x\rightarrow0}{=}\tan\left(x-\frac{x^3}{6}+\frac{x^5}{120}-\frac{x^7}{5040}+o(x^7)\right)\\
 &=\left(x-\frac{x^3}{6}+\frac{x^5}{120}-\frac{x^7}{5040}\right)+\frac{1}{3}\left(x-\frac{x^3}{6}+\frac{x^5}{120}\right)^3+\frac{2}{15}\left(x-\frac{x^3}{6}\right)^5+\frac{17}{315}x^7+o(x^7)\\
 &=x+\frac{x^3}{6}-\frac{x^5}{40}+\left(-\frac{1}{5040}+\frac{1}{3}\left(3\times\frac{1}{120}+3\times\frac{1}{36}\right)+\frac{2}{15}\left(-\frac{5}{6}\right)+\frac{17}{315}\right)x^7+o(x^7)\\
 &=x+\frac{x^3}{6}-\frac{x^5}{40}+\left(-\frac{1}{5040}+\frac{1}{120}+\frac{1}{36}-\frac{1}{9}+\frac{17}{315}\right)x^7+o(x^7),
\end{align*}
et,

\begin{align*}\ensuremath
\sin(\tan x)&\underset{x\rightarrow0}{=}\sin\left(x+\frac{x^3}{3}+\frac{2x^5}{15}+\frac{17}{x^7}{315}+o(x^7)\right)\\
 &=\left(x+\frac{x^3}{3}+\frac{2x^5}{15}+\frac{17x^7}{315}\right)-\frac{1}{6}\left(x+\frac{x^3}{3}+\frac{2x^5}{15}\right)^3+\frac{1}{120}\left(x+\frac{x^3}{3}\right)^5-\frac{1}{5040}(x)^7+o(x^7)\\
 &= x+\frac{x^3}{6}-\frac{x^5}{40}+\left(\frac{17}{315}
 -\frac{1}{6}\left(3\times\frac{2}{15}+3\times\frac{1}{9}\right)+\frac{1}{120}\times\frac{5}{3}-\frac{1}{5040}\right)x^7+o(x^7)\\
 &=x+\frac{x^3}{6}-\frac{x^5}{40}+\left(\frac{17}{315}-\frac{1}{15}-\frac{1}{18}+\frac{1}{72}-\frac{1}{5040}\right)x^7+o(x^7).
\end{align*}
Finalement,

\begin{align*}\ensuremath
\tan(\sin x)-\sin(\tan x)&\underset{x\rightarrow0}{=}\left(\frac{1}{120}+\frac{1}{36}
-\frac{1}{9}+\frac{1}{15}+\frac{1}{18}-\frac{1}{72}\right)x^7+o(x^7)
=\frac{x^7}{30}+o(x^7),
\end{align*}
et donc

\begin{center}
\shadowbox{
$\tan(\sin x)-\sin(\tan x)\underset{x\rightarrow0}{\sim}\frac{x^7}{30}$.
}
\end{center}
\end{enumerate}
\fincorrection
\correction{005432}
Pour $n\geq 5$, on a

$$u_n=1+\frac{1}{n}+\frac{1}{n(n-1)}+\frac{1}{n(n-1)(n-2)}+\frac{1}{n(n-1)(n-2)(n-3)}+\sum_{k=0}^{n-5}\frac{1}{n(n-1)...(k+1)}.$$
Ensuite, 

$$0\leq n^3\sum_{k=0}^{n-5}\frac{1}{n(n-1)...(k+1)}\leq n^3(n-4)\frac{1}{n(n-1)(n-2)(n-3)(n-4)}\underset{n\rightarrow+\infty}{\sim}\frac{1}{n}\underset{n\rightarrow+\infty}{\rightarrow}0.$$
Donc, $\sum_{k=0}^{n-5}\frac{1}{n(n-1)...(k+1)}\underset{n\rightarrow+\infty}{=}o\left(\frac{1}{n^3}\right)$ et de même $\frac{1}{n(n-1)(n-2)(n-3)}\underset{n\rightarrow+\infty}{=} o\left(\frac{1}{n^3}\right)$. Il reste

\begin{align*}\ensuremath
u_n&\underset{n\rightarrow+\infty}{=}1+\frac{1}{n}+\frac{1}{n^2}\left(1-\frac{1}{n}\right)^{-1}+\frac{1}{n^3}+o\left(\frac{1}{n^3}\right)=1+\frac{1}{n}
+\frac{1}{n^2}\left(1+\frac{1}{n}\right)+\frac{1}{n^3}+o\left(\frac{1}{n^3}\right)\\
 &=1+\frac{1}{n}+\frac{1}{n^2}+\frac{2}{n^3}+o\left(\frac{1}{n^3}\right).
\end{align*}

\begin{center}
\shadowbox{
$\frac{1}{n!}\sum_{k=0}^{n}k!\underset{n\rightarrow+\infty}{=}1+\frac{1}{n}+\frac{1}{n^2}+\frac{2}{n^3}+o\left(\frac{1}{n^3}\right)$.
}
\end{center}
\fincorrection
\correction{005433}
\begin{enumerate}
 \item 
\begin{align*}\ensuremath
\frac{1}{x(e^x-1)}-\frac{1}{x^2}&\underset{x\rightarrow0}{=}\frac{1}{x\left(x+\frac{x^2}{2}+\frac{x^3}{6}+\frac{x^4}{24}+\frac{x^5}{120}+o(x^5)\right)}-\frac{1}{x^2}=\frac{1}{x^2}\left(\left(1+\frac{x}{2}+\frac{x^2}{6}+\frac{x^3}{24}+\frac{x^4}{120}+o(x^4)\right)^{-1}-1\right)\\
 &=\frac{1}{x^2}\left(-\left(\frac{x}{2}+\frac{x^2}{6}+\frac{x^3}{24}+\frac{x^4}{120}\right)+\left(\frac{x}{2}+\frac{x^2}{6}+\frac{x^3}{24}\right)^2-\left(\frac{x}{2}+\frac{x^2}{6}\right)^3+\left(\frac{x}{2}\right)^4+o(x^4)\right)\\
 &=\frac{1}{x^2}\left(-\frac{x}{2}+x^2\left(-\frac{1}{6}+\frac{1}{4}\right)+x^3\left(-\frac{1}{24}+\frac{1}{6}-\frac{1}{8}\right)
 +x^4\left(-\frac{1}{120}+\left(\frac{1}{36}+\frac{1}{24}\right)-\frac{1}{8}+\frac{1}{16}\right)+o(x^4)\right)\\
 &=-\frac{1}{2x}+\frac{1}{12}-\frac{x^2}{720}+o(x^2).
\end{align*}

\begin{center}
\shadowbox{
$\frac{1}{x(e^x-1)}-\frac{1}{x^2}\underset{x\rightarrow0}{=}-\frac{1}{2x}+\frac{1}{12}-\frac{x^2}{720}+o(x^2)$.
}
\end{center}
 \item 
 
\begin{align*}\ensuremath
x\ln(x+1)-(x+1)\ln x&=x\left(\ln x+\ln\left(1+\frac{1}{x}\right)\right)-(x+1)\ln x\underset{x\rightarrow+\infty}{=}-\ln
 x+x\left(\frac{1}{x}-\frac{1}{2x^2}+\frac{1}{3x^3}-\frac{1}{4x^4}+o\left(\frac{1}{x^4}\right)\right)\\
 &=-\ln x+1-\frac{1}{2x}+\frac{1}{3x^2}-\frac{1}{4x^3}+o\left(\frac{1}{x^3}\right).
\end{align*}

\begin{center}
\shadowbox{
$x\ln(x+1)-(x+1)\ln x\underset{x\rightarrow+\infty}{=}-\ln x+1-\frac{1}{2x}+\frac{1}{3x^2}-\frac{1}{4x^3}+o\left(\frac{1}{x^3}\right)$.
}
\end{center}
\end{enumerate}
\fincorrection
\correction{005434}
\begin{enumerate}
 \item  

$$f_n(a)=\text{exp}\left(n\ln\left(1+\frac{a}{n}\right)\right)\underset{n\rightarrow+\infty}{=}\text{exp}\left(a-\frac{a^2}{2n}+o\left(\frac{1}{n}\right)\right)= e^a\left(1-\frac{a^2}{2n}+o\left(\frac{1}{n}\right)\right).$$
En remplaçant $a$ par $b$ ou $a+b$, on obtient

\begin{align*}\ensuremath
f_n(a+b)-f_n(a)f_n(b)&\underset{n\rightarrow+\infty}{=}e^{a+b}\left(1-\frac{(a+b)^2}{2n}\right)-e^a\left(1-\frac{a^2}{2n}\right)e^b\left(1-\frac{b^2}{2n}\right)+o\left(\frac{1}{n}\right)\\
 &=e^{a+b}\frac{-(a+b)^2+a^2+b^2}{2n}+o\left(\frac{1}{n}\right)=-\frac{ab\;e^{a+b}}{n}+o\left(\frac{1}{n}\right).
\end{align*}
Donc, si $ab\neq0$, $f_n(a+b)-f_n(a)f_n(b)\underset{n\rightarrow+\infty}{\sim}-\frac{ab\;e^{a+b}}{n}$. Si $ab=0$, il est clair que $f_n(a+b)-f_n(a)f_n(b)=0$.
 \item  $e^{-a}f_n(a)\underset{n\rightarrow+\infty}{=}\text{exp}\left(-a+\left(a-\frac{a^2}{2n}+\frac{a^3}{3n^2}\right)+o\left(\frac{1}{n^2}\right)\right)
=1+\left(-\frac{a^2}{2n}+\frac{a^3}{3n^2}\right)+\frac{1}{2}\left(-\frac{a^2}{2n}\right)^2+o\left(\frac{1}{n^2}\right)$, et donc

$$e^{-a}f_n(a)-1+\frac{a^2}{2n}\underset{n\rightarrow+\infty}{\sim}\left(\frac{a^3}{3}+\frac{a^4}{8}\right)\frac{1}{n^2}.$$
\end{enumerate}
\fincorrection
\correction{005435}
\begin{enumerate}
 \item  
Pour $x\in\left[0,\frac{\pi}{2}\right]$, posons $f(x)=\sin x$. On a $f\left(\left]0,\frac{\pi}{2}\right]\right)=]0,1]\subset\left]0,\frac{\pi}{2}\right]$. Donc, puisque $u_0\in\left]0,\frac{\pi}{2}\right]$, on en déduit que $\forall n\in\Nn,\;u_n\in\left]0,\frac{\pi}{2}\right]$.\rule[-5mm]{0mm}{10mm}
Il est connu que $\forall x\in\left]0,\frac{\pi}{2}\right]$, $\sin x<x$ et de plus, pour $x\in\left[0,\frac{\pi}{2}\right]$, $\sin x=x\Leftrightarrow x=0$.
La suite $u$ est à valeurs dans $\left]0,\frac{\pi}{2}\right]$ et donc $\forall n\in\Nn,\;u_{n+1}=\sin(u_n)<u_n$. La suite $u$ est donc strictement décroissante et, étant minorée par $0$, converge vers un réel $\ell$ de $\left[0,\frac{\pi}{2}\right]$ qui vérifie ($f$ étant continue sur le segment $\left[0,\frac{\pi}{2}\right]$) $f(\ell)=\ell$ ou encore $\ell=0$.
En résumé,

\begin{center}
\shadowbox{
la suite $u$ est strictement positive, strictement décroissante et converge vers $0$.
}
\end{center}
 \item  Soit $\alpha$ un réel quelconque. Puisque la suite $u$ tend vers 0 , on a

\begin{align*}\ensuremath
u_{n+1}^{\alpha}-u_n^{\alpha}=(\sin u_n)^{\alpha}-u_n^{\alpha}&\underset{n\rightarrow+\infty}{=}\left(u_n-\frac{u_n^3}{6}+o(u_n^3)\right)^{\alpha}-u_n^{\alpha}\\
 &=u_n^{\alpha}\left(\left(1-\frac{u_n^2}{6}+o(u_n^2)\right)^{\alpha}-1\right)=u_n^{\alpha}\left(-\alpha\frac{u_n^2}{6}+o(u_n^2)\right)\\
 &=-\alpha\frac{u_n^{\alpha+2}}{6}+o(u_n^{\alpha+2})
\end{align*}
Pour $\alpha=-2$ on a donc 

$$\frac{1}{u_{n+1}^2}-\frac{1}{u_n^2}=\frac{1}{3}+o(1).$$
D'après le lemme de \textsc{Cesaro}, $\frac{1}{n}\sum_{k=0}^{n-1}\left(\frac{1}{u_{k+1}^2}-\frac{1}{u_k^2}\right)=\frac{1}{3}+o(1)$ ou encore $\frac{1}{n}\left(\frac{1}{u_n^2}-\frac{1}{u_0^2}\right)=\frac{1}{3}+o(1)$ ou enfin, 

\begin{center}
$\frac{1}{u_n^2}\underset{n\rightarrow+\infty}{=}\frac{n}{3}+\frac{1}{u_0^2}+o(n)\underset{n\rightarrow+\infty}{=}\frac{n}{3}+o(n)\underset{n\rightarrow+\infty}{\sim}\frac{n}{3}$.
\end{center}
Par suite, puisque la suite $u$ est strictement positive, 

\begin{center}
\shadowbox{
$u_n\underset{n\rightarrow+\infty}{\sim}\sqrt{\frac{3}{n}}.$
}
\end{center}
\end{enumerate}
\fincorrection
\correction{005436}
Il est immédiat par récurrence que $\forall n\in\Nn,\;u_n>0$. Donc, $\forall n\in\Nn^,\;\frac{u_{n+1}}{u_n}=e^{-u_n}<1$ et donc, puisque la suite $u$ est stritement positive, $u_{n+1}<u_n$. La suite $u$ est strictement décroissante, minorée par $0$ et donc converge vers un réel $\ell$ vérifiant $\ell=\ell e^{-\ell}$ ou encore $\ell(1-e^{-\ell})=0$ ou encore $\ell=0$.

\begin{center}
\shadowbox{
$u$ est strictement positive, strictement décroissante et converge vers $0$.
}
\end{center}
Soit $\alpha$ un réel quelconque. Puisque la suite $u$ tend vers $0$,

$$u_{n+1}^{\alpha}-u_n^{\alpha}=u_n^{\alpha}(e^{-\alpha u_n}-1)=u_n^{\alpha}(-\alpha u_n+o(u_n))=-\alpha u_n^{\alpha+1}+o(u_n^{\alpha+1}).$$
Pour $\alpha=-1$, on obtient en particulier $\frac{1}{u_{n+1}}-\frac{1}{u_n}=1+o(1)$. Puis, comme au numéro précédent, $\frac{1}{u_n}=n+\frac{1}{u_0}+o(n)\underset{n\rightarrow+\infty}{\sim}n$ et donc 

\begin{center}
\shadowbox{
$u_n\underset{n\rightarrow+\infty}{\sim}\frac{1}{n}$.
}
\end{center}
\fincorrection
\correction{005437}
Pour $n$ entier naturel donné, posons $I_n=\left]-\frac{\pi}{2}+n\pi,\frac{\pi}{2}+n\pi\right[$.
\textbullet~Soit $n\in\Nn$. Pour $x\in I_n$, posons $f(x)=\tan x-x$. $f$ est dérivable sur $I_n$ et pour $x$ dans $I_n$, $f'(x)=\tan^2x$. Ainsi, $f$ est dérivable sur $I_n$ et $f'$ est strictement positive sur $I_n\setminus\{n\pi\}$. Donc $f$ est strictement croissante sur $I_n$.

\textbullet~Soit $n\in\Nn$. $f$ est continue et strictement croissante sur $I_n$ et réalise donc une bijection de $I_n$ sur $f(I_n)=\Rr$. En particulier, $\forall n\in\Nn,\;\exists!x_n\in I_n/\;f(x_n)=0$ (ou encore tel que $\tan x_n=x_n$.
\textbullet~On a $x_0=0$ puis pour $n\in\Nn^*$, $f(n\pi)=-n\pi<0$ et donc, $\forall n\in\Nn^*,\;x_n\in]n\pi,\frac{\pi}{2}+n\pi[$. En particulier,

\begin{center}
\shadowbox{
$x_n\underset{n\rightarrow+\infty}{=}n\pi+O(1)$.
}
\end{center}
\textbullet~Posons alors $y_n=x_n-n\pi$. $\forall n\in\Nn^*,\;y_n\in\left]0,\frac{\pi}{2}\right[$. De plus, $\tan(y_n)=\tan(x_n)=n\pi+y_n$ et donc, puisque $y_n\in\left]0,\frac{\pi}{2}\right[$,

$$\frac{\pi}{2}>y_n=\Arctan(y_n+n\pi)\geq\Arctan(n\pi).$$ 
Puisque $\Arctan(n\pi)$ tend vers $\frac{\pi}{2}$, on a $y_n=\frac{\pi}{2}+o(1)$ ou encore

\begin{center}
\shadowbox{
$x_n\underset{n\rightarrow+\infty}{=}n\pi+\frac{\pi}{2}+o(1)$.
}
\end{center}
\textbullet~Posons maintenant $z_n=y_n-\frac{\pi}{2}=x_n-n\pi-\frac{\pi}{2}$.
D'après ce qui précède, $\forall n\in\Nn^*,\;z_n\in\left]-\frac{\pi}{2},0\right[$ et d'autre part $z_n\underset{n\rightarrow+\infty}{=}o(1)$.
Ensuite, $\tan\left(z_n+\frac{\pi}{2}\right)=n\pi+\frac{\pi}{2}+z_n$ et donc $-\cotan(z_n)=n\pi+\frac{\pi}{2}+z_n\underset{n\rightarrow+\infty}{\sim}n\pi$. Puisque $z_n$ tend vers $0$, on en déduit que

\begin{center}
$-\frac{1}{z_n}\underset{n\rightarrow+\infty}{\sim}-\cotan(z_n)\underset{n\rightarrow+\infty}{\sim}n\pi$,
\end{center}
ou encore $z_n\underset{n\rightarrow+\infty}{=}-\frac{1}{n\pi}+o\left(\frac{1}{n}\right)$. Ainsi,

\begin{center}
\shadowbox{$x_n\underset{n\rightarrow+\infty}{=}n\pi+\frac{\pi}{2}-\frac{1}{n\pi}+o\left(\frac{1}{n}\right).$}
\end{center}
\textbullet~Posons enfin $t_n=z_n+\frac{1}{n\pi}=x_n-n\pi-\frac{\pi}{2}+\frac{1}{n\pi}$. On sait que $t_n=o\left(\frac{1}{n}\right)$ et que 

\begin{center}
$-\cotan\left(t_n-\frac{1}{n\pi}\right)=-\cotan(z_n)=n\pi+\frac{\pi}{2}+z_n=n\pi+\frac{\pi}{2}-\frac{1}{n\pi}+o(\frac{1}{n})$.
\end{center} Par suite,
 
$$-\tan\left(t_n-\frac{1}{n\pi}\right)=\frac{1}{n\pi}\left(1+\frac{1}{2n}+o(\frac{1}{n})\right)^{-1}=\frac{1}{n\pi}-\frac{1}{2n^2\pi}+o\left(\frac{1}{n^2}\right),$$
puis,

$$\frac{1}{n\pi}-t_n=\Arctan\left(\frac{1}{n\pi}-\frac{1}{2n^2\pi}+o(\frac{1}{n^2})\right)=\frac{1}{n\pi}-\frac{1}{2n^2\pi}+o\left(\frac{1}{n^2}\right),$$ 
et donc $t_n=\frac{1}{2n^2\pi}+o\left(\frac{1}{n^2}\right)$. Finalement, 

\begin{center}
\shadowbox{
$x_n\underset{n\rightarrow+\infty}{=}n\pi+\frac{\pi}{2}-\frac{1}{n\pi}+\frac{1}{2n^2\pi}+o\left(\frac{1}{n^2}\right).$
}
\end{center}

\fincorrection
\correction{005438}
\begin{enumerate}
 \item  Pour $x>0$, posons $f(x)=x+\ln x$. $f$ est continue sur $]0,+\infty[$, strictement croissante sur $]0,+\infty[$ en tant que somme de deux fonctions continues et strictement croissantes sur $]0,+\infty[$. $f$ réalise donc une bijection de $]0,+\infty[$ sur $f\left(]0,+\infty[\right)=\left]\lim_{x\rightarrow 0,\;x>0}f(x),\lim_{x\rightarrow +\infty}f(x)\right[=]-\infty,+\infty[=\Rr$. En particulier,

\begin{center}
\shadowbox{
$\forall k\in\Rr,\;\exists!x_k\in]0,+\infty[/\;f(x_k)=k.$
}
\end{center}
 \item  $f\left(\frac{k}{2}\right)=\frac{k}{2}+\ln\frac{k}{2}<k$ pour $k$ suffisament grand (car $k-(\frac{k}{2}+\ln\frac{k}{2})=\frac{k}{2}-\ln\frac{k}{2}\underset{k\rightarrow+\infty}{\rightarrow}+\infty$ d'après les théorèmes de croissances comparées). Donc, pour $k$ suffisament grand, $f\left(\frac{k}{2}\right)<f(x_k)$. Puisque $f$ est strictement croissante sur $]0,+\infty[$, on en déduit que $x_k>\frac{k}{2}$ pour $k$ suffisament grand et donc que $\lim_{k\rightarrow +\infty}x_k=+\infty$. Mais alors, $k=x_k+\ln x_k\sim x_k$ et donc, quand $k$ tend vers $+\infty$,

\begin{center}
\shadowbox{
$x_k\underset{k\rightarrow+\infty}{=}k+o(k).$
}
\end{center}
Posons $y_k=x_k-k$. On a $y_k=o(k)$ et de plus $y_k+\ln(y_k+k)=0$ ce qui s'écrit~:

$$y_k=-\ln(k+y_k)=-\ln(k+o(k))=-\ln k+\ln(1+o(1))=-\ln k+o(1).$$
Donc,

\begin{center}
\shadowbox{
$x_k\underset{k\rightarrow+\infty}{=}k-\ln k+o(1).$
}
\end{center}
Posons $z_k=y_k+\ln k=x_k-k+\ln k$. Alors, $z_k=o(1)$ et $-\ln k+z_k=-\ln(k-\ln k+z_k)$. Par suite,

$$z_k=\ln k-\ln(k-\ln k+o(1))=-\ln\left(1-\frac{\ln k}{k}+o\left(\frac{\ln k}{k}\right)\right)=\frac{\ln k}{k}+o\left(\frac{\ln k}{k}\right).$$

Finalement,

\begin{center}
\shadowbox{
$x_k\underset{k\rightarrow+\infty}{=}k-\ln k+\frac{\ln k}{k}+o\left(\frac{\ln k}{k}\right).$
}
\end{center}
\end{enumerate}
\fincorrection
\correction{005439}
\begin{enumerate}
 \item  $x^3\sin\frac{1}{x^2}\underset{x\rightarrow0}{=}O(x^3)$ et en particulier $x^3\sin\frac{1}{x}\underset{x\rightarrow0}{=}o(x^2)$. Donc, en tenant compte de $f(0)=1$,

\begin{center}
$f(x)\underset{x\rightarrow0}{=}1+x+x^2+o(x^2)$.
\end{center}
$f$ admet en $0$ un développement limité d'ordre $2$.
 \item  $f(x)\underset{x\rightarrow0}{=}1+x+o(x)$. Donc, $f$ admet en $0$ un développement limité d'ordre $1$. On en déduit que $f$ est continue et dérivable en $0$ avec $f(0)=f'(0)=1$. $f$ est d'autre part dérivable sur $\Rr^*$ en vertu de théorèmes généraux (et donc sur $\Rr$) et pour $x\neq 0$, $f'(x)=1+2x+3x^2\sin\frac{1}{x^2}-2\cos\frac{1}{x^2}$.
 \item  $f'$ est définie sur $\Rr$ mais n'a pas de limite en $0$. $f'$ n'admet donc même pas un développement limité d'ordre $0$ en $0$.
\end{enumerate}
\fincorrection
\correction{005440}
$\frac{1}{\sqrt{1-x^2}}\underset{x\rightarrow0}{=}1+\frac{x^2}{2}+\frac{3x^4}{8}+o(x^4)$, et donc 

$$\Arcsin x\underset{x\rightarrow0}{=}x+\frac{x^3}{6}+\frac{3x^5}{40}+o(x^5).$$
Puis,

$$\frac{1}{\Arcsin x}\underset{x\rightarrow0}{=}\frac{1}{x}\left(1+\frac{x^2}{6}+\frac{3x^4}{40}+o(x^4)\right)^{-1}=\frac{1}{x}\left(1-\frac{x^2}{6}-\frac{3x^4}{40}+\frac{x^4}{36}+o(x^4)\right)=\frac{1}{x}-\frac{x}{6}-\frac{17x^3}{360}+o(x^3),$$
et donc,

\begin{center}
\shadowbox{
$\frac{1}{x}-\frac{1}{\Arcsin x}\underset{x\rightarrow0}{=}\frac{x}{6}+\frac{17x^3}{360}+o(x^3).$
}
\end{center}
La fonction $f$ proposée se prolonge donc par continuité en $0$ en posant $f(0)=0$. Le prolongement est dérivable en $0$ et $f'(0)=\frac{1}{6}$. La courbe représentative de $f$ admet à l'origine une tangente d'équation $y=\frac{x}{6}$. Le signe de la différence $f(x)-\frac{x}{6}$ est, au voisinage de $0$, le signe de $\frac{17x^3}{360}$. La courbe représentative de $f$ admet donc à l'origine une tangente d'inflexion d'équation $y=\frac{x}{6}$.
\fincorrection
\correction{005441}
\begin{enumerate}
 \item  $\Arccos x\underset{x\rightarrow1^-}{=}o(1)$ (développement limité à l'ordre $0$). Mais la fonction $x\mapsto\Arccos x$ n'est pas dérivable en $1$ et n'admet donc pas en $1$ un développemement limité d'ordre $1$.

{2)} Puisque $\Arccos x\underset{x\rightarrow1^-}{=}o(1)$, 

$$\Arccos x\underset{x\rightarrow1^-}{\sim}\sin(\Arccos x)=\sqrt{1-x^2}=\sqrt{(1+x)(1-x)}\underset{x\rightarrow1^-}{\sim}\sqrt{2}\sqrt{1-x}.$$

\begin{center}
\shadowbox{
$\Arccos x\underset{x\rightarrow1^-}{\sim}\sqrt{2}\sqrt{1-x}.$
}
\end{center}
\end{enumerate}
\fincorrection
\correction{005442}
\begin{enumerate}
 \item  Quand $x$ tend vers $0$,
  
\begin{align*}\ensuremath
\frac{1}{(1-x)^2(1+x)}&=\frac{1}{4}\frac{1}{1-x}+\frac{1}{2}\frac{1}{(1-x)^2}+\frac{1}{4}\frac{1}{1+x}
\underset{x\rightarrow0}{=}\frac{1}{4}\left(\sum_{k=0}^{n}x^k+2\sum_{k=0}^{n}(k+1)x^k+\sum_{k=0}^{n}(-1)^kx^k\right)+o(x^n)\\
 &=\sum_{k=0}^{n}\frac{2k+3+(-1)^k}{4}x^k+o(x^n).
\end{align*}
 \item  On a aussi,

\begin{align*}\ensuremath
\frac{1}{(1-x)^2(1+x)}&=\frac{1}{(1-x)(1-x^2)}\underset{x\rightarrow0}{=}\left(\sum_{k=0}^{n}x^p\right)\left(\sum_{k=0}^{n}x^{2q}\right)+o(x^n)\\
 &=\sum_{k=0}^{n}\left(\sum_{p+2q=k}^{}1\right)x^k+o(x^n)=\sum_{k=0}^{n}a_kx^k+o(x^n).
\end{align*}
Par unicité des coefficients d'un développement limité, on a donc

\begin{center}
\shadowbox{
$\forall k\in\Nn$, $a_k=\frac{2k+3+(-1)^k}{4}$.
}
\end{center}
($a_k$ est le nombre de façons de payer $k$ euros en pièces de $1$ et $2$ euros).
\end{enumerate}
\fincorrection


\end{document}

