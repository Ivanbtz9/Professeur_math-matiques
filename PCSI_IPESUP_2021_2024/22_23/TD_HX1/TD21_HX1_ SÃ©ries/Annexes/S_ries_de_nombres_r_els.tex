\documentclass[11pt]{article}

 %Configuration de la feuille 
 
\usepackage{amsmath,amssymb,enumerate,graphicx,pgf,tikz,fancyhdr}
\usepackage[utf8]{inputenc}
\usetikzlibrary{arrows}
\usepackage{geometry}
\usepackage{tabvar}
\geometry{hmargin=2.2cm,vmargin=1.5cm}\pagestyle{fancy}
\lfoot{\bfseries http://www.bibmath.net}
\rfoot{\bfseries\thepage}
\cfoot{}
\renewcommand{\footrulewidth}{0.5pt} %Filet en bas de page

 %Macros utilisées dans la base de données d'exercices 

\newcommand{\mtn}{\mathbb{N}}
\newcommand{\mtns}{\mathbb{N}^*}
\newcommand{\mtz}{\mathbb{Z}}
\newcommand{\mtr}{\mathbb{R}}
\newcommand{\mtk}{\mathbb{K}}
\newcommand{\mtq}{\mathbb{Q}}
\newcommand{\mtc}{\mathbb{C}}
\newcommand{\mch}{\mathcal{H}}
\newcommand{\mcp}{\mathcal{P}}
\newcommand{\mcb}{\mathcal{B}}
\newcommand{\mcl}{\mathcal{L}}
\newcommand{\mcm}{\mathcal{M}}
\newcommand{\mcc}{\mathcal{C}}
\newcommand{\mcmn}{\mathcal{M}}
\newcommand{\mcmnr}{\mathcal{M}_n(\mtr)}
\newcommand{\mcmnk}{\mathcal{M}_n(\mtk)}
\newcommand{\mcsn}{\mathcal{S}_n}
\newcommand{\mcs}{\mathcal{S}}
\newcommand{\mcd}{\mathcal{D}}
\newcommand{\mcsns}{\mathcal{S}_n^{++}}
\newcommand{\glnk}{GL_n(\mtk)}
\newcommand{\mnr}{\mathcal{M}_n(\mtr)}
\DeclareMathOperator{\ch}{ch}
\DeclareMathOperator{\sh}{sh}
\DeclareMathOperator{\vect}{vect}
\DeclareMathOperator{\card}{card}
\DeclareMathOperator{\comat}{comat}
\DeclareMathOperator{\imv}{Im}
\DeclareMathOperator{\rang}{rg}
\DeclareMathOperator{\Fr}{Fr}
\DeclareMathOperator{\diam}{diam}
\DeclareMathOperator{\supp}{supp}
\newcommand{\veps}{\varepsilon}
\newcommand{\mcu}{\mathcal{U}}
\newcommand{\mcun}{\mcu_n}
\newcommand{\dis}{\displaystyle}
\newcommand{\croouv}{[\![}
\newcommand{\crofer}{]\!]}
\newcommand{\rab}{\mathcal{R}(a,b)}
\newcommand{\pss}[2]{\langle #1,#2\rangle}
 %Document 

\begin{document} 

\begin{center}\textsc{{\huge }}\end{center}

% Exercice 1110


\vskip0.3cm\noindent\textsc{Exercice 1} - Équivalents et majorations - 2
\vskip0.2cm
\'Etudier la convergence des séries $\sum u_n$ suivantes :
$$\begin{array}{lllll}
\displaystyle \mathbf 1.\ u_n=\left(\frac{1}{2}\right)^{\sqrt{n}}&&
\displaystyle \mathbf 2.\ u_n=a^n n!,\ a\in\mathbb R_+&&\displaystyle \mathbf 3. \ u_n=ne^{-\sqrt n}\\
\displaystyle {\bf 4.}
\ u_n=\frac{\ln(n^2+3)\sqrt{2^n+1}}{4^n}.&&
\displaystyle {\bf 5}.\ 
\ u_n=\frac{\ln n}{\ln(e^n -1)}&&
\displaystyle \mathbf 6.\ u_n=\left(\frac 1n\right)^{1+\frac 1n}\\
\ \displaystyle \mathbf 7.\ u_n=\frac{(n!)^3}{(3n)!}.
\end{array}$$



% Exercice 1127


\vskip0.3cm\noindent\textsc{Exercice 2} - Une erreur classique...
\vskip0.2cm
\begin{enumerate}
\item Démontrer que la série $\sum_n \frac{(-1)^n}{\sqrt n}$ converge.
\item Démontrer que $\displaystyle \frac{(-1)^n}{\sqrt n+(-1)^n}=\frac{(-1)^n}{\sqrt n}-\frac1n+\frac{(-1)^n}{n\sqrt
n}+o\left(\frac 1{n\sqrt n}\right)$.
\item \'Etudier la convergence de la série $\displaystyle \sum_n \frac{(-1)^n}{\sqrt n+(-1)^n}$.
\item Qu'a-t-on voulu mettre en évidence dans cet exercice?
\end{enumerate}


% Exercice 1115


\vskip0.3cm\noindent\textsc{Exercice 3} - Séries de Bertrand
\vskip0.2cm
On souhaite étudier, suivant la valeur de $\alpha,\beta\in\mathbb R$, la convergence de la série de terme général
$$u_n=\frac{1}{n^\alpha(\ln n)^\beta}.$$
\begin{enumerate}
\item Démontrer que la série converge si $\alpha>1$.
\item Traiter le cas $\alpha<1$.
\item On suppose que $\alpha=1$. 
On pose $T_n=\int_2^n \frac{dx}{x(\ln x)^\beta}$.
\begin{enumerate}
\item Montrer si $\beta\leq 0$, alors la série de terme général $u_n$ est divergente.
\item Montrer que si $\beta>1$, alors la suite $(T_n)$ est bornée, alors que si $\beta\leq 1$, la suite $(T_n)$ tend vers $+\infty$.
\item Conclure pour la série de terme général $u_n$, lorsque $\alpha=1$.
\end{enumerate}
\end{enumerate}


% Exercice 1153


\vskip0.3cm\noindent\textsc{Exercice 4} - Développement asymptotique de la série harmonique
\vskip0.2cm
On pose $H_n=1+\frac12+\dots+\frac1n$. 
\begin{enumerate}
\item Prouver que $H_n\sim_{+\infty}\ln n$.
\item On pose $u_n=H_n-\ln n$, et $v_n=u_{n+1}-u_n$.
\'Etudier la nature de la série $\sum_n v_n$. En déduire que la suite $(u_n)$ est convergente. On notera $\gamma$ sa limite.
\item Soit $R_n=\sum_{k=n}^{+\infty} \frac{1}{k^2}$. Donner un équivalent de $R_n$.
\item Soit $w_n$ tel que $H_n=\ln n+\gamma+w_n$, et soit 
$t_n=w_{n+1}-w_n$. Donner un équivalent du reste $\sum_{k\geq n}t_k$.
En déduire que $H_n=\ln n+\gamma+\frac{1}{2n}+o\left(\frac1n\right)$.
\end{enumerate}


% Exercice 1160


\vskip0.3cm\noindent\textsc{Exercice 5} - Formule de Stirling
\vskip0.2cm
\begin{enumerate}
\item Soit $(x_n)$ une suite de réels et soit $(y_n)$ définie par $y_n=x_{n+1}-x_n$. Démontrer que la série $\sum_n y_n$ et la suite
$(x_n)$ sont de même nature.
\item On pose  $(u_n)$ la suite définie par $\dis u_n=\frac{n^ne^{-n}\sqrt{n}}{n!}$. 
Donner la nature de la série de terme général $\dis v_n=\ln\left(\frac{u_{n+1}}{u_n}\right)$. 
\item En déduire l'existence d'une constante $C>0$ telle que :
$$n!\sim_{+\infty} C\sqrt{n}n^ne^{-n}.$$
\end{enumerate}




\vskip0.5cm
\noindent{\small Cette feuille d'exercices a été conçue à l'aide du site \textsf{https://www.bibmath.net}}

%Vous avez accès aux corrigés de cette feuille par l'url : https://www.bibmath.net/ressources/justeunefeuille.php?id=29273
\end{document}