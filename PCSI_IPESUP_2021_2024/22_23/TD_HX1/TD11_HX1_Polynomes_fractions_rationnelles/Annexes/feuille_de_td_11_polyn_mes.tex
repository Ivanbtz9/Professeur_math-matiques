\documentclass[11pt]{article}

 %Configuration de la feuille 
 
\usepackage{amsmath,amssymb,enumerate,graphicx,pgf,tikz,fancyhdr}
\usepackage[utf8]{inputenc}
\usetikzlibrary{arrows}
\usepackage{geometry}
\usepackage{tabvar}
\geometry{hmargin=2.2cm,vmargin=1.5cm}\pagestyle{fancy}
\lfoot{\bfseries http://www.bibmath.net}
\rfoot{\bfseries\thepage}
\cfoot{}
\renewcommand{\footrulewidth}{0.5pt} %Filet en bas de page

 %Macros utilisées dans la base de données d'exercices 

\newcommand{\mtn}{\mathbb{N}}
\newcommand{\mtns}{\mathbb{N}^*}
\newcommand{\mtz}{\mathbb{Z}}
\newcommand{\mtr}{\mathbb{R}}
\newcommand{\mtk}{\mathbb{K}}
\newcommand{\mtq}{\mathbb{Q}}
\newcommand{\mtc}{\mathbb{C}}
\newcommand{\mch}{\mathcal{H}}
\newcommand{\mcp}{\mathcal{P}}
\newcommand{\mcb}{\mathcal{B}}
\newcommand{\mcl}{\mathcal{L}}
\newcommand{\mcm}{\mathcal{M}}
\newcommand{\mcc}{\mathcal{C}}
\newcommand{\mcmn}{\mathcal{M}}
\newcommand{\mcmnr}{\mathcal{M}_n(\mtr)}
\newcommand{\mcmnk}{\mathcal{M}_n(\mtk)}
\newcommand{\mcsn}{\mathcal{S}_n}
\newcommand{\mcs}{\mathcal{S}}
\newcommand{\mcd}{\mathcal{D}}
\newcommand{\mcsns}{\mathcal{S}_n^{++}}
\newcommand{\glnk}{GL_n(\mtk)}
\newcommand{\mnr}{\mathcal{M}_n(\mtr)}
\DeclareMathOperator{\ch}{ch}
\DeclareMathOperator{\sh}{sh}
\DeclareMathOperator{\vect}{vect}
\DeclareMathOperator{\card}{card}
\DeclareMathOperator{\comat}{comat}
\DeclareMathOperator{\imv}{Im}
\DeclareMathOperator{\rang}{rg}
\DeclareMathOperator{\Fr}{Fr}
\DeclareMathOperator{\diam}{diam}
\DeclareMathOperator{\supp}{supp}
\newcommand{\veps}{\varepsilon}
\newcommand{\mcu}{\mathcal{U}}
\newcommand{\mcun}{\mcu_n}
\newcommand{\dis}{\displaystyle}
\newcommand{\croouv}{[\![}
\newcommand{\crofer}{]\!]}
\newcommand{\rab}{\mathcal{R}(a,b)}
\newcommand{\pss}[2]{\langle #1,#2\rangle}
 %Document 

\begin{document} 

\begin{center}\textsc{{\huge }}\end{center}

% Exercice 657


\vskip0.3cm\noindent\textsc{Exercice 1} - Un calcul de somme
\vskip0.2cm
\begin{enumerate}
\item Décomposer en éléments simples la fraction rationnelle $\displaystyle\frac{1}{X(X+1)(X+2)}$.
\item En déduire la limite de la suite $(S_n)$ suivante : $\displaystyle S_n=\sum_{k=1}^n \frac{1}{k(k+1)(k+2)}$.
\end{enumerate}


% Exercice 658


\vskip0.3cm\noindent\textsc{Exercice 2} - Un calcul de somme
\vskip0.2cm
Soit $P\in\mathbb R[X]$ un polynôme de degré $n\geq 1$ possédant $n$ racines distinctes $x_1,\dots,x_n$ non-nulles.
\begin{enumerate}
\item Décomposer en éléments simples la fraction rationnelle $\displaystyle \frac1{XP(X)}$.
\item En déduire que $\displaystyle\sum_{k=1}^n \frac{1}{x_k P'(x_k)}=\frac{-1}{P(0)}$.
\end{enumerate}



% Exercice 660


\vskip0.3cm\noindent\textsc{Exercice 3} - Enveloppe convexe des zéros
\vskip0.2cm
Soit $P\in\mathbb C_n[X]$ admettant $n$ racines simples $\alpha_1,\dots,\alpha_n$.
Soient $A_1,\dots,A_n$ les points du plan complexe d'affixe respectives $\alpha_1,\dots,\alpha_n$.
\begin{enumerate}
\item Décomposer la fraction rationnelle $P'/P$ en éléments simples.
\item Soit $\beta$ une racine de $P'$, et soit $B$ son image dans le plan complexe. Déduire de la question précédente que 
$$\sum_{j=1}^n \frac{1}{\beta-\alpha_j}=0.$$
\item En déduire que $B$ est un barycentre de la famille de points $(A_1,\dots,A_n)$, avec des coefficients positifs.
Interpréter géométriquement cette propriété.
\end{enumerate}


% Exercice 649


\vskip0.3cm\noindent\textsc{Exercice 4} - Tout polynôme positif est somme de deux carrés
\vskip0.2cm
Soit $P\in\mathbb R[X]$ non constant tel que $P(x)\geq 0$ pour tout réel $x$.
\begin{enumerate}
 \item Montrer que le coefficient dominant de $P$ est positif et que les racines réelles de $P$ sont de multiplicité paire.
 \item Montrer qu'il existe un polynôme $C\in\mathbb C[X]$ tel que $P=C\overline{C}$.
 \item En déduire qu'il existe $A$ et $B$ dans $\mathbb R[X]$ tels que $P=A^2+B^2$.
\end{enumerate}


% Exercice 633


\vskip0.3cm\noindent\textsc{Exercice 5} - Racines rationnelles
\vskip0.2cm
Soit $P(X)=a_nX^n+\dots+a_0$ un polynôme à coefficients dans $\mathbb Z$, avec
$a_n\neq 0$ et $a_0\neq 0$. On suppose que $P$ admet une racine rationnelle $p/q$
avec $p\wedge q=1$. Démontrer que $p|a_0$ et que $q|a_n$. Le polynôme $P(X)=X^5-X^2+1$
admet-il des racines dans $\mathbb Q$?


% Exercice 632


\vskip0.3cm\noindent\textsc{Exercice 6} - Somme des racines
\vskip0.2cm
Soit $P\in\mathbb C[X]$. On note, pour $p<n$, $u_p$ la somme des racines de $P^{(p)}$. Démontrer que
$u_0,\dots,u_{n-1}$ forme une progression arithmétique.




\vskip0.5cm
\noindent{\small Cette feuille d'exercices a été conçue à l'aide du site \textsf{https://www.bibmath.net}}

%Vous avez accès aux corrigés de cette feuille par l'url : https://www.bibmath.net/ressources/justeunefeuille.php?id=26589
\end{document}