\documentclass[11pt]{article}

 %Configuration de la feuille 
 
\usepackage{amsmath,amssymb,enumerate,graphicx,pgf,tikz,fancyhdr}
\usepackage[utf8]{inputenc}
\usetikzlibrary{arrows}
\usepackage{geometry}
\usepackage{tabvar}
\geometry{hmargin=2.2cm,vmargin=1.5cm}\pagestyle{fancy}
\lfoot{\bfseries http://www.bibmath.net}
\rfoot{\bfseries\thepage}
\cfoot{}
\renewcommand{\footrulewidth}{0.5pt} %Filet en bas de page

 %Macros utilisées dans la base de données d'exercices 

\newcommand{\mtn}{\mathbb{N}}
\newcommand{\mtns}{\mathbb{N}^*}
\newcommand{\mtz}{\mathbb{Z}}
\newcommand{\mtr}{\mathbb{R}}
\newcommand{\mtk}{\mathbb{K}}
\newcommand{\mtq}{\mathbb{Q}}
\newcommand{\mtc}{\mathbb{C}}
\newcommand{\mch}{\mathcal{H}}
\newcommand{\mcp}{\mathcal{P}}
\newcommand{\mcb}{\mathcal{B}}
\newcommand{\mcl}{\mathcal{L}}
\newcommand{\mcm}{\mathcal{M}}
\newcommand{\mcc}{\mathcal{C}}
\newcommand{\mcmn}{\mathcal{M}}
\newcommand{\mcmnr}{\mathcal{M}_n(\mtr)}
\newcommand{\mcmnk}{\mathcal{M}_n(\mtk)}
\newcommand{\mcsn}{\mathcal{S}_n}
\newcommand{\mcs}{\mathcal{S}}
\newcommand{\mcd}{\mathcal{D}}
\newcommand{\mcsns}{\mathcal{S}_n^{++}}
\newcommand{\glnk}{GL_n(\mtk)}
\newcommand{\mnr}{\mathcal{M}_n(\mtr)}
\DeclareMathOperator{\ch}{ch}
\DeclareMathOperator{\sh}{sh}
\DeclareMathOperator{\vect}{vect}
\DeclareMathOperator{\card}{card}
\DeclareMathOperator{\comat}{comat}
\DeclareMathOperator{\imv}{Im}
\DeclareMathOperator{\rang}{rg}
\DeclareMathOperator{\Fr}{Fr}
\DeclareMathOperator{\diam}{diam}
\DeclareMathOperator{\supp}{supp}
\newcommand{\veps}{\varepsilon}
\newcommand{\mcu}{\mathcal{U}}
\newcommand{\mcun}{\mcu_n}
\newcommand{\dis}{\displaystyle}
\newcommand{\croouv}{[\![}
\newcommand{\crofer}{]\!]}
\newcommand{\rab}{\mathcal{R}(a,b)}
\newcommand{\pss}[2]{\langle #1,#2\rangle}
 %Document 

\begin{document} 

\begin{center}\textsc{{\huge }}\end{center}

% Exercice 655


\vskip0.3cm\noindent\textsc{Exercice 1} - Pôles simples
\vskip0.2cm
Décomposer en éléments simples les fractions rationnelles suivantes : 
$$\begin{array}{lll}
\displaystyle\mathbf{1.}\quad\frac{1}{X^3-X}&\quad\quad\mathbf{2.}\quad  \displaystyle \frac{X^3}{(X-1)(X-2)(X-3)}
\end{array}$$


% Exercice 3225


\vskip0.3cm\noindent\textsc{Exercice 2} - Tous les cas possibles
\vskip0.2cm
Décomposer sur $\mathbb R$ les fractions rationnelles suivantes :
$$\begin{array}{lll}
\displaystyle\mathbf{1.}\quad\frac{X^2+2X +5}{X^2-3X+2}&\quad\quad\mathbf{2.}\quad \displaystyle\frac{X^2+3X+1}{(X-1)^2(X-2)}&\quad\quad\mathbf{3.}\quad \displaystyle \frac 1{X^4-1}
\end{array}$$


% Exercice 3226


\vskip0.3cm\noindent\textsc{Exercice 3} - Pôles multiples
\vskip0.2cm
Décomposer en éléments simples les fractions rationnelles suivantes : 
$$\begin{array}{lll}
\displaystyle\mathbf{1.}\quad\frac{2X^2+1}{(X^2-1)^2}&
\quad\quad\mathbf{2.}\quad\displaystyle\frac{X^3+1}{(X-1)^3}
\end{array}$$


% Exercice 3227


\vskip0.3cm\noindent\textsc{Exercice 4} - Pôle multiple et facteur irréductible de degré $2$
\vskip0.2cm
Décomposer en éléments simples sur $\mathbb R$ la fraction rationnelle suivante : 
$$\frac{X^4+1}{(X+1)^2(X^2+1)}$$




\vskip0.5cm
\noindent{\small Cette feuille d'exercices a été conçue à l'aide du site \textsf{https://www.bibmath.net}}

%Vous avez accès aux corrigés de cette feuille par l'url : https://www.bibmath.net/ressources/justeunefeuille.php?id=26597
\end{document}