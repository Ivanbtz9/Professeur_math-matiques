
%%%%%%%%%%%%%%%%%% PREAMBULE %%%%%%%%%%%%%%%%%%

\documentclass[11pt,a4paper]{article}

\usepackage{amsfonts,amsmath,amssymb,amsthm}
\usepackage[utf8]{inputenc}
\usepackage[T1]{fontenc}
\usepackage[francais]{babel}
\usepackage{mathptmx}
\usepackage{fancybox}
\usepackage{graphicx}
\usepackage{ifthen}

\usepackage{tikz}   

\usepackage{hyperref}
\hypersetup{colorlinks=true, linkcolor=blue, urlcolor=blue,
pdftitle={Exo7 - Exercices de mathématiques}, pdfauthor={Exo7}}

\usepackage{geometry}
\geometry{top=2cm, bottom=2cm, left=2cm, right=2cm}

%----- Ensembles : entiers, reels, complexes -----
\newcommand{\Nn}{\mathbb{N}} \newcommand{\N}{\mathbb{N}}
\newcommand{\Zz}{\mathbb{Z}} \newcommand{\Z}{\mathbb{Z}}
\newcommand{\Qq}{\mathbb{Q}} \newcommand{\Q}{\mathbb{Q}}
\newcommand{\Rr}{\mathbb{R}} \newcommand{\R}{\mathbb{R}}
\newcommand{\Cc}{\mathbb{C}} \newcommand{\C}{\mathbb{C}}
\newcommand{\Kk}{\mathbb{K}} \newcommand{\K}{\mathbb{K}}

%----- Modifications de symboles -----
\renewcommand{\epsilon}{\varepsilon}
\renewcommand{\Re}{\mathop{\mathrm{Re}}\nolimits}
\renewcommand{\Im}{\mathop{\mathrm{Im}}\nolimits}
\newcommand{\llbracket}{\left[\kern-0.15em\left[}
\newcommand{\rrbracket}{\right]\kern-0.15em\right]}
\renewcommand{\ge}{\geqslant} \renewcommand{\geq}{\geqslant}
\renewcommand{\le}{\leqslant} \renewcommand{\leq}{\leqslant}

%----- Fonctions usuelles -----
\newcommand{\ch}{\mathop{\mathrm{ch}}\nolimits}
\newcommand{\sh}{\mathop{\mathrm{sh}}\nolimits}
\renewcommand{\tanh}{\mathop{\mathrm{th}}\nolimits}
\newcommand{\cotan}{\mathop{\mathrm{cotan}}\nolimits}
\newcommand{\Arcsin}{\mathop{\mathrm{arcsin}}\nolimits}
\newcommand{\Arccos}{\mathop{\mathrm{arccos}}\nolimits}
\newcommand{\Arctan}{\mathop{\mathrm{arctan}}\nolimits}
\newcommand{\Argsh}{\mathop{\mathrm{argsh}}\nolimits}
\newcommand{\Argch}{\mathop{\mathrm{argch}}\nolimits}
\newcommand{\Argth}{\mathop{\mathrm{argth}}\nolimits}
\newcommand{\pgcd}{\mathop{\mathrm{pgcd}}\nolimits} 

%----- Structure des exercices ------

\newcommand{\exercice}[1]{\video{0}}
\newcommand{\finexercice}{}
\newcommand{\noindication}{}
\newcommand{\nocorrection}{}

\newcounter{exo}
\newcommand{\enonce}[2]{\refstepcounter{exo}\hypertarget{exo7:#1}{}\label{exo7:#1}{\bf Exercice \arabic{exo}}\ \  #2\vspace{1mm}\hrule\vspace{1mm}}

\newcommand{\finenonce}[1]{
\ifthenelse{\equal{\ref{ind7:#1}}{\ref{bidon}}\and\equal{\ref{cor7:#1}}{\ref{bidon}}}{}{\par{\footnotesize
\ifthenelse{\equal{\ref{ind7:#1}}{\ref{bidon}}}{}{\hyperlink{ind7:#1}{\texttt{Indication} $\blacktriangledown$}\qquad}
\ifthenelse{\equal{\ref{cor7:#1}}{\ref{bidon}}}{}{\hyperlink{cor7:#1}{\texttt{Correction} $\blacktriangledown$}}}}
\ifthenelse{\equal{\myvideo}{0}}{}{{\footnotesize\qquad\texttt{\href{http://www.youtube.com/watch?v=\myvideo}{Vidéo $\blacksquare$}}}}
\hfill{\scriptsize\texttt{[#1]}}\vspace{1mm}\hrule\vspace*{7mm}}

\newcommand{\indication}[1]{\hypertarget{ind7:#1}{}\label{ind7:#1}{\bf Indication pour \hyperlink{exo7:#1}{l'exercice \ref{exo7:#1} $\blacktriangle$}}\vspace{1mm}\hrule\vspace{1mm}}
\newcommand{\finindication}{\vspace{1mm}\hrule\vspace*{7mm}}
\newcommand{\correction}[1]{\hypertarget{cor7:#1}{}\label{cor7:#1}{\bf Correction de \hyperlink{exo7:#1}{l'exercice \ref{exo7:#1} $\blacktriangle$}}\vspace{1mm}\hrule\vspace{1mm}}
\newcommand{\fincorrection}{\vspace{1mm}\hrule\vspace*{7mm}}

\newcommand{\finenonces}{\newpage}
\newcommand{\finindications}{\newpage}


\newcommand{\fiche}[1]{} \newcommand{\finfiche}{}
%\newcommand{\titre}[1]{\centerline{\large \bf #1}}
\newcommand{\addcommand}[1]{}

% variable myvideo : 0 no video, otherwise youtube reference
\newcommand{\video}[1]{\def\myvideo{#1}}

%----- Presentation ------

\setlength{\parindent}{0cm}

\definecolor{myred}{rgb}{0.93,0.26,0}
\definecolor{myorange}{rgb}{0.97,0.58,0}
\definecolor{myyellow}{rgb}{1,0.86,0}

\newcommand{\LogoExoSept}[1]{  % input : echelle       %% NEW
{\usefont{U}{cmss}{bx}{n}
\begin{tikzpicture}[scale=0.1*#1,transform shape]
  \fill[color=myorange] (0,0)--(4,0)--(4,-4)--(0,-4)--cycle;
  \fill[color=myred] (0,0)--(0,3)--(-3,3)--(-3,0)--cycle;
  \fill[color=myyellow] (4,0)--(7,4)--(3,7)--(0,3)--cycle;
  \node[scale=5] at (3.5,3.5) {Exo7};
\end{tikzpicture}}
}


% titre
\newcommand{\titre}[1]{%
\vspace*{-4ex} \hfill \hspace*{1.5cm} \hypersetup{linkcolor=black, urlcolor=black} 
\href{http://exo7.emath.fr}{\LogoExoSept{3}} 
 \vspace*{-5.7ex}\newline 
\hypersetup{linkcolor=blue, urlcolor=blue}  {\Large \bf #1} \newline 
 \rule{12cm}{1mm} \vspace*{3ex}}

%----- Commandes supplementaires ------



\begin{document}

%%%%%%%%%%%%%%%%%% EXERCICES %%%%%%%%%%%%%%%%%%
\fiche{f00007, bodin, 2007/09/01}

\titre{Polynômes}

Corrections de Léa Blanc-Centi.

\section{Opérations sur les polynômes}

\exercice{427, cousquer, 2003/10/01}
\video{59DItk_pxuc}
\enonce{000427}{}

Trouver le polynôme $P$ de degré inférieur ou égal à $3$ tel que :
$$P(0)=1\quad\text{et}\quad P(1)=0\quad\text{et}\quad P(-1)=-2\quad\text{et}\quad P(2)=4.$$
\finenonce{000427} 


\finexercice
\section{Division, pgcd}

\exercice{6955, exo7, 2014/04/01}
\video{c08DgsqArHw}
% Mélange de 364 (bodin), 375 (ridde), 366 (bodin), 371 (cousquer)
\enonce{006955}{}
\begin{enumerate}
\item Effectuer la division euclidienne de $A$ par $B$ :
\begin{enumerate}
\item $A=3X^5+4X^2+1,\ B=X^2+2X+3$
\item $A=3X^5+2X^4-X^2+1,\ B=X^3+X+2$
\item $A=X^4-X^3+X-2,\ B=X^2-2X+4$
\item $A=X^5-7X^4-X^2-9X+9,\ B=X^2-5X+4$
\end{enumerate}
\item Effectuer la division selon les puissances croissantes de $A$ 
par $B$ à l'ordre $k$ (c'est-à-dire tel que le reste soit divisible 
par $X^{k+1}$) :
\begin{enumerate}
\item $A=1-2X+X^3+X^4,\ B=1+2X+X^2,\ k=2$
\item $A=1+X^3-2X^4+X^6,\ B=1+X^2+X^3,\ k=4$
\end{enumerate}
\end{enumerate}
\finenonce{006955} 


\finexercice\exercice{6956, blanc-centi, 2014/04/01}
\video{J1GwUtSQ9D4}
\enonce{006956}{}
\`A quelle condition sur $a,b,c\in\Rr$ le polynôme 
$X^4+aX^2+bX+c$ est-il divisible par $X^2+X+1$ ?
\finenonce{006956} 


\finexercice\exercice{6957, exo7, 2014/04/01}
\video{DYun3S4_zgw}
% Mélange de 379 (bodin), 380 (bodin), 387 (cousquer)
\enonce{006957}{}
\begin{enumerate}
\item Déterminer les pgcd des polynômes suivants:
\begin{enumerate}
\item $X^3-X^2-X-2$ et $X^5-2X^4+X^2-X-2$
\item $X^4+X^3-2X+1$ et $X^3+X+1$
\item $X^5+3X^4+X^3+X^2+3X+1$ et $X^4+2X^3+X+2$
\item $nX^{n+1}-(n+1)X^n+1$ et $X^n-nX+n-1$ ($n\in\N^*)$
\end{enumerate}
\item Calculer le pgcd $D$ des polynômes $A$ et $B$ ci-dessous. 
Trouver des polynômes $U$ et $V$ tels que $AU+BV=D$.
\begin{enumerate}
\item $A=X^5+3X^4+2X^3-X^2-3X-2$\\ et $B=X^4+2X^3+2X^2+7X+6$
\item $A=X^6-2X^5+2X^4-3X^3+3X^2-2X$\\ et $B=X^4-2X^3+X^2-X+1$
\end{enumerate}
\end{enumerate}
\finenonce{006957} 


\finexercice\exercice{6958, blanc-centi, 2014/04/01}
\video{FljXmG2ie3Q}
\enonce{006958}{}

\
\begin{enumerate}
\item Montrer que si $A$ et $B$ sont deux polynômes à coefficients 
dans $\Q$, alors le quotient et le reste de la division 
euclidienne de $A$ par $B$, ainsi que $\pgcd(A,B)$, sont 
aussi à coefficients dans $\Qq$.

\item Soit $a,b,c\in\Cc^*$ distincts, et $0<p<q<r$ des entiers. 
Montrer que si $P(X)=(X-a)^p(X-b)^q(X-c)^r$ est à coefficients 
dans $\Qq$, alors $a,b,c \in \Qq$.
\end{enumerate}
\finenonce{006958} 



\section{Racines et factorisation}

\exercice{6959, exo7, 2014/04/01}
\video{aYIfJ9ze69o}
% Mélange 412 (cousquer), 423 (ridde)
\enonce{006959}{}
\begin{enumerate}
\item Factoriser dans $\Rr[X]$ et $\Cc[X]$ les polynômes suivants :
$$a)\ X^3-3\quad\quad b)\ X^{12}-1\quad\quad c)\ X^6+1\quad\quad d)\ X^9+X^6+X^3+1$$

\item Factoriser les polynômes suivants :
$$a)\ X^2+(3i-1)X-2-i\quad\quad b)\ X^3+(4+i)X^2+(5-2i)X+2-3i$$
\end{enumerate}
\finenonce{006959} 


\finexercice\exercice{410, cousquer, 2003/10/01}
\video{ywVn-MdHV0A}
\enonce{000410}{}
Pour quelles valeurs de $a$ le polynôme $(X+1)^7-X^7-a$ admet-il une racine multiple réelle?
\finenonce{000410} 


\finexercice
\exercice{370, cousquer, 2003/10/01}
\video{ndl3qcjN2vw} 
\enonce{000370}{}
Chercher tous les polynômes $P$ tels que $P+1$ soit divisible par $(X-1)^4$ et $P-1$ par $(X+1)^4$. 

\medskip
 

\emph{Indications.} 
Commencer par trouver une solution particulière $P_0$ avec l'une des méthode suivantes :
\begin{enumerate}
\item à partir de la relation de Bézout entre  $(X-1)^4$ et $(X+1)^4$;
\item en considérant le polynôme dérivé $P_0'$ et en cherchant un polynôme de degré minimal.
\end{enumerate}
Montrer que $P$ convient si et seulement si le polynôme $P-P_0$ est divisible par 
$(X-1)^4(X+1)^4$, et en déduire toutes les solutions du problème.
\finenonce{000370} 


\finexercice\exercice{378, gourio, 2001/09/01}
\video{ypZfDoacMUs}
\enonce{000378}{}
Quels sont les polynômes $P\in\Cc[X]$ tels que $P'$ divise $P$?
\finenonce{000378} 


\finexercice
\exercice{6960, blanc-centi, 2014/04/01}
\video{nC2gu_mKDR4}
\enonce{006960}{}
Trouver tous les polynômes $P$ qui vérifient la relation 
$$P(X^2)=P(X)P(X+1)$$
\finenonce{006960} 


\finexercice
\exercice{6961, blanc-centi, 2014/04/01}
\video{8EYEgxhsGFA}
\enonce{006961}{}
Soit $n\in\Nn$. Montrer qu'il existe un unique $P\in\Cc[X]$ tel que 
$$\forall z\in\Cc^* \qquad  P\left(z+\frac{1}{z}\right) = z^n+\frac{1}{z^n}$$

Montrer alors que toutes les racines de $P$ sont réelles, simples, 
et appartiennent à l'intervalle $[-2,2]$.
\finenonce{006961} 


\finexercice
\exercice{6962, blanc-centi, 2014/04/01}
\video{uMsrg-zPUko}
\enonce{006962}{}
\begin{enumerate}
\item Soit $P=X^n+a_{n-1}X^{n-1}+\cdots+a_1X+a_0$ un polynôme de degré $n\ge 1$ 
à coefficients dans $\Zz$. Démontrer que si $P$ admet une racine dans $\Zz$, 
alors celle-ci divise $a_0$.

\item Les polynômes $X^3-X^2-109X-11$ et $X^{10}+X^5+1$ ont-ils des racines dans $\Z$?
\end{enumerate}
\finenonce{006962} 


\finexercice
\exercice{6963, blanc-centi, 2014/04/01}
\video{Ez8FgWWwzVE}
\enonce{006963}{}

Soient $a_0,\ldots,a_n$ des réels deux à deux distincts.
 Pour tout $i=0,\ldots,n$, on pose
$$L_i(X)=\prod_{\substack{1\le j\le n \\ j\not= i}}\frac{X-a_j}{a_i-a_j}$$
(les $L_i$ sont appelés \emph{polynômes interpolateurs de Lagrange}).
Calculer $L_i(a_j)$.

Soient $b_0,\ldots,b_n$ des réels fixés. 
Montrer que $P(X)=\sum_{i=0}^nb_iL_i(X)$ est l'unique polynôme de degré inférieur ou égal à $n$ qui vérifie:
$$P(a_j)=b_j  \quad \text{ pour tout }j=0,\ldots,n.$$


\emph{Application.} Trouver le polynôme $P$ de degré inférieur ou égal à $3$ tel que 
$$P(0)=1\quad\text{et}\quad P(1)=0\quad\text{et}\quad P(-1)=-2\quad\text{et}\quad P(2)=4.$$
\finenonce{006963} 


\finexercice

% Old :
%\insertion{364, 366, 370, 371, 375, 378, 379, 380, 387, 401, 409, 410, 412, 423, 426, 427}

\finfiche 

 \finenonces 



 \finindications 

\noindication
\noindication
\noindication
\indication{006957}
Le calcul du pgcd se fait par l'algorithme d'Euclide, et la "remontée" 
de l'algorithme permet d'obtenir $U$ et $V$.
\finindication
\indication{006958}
Calculer $\pgcd(P,P')$.
\finindication
\noindication
\noindication
\noindication
\indication{000378}
Si $P=P'Q$ avec $P\not=0$, regarder le degré de $Q$.
\finindication
\indication{006960}
Montrer que si $P$ est un polynôme non constant vérifiant la relation, 
alors ses seules racines possibles sont $0$ et $1$.
\finindication
\indication{006961}
Pour l'existence, preuve par récurrence sur $n$. Pour les racines,
montrer que $P(x)=2\cos(n\Arccos(x/2))$.
\finindication
\noindication
\noindication


\newpage

\correction{000427}
On cherche $P$ sous la forme $P(X)=aX^3+bX^2+cX+d$, ce qui donne le système linéaire suivant à résoudre:
$$\left\{\begin{array}{rcrcrcrcr} &&&&&&d&=&1\\a&+&b&+&c&+&d&=&0\\-a&+&b&-&c&+&d&=&-2\\8a&+&4b&+&2c&+&d&=&4\end{array}\right.$$
Après calculs, on trouve une unique solution :
 $a=\frac{3}{2}$, $b=-2$, $c=-\frac{1}{2}$, $d=1$ c'est-à-dire 
 $$P(X)=\frac{3}{2}X^3-2X^2-\frac{1}{2}X+1.$$
\fincorrection
\correction{006955}
\
\begin{enumerate}
\item \begin{enumerate}
\item $3X^5+4X^2+1=(X^2+2X+3)(3X^3-6X^2+3X+16)-41X-47$
\item $3X^5+2X^4-X^2+1=(X^3+X+2)(3X^2+2X-3)-9X^2-X+7$
\item $X^4-X^3+X-2=(X^2-2X+4)(X^2+X-2)-7X+6$
\item $X^5-7X^4-X^2-9X+9$\\
\ \ \ $=(X^2-5X+4)(X^3-2X^2-14X-63)-268X+261$
\end{enumerate}
\item \begin{enumerate}
\item $1-2X+X^3+X^4=(1+2X+X^2)(1-4X+7X^2)+X^3(-9-6X)$
\item $1+X^3-2X^4+X^6=(1+X^2+X^3)(1-X^2-X^4)+X^5(1+2X+X^2)$
\end{enumerate}
\end{enumerate}
\fincorrection
\correction{006956}
La division euclidienne de $A=X^4+aX^2+bX+c$ par $B=X^2+X+1$ donne
$$X^4+aX^2+bX+c=(X^2+X+1)(X^2-X+a)+(b-a+1)X+c-a$$
Or $A$ est divisible par $B$ si et seulement si le reste 
$R=(b-a+1)X+c-a$ est le polynôme nul, 
c'est-à-dire si et seulement si $b-a+1=0$ et $c-a=0$.
\fincorrection
\correction{006957}
\
\begin{enumerate}
\item L'algorithme d'Euclide permet de calculer le pgcd par 
une suite de divisions euclidiennes.

\begin{enumerate}
\item $X^5-2X^4+X^2-X-2=(X^3-X^2-X-2)(X^2-X)+2X^2-3X-2$ 

puis $X^3-X^2-X-2=(2X^2-3X-2)(\frac{1}{2}X+\frac{1}{4})+\frac{3}{4}X-\frac{3}{2}$ 

puis $2X^2-3X-2=(\frac{3}{4}X-\frac{3}{2})(\frac{8}{3}X+\frac{4}{3})$


Le pgcd est le dernier reste non nul, 
divisé par son coefficient dominant: 
$$\pgcd(X^3-X^2-X-2,X^5-2X^4+X^2-X-2)=X-2$$


\item $X^4+X^3-2X+1=(X^3+X+1)(X+1)-X^2-4X$

puis $X^3+X+1=(-X^2-4X)(-X+4)+17X+1$

$$\begin{array}{rl}\text{donc}\ \pgcd&(X^4+X^3-2X+1,X^3+X+1)\\
&=\pgcd(-X^2-4X,17X+1)=1\end{array}$$

car $-X^2-4X$ et $17X+1$ n'ont pas de racine (même complexe) commune.


\item $X^5+3X^4+X^3+X^2+3X+1=(X^4+2X^3+X+2)(X+1)-X^3-1$

puis $X^4+2X^3+X+2=(-X^3-1)(-X-2)+2X^3+2$

$$\pgcd(X^5+3X^4+X^3+X^2+3X+1,X^4+2X^3+X+2)=X^3+1$$


\item $nX^{n+1}-(n+1)X^n+1$

\ \ \ \ \ \ \ \ \ \ \ $=(X^n-nX+n-1)(nX-(n+1))+n^2(X-1)^2$ 

Si $n=1$ alors $X^n-nX+n-1=0$ et le $\pgcd$ vaut $(X-1)^2$.
On constate que $1$ est racine de $X^n-nX+n-1$, 
et on trouve $X^n-nX+n-1=(X-1)(X^{n-1}+X^{n-2}+\cdots+X^2+X-(n-1))$.

Si $n\ge 2$: $1$ est racine de $X^{n-1}+X^{n-2}+\cdots+X^2+X-(n-1)$ et on trouve 

$X^{n-1}+X^{n-2}+\cdots+X^2+X-(n-1)$

\ \ \ \ \ \ \ $=(X-1)(X^{n-2}+2X^{n-3}+\cdots+(n-1)X^2+nX+(n+1))$, donc finalement 
$(X-1)^2$ divise $X^n-nX+n-1$ (on pourrait aussi remarquer 
que $1$ est racine de multiplicité au moins deux de 
$X^n-nX+n-1$, puisqu'il est racine de ce polynôme et de sa dérivée). 
Ainsi 
$$\text{si}\ n\ge 2,\ \pgcd(nX^{n+1}-(n+1)X^n+1,X^n-nX+n-1)=(X-1)^2$$


\end{enumerate}


\item \begin{enumerate}
\item $A=X^5+3X^4+2X^3-X^2-3X-2$ et $B=X^4+2X^3+2X^2+7X+6$

donc $A=BQ_1+R_1$ avec $Q_1=X+1$, $R_1=-2X^3-10X^2-16X-8$

puis $B=R_1Q_2+R_2$ avec $Q_2=-\frac{1}{2}X+\frac{3}{2}$ et $R_2=9X^2+27X+18$

et enfin $R_1=R_2Q_3$ avec $Q_3=-\frac{2}{9}X-\frac{4}{9}$

Donc $D=X^2+3X+2$, et on obtient
$$9D=B-R_1Q_2=B-(A-BQ_1)Q_2=-AQ_2+B(1+Q_1Q_2)$$
soit 
$$\left\{\begin{array}{l}U=\frac{1}{9}(-Q_2)=\frac{1}{18}X-\frac{1}{6}\\
V=\frac{1}{9}(1+Q_1Q2)=-\frac{1}{18}X^2+\frac{1}{9}X+\frac{5}{18}
\end{array}\right.$$

\item On a $A=BQ_1+R_1$ avec $Q_1=X^2+1$, $R_1=X^2-X-1$

puis $B=R_1Q_2+R_2$ avec $Q_2=X^2-X+1$ et $R_2=-X+2$

et enfin $R_1=R_2Q_3+R_3$ avec $Q_3=-X-1$ et $R_3=1$

Donc $D=1$, et on obtient
\begin{eqnarray*}
1&=&R_1-R_2Q_3=R_1-(B-R_1Q_2)Q_3=R_1(1+Q_2Q_3)-BQ_3\\
 &=&(A-BQ_1)(1+Q_2Q_3)-BQ_3\\
 &=&A(1+Q_2Q_3)-B(Q_1(1+Q_2Q_3)+Q_3)
\end{eqnarray*}
soit 
$$\left\{\begin{array}{l}
U=1+Q_2Q_3=-X^3\\
V=-Q_1(1+Q_2Q_3)-Q_3=1+X+X^3+X^5\end{array}\right.$$
\end{enumerate}
\end{enumerate}
\fincorrection
\correction{006958}
\
\begin{enumerate}
\item Lorsqu'on effectue la division euclidienne $A=BQ+R$, 
les coefficients de $Q$ sont obtenus par des opérations 
élémentaires (multiplication, division, addition) à partir 
des coefficients de $A$ et $B$ : ils restent donc dans $\Qq$. 
De plus, $R=A-BQ$ est alors encore à coefficients rationnels. 

Alors $\pgcd(A,B)=\pgcd(B,R)$ et pour l'obtenir, on fait 
la division euclidienne de $B$ par $R$ (dont le quotient 
et le reste sont encore à coefficients dans $\Qq$), puis on 
recommence... Le pgcd est le dernier reste non nul, c'est 
donc encore un polynôme à coefficients rationnels.

\item Notons $P_1=\pgcd(P,P')$: comme $P$ est à coefficients 
rationnels, $P'$ aussi et donc $P_1$ aussi. 
Or $P_1(X)=(X-a)^{p-1}(X-b)^{q-1}(X-c)^{r-1}$. En itérant 
le processus, on obtient que $P_{r-1}(X)=(X-c)$ est à 
coefficients rationnels, donc $c\in\Qq$.

On remonte alors les étapes: $P_{q-1}(X)=(X-b)(X-c)^{r-q+1}$ 
est à coefficients rationnels, et $X-b$ aussi en tant que 
quotient de $P_{q-1}$ par le polynôme à coefficients rationnels 
$(X-c)^{r-q+1}$, donc $b\in\Qq$. De même, en considérant 
$P_{p-1}$, on obtient $a\in\Qq$.
\end{enumerate}
\fincorrection\correction{006959}
\
\begin{enumerate}
\item \begin{enumerate}
\item $X^3-3=(X-3^{1/3})(X^2+3^{1/3}X+3^{2/3})$ où $X^2+3^{1/3}X+3^{2/3}$ 
est irréductible sur $\R$. On cherche ses racines complexes 
pour obtenir la factorisation sur $\Cc$ :
$$X^3-3=(X-3^{1/3})(X+\frac{1}{2}3^{1/3}-\frac{i}{2}3^{5/6})(X+\frac{1}{2}3^{1/3}+\frac{i}{2}3^{5/6})$$

\item Passons à $X^{12}-1$. $z=re^{i\theta}$ vérifie $z^{12}=1$ si et seulement 
si $r=1$ et $12\theta\equiv 0 [2\pi]$, on obtient donc comme racines 
complexes les $e^{ik\pi/6}$ ($k=0,\ldots,11$), 
parmi lesquelles il y en a deux réelles ($-1$ et $1$) et cinq couples de 
racines complexes conjuguées 
($e^{i\pi/6}$ et $e^{11i\pi/6}$, $e^{2i\pi/6}$ et $e^{10i\pi/6}$, 
$e^{3i\pi/6}$ et $e^{9i\pi/6}$, $e^{4i\pi/6}$ et $e^{8i\pi/6}$, 
$e^{5i\pi/6}$ et $e^{7i\pi/6}$), d'où la factorisation sur $\Cc[X]$:
$$\begin{array}{rl}
X^{12}-1=&(X-1)(X+1)(X-e^{i\pi/6})(X-e^{11i\pi/6})(X-e^{2i\pi/6})\\
         & \ \ (X-e^{10i\pi/6})(X-e^{3i\pi/6})(X-e^{9i\pi/6})(X-e^{4i\pi/6})\\
 & \ \ \ \ \ \ \ \ \ \ \ \ \ (X-e^{8i\pi/6})(X-e^{5i\pi/6})(X-e^{7i\pi/6})
\end{array}$$

Comme $(X-e^{i\theta})(X-e^{-i\theta})=(X^2-2\cos(\theta)X+1)$, on en déduit la factorisation dans $\Rr[X]$ :
$$\begin{array}{rl}
X^{12}-1&=(X-1)(X+1)(X^2-2\cos(\pi/6)X+1)\\ 
 &\ \ \ (X^2-2\cos(2\pi/6)X+1)(X^2-2\cos(3\pi/6)X+1)\\
 &\ \ \ \ \ \ \ \ (X^2-2\cos(4\pi/6)X+1)(X^2-2\cos(5\pi/6)X+1)\\
 &=(X-1)(X+1)(X^2-\sqrt{3}X+1)\\
 &\ \ \ \ \ \ (X^2-X+1)(X^2+1)(X^2+X+1)(X^2+\sqrt{3}X+1)
\end{array}$$

\item Pour $X^6+1$, $z=re^{i\theta}$ vérifie $z^{6}=-1$ si et seulement 
si $r=1$ et $6\theta\equiv \pi [2\pi]$, on obtient donc comme racines 
complexes les $e^{i(\pi+2k\pi)/6}$ ($k=0,\ldots,5$). D'où la factorisation dans $\Cc[X]$ :
$$\begin{array}{rl}
X^6+1 &=(X-e^{i\pi/6})(X-e^{3i\pi/6})(X-e^{5i\pi/6})(X-e^{7i\pi/6})\\
  &\ \ \ \ \ (X-e^{9i\pi/6})(X-e^{11i\pi/6})
\end{array}$$

Pour obtenir la factorisation dans $\Rr[X]$, on regroupe les paires de racines complexes conjuguées :
$$X^6+1=(X^2+1)(X^2-\sqrt{3}X+1)(X^2+\sqrt{3}X+1)$$

\item $X^9+X^6+X^3+1=P(X^3)$ où $P(X)=X^3+X^2+X+1=\frac{X^4-1}{X-1}$ : 
les racines de $P$ sont donc les trois racines quatrièmes de l'unité 
différentes de $1$ ($i$, $-i$, $-1$) et 
$$\begin{array}{rcl}
X^9+X^6+X^3+1&=&P(X^3)\\
 &=&(X^3+1)(X^3-i)(X^3+i)\\
 &=&(X^3+1)(X^6+1)
\end{array}$$
On sait déjà factoriser $X^6+1$, il reste donc à factoriser le polynôme 
$X^3+1=(X+1)(X^2-X+1)$, où $X^2-X+1$ n'a pas de racine réelle. Donc
$$\begin{array}{rl}
X^9+X^6+X^3+1&=(X+1)(X^2-X+1)(X^2+1)\\
 & \ \ \ \ \ (X^2-\sqrt{3}X+1)(X^2+\sqrt{3}X+1)
\end{array}$$

Pour la factorisation sur $\Cc$ : les racines de $X^2-X+1$ sont $e^{i\pi/3}$ et $e^{5i\pi/3}$, ce qui donne
$$\begin{array}{rl}
X^9+X^6+X^3+1&=(X+1)(X-e^{i\pi/3})(X-e^{5i\pi/3})\\
 &\ \ \ (X-e^{i\pi/6})(X-e^{3i\pi/6})(X-e^{5i\pi/6})\\
 &\ \ \ (X-e^{7i\pi/6})(X-e^{9i\pi/6})(X-e^{11i\pi/6})
\end{array}$$
\end{enumerate}

\item 
\begin{enumerate}
\item Pour $X^2+(3i-1)X-2-i$, on calcule le discriminant 
$$\Delta=(3i-1)^2-4(-2-i)=-2i$$ 
et on cherche les racines carrées (complexes!) de $\Delta$: $w=a+ib$ 
vérifie $w^2=\Delta$ si et seulement si $w=1-i$ ou $w=-1+i$. 
Les racines du polynômes sont donc $\frac{1}{2}(-(3i-1)\pm(1-i))$ 
et $P(X)=(X+i)(X-1+2i)$.

\item Pour $X^3+(4+i)X^2+(5-2i)X+2-3i$: $-1$ est racine évidente, et $P(X)=(X+1)(X^2+(3+i)X+2-3i)$. 
Le discriminant du polynôme $X^2+(3+i)X+2-3i$ vaut $\Delta=18i$, 
ses deux racines carrées complexes sont $\pm(3+3i)$ et finalement 
on obtient $P(X)=(X+1)(X-i)(X+3+2i)$.
\end{enumerate}
\end{enumerate}
\fincorrection
\correction{000410}
Soit $x\in\R$ ; $x$ est une racine multiple de $P$ si et seulement si $P(x)=0$ et $P'(x)=0$:
$$
\begin{array}{rcl}
P(x)=P'(x) 0 
&\iff& \left\{\begin{array}{l}(x+1)^7-x^7-a=0\\7(x+1)^6-7x^6=0\end{array}\right.\\
&\iff& \left\{\begin{array}{l}(x+1)x^6-x^7-a=0\qquad \text{ en utilisant la deuxième équation}\\(x+1)^6=x^6\end{array}\right.\\
&\iff& \left\{\begin{array}{l}x^6=a\\(x+1)^3=\pm x^3 \qquad \text{ en prenant la racine carrée} \end{array}\right.\\
&\iff& \left\{\begin{array}{l}x^6=a\\x+1=\pm x \qquad \text{ en prenant la racine cubique} \end{array}\right.\\
\end{array}$$
qui admet une solution ($x=-\frac{1}{2}$) si et seulement si $a=\frac{1}{64}$.
\fincorrection
\correction{000370}
\
\begin{enumerate}
\item On remarque que si $P$ est solution, alors $P+1=(X-1)^4A$ et par ailleurs $P-1=(X+1)^4B$, ce qui donne $1=\frac{A}{2}(X-1)^4+\frac{-B}{2}(X+1)^4$. Cherchons des polynômes $A$ et $B$ qui conviennent: pour cela, on écrit la relation de Bézout entre $(X-1)^4$ et $(X+1)^4$ qui sont premiers entre eux, et on obtient 
$$\frac{A}{2}=\frac{5}{32}X^3+\frac{5}{8}X^2+\frac{29}{32}X+\frac{1}{2}$$
$$\frac{-B}{2}=-\frac{5}{32}X^3+\frac{5}{8}X^2-\frac{29}{32}X+\frac{1}{2}$$
On a alors par construction
$$(X-1)^4A-1=2\bigg(1+(X+1)^4\frac{-B}{2}\bigg)=1+(X+1)^4B$$
et $P_0=(X-1)^4A-1$ convient. En remplaçant, on obtient après calculs :
$$P_0 = \frac{5}{16}X^7-\frac{21}{16}X^5+\frac{35}{16}X^3-\frac{35}{16}X$$


\item Si $(X-1)^4$ divise $P+1$, alors $1$ est racine de multiplicité au moins $4$
de $P+1$, et donc racine de multiplicité au moins $3$ de $P'$ : 
alors $(X-1)^3$ divise $P'$. De même $(X+1)^3$ divise $P'$. 
Comme  $(X-1)^3$ et  $(X+1)^3$ sont premiers entre eux, nécessairement $(X-1)^3(X+1)^3$ divise $P'$. 
Cherchons un polynôme de degré minimal : on remarque que les primitives de 
$$\lambda(X-1)^3(X+1)^3=\lambda(X^2-1)^3=\lambda(X^6-3X^4+3X^2-1)$$
sont de la forme $P(X)=\lambda(\frac{1}{7}X^7-\frac{3}{5}X^5+X^3-X+a)$. 
Si $P$ convient, nécessairement $1$ est racine de $P+1$ et $-1$ est racine de $P-1$, ce qui donne
$\lambda(\frac{-16}{35}+a)=-1$ et $\lambda(\frac{16}{35}+a)=1$. 
D'où $\lambda a=0$ et comme on cherche $P$ non nul, il faut $a=0$ et $\lambda=\frac{35}{16}$. 
On vérifie que 
$$P_0(X)=\frac{35}{16}(\frac{1}{7}X^7-\frac{3}{5}X^5+X^3-X)=\frac{5}{16}X^7-\frac{21}{16}X^5
+\frac{35}{16}X^3-\frac{35}{16}X$$
 est bien solution du problème: le polynôme $A=P_0+1$ admet $1$ comme racine, i.e. $A(1)=0$, 
 et sa dérivée admet $1$ comme racine triple donc $A'(1)=A''(1)=A'''(1)=0$, 
 ainsi $1$ est racine de multiplicité au moins $4$ de $A$ et donc $(X-1)^4$ divise $A=P+1$. 
 De même, $(X+1)^4$ divise $P-1$.
\end{enumerate}


Supposons que $P$ soit une solution du problème. On note toujours $P_0$ la solution particulière obtenue ci-dessus.
Alors $P+1$ et $P_0+1$ sont divisibles par $(X-1)^4$, et $P-1$ et $P_0-1$ sont divisibles par $(X+1)^4$. 
Ainsi $P-P_0=(P+1)-(P_0+1)=(P-1)-(P_0-1)$ est divisible par $(X-1)^4$ et par $(X+1)^4$. 
Comme $(X-1)^4$ et $(X+1)^4$ sont premiers entre eux, nécessairement $P-P_0$ est divisible 
par $(X-1)^4(X+1)^4$. Réciproquement, si $P=P_0+(X-1)^4(X+1)^4A$, alors $P+1$ est 
bien divisible par $(X-1)^4$ et $P-1$ est divisible par $(X+1)^4$. 

Ainsi les solutions sont exactement les polynômes de la forme
$$P_0(X)+(X-1)^4(X+1)^4A(X)$$
où $P_0$ est la solution particulière trouvée précédemment, et $A$ un polynôme quelconque.
\fincorrection
\correction{000378}
Le polynôme nul convient. Dans la suite on suppose que $P$ n'est pas le polynôme nul.

Notons $n = \deg P$ son degré. Comme $P'$ divise $P$, alors $P$ est non constant, 
donc $n\ge 1$. Soit $Q\in\Cc[X]$ tel que $P=P'Q$.
Puisque $\deg(P')=\deg(P)-1\ge 0$, alors $Q$ est de degré $1$.
Ainsi $Q(X)=aX+b$ avec $a\neq0$, et donc  
$$P(X)=P'(X)(aX+b)=aP'(X)(X+\frac{b}{a})$$
Donc si $z\not=\frac{-b}{a}$ et si $z$ est racine de $P$ de multiplicité $k \ge 1$, 
alors $z$ est aussi racine de $P'$ avec la même multiplicité, ce qui est impossible. 
Ainsi la seule racine possible de $P$ est $\frac{-b}{a}$.

Réciproquement, soit $P$ un polynôme avec une seule racine $z_0 \in \Cc$ : 
il existe $\lambda\not=0$, $n\ge 1$ tels que $P=\lambda(X-z_0)^n$, 
qui est bien divisible par son polynôme dérivé.
\fincorrection
\correction{006960}
Si $P$ est constant égal à $c$, il convient si et seulement si $c=c^2$,
et alors $c\in\{0;1\}$. 

Dans la suite on suppose $P$ non constant.
Notons $Z$ l'ensemble des racines de $P$. On sait que $Z$ est un ensemble non vide, fini.


\emph{Analyse}

Si $z\in Z$, alors $P(z)=0$ et la relation $P(X^2)=P(X)P(X+1)$ implique $P(z^2)=0$, donc $z^2\in Z$.
En itérant, on obtient $z^{2k}\in Z$ (pour tout $k\in\N^*$). 
Si $|z|>1$, la suite $(|z^{2k}|)_k$ est strictement croissante 
donc $Z$ contient une infinité d'éléments, ce qui est impossible. 
De même si $0<|z|<1$, la suite $(|z^{2k}|)_k$ est strictement décroissante, 
ce qui est impossible pour la même raison. 
Donc les éléments de $Z$ sont soit $0$, soit des nombres complexes de module $1$.

De plus, si $P(z)=0$, alors toujours par la relation $P(X^2)=P(X)P(X+1)$,
on a que $P((z-1)^2)=0$ donc $(z-1)^2\in Z$. Par le même raisonnement que précédemment,
alors ou bien $z-1=0$ ou bien $|z-1|=1$. 

En écrivant $z=a+ib$, on vérifie que $|z|=|z-1|=1$ équivaut à 
$z=e^{\pm i\pi/3}$. Finalement, $Z\subset\big\{0,1,e^{i\pi/3},e^{-i\pi/3}\big\}$. 
Or si $e^{\pm i\pi/3}$ était racine de $P$, alors $(e^{\pm i\pi/3})^2$ devrait aussi être dans $Z$, 
mais ce n'est aucun des quatre nombres complexes listés ci-dessus. 
Donc ni $e^{i\pi/3}$, ni $e^{-i\pi/3}$ ne sont dans $Z$. 
Les deux seules racines (complexes) possibles sont donc $0$ et $1$. 
Conclusion : le polynôme $P$ est nécessairement de la forme $\lambda X^k(X-1)^\ell$. 

\bigskip

\emph{Synthèse}

La condition $P(X^2)=P(X)P(X+1)$ devient
$$\lambda X^{2k}(X^2-1)^\ell=\lambda^2X^k(X-1)^\ell(X+1)^kX^\ell$$
qui équivaut à $\left\{\begin{array}{l}\lambda^2=\lambda\\2k=k+\ell\\k=\ell\end{array}\right.$.

Autrement dit $k=\ell$ et $\lambda=1$ (puisqu'on a supposé $P$ non constant). 

{\it Conclusion}
Finalement, les solutions sont le polynôme nul et les polynômes $(X^2-X)^k$, 
$k\in\Nn$ ($k=0$ donne le polynôme $1$).
\fincorrection
\correction{006961}
\begin{enumerate}
  \item Commençons par remarquer que si $P$ et $Q$ sont deux polynômes 
  qui conviennent, alors pour tout $z\in\Cc^*$, 
$P\left(z+\frac{1}{z}\right)-Q\left(z+\frac{1}{z}\right)=0$. 
En appliquant cette égalité à $z=e^{i\theta}$, on obtient 
$(P-Q)(2\cos\theta)=0$. Le polynôme $P-Q$ a une infinité 
de racines, donc il est nul, ce qui montre $P=Q$.
  
  \item Montrons l'existence de $P$ par récurrence forte sur $n$:
  \begin{itemize}
    \item Pour $n=0$, $P=2$ convient et pour $n=1$, $P=X$ convient.
    
    \item Passage des rangs $k\le n$ au rang $n+1$. 
    Si on note $P_k$ le polynôme construit pour $k\le n$, on a 
$$z^{n+1}+\frac{1}{z^{n+1}}=(z+\frac{1}{z})(z^n+\frac{1}{z^n})-(z^{n-1}+\frac{1}{z^{n-1}})
=(z+\frac{1}{z})P_n(z+\frac{1}{z})-P_{n-1}(z+\frac{1}{z})$$
donc $P_{n+1}(X)=XP_n(X)-P_{n-1}(X)$ convient.

    \item On a ainsi construit $P_n$ pour tout $n$ (avec $\deg P_n =n$). 
  \end{itemize}

  \item Fixons $n$ et notons $P$ le polynôme obtenu.
  Pour tout $\theta\in\R$, $P(e^{i\theta}+e^{-i\theta})=e^{in\theta}+e^{-in\theta}$ 
  donc $P(2\cos(\theta))=2\cos(n\theta)$. 
  
  En posant $x=2\cos(\theta)$ et donc $\theta = \Arccos(\frac x2)$ on obtient la relation
  Ainsi,
$$P(x)=2\cos(n\Arccos(\frac x2)) \qquad \forall x\in[-2,2]$$
Le polynôme dérivée est $P'(x)=\frac{n}{\sqrt{1-(\frac{x}{2})^2}}\sin(n\Arccos(\frac x2))$, 
il s'annule en changeant de signe en chaque 
$\alpha_k = 2\cos(\frac{k\pi}{n})$, ainsi $P'(\alpha_k)=0$ pour $k = 0,\ldots,n$.

On calcule aussi que $P(\alpha_k) = \pm 2$.
Le tableau de signe montre que $P$ est alternativement croissante 
(de $-2$ à $+2$) puis décroissante (de $+2$ à $-2$) 
sur chaque intervalle
$[\alpha_{k+1}, \alpha_k]$, qui forment une partition de $[-2,2]$.
D'après le théorème des valeurs intermédiaires, $P$ possède $n$ racines simples 
(une dans chaque intervalle $[\alpha_{k+1}, \alpha_k]$) dans $[-2,2]$. 
Puisque $P$ est de degré $n$, on a ainsi obtenu toutes ses racines.

\end{enumerate}
\fincorrection
\correction{006962}
\begin{enumerate}
\item Si $k\in\Zz$ est racine de $P$, alors $k^n+a_{n-1}k^{n-1}+\cdots+a_1k=-a_0$ 
ce qui donne $k(k^{n-1}+\cdots+a_1)=-a_0$, donc $k$ divise $a_0$.

\item Si $X^3-X^2-109X-11$ a une racine $k\in\Zz$, nécessairement $k$ divise $11$, 
donc $k$ vaut $-1$, $1$, $-11$ ou $11$. En testant ces quatre valeurs, on trouve 
que seul $11$ est racine. 

De même, si $X^{10}+X^5+1$ admettait une racine entière $k$, celle-ci diviserait $1$ donc vaut 
$k = \pm 1$, or on vérifie que ni $+1$, ni $-1$ ne sont racines.
Ainsi $X^{10}+X^5+1$ n'a pas de racine entière.
\end{enumerate}
\fincorrection
\correction{006963}
On a 
$$L_i(a_i)=\prod_{\substack{1\le j\le n \\ j\not= i}}\frac{a_i-a_j}{a_i-a_j}=1
\qquad \text{ et } \quad L_i(a_j)=0  \text{ si } j\not=i$$
puisque le produit contient un facteur qui est nul: $(a_j-a_j)$. 
Puisque les $L_i$ sont tous de degré $n$, le polynôme $P$ est de degré inférieur ou égal à $n$, et 
$P(a_j)=\sum_{i=0}^nb_iL_i(a_j)=b_i$. 

Il reste à montrer qu'un tel polynôme est unique. Supposons que $Q$ convienne aussi, 
alors $P-Q$ est de degré inférieur ou égal à 
$n$ et s'annule en $n+1$ points (les $a_i$), donc il est identiquement nul, i.e. $P=Q$.

\bigskip

Pour l'application on utilise utilise les polynômes interpolateurs de Lagrange avec
$a_0=0$, $b_0=1$ ; $a_1=1$, $b_1=0$ ; $a_2=-1$, $b_2=-2$ ; $a_3=2$, $b_3=4$. 
On sait qu'un tel polynôme $P(X)$ est unique et s'écrit 
$$P(X)=1\cdot L_0(X)+0\cdot L_1(X)-2\cdot L_2(X)+4L_3(X)$$
où

$$L_0(X)=\frac{(X-1)(X+1)(X-2)}{(0-1)(0+1)(0-2)}=\frac{1}{2}(X^3-2X^2-X+2)$$

$$L_1(X)=\frac{(X-0)(X+1)(X-2)}{(1-0)(1+1)(1-2)}=\frac{-1}{2}(X^3-X^2-2X)$$

$$L_2(X)=\frac{(X-0)(X-1)(X-2)}{(-1-0)(-1-1)(-1-2)}=\frac{-1}{6}(X^3-3X^2+2X)$$

$$L_3(X)=\frac{(X-0)(X-1)(X+1)}{(2-0)(2-1)(2+1)}=\frac{1}{6}(X^3-X)$$

Ainsi :
$$P(X)=\frac{3}{2}X^3-2X^2-\frac{1}{2}X+1.$$

\fincorrection


\end{document}

