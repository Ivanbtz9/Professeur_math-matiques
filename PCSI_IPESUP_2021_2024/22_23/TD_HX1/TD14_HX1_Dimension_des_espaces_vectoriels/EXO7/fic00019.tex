
%%%%%%%%%%%%%%%%%% PREAMBULE %%%%%%%%%%%%%%%%%%

\documentclass[11pt,a4paper]{article}

\usepackage{amsfonts,amsmath,amssymb,amsthm}
\usepackage[utf8]{inputenc}
\usepackage[T1]{fontenc}
\usepackage[francais]{babel}
\usepackage{mathptmx}
\usepackage{fancybox}
\usepackage{graphicx}
\usepackage{ifthen}

\usepackage{tikz}   

\usepackage{hyperref}
\hypersetup{colorlinks=true, linkcolor=blue, urlcolor=blue,
pdftitle={Exo7 - Exercices de mathématiques}, pdfauthor={Exo7}}

\usepackage{geometry}
\geometry{top=2cm, bottom=2cm, left=2cm, right=2cm}

%----- Ensembles : entiers, reels, complexes -----
\newcommand{\Nn}{\mathbb{N}} \newcommand{\N}{\mathbb{N}}
\newcommand{\Zz}{\mathbb{Z}} \newcommand{\Z}{\mathbb{Z}}
\newcommand{\Qq}{\mathbb{Q}} \newcommand{\Q}{\mathbb{Q}}
\newcommand{\Rr}{\mathbb{R}} \newcommand{\R}{\mathbb{R}}
\newcommand{\Cc}{\mathbb{C}} \newcommand{\C}{\mathbb{C}}
\newcommand{\Kk}{\mathbb{K}} \newcommand{\K}{\mathbb{K}}

%----- Modifications de symboles -----
\renewcommand{\epsilon}{\varepsilon}
\renewcommand{\Re}{\mathop{\mathrm{Re}}\nolimits}
\renewcommand{\Im}{\mathop{\mathrm{Im}}\nolimits}
\newcommand{\llbracket}{\left[\kern-0.15em\left[}
\newcommand{\rrbracket}{\right]\kern-0.15em\right]}
\renewcommand{\ge}{\geqslant} \renewcommand{\geq}{\geqslant}
\renewcommand{\le}{\leqslant} \renewcommand{\leq}{\leqslant}

%----- Fonctions usuelles -----
\newcommand{\ch}{\mathop{\mathrm{ch}}\nolimits}
\newcommand{\sh}{\mathop{\mathrm{sh}}\nolimits}
\renewcommand{\tanh}{\mathop{\mathrm{th}}\nolimits}
\newcommand{\cotan}{\mathop{\mathrm{cotan}}\nolimits}
\newcommand{\Arcsin}{\mathop{\mathrm{arcsin}}\nolimits}
\newcommand{\Arccos}{\mathop{\mathrm{arccos}}\nolimits}
\newcommand{\Arctan}{\mathop{\mathrm{arctan}}\nolimits}
\newcommand{\Argsh}{\mathop{\mathrm{argsh}}\nolimits}
\newcommand{\Argch}{\mathop{\mathrm{argch}}\nolimits}
\newcommand{\Argth}{\mathop{\mathrm{argth}}\nolimits}
\newcommand{\pgcd}{\mathop{\mathrm{pgcd}}\nolimits} 

%----- Structure des exercices ------

\newcommand{\exercice}[1]{\video{0}}
\newcommand{\finexercice}{}
\newcommand{\noindication}{}
\newcommand{\nocorrection}{}

\newcounter{exo}
\newcommand{\enonce}[2]{\refstepcounter{exo}\hypertarget{exo7:#1}{}\label{exo7:#1}{\bf Exercice \arabic{exo}}\ \  #2\vspace{1mm}\hrule\vspace{1mm}}

\newcommand{\finenonce}[1]{
\ifthenelse{\equal{\ref{ind7:#1}}{\ref{bidon}}\and\equal{\ref{cor7:#1}}{\ref{bidon}}}{}{\par{\footnotesize
\ifthenelse{\equal{\ref{ind7:#1}}{\ref{bidon}}}{}{\hyperlink{ind7:#1}{\texttt{Indication} $\blacktriangledown$}\qquad}
\ifthenelse{\equal{\ref{cor7:#1}}{\ref{bidon}}}{}{\hyperlink{cor7:#1}{\texttt{Correction} $\blacktriangledown$}}}}
\ifthenelse{\equal{\myvideo}{0}}{}{{\footnotesize\qquad\texttt{\href{http://www.youtube.com/watch?v=\myvideo}{Vidéo $\blacksquare$}}}}
\hfill{\scriptsize\texttt{[#1]}}\vspace{1mm}\hrule\vspace*{7mm}}

\newcommand{\indication}[1]{\hypertarget{ind7:#1}{}\label{ind7:#1}{\bf Indication pour \hyperlink{exo7:#1}{l'exercice \ref{exo7:#1} $\blacktriangle$}}\vspace{1mm}\hrule\vspace{1mm}}
\newcommand{\finindication}{\vspace{1mm}\hrule\vspace*{7mm}}
\newcommand{\correction}[1]{\hypertarget{cor7:#1}{}\label{cor7:#1}{\bf Correction de \hyperlink{exo7:#1}{l'exercice \ref{exo7:#1} $\blacktriangle$}}\vspace{1mm}\hrule\vspace{1mm}}
\newcommand{\fincorrection}{\vspace{1mm}\hrule\vspace*{7mm}}

\newcommand{\finenonces}{\newpage}
\newcommand{\finindications}{\newpage}


\newcommand{\fiche}[1]{} \newcommand{\finfiche}{}
%\newcommand{\titre}[1]{\centerline{\large \bf #1}}
\newcommand{\addcommand}[1]{}

% variable myvideo : 0 no video, otherwise youtube reference
\newcommand{\video}[1]{\def\myvideo{#1}}

%----- Presentation ------

\setlength{\parindent}{0cm}

\definecolor{myred}{rgb}{0.93,0.26,0}
\definecolor{myorange}{rgb}{0.97,0.58,0}
\definecolor{myyellow}{rgb}{1,0.86,0}

\newcommand{\LogoExoSept}[1]{  % input : echelle       %% NEW
{\usefont{U}{cmss}{bx}{n}
\begin{tikzpicture}[scale=0.1*#1,transform shape]
  \fill[color=myorange] (0,0)--(4,0)--(4,-4)--(0,-4)--cycle;
  \fill[color=myred] (0,0)--(0,3)--(-3,3)--(-3,0)--cycle;
  \fill[color=myyellow] (4,0)--(7,4)--(3,7)--(0,3)--cycle;
  \node[scale=5] at (3.5,3.5) {Exo7};
\end{tikzpicture}}
}


% titre
\newcommand{\titre}[1]{%
\vspace*{-4ex} \hfill \hspace*{1.5cm} \hypersetup{linkcolor=black, urlcolor=black} 
\href{http://exo7.emath.fr}{\LogoExoSept{3}} 
 \vspace*{-5.7ex}\newline 
\hypersetup{linkcolor=blue, urlcolor=blue}  {\Large \bf #1} \newline 
 \rule{12cm}{1mm} \vspace*{3ex}}

%----- Commandes supplementaires ------



\begin{document}

%%%%%%%%%%%%%%%%%% EXERCICES %%%%%%%%%%%%%%%%%%
\fiche{f00019, bodin, 2007/09/01} 

\titre{Espaces vectoriels de dimension finie}

\section{Base} 
\exercice{981, liousse, 2003/10/01}
\video{vlr-D0jhb-Q}
\enonce{000981}{}
\begin{enumerate}
\item Montrer que les vecteurs 
$v_1=(0,1,1)$, $v_2=(1,0,1)$ et $v_3=(1,1,0)$ forment une base de 
$\Rr^3$. Trouver les composantes du vecteur $w=(1,1,1)$ dans cette base $(v_1,v_2,v_3)$.

\item Montrer que les vecteurs 
$v_1=(1,1,1)$, $v_2=(-1,1,0)$ et $v_3=(1,0,-1)$ forment une base de 
$\Rr^3$. Trouver les composantes du vecteur $e_1=(1,0,0)$,
$e_2=(0,1,0)$, $e_3=(0,0,1)$ et $w=(1,2,-3)$ dans cette base $(v_1,v_2,v_3)$.

\item Dans $\Rr^3$, donner un exemple de famille libre qui n'est pas
g\'en\'eratrice.

\item Dans $\Rr^3$, donner un exemple de famille g\'en\'eratrice qui n'est pas libre.
\end{enumerate}
\finenonce{000981} 


\finexercice
\exercice{901, liousse, 2003/10/01}
\video{A-ieYnIs2KA}
\enonce{000901}{}
 Dans $\Rr^4$ on consid\`ere
l'ensemble $E$ des vecteurs $(x_1,x_2,x_3,x_4)$ v\'erifiant
$x_1+x_2+x_3+x_4=0$. L'ensemble $E$ est-il un sous-espace vectoriel de
$\Rr^4$ ? Si oui, en donner une base.
\finenonce{000901} 


\finexercice
\exercice{996, legall, 1998/09/01}
\video{Pmhg9aR10aA}
\enonce{000996}{}
D\'eterminer pour quelles valeurs de $ t\in {\Rr} $ les
vecteurs  
$$\big\{(1, 0, t),  (1, 1, t), (t,0,1)\big\}$$
forment une base de  $\Rr^3$.
\finenonce{000996} 


\finexercice
\exercice{1006, legall, 1998/09/01}
\video{G-Xa2ZmQUwQ}
\enonce{001006}{}
\begin{enumerate}
    \item Montrer que les vecteurs  $v_1 =(1,-1,i)$, $v_2=(-1,i,1)$, $v_3 =(i,1,-1)$ forment une base de $\Cc^3$.
    \item Calculer les coordonnées de $v = (1+i,1-i,i)$ dans cette base.
\end{enumerate}
\finenonce{001006} 


\finexercice
\exercice{992, cousquer, 2003/10/01}
\video{FZf39lWtpZw}
\enonce{000992}{}
\begin{enumerate}
  \item Soit $E=\Rr_n[X]$ l'espace vectoriel des polynômes de degré inférieur ou égal à $n$.
Montrer que toute famille de polyn\^omes $\{P_0,P_1,\ldots,P_n\}$ avec $\deg P_i = i$ 
(pour $i=0,1,\ldots,n$) forme une base de $E$.

\item  \'Ecrire le polyn\^ome $F=3X-X^2+8X^3$ sous la forme $F=a+b(1-X)+c(X-X^2)+d(X^2-X^3)$
($a,b,c,d \in \Rr$) puis sous la forme
$F=\alpha+\beta(1+X)+\gamma(1+X+X^2)+\delta(1+X+X^2+X^3)$
($\alpha,\beta,\gamma,\delta \in \Rr$).

\end{enumerate}
\finenonce{000992} 


\finexercice

\section{Dimension}
\exercice{1015, cousquer, 2003/10/01}
\video{dSfb2GXKgeU}
\enonce{001015}{}
Soit $E$ est un espace vectoriel de dimension finie et $F$ et $G$ deux
sous-espaces vectoriels de $E$. Montrer que : 
$$\dim(F+G) = \dim F+\dim G - \dim(F\cap G).$$
\finenonce{001015} 


\finexercice
\exercice{1019, legall, 1998/09/01}
\video{mHpyqFvkVLI}
\enonce{001019}{}
On consid\`ere, dans  ${\Rr}^4$, les vecteurs : 
$$v_1=(1, 2, 3, 4),\quad v_2= (1, 1, 1, 3),\quad v_3= (2, 1, 1, 1),\quad v_4=(-1, 0, -1, 2),\quad v_5=(2, 3, 0, 1).$$

Soit  $F$  l'espace vectoriel engendr\'e par  $\{v_1, v_2  ,  v_3\} $  et soit $G$  celui engendr\'e par  $\{v_4, v_5\}$.
Calculer les dimensions respectives de  $F$, $G$, $F\cap G$, $F+G$.
\finenonce{001019} 


\finexercice
\exercice{1016, legall, 1998/09/01}
\video{QdZN5i_YucI}
\enonce{001016}{}
Montrer que tout sous-espace vectoriel d'un espace vectoriel de dimension finie
est de dimension finie.
\finenonce{001016} 


\finexercice

\finfiche

 \finenonces 



 \finindications 

\indication{000981}
\^Etre une base, c'est être libre et génératrice.
Chacune de ces conditions se vérifie par un système linéaire.
\finindication
\indication{000901}
$E$ est un sous-espace vectoriel de $\Rr^4$. Une base comporte trois vecteurs.
\finindication
\indication{000996}
C'est une base pour $t\neq \pm 1$.
\finindication
\indication{001006}
Il n'y a aucune difficulté. C'est comme dans $\Rr^3$ sauf qu'ici les coefficients sont des nombres complexes.
\finindication
\indication{000992}
Il suffit de montrer que la famille est libre (pourquoi ?). 
Prendre ensuite une combinaison linéaire nulle et regarder le terme de plus haut degré.
\finindication
\indication{001015}
Partir d'une base $(e_1,\ldots, e_k)$ de $F\cap G$ et la compl\'eter par des vecteurs $(f_1,\ldots,f_\ell)$ en une base de $F$.
Repartir de $(e_1,\ldots, e_k)$ pour la compléter par  des vecteurs $(g_1,\ldots,g_m)$ en une base de $G$.
Montrer que $(e_1,\ldots, e_k,f_1,\ldots,f_\ell,g_1,\ldots,g_m)$ est une base de $F+G$.
\finindication
\indication{001019}
Calculer d'abord les dimensions de $F$ et $G$.
Pour celles de $F\cap G$ et $F+G$ servez-vous de la formule $\dim (F+G) = \dim F + \dim G - \dim (F\cap G)$.
\finindication
\indication{001016}
On peut utiliser des familles libres.
\finindication


\newpage

\correction{000981}
\begin{enumerate}
  \item Pour montrer que la famille $\{ v_1, v_2, v_3\}$ est une base nous allons 
montrer que cette famille est libre et génératrice.


  \begin{enumerate}
    \item Montrons que la famille $\{ v_1, v_2, v_3\}$ est libre.
Soit une combinaison linéaire nulle $a v_1+b v_2 + c v_3 = 0$, nous devons montrer qu'alors
les coefficients $a,b,c$ sont nuls. Ici le vecteur nul est $0=(0,0,0)$

\begin{align*}
  &  a v_1+b v_2 + c v_3 = (0,0,0) \\
\iff & a (0,1,1) + b(1,0,1) + c(1,1,0) = (0,0,0) \\
\iff & (b+c,a+c,a+b)=(0,0,0) \\
\iff & \begin{cases}
       b+c = 0 \\
       a+c = 0 \\
       a+b = 0  \\
       \end{cases} 
\iff \begin{cases}
       a = 0 \\
       b = 0 \\
       c = 0  \\
       \end{cases} \\
\end{align*}
Ainsi les coefficients vérifient $a=b=c=0$, cela prouve que la famille est libre.

    \item Montrons que la famille $\{ v_1, v_2, v_3\}$ est génératrice. Pour n'importe quel
vecteur $v=(x,y,z)$ de $\Rr^3$ on doit trouver $a,b,c\in\Rr$ tels que $a v_1+b v_2 + c v_3 = v$.
\begin{align*}
  &  a v_1+b v_2 + c v_3 = v \\
\iff & a (0,1,1) + b(1,0,1) + c(1,1,0) = (x,y,z) \\
\iff & (b+c,a+c,a+b)=(x,y,z) \\
\iff & \begin{cases}
       b+c = x \\
       a+c = y \quad (L_2) \\
       a+b = z \quad (L_3) \\
       \end{cases} 
\iff  \begin{cases}
       b+c = x \quad (L_1')\\
       a+c = y \\
       b-c = z-y  \quad(L_3')=(L_3-L_2)\\
       \end{cases} \\
\iff & \begin{cases}
       2b = x+z-y \quad (L_1'+L_3') \\
       a+c = y \\
       2c = x-(z-y) \quad  (L_1'-L_3')  \\
       \end{cases} 
\iff  \begin{cases}
       a = \frac12 (-x+y+z) \\
       b = \frac12 (x-y+z) \\
       c = \frac12 (x+y-z) \\
       \end{cases} \\
\end{align*}


Pour $a= \frac12 (-x+y+z)$, $b= \frac12 (x-y+z)$, $c = \frac12 (x+y-z)$ nous avons donc
la relation $av_1+bv_2+cv_3=(x,y,z)=v$. Donc la famille $\{ v_1, v_2, v_3\}$ est génératrice.


    \item La famille est libre et génératrice  donc c'est une base.

    \item Pour écrire  $w=(1,1,1)$ dans la base $(v_1,v_2,v_3)$  on peut résoudre le système correspondant
à la relation $a v_1+b v_2 + c v_3 = w$.
Mais en fait nous l'avons déjà résolu pour tout vecteur $(x,y,z)$, en particulier pour le vecteur
$(1,1,1)$ la solution est $a=\frac 12$, $b=\frac 12$, $c=\frac 12$.
Autrement dit $\frac 12 v_1+\frac 12 v_2 + \frac 12 v_3 = w$. Les coordonnées de $w$ dans la base $(v_1,v_2,v_3)$
sont donc $(\frac12,\frac12,\frac12)$.

  \end{enumerate} 


  \item Pour montrer que la famille est libre et génératrice les calculs 
sont similaires à ceux de la question précédente. Notons $\mathcal{B}$ la base $(v_1,v_2,v_3)$.

Exprimons ensuite $e_1$ dans cette base, les calculs donnent : $e_1 = \frac 13 v_1-\frac 13 v_2 +\frac 13 v_3$. 
Ses coordonnées dans la base $\mathcal{B}$ sont $(\frac13,-\frac13,\frac13)$.


$e_2 = \frac 13 v_1+\frac 23 v_2 +\frac 13 v_3$. 
Ses coordonnées dans $\mathcal{B}$ sont $(\frac13,\frac23,\frac13)$.

$e_3 = \frac 13 v_1-\frac 13 v_2 -\frac 23 v_3$. 
Ses coordonnées dans  $\mathcal{B}$ sont $(\frac13,-\frac13,-\frac23)$.


Les calculs sont ensuite terminés, on remarque que $w=(1,2,-3)$ vaut en fait
$w=e_1+2e_2-3e_3$ donc par nos calculs précédents $w=\frac 13 v_1-\frac 13 v_2 +\frac 13 v_3
+2(\frac 13 v_1+\frac 23 v_2 +\frac 13 v_3)-3(\frac 13 v_1-\frac 13 v_2 -\frac 23 v_3)
=  2v_2 +3 v_3$. Les coordonnées de $w$ dans  $\mathcal{B}$ sont $(0,2, 3)$.

  \item Par exemple la famille $\{(1,0,0),(0,1,0)\}$ est libre dans $\Rr^3$ mais pas g\'en\'eratrice.

  \item La famille  $\{(1,0,0),(0,1,0),(0,0,1),(1,1,1)\}$ est g\'en\'eratrice dans $\Rr^3$ mais pas libre.
\end{enumerate}
\fincorrection
\correction{000901}
\begin{enumerate}
\item On v\'erifie les propri\'et\'es qui font de $E$ un sous-espace
  vectoriel de $\Rr^4$ : 
  \begin{enumerate}
    \item l'origine $(0,0,0,0)$ est dans $E$,
    \item si $v=(x_1,x_2,x_3,x_4) \in E$ et $v'=(x_1',x_2',x_3',x_4')\in E$ alors $v+v'=(x_1+x_1',x_2+x_2',x_3+x_3',x_4+x_4')$
a des coordonnées qui vérifient l'équation et donc $v+v' \in E$.
    \item si $v=(x_1,x_2,x_3,x_4) \in E$ et $\lambda \in \Rr$ alors les coordonnées de 
$\lambda\cdot v = (\lambda x_1,\lambda x_2,\lambda x_3,\lambda x_4)$ vérifient 
l'équation et donc $\lambda \cdot v \in E$.
  \end{enumerate}

\item Il faut trouver une famille libre de vecteurs qui engendrent
  $E$. Comme $E$ est dans $\Rr^4$, il y aura moins de $4$ vecteurs
  dans cette famille. On prend un vecteur de $E$ (au hasard), par
  exemple $v_1 = (1,-1,0,0)$. Il est bien clair que $v_1$ n'engendre
  pas tout $E$, on cherche donc un vecteur $v_2$ lin\'eairement
  ind\'ependant de $v_1$, prenons $v_2 = (1,0,-1,0)$. Alors $\{v_1,v_2\}$
  n'engendrent pas tout $E$ ; par exemple $v_3 = (1,0,0,-1)$ est dans $E$
  mais n'est pas engendr\'e par $v_1$ et $v_2$. Montrons que
  $(v_1,v_2,v_3)$ est une base de $E$.
  \begin{enumerate}
  \item $(v_1,v_2,v_3)$ est une famille libre. En effet soient
    $\alpha,\beta,\gamma \in \Rr$ tels que $\alpha v_1+\beta
    v_2+\gamma v_3=0$. Nous obtenons donc :
\begin{align*}
& \alpha v_1+\beta v_2+\gamma v_3=0 \\
&\Rightarrow \alpha \begin{pmatrix}1 \cr -1 \cr 0 \cr 0 \cr
\end{pmatrix} + \beta
 \begin{pmatrix}1 \cr 0 \cr -1 \cr 0 \cr \end{pmatrix} + \gamma 
\begin{pmatrix}1 \cr 0 \cr 0 \cr -1 \cr \end{pmatrix}= 
\begin{pmatrix}0 \cr 0 \cr 0 \cr 0 \cr \end{pmatrix} \\
 &\Rightarrow \begin{cases}
  \alpha+\beta +\gamma &= 0 \\
  -\alpha &= 0 \\
  -\beta &=  0 \\
  -\gamma &=  0 \\
\end{cases}\\
 &\Rightarrow \alpha=0, \beta=0, \gamma = 0 \\
\end{align*}
Donc la famille est libre.

\item Montrons que la famille est g\'en\'eratrice : soit $v =
  (x_1,x_2,x_3,x_4) \in E$. Il faut \'ecrire $v$ comme combinaison
  lin\'eaire de $v_1,v_2,v_3$.  On peut r\'esoudre un syst\`eme comme
  ci-dessus (mais avec second membre) en cherchant
  $\alpha,\beta,\gamma$ tels que $\alpha v_1+\beta v_2+\gamma v_3=v$.
  On obtient que $v = -x_2v_1-x_3v_2-x_4v_4$ (on utilise
  $x_1+x_2+x_3+x_4=0$).
  \end{enumerate}
  
  Bien s\^ur vous pouvez choisir d'autres vecteurs de base (la seule
  chose qui reste ind\'ependante des choix est le nombre de vecteurs
  dans une base : ici $3$).
\end{enumerate}
\fincorrection
\correction{000996}

Quand le nombre de vecteurs égal la dimension de l'espace  nous avons les équivalences, entre 
\emph{être une famille libre} et \emph{être une famille génératrice} et donc aussi \emph{être une base}.

Trois vecteurs dans $\Rr^3$ forment donc une base si et seulement s'ils forment une famille libre.
Vérifions quand c'est le cas.


\begin{align*}
      & a (1, 0, t) + b (1, 1, t) + c (t,0,1) = (0,0,0) \\
\iff  & (a+b+tc,b,at+bt+c)=(0,0,0) \\
\iff  & \begin{cases}
         a+b+tc = 0 \\
         b = 0 \\
         at+bt+c = 0 \\
        \end{cases} 
\iff \begin{cases}
         b = 0 \\
         a+tc = 0 \\
         at+c = 0 \\
        \end{cases} \\
\iff & \begin{cases}
         b = 0 \\
         a=-tc \\
         (-tc)t+c = 0 \\
        \end{cases} 
\iff \begin{cases}
         b = 0 \\
         a=-tc \\
         (t^2-1)c = 0 \\
        \end{cases} \\
\end{align*}

Premier cas : si $t\neq \pm 1$. Alors $t^2-1 \neq 0$ et donc
la seule solution du système est $(a=0,b=0,c=0)$.
Dans ce cas la famille est libre et est donc aussi une base.


Deuxième cas : si $t=\pm1$. Alors la dernière ligne du système disparaît
et il existe des solutions non triviales (par exemple si $t=1$, $(a=1,b=0,c=-1)$ est une solution).
La famille n'est pas libre et n'est donc pas une base. 
\fincorrection
\correction{001006}
\begin{enumerate}
  \item C'est bien une base. Comme nous avons trois vecteurs et nous souhaitons 
montrer qu'ils forment un base d'un espace vectoriel de dimension $3$, 
il suffit de montrer que soit la famille est libre, soit elle est génératrice
(ces conditions sont équivalentes pour $n$ vecteurs dans un espace vectoriel de dimension $n$).

Il est plus simple de montrer que la famille est libre. Soit une combinaison linéaire nulle
$a v_1+b v_2+c v_3=0$  il faut montrer que $a=b=c=0$. Mais attention ici le corps de base est $K=\Cc$
donc $a,b,c$ sont des nombres complexes.

\begin{align*}
     &  a v_1+b v_2+c v_3=0 \\
\iff & a (1,-1,i) + b (-1,i,1) + c (i,1,-1) = (0,0,0) \\
\iff & (a-b+ic,-a+ib+c,ia+b-c)=(0,0,0) \\
\iff & \begin{cases}
        a-b+ic  = 0 \\
        -a+ib+c = 0  \\
        ia+b-c = 0 \\
       \end{cases} \\
\iff & \cdots \text{ on résout le système } \\
\iff & a=0, b=0, c=0 \\
\end{align*}

La famille $(v_1,v_2,v_3)$ est libre, donc aussi génératrice ; c'est donc une base de $\Cc^3$.




  \item On cherche $a,b,c \in \Cc$ tels que $a v_1+b v_2+c v_3=v$.
Il s'agit donc de r\'esoudre le syst\`eme :
$$ \begin{cases} 
a  -b +ic = 1+i \\ 
-a + ib  +c = 1-i \\ 
ia+b-c = i \\
\end{cases}$$
On trouve $a=0$, $b=\frac12(1-i)$, $c=\frac12(1-3i)$.
Nous avons donc $v = \frac12(1-i) v_2 + \frac12(1-3i) v_3$ et ainsi
 les coordonn\'ees de $v$ dans la base $(v_1,v_2,v_3)$ sont
$(0,\frac12(1-i),\frac12(1-3i))$.
\end{enumerate}
\fincorrection
\correction{000992}
\begin{enumerate}
  \item Tout d'abord la famille $\{P_0,P_1,P_2,\ldots,P_n\}$ contient $n+1$ vecteurs 
dans l'espace $E=\Rr_n[X]$ de dimension $n+1$. Ici un vecteur est un polynôme :
$P_0$ est un polynôme constant non nul, $P_1$ est un polynôme de degré exactement $1$,...
Rappelons que lorsque le nombre de vecteurs égal la dimension de l'espace  nous avons les équivalences, entre 
\emph{être une famille libre} et \emph{être une famille génératrice} et donc aussi \emph{être une base}.

\medskip

Nous allons donc montrer que $\{P_0,P_1,\ldots,P_n\}$ est une famille libre.
Soit une combinaison linéaire nulle :
$$\lambda_0 P_0+\lambda_1 P_1 + \cdots + \lambda_n P_n = 0.$$

Introduisons l'hypothèse concernant les degrés :
$\deg P_0 = 0$, $\deg P_1=1$, \ldots, $\deg P_n=n$.
Définissons le polynôme $P(X) = \lambda_0 P_0+\lambda_1 P_1 + \cdots + \lambda_n P_n$.

\medskip

Nous allons montrer successivement $\lambda_n=0$ puis $\lambda_{n-1}=0$,\ldots, $\lambda_0=0$.

Par l'absurde supposons $\lambda_n \neq 0$ et écrivons 
$P_n(X) = a_n X^n + a_{n-1}X^{n-1}+\cdots+a_1X+a_0$, comme
$\deg P_n(X)=n$ alors $a_n\neq0$.
Maintenant 
$P(X)$ est aussi un polynôme de degré exactement $n$ qui s'écrit
$$P(X) = \lambda_n \cdot a_n \cdot X^n + \text{termes de plus bas degré}$$
La combinaison linéaire nulle implique que $P(X)=0$ (le polynôme nul).
Donc en identifiant les coefficients devant $X^n$ on obtient $\lambda_n\cdot a_n=0$
On obtient  $a_n=0$ ou $\lambda_n=0$. Ce qui est une contradiction.
Conclusion $\lambda_n=0$.

\medskip

Maintenant la combinaison linéaire nulle s'écrit $\lambda_0 P_0+\lambda_1 P_1 + \cdots + \lambda_{n-1} P_{n-1}=0$.
Par récurrence descendante on trouve $\lambda_{n-1}=0$, \ldots, jusqu'à $\lambda_0=0$.

Bilan : $\lambda_0=0$, \ldots, $\lambda_n=0$ donc la famille $\{P_0,P_1,\ldots,P_n\}$ est libre,
elle donc aussi génératrice ; ainsi $\{P_0,P_1,\ldots,P_n\}$ est une base de $E=\Rr_n[X]$.

\medskip

Un point que nous avons utilisé et qu'il est peut-être utile de détailler est le suivant : 
si un polynôme égal le polynôme nul alors tous ces coefficients sont nul.

Voici une justification : écrivons $a_nX^n+ a_{n-1}X^{n-1}+\cdots+a_1X+a_0=0$
et divisons par $X^n$ :
$$a_n + \frac {a_{n-1}}{X} + \frac{a_{n-2}}{X^2} + \cdots + \frac{a_1}{X^{n-1}} + \frac{a_0}{X^n} = 0$$
Lorsque l'on fait tendre $X$ vers $+\infty$ alors le terme de gauche tend vers $a_n$ et celui de droite vaut $0$ donc 
par unicité de la limite $a_n=0$.
On fait ensuite une récurrence descendante pour prouver $a_{n-1}=0$,\ldots, $a_0=0$.

Une conséquence est que si deux polynômes sont égaux alors leurs coefficients sont égaux.
Et une autre formulation est de dire que $\{1,X,X^2,\ldots,X^n\}$ est une base de $\Rr_n[X]$.

  \item On trouve $a=10, b= -10, c = -7, d= -8$.
Puis $\alpha=-3,\beta=4,\gamma=-9,\delta=8$.
\end{enumerate}
\fincorrection
\correction{001015}
\begin{enumerate}
  \item
$F\cap G$ est un sous-espace vectoriel de $E$ donc est de dimension finie.
Soit $(e_1,\ldots, e_k)$ une base de $F\cap G$ avec $k=\dim F\cap G$.

$(e_1,\ldots, e_k)$ est une famille libre dans $F$ donc on peut la compl\'eter en une base de $F$ 
par le th\'eor\`eme de la base incompl\`ete.
Soient donc $(f_1,\ldots,f_\ell)$ des vecteurs de $F$ tels que 
$(e_1,\ldots, e_k,f_1,\ldots,f_\ell)$ soit une base de $F$.
Nous savons que $k+\ell = \dim F$.
Remarquons que les vecteurs $f_i$ sont dans $F\setminus G$ (car ils sont dans $F$ mais pas dans $F\cap G$).

Nous repartons de la famille $(e_1,\ldots, e_k)$ mais cette fois nous la compl\'etons en une base de $G$ : soit donc $(g_1,\ldots,g_m)$ des vecteurs de $G$ tels que $(e_1,\ldots, e_k,g_1,\ldots,g_m)$ soit une base de $G$.
Nous savons que $k+m = \dim G$.
Remarquons que cette fois les vecteurs $g_i$ sont dans $G\setminus F$.

\item Montrons que $\mathcal{B}=(e_1,\ldots, e_k,f_1,\ldots,f_\ell,g_1,\ldots,g_m)$ est une base de $F+G$.

C'est une famille g\'en\'eratrice car $F=\mathrm{Vect}(e_1,\ldots, e_k,f_1,\ldots,f_\ell) \subset \mathrm{Vect}(\mathcal{B})$ et $G=\mathrm{Vect}(e_1,\ldots, e_k,g_1,\ldots,g_m) \subset \mathrm{Vect}(\mathcal{B})$. Donc $F+G \subset  \mathrm{Vect}(\mathcal{B})$.

C'est une famille libre : en effet soit une combinaison lin\'eaire nulle
$$a_1e_1+\cdots+ a_ke_k\quad + \quad b_1f_1+\cdots +b_\ell f_\ell \quad + \quad c_1 g_1+\cdots +c_m g_m=0.$$
Notons $e=a_1e_1+\ldots +a_ke_k$, $f=b_1f_1+\cdots +b_\ell f_\ell$, $g=c_1 g_1+\cdots +c_m g_m$.
Donc la combinaison lin\'eaire devient :
$$e+f+g=0.$$
Donc $g=-e-f$, or $e$ et $f$ sont dans $F$ donc $g$ appartient \`a $F$.
Or les vecteurs $g_i$ ne sont pas dans $F$. Donc 
$g=c_1 g_1+\cdots+ c_m g_m$ est n\'ecessairement le vecteur nul.
Nous obtenons $c_1 g_1+\cdots+ c_m g_m=0$ c'est donc une combinaison lin\'eaire nulle pour la famille libre 
$(g_1,\ldots,g_m)$. Donc tous les coefficients $c_1,\ldots,c_m$ sont nuls.

Le reste de l'\'equation devient $a_1e_1+\cdots +a_ke_k+b_1f_1+\cdots +b_\ell f_\ell=0$,
or $(e_1,\ldots, e_k,f_1,\ldots,f_\ell)$ est une base de $F$ donc tous les coefficients 
$a_1,\ldots,a_k,b_1,\ldots,b_\ell$ sont nuls.

Bilan : tous les coefficients sont nuls donc la famille est libre. Comme elle \'etait g\'en\'eratrice, c'est une base.


  \item Puisque $\mathcal{B}$ est une base de $F+G$ alors la dimension de $F+G$ est le nombre de vecteurs de la base $\mathcal{B}$:
$$\dim (F+G) = k + \ell + m.$$
Or $k=\dim F\cap G$, $\ell = \dim F - k$, $m=\dim G - k$, donc 
$$\dim (F+G) = \dim F + \dim G - \dim (F\cap G).$$

\end{enumerate}
\fincorrection
\correction{001019}
\begin{enumerate}
  \item $G$  est engendr\'e par deux vecteurs donc $\dim G \leq 2$. Clairement  $v_4$ et  $v_5$  ne sont pas li\'es donc
$\dim G \geq 2$  c'est-\`a-dire  $\dim G=2$. 

  \item $F$  est engendr\'e par trois vecteurs donc  $\dim F \leq 3$.
Un calcul montre que la famille $\{ v_1,  v_2, v_3 \} $  est libre, 
d'où $\dim F\geq 3$  et donc $\dim F= 3$.

  \item Essayons d'abord d'estimer la dimension de $F\cap G$. 
D'une part $F\cap G \subset G$  donc  $\dim(F\cap G)\leq 2$.
Utilisons d'autre part la formule $\dim(F+G) =\dim F + \dim G - \dim(F\cap G)$. Comme  $F+G\subset  {\R}^4$, on a
$\dim(F+G)\leq 4$  d'o\`u on tire l'in\'egalit\' e  
$\dim(F\cap G) \ge 1$. Donc soit  $\dim(F\cap G)=1$
ou bien $\dim(F\cap G)=2$.

Supposons que  $\dim(F\cap G)$  soit \'egale \`a $2$. 
Comme  $F\cap G \subset G$  on aurait dans ce cas  $F\cap G =
G$ et donc $G\subset F$.  En particulier il existerait  $\alpha  , \beta , \gamma \in
{\R}$  tels que  $v_4=\alpha v_1 + \beta v_2 + \gamma v_3$. On v\'erifie ais\'ement que ce n'est pas le cas, 
ainsi $\dim (F\cap G)$  n'est pas \'egale \`a  $2$.
On peut donc conclure $\dim(F\cap G)=1$


  \item Par la formule $\dim (F+G) = \dim F + \dim G - \dim (F\cap G)$,
on obtient $\dim (F+G) = 2+3-1=4$. Cela entraîne $F+G=\Rr^4$.
\end{enumerate}
\fincorrection
\correction{001016}
Soit $E$ un espace vectoriel de dimension $n$ et $F$ un sous-espace vectoriel de $E$.
Par l'absurde supposons que $F$ ne soit pas de dimension finie, 
alors il existe $v_1,\ldots,v_{n+1}$, $n+1$ vecteurs de $F$ lin\'eairement ind\'ependants dans $F$.
Mais ils sont aussi lin\'eairement ind\'ependants dans $E$.
Donc la dimension de $E$ est au moins $n+1$. Contradiction.

\bigskip

Deux remarques :
\begin{itemize}
  \item En fait on a m\^eme  montré que la dimension de $F$ est plus petite que la dimension de $E$.
  \item On a utilisé le r\'esultat suivant : si $E$ admet une famille libre \`a $k$ \'el\'ements alors la dimension de $E$ 
est plus grande que $k$ (ou est infini). Ce r\'esultat est une cons\'equence imm\'ediate du th\'eor\`eme de la base incompl\`ete.
\end{itemize}
\fincorrection


\end{document}

