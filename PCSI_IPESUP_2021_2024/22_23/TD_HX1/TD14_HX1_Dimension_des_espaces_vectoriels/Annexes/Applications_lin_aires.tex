
 %Macros utilisées dans la base de données d'exercices 

\newcommand{\mtn}{\mathbb{N}}
\newcommand{\mtns}{\mathbb{N}^*}
\newcommand{\mtz}{\mathbb{Z}}
\newcommand{\mtr}{\mathbb{R}}
\newcommand{\mtk}{\mathbb{K}}
\newcommand{\mtq}{\mathbb{Q}}
\newcommand{\mtc}{\mathbb{C}}
\newcommand{\mch}{\mathcal{H}}
\newcommand{\mcp}{\mathcal{P}}
\newcommand{\mcb}{\mathcal{B}}
\newcommand{\mcl}{\mathcal{L}}
\newcommand{\mcm}{\mathcal{M}}
\newcommand{\mcc}{\mathcal{C}}
\newcommand{\mcmn}{\mathcal{M}}
\newcommand{\mcmnr}{\mathcal{M}_n(\mtr)}
\newcommand{\mcmnk}{\mathcal{M}_n(\mtk)}
\newcommand{\mcsn}{\mathcal{S}_n}
\newcommand{\mcs}{\mathcal{S}}
\newcommand{\mcd}{\mathcal{D}}
\newcommand{\mcsns}{\mathcal{S}_n^{++}}
\newcommand{\glnk}{GL_n(\mtk)}
\newcommand{\mnr}{\mathcal{M}_n(\mtr)}
\DeclareMathOperator{\ch}{ch}
\DeclareMathOperator{\sh}{sh}
\DeclareMathOperator{\vect}{vect}
\DeclareMathOperator{\card}{card}
\DeclareMathOperator{\comat}{comat}
\DeclareMathOperator{\imv}{Im}
\DeclareMathOperator{\rang}{rg}
\DeclareMathOperator{\Fr}{Fr}
\DeclareMathOperator{\diam}{diam}
\DeclareMathOperator{\supp}{supp}
\newcommand{\veps}{\varepsilon}
\newcommand{\mcu}{\mathcal{U}}
\newcommand{\mcun}{\mcu_n}
\newcommand{\dis}{\displaystyle}
\newcommand{\croouv}{[\![}
\newcommand{\crofer}{]\!]}
\newcommand{\rab}{\mathcal{R}(a,b)}
\newcommand{\pss}[2]{\langle #1,#2\rangle}
 %Document 

\begin{document} 

\begin{center}\textsc{{\huge }}\end{center}

% Exercice 875


\vskip0.3cm\noindent\textsc{Exercice 1} - Application linéaire donnée par l'image d'une base
\vskip0.2cm
Soit $E=\mathbb R^3$. On note ${\cal B}=\{e_1,e_2,e_3\}$ la base canonique de $E$ et $u$ l'endomorphisme de $\mathbb R^3$ défini par la donnée des images des vecteurs de la base : 
$$u(e_1) = -2e_1 +2e_3 \; , u(e_2)=3e_2 \; , u(e_3)=-4e_1 + 4e_3.$$ 
\begin{enumerate}
\item Déterminer une base de $\ker~u$. 
$u$ est-il injectif? peut-il être surjectif? Pourquoi?
\item Déterminer une base de $\textrm{Im}~u$. Quel est le rang de u ?
\item Montrer que $E=\ker~u\bigoplus \textrm{Im}~u$.
\end{enumerate}


% Exercice 892


\vskip0.3cm\noindent\textsc{Exercice 2} - Noyau prescrit?
\vskip0.2cm
Soit $E=\mathbb R^4$ et $F=\mathbb R^2$. On considère
$H=\{(x,y,z,t)\in\mathbb R^4;\ x=y=z=t\}$. Existe-t-il des applications linéaires de $E$ dans $F$
dont le noyau est $H$?


% Exercice 878


\vskip0.3cm\noindent\textsc{Exercice 3} - A noyau fixé
\vskip0.2cm
Soit $E$ le sous-espace vectoriel de $\mathbb R^3$ engendré par les vecteurs
$u=(1,0,0)$ et $v=(1,1,1)$. Trouver un endomorphisme $f$ de $\mathbb R^3$ dont le noyau est $E$.


% Exercice 899


\vskip0.3cm\noindent\textsc{Exercice 4} - Du local au global...
\vskip0.2cm
Soit $E$ un espace vectoriel de dimension finie et $f\in\mathcal L(E)$.
On suppose que, pour tout $x\in E$, il existe un entier $n_x\in\mathbb N$ tel
que $f^{n_x}(x)=0.$ Montrer qu'il existe un entier $n$ tel que $f^n=0$.


% Exercice 908


\vskip0.3cm\noindent\textsc{Exercice 5} - Suite exacte
\vskip0.2cm
Soient $E_0,\dots,E_n$ des espaces vectoriels de dimensions finies respectivement égales à $a_0,\dots,a_n$. 
On suppose qu'il existe $n$ applications linéaires $f_0,\dots,f_{n-1}$ telles que, pour chaque $k\in\{0,\dots,n-1\}$,
$f_k$ est une application linéaire de $E_k$ dans $E_{k+1}$ et 
\begin{enumerate}
\item $f_0$ est injective;
\item $\ker(f_k)=\textrm{Im}(f_{k-1})$ pour tout $k=1,\dots,n-1$;
\item $f_{n-1}$ est surjective.
\end{enumerate}
Prouver que $\sum_{k=0}^n (-1)^k a_k=0$.




\vskip0.5cm
\noindent{\small Cette feuille d'exercices a été conçue à l'aide du site \textsf{https://www.bibmath.net}}

%Vous avez accès aux corrigés de cette feuille par l'url : https://www.bibmath.net/ressources/justeunefeuille.php?id=27362
\end{document}