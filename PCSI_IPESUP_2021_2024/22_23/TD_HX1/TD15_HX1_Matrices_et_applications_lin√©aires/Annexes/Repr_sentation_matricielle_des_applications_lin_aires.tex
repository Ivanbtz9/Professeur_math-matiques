\documentclass[11pt]{article}

 %Configuration de la feuille 
 
\usepackage{amsmath,amssymb,enumerate,graphicx,pgf,tikz,fancyhdr}
\usepackage[utf8]{inputenc}
\usetikzlibrary{arrows}
\usepackage{geometry}
\usepackage{tabvar}
\geometry{hmargin=2.2cm,vmargin=1.5cm}\pagestyle{fancy}
\lfoot{\bfseries http://www.bibmath.net}
\rfoot{\bfseries\thepage}
\cfoot{}
\renewcommand{\footrulewidth}{0.5pt} %Filet en bas de page

 %Macros utilisées dans la base de données d'exercices 

\newcommand{\mtn}{\mathbb{N}}
\newcommand{\mtns}{\mathbb{N}^*}
\newcommand{\mtz}{\mathbb{Z}}
\newcommand{\mtr}{\mathbb{R}}
\newcommand{\mtk}{\mathbb{K}}
\newcommand{\mtq}{\mathbb{Q}}
\newcommand{\mtc}{\mathbb{C}}
\newcommand{\mch}{\mathcal{H}}
\newcommand{\mcp}{\mathcal{P}}
\newcommand{\mcb}{\mathcal{B}}
\newcommand{\mcl}{\mathcal{L}}
\newcommand{\mcm}{\mathcal{M}}
\newcommand{\mcc}{\mathcal{C}}
\newcommand{\mcmn}{\mathcal{M}}
\newcommand{\mcmnr}{\mathcal{M}_n(\mtr)}
\newcommand{\mcmnk}{\mathcal{M}_n(\mtk)}
\newcommand{\mcsn}{\mathcal{S}_n}
\newcommand{\mcs}{\mathcal{S}}
\newcommand{\mcd}{\mathcal{D}}
\newcommand{\mcsns}{\mathcal{S}_n^{++}}
\newcommand{\glnk}{GL_n(\mtk)}
\newcommand{\mnr}{\mathcal{M}_n(\mtr)}
\DeclareMathOperator{\ch}{ch}
\DeclareMathOperator{\sh}{sh}
\DeclareMathOperator{\vect}{vect}
\DeclareMathOperator{\card}{card}
\DeclareMathOperator{\comat}{comat}
\DeclareMathOperator{\imv}{Im}
\DeclareMathOperator{\rang}{rg}
\DeclareMathOperator{\Fr}{Fr}
\DeclareMathOperator{\diam}{diam}
\DeclareMathOperator{\supp}{supp}
\newcommand{\veps}{\varepsilon}
\newcommand{\mcu}{\mathcal{U}}
\newcommand{\mcun}{\mcu_n}
\newcommand{\dis}{\displaystyle}
\newcommand{\croouv}{[\![}
\newcommand{\crofer}{]\!]}
\newcommand{\rab}{\mathcal{R}(a,b)}
\newcommand{\pss}[2]{\langle #1,#2\rangle}
 %Document 

\begin{document} 

\begin{center}\textsc{{\huge }}\end{center}

% Exercice 3004


\vskip0.3cm\noindent\textsc{Exercice 1} - Matrices, produits et composition
\vskip0.2cm
Soient $S$ et $T$ les deux endomorphismes de $\mathbb R^2$ définis par

$$
S(x,y)=(2x-5y,\ -3x+4y)\quad\text{et}\quad T(x,y)=(-8y,\ 7x+y).
$$
\begin{enumerate}
\item Déterminer les matrices de $S$ et $T$ dans la base canonique de $\mathbb R^2$.
\item Déterminer les applications linéaires $S+T$, $S\circ T$, $T\circ S$ et $S\circ S$ ainsi que leurs matrices dans la base canonique de $\mathbb R^2$.
\end{enumerate}


% Exercice 882


\vskip0.3cm\noindent\textsc{Exercice 2} - Réduction
\vskip0.2cm
On considère l'endomorphisme $f$ de $\mathbb R^3$ dont la matrice 
dans la base canonique est :
$$M=\left(
\begin{array}{ccc}
1&1&-1\\
-3&-3&3\\
-2&-2&2
\end{array}\right).$$
Donner une base de $\ker(f)$ et de $\textrm{Im}(f)$. En déduire que $M^n=0$ pour tout $n\geq 2$.


% Exercice 886


\vskip0.3cm\noindent\textsc{Exercice 3} - Application linéaire définie sur les matrices
\vskip0.2cm
Soient $A=\left(\begin{array}{cc}-1&2\\1&0\end{array}\right)$
et $f$ l'application de $M_2(\mathbb R)$ dans $M_2(\mathbb R)$
définie par $f(M)=AM$.
\begin{enumerate}
\item Montrer que $f$ est linéaire.
\item Déterminer sa matrice dans la base canonique de $M_2(\mathbb R)$.
\end{enumerate}


% Exercice 885


\vskip0.3cm\noindent\textsc{Exercice 4} - Surjective?
\vskip0.2cm
Soient $\alpha,\beta$ deux réels et 
$$M_{\alpha,\beta}=\left(\begin{array}{cccc}
1&3&\alpha&\beta\\
2&-1&2&1\\
-1&1&2&0
\end{array}\right).$$
Déterminer les valeurs de $\alpha$ et $\beta$ pour lesquelles l'application linéaire associée à $M_{\alpha,\beta}$
est surjective.


% Exercice 942


\vskip0.3cm\noindent\textsc{Exercice 5} - Rang par blocs
\vskip0.2cm
Soit $B$ la matrice diagonale par blocs 
$$B=\left(
\begin{array}{cccc}
A_1&0&\dots&0\\
0&A_2&\ddots&\vdots\\
\vdots&\dots&\ddots&\vdots\\
0&\dots&0&A_n
\end{array}
\right).$$
Calculer le rang de $B$ en fonction du rang des $A_i$.





\vskip0.5cm
\noindent{\small Cette feuille d'exercices a été conçue à l'aide du site \textsf{https://www.bibmath.net}}

%Vous avez accès aux corrigés de cette feuille par l'url : https://www.bibmath.net/ressources/justeunefeuille.php?id=27579
\end{document}