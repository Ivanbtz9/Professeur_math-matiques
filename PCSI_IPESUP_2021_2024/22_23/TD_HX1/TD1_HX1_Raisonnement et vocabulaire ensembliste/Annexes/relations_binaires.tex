\documentclass[11pt]{article}

 %Configuration de la feuille 
 
\usepackage{amsmath,amssymb,enumerate,graphicx,pgf,tikz,fancyhdr}
\usepackage[utf8]{inputenc}
\usetikzlibrary{arrows}
\usepackage{geometry}
\usepackage{tabvar}
\geometry{hmargin=2.2cm,vmargin=1.5cm}\pagestyle{fancy}
\lfoot{\bfseries http://www.bibmath.net}
\rfoot{\bfseries\thepage}
\cfoot{}
\renewcommand{\footrulewidth}{0.5pt} %Filet en bas de page

 %Macros utilisées dans la base de données d'exercices 

\newcommand{\mtn}{\mathbb{N}}
\newcommand{\mtns}{\mathbb{N}^*}
\newcommand{\mtz}{\mathbb{Z}}
\newcommand{\mtr}{\mathbb{R}}
\newcommand{\mtk}{\mathbb{K}}
\newcommand{\mtq}{\mathbb{Q}}
\newcommand{\mtc}{\mathbb{C}}
\newcommand{\mch}{\mathcal{H}}
\newcommand{\mcp}{\mathcal{P}}
\newcommand{\mcb}{\mathcal{B}}
\newcommand{\mcl}{\mathcal{L}}
\newcommand{\mcm}{\mathcal{M}}
\newcommand{\mcc}{\mathcal{C}}
\newcommand{\mcmn}{\mathcal{M}}
\newcommand{\mcmnr}{\mathcal{M}_n(\mtr)}
\newcommand{\mcmnk}{\mathcal{M}_n(\mtk)}
\newcommand{\mcsn}{\mathcal{S}_n}
\newcommand{\mcs}{\mathcal{S}}
\newcommand{\mcd}{\mathcal{D}}
\newcommand{\mcsns}{\mathcal{S}_n^{++}}
\newcommand{\glnk}{GL_n(\mtk)}
\newcommand{\mnr}{\mathcal{M}_n(\mtr)}
\DeclareMathOperator{\ch}{ch}
\DeclareMathOperator{\sh}{sh}
\DeclareMathOperator{\vect}{vect}
\DeclareMathOperator{\card}{card}
\DeclareMathOperator{\comat}{comat}
\DeclareMathOperator{\imv}{Im}
\DeclareMathOperator{\rang}{rg}
\DeclareMathOperator{\Fr}{Fr}
\DeclareMathOperator{\diam}{diam}
\DeclareMathOperator{\supp}{supp}
\newcommand{\veps}{\varepsilon}
\newcommand{\mcu}{\mathcal{U}}
\newcommand{\mcun}{\mcu_n}
\newcommand{\dis}{\displaystyle}
\newcommand{\croouv}{[\![}
\newcommand{\crofer}{]\!]}
\newcommand{\rab}{\mathcal{R}(a,b)}
\newcommand{\pss}[2]{\langle #1,#2\rangle}
 %Document 

\begin{document} 

\begin{center}\textsc{{\huge relations binaires }}\end{center}

% Exercice 518


\vskip0.3cm\noindent\textsc{Exercice 1} - Nature des relations
\vskip0.2cm
Dire si les relations suivantes sont réflexives, symétriques, antisymétriques, transitives :
\begin{enumerate}
\item $E=\mathbb Z$ et $x\mathcal R y\iff x=-y$;
\item $E=\mathbb R$ et $x\mathcal R y\iff \cos^2 x+\sin^2 y=1$;
\item $E=\mathbb N$ et $x\mathcal R y\iff \exists p,q\geq 1,\ y=px^q$ ($p$ et $q$ sont des entiers).
\end{enumerate}
Quelles sont parmi les exemples précédents les relations d'ordre et les relations d'équivalence?


% Exercice 520


\vskip0.3cm\noindent\textsc{Exercice 2} - Relation d'équivalence et fonction
\vskip0.2cm
On définit sur $\mathbb R$ la relation $x\mathcal R y$ si et seulement si
$x^2-y^2=x-y$.
\begin{enumerate}
\item Montrer que $\mathcal R$ est une relation d'équivalence.
\item Calculer la classe d'équivalence d'un élément $x$ de $\mathbb R$.
Combien y-a-t-il d'éléments dans cette classe?
\end{enumerate}


% Exercice 2516


\vskip0.3cm\noindent\textsc{Exercice 3} - 
\vskip0.2cm
On munit  l'ensemble $E=\mathbb R^2$ de la relation $\cal R$ définie par 
$$(x, y)\ {\cal R}\ (x',y')\iff\exists a>0,\ \exists b>0\mid x'=ax{\rm \ et\ }y'=by.$$
\begin{enumerate}
\item Montrer que $\cal R$ est une relation d'équivalence.
\item Donner la classe d'équivalence des éléments $A=(1,0)$, $B=(0,-1)$ et $C=(1,1)$.
\item Déterminer les classes d'équivalence de $\mathcal{R}$.
\end{enumerate}


% Exercice 526


\vskip0.3cm\noindent\textsc{Exercice 4} - Ordre lexicographique
\vskip0.2cm
On définir sur $\mathbb R^2$ la relation $\prec$ par 
$$(x,y)\prec (x',y')\iff \big( (x<x')\textrm{ ou }(x=x'\textrm{ et }y\leq y')\big).$$
Démontrer que ceci définit une relation d'ordre sur $\mathbb R^2$. 


% Exercice 524


\vskip0.3cm\noindent\textsc{Exercice 5} - Une relation d'ordre sur les entiers
\vskip0.2cm
On définit la relation $\mathcal R$ sur $\mathbb N^*$ par $p\mathcal R q\iff \exists k\in\mathbb N^*,\ q=p^k$. Montrer que $\mathcal R$ définit un ordre partiel sur $\mathbb N^*$.
Déterminer les majorants de $\{2,3\}$ pour cet ordre.


% Exercice 2171


\vskip0.3cm\noindent\textsc{Exercice 6} - Pas d'élément maximal
\vskip0.2cm
Soit $E$ un ensemble ordonné. Démontrer que toute partie de $E$ admet un élément maximal si et seulement si toute suite croissante de $E$ est stationnaire.




\vskip0.5cm
\noindent{\small Cette feuille d'exercices a été conçue à l'aide du site \textsf{http://www.bibmath.net}}

%Vous avez accès aux corrigés de cette feuille par l'url : http://www.bibmath.net/ressources/justeunefeuille.php?id=24408
\end{document}