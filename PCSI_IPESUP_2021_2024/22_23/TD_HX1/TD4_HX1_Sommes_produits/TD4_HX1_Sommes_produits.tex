\documentclass[a4paper,11pt]{article}

\usepackage{inputenc}
\usepackage[T1]{fontenc}
\usepackage[frenchb]{babel}
\usepackage{fancyhdr,fancybox} % pour personnaliser les en-têtes
\usepackage{lastpage,setspace}
\usepackage{amsfonts,amssymb,amsmath,amsthm,mathrsfs}
\usepackage{relsize,exscale,bbold}
\usepackage{paralist}
\usepackage{xspace,multicol,diagbox,array}
\usepackage{xcolor}
\usepackage{variations}
\usepackage{xypic}
\usepackage{eurosym,stmaryrd}
\usepackage{graphicx}
\usepackage[np]{numprint}
\usepackage{hyperref} 
\usepackage{tikz}
\usepackage{colortbl}
\usepackage{multirow}
\usepackage{MnSymbol,wasysym}
\usepackage[top=1.5cm,bottom=1.5cm,right=1.2cm,left=1.25cm]{geometry}
\usetikzlibrary{calc, arrows, plotmarks, babel,decorations.pathreplacing}
\setstretch{1.25}
%\usepackage{lipsum} %\usepackage{enumitem} %\setlist[enumerate]{itemsep=1mm} bug avec enumerate



\newtheorem{thm}{Théorème}
\newtheorem{rmq}{Remarque}
\newtheorem{prop}{Propriété}
\newtheorem{cor}{Corollaire}
\newtheorem{lem}{Lemme}
\newtheorem{prop-def}{Propriété-définition}

\theoremstyle{definition}

\newtheorem{defi}{Définition}
\newtheorem{ex}{Exemple}
\newtheorem*{rap}{Rappel}
\newtheorem{cex}{Contre-exemple}
\newtheorem{exo}{Exercice} % \large {\fontfamily{ptm}\selectfont EXERCICE}
\newtheorem{nota}{Notation}
\newtheorem{ax}{Axiome}
\newtheorem{appl}{Application}
\newtheorem{csq}{Conséquence}
\def\di{\displaystyle}



\renewcommand{\thesection}{\Roman{section}}\renewcommand{\thesubsection}{\arabic{subsection} }\renewcommand{\thesubsubsection}{\alph{subsubsection} }


\newcommand{\bas}{~\backslash}\newcommand{\ba}{\backslash}
\newcommand{\C}{\mathbb{C}}\newcommand{\R}{\mathbb{R}}\newcommand{\Q}{\mathbb{Q}}\newcommand{\Z}{\mathbb{Z}}\newcommand{\N}{\mathbb{N}}\newcommand{\V}{\overrightarrow}\newcommand{\Cs}{\mathscr{C}}\newcommand{\Ps}{\mathscr{P}}\newcommand{\Rs}{\mathscr{R}}\newcommand{\Gs}{\mathscr{G}}\newcommand{\Ds}{\mathscr{D}}\newcommand{\happy}{\huge\smiley}\newcommand{\sad}{\huge\frownie}\newcommand{\danger}{\begin{tikzpicture}[x=1.5pt,y=1.5pt,rotate=-14.2]
	\definecolor{myred}{rgb}{1,0.215686,0}
	\draw[line width=0.1pt,fill=myred] (13.074200,4.937500)--(5.085940,14.085900)..controls (5.085940,14.085900) and (4.070310,15.429700)..(3.636720,13.773400)
	..controls (3.203130,12.113300) and (0.917969,2.382810)..(0.917969,2.382810)
	..controls (0.917969,2.382810) and (0.621094,0.992188)..(2.097660,1.359380)
	..controls (3.574220,1.726560) and (12.468800,3.984380)..(12.468800,3.984380)
	..controls (12.468800,3.984380) and (13.437500,4.132810)..(13.074200,4.937500)
	--cycle;
	\draw[line width=0.1pt,fill=white] (11.078100,5.511720)--(5.406250,11.875000)..controls (5.406250,11.875000) and (4.683590,12.812500)..(4.367190,11.648400)
	..controls (4.050780,10.488300) and (2.375000,3.675780)..(2.375000,3.675780)
	..controls (2.375000,3.675780) and (2.156250,2.703130)..(3.214840,2.964840)
	..controls (4.273440,3.230470) and (10.640600,4.847660)..(10.640600,4.847660)
	..controls (10.640600,4.847660) and (11.332000,4.953130)..(11.078100,5.511720)
	--cycle;
	\fill (6.144520,8.839900)..controls (6.460940,7.558590) and (6.464840,6.457090)..(6.152340,6.378910)
	..controls (5.835930,6.300840) and (5.320300,7.277400)..(5.003900,8.554750)
	..controls (4.683590,9.835940) and (4.679690,10.941400)..(4.996090,11.019600)
	..controls (5.312490,11.097700) and (5.824210,10.121100)..(6.144520,8.839900)
	--cycle;
	\fill (7.292960,5.261780)..controls (7.382800,4.898500) and (7.128900,4.523500)..(6.730460,4.421880)
	..controls (6.328120,4.324220) and (5.929680,4.535220)..(5.835930,4.898500)
	..controls (5.746080,5.261780) and (5.999990,5.640630)..(6.402340,5.738340)
	..controls (6.804690,5.839840) and (7.203110,5.625060)..(7.292960,5.261780)
	--cycle;
	\end{tikzpicture}}\newcommand{\alors}{\Large\Rightarrow}\newcommand{\equi}{\Leftrightarrow}
\newcommand{\fonction}[5]{\begin{array}{l|rcl}
		#1: & #2 & \longrightarrow & #3 \\
		& #4 & \longmapsto & #5 \end{array}}


\definecolor{vert}{RGB}{11,160,78}
\definecolor{rouge}{RGB}{255,120,120}
\definecolor{bleu}{RGB}{15,5,107}



\pagestyle{fancy}
\lhead{Groupe IPESUP}\chead{}\rhead{Année~2022-2023}\lfoot{M. Botcazou \& M.Dupré}\cfoot{\thepage/5}\rfoot{PCSI }\renewcommand{\headrulewidth}{0.4pt}\renewcommand{\footrulewidth}{0.4pt}


\begin{document}
 	
	

\noindent\shadowbox{
	\begin{minipage}{1\linewidth}
		\centering
		\huge{\textbf{ TD 4 :  Sommes et produits }}
	\end{minipage}
}
\bigskip

%https://www.bibmath.net/ressources/index.php?action=affiche&quoi=bde/algebre/calculalgebrique&type=fexo
\raggedright

\section*{Sommes et produits classiques:}\hfill\\%[-0.25cm]


\begin{minipage}{1\linewidth}
	\begin{minipage}[t]{0.48\linewidth}
		\raggedright
\subsection*{Exemples de sommes et de produits}		
\begin{exo}\textbf{(*)}\quad\\[0.2cm]
	Calculer les sommes suivantes pour $n\in\N$.
	\begin{multicols}{2}
		\begin{enumerate}
			\item$ \quad\mathlarger\sum\limits_{k=0}^{n}k(3k+1)$
			\item$ \quad\mathlarger\sum\limits_{k=0}^{n} \frac{2^{k-1}}{3^{k+1}}$
			\item$ \quad\mathlarger\sum\limits_{k=0}^{n}(-1)^kk$
			\item$ \quad\mathlarger\sum\limits_{k=0}^{n}k(k+1)(k-1)$
		\end{enumerate}
		  
	\end{multicols}
	
	\centering
	\rule{1\linewidth}{0.6pt}
\end{exo}

\begin{exo}\textbf{(*)}\quad\\[0.2cm]
	Simplifier les produits suivants:
	\begin{multicols}{2}
		\begin{enumerate}
			\item $\mathlarger\prod\limits_{k=1}^{n}\sqrt{k(k+1)}$
			\item$\mathlarger\prod\limits_{k=1}^{n}(-5)^{k^2-k}$
		\end{enumerate}
	\end{multicols}
	
	\centering
	\rule{1\linewidth}{0.6pt}
\end{exo}




\begin{exo}\textbf{(*)}\quad\\[0.2cm]
	Montrer en raisonnant par récurrence que\\ pour tout $ n\in\N^*$
	$$\sum\limits_{k=1}^{2n} \frac{(-1)^{k-1}}{k} \ = \ \sum\limits_{k=1}^{n} \frac{1}{n+k} $$ 
	
	\centering
	\rule{1\linewidth}{0.6pt}
\end{exo}

\begin{exo}\textbf{(*)}\quad\\[0.2cm]
	Calculer $\mathlarger\sum\limits_{k=2}^{n-1} \frac{3^k}{2^{2k-1}}$ pour tout     $n\in\N,~n\geq 3$\\ \quad\\
	
	En déduire:  
	$$\lim\limits_{n \rightarrow +\infty} ~\sum\limits_{k=2}^{n-1} \frac{3^k}{2^{2k-1}} $$
	
	\centering
	\rule{1\linewidth}{0.6pt}
\end{exo}
	




\begin{exo}\textbf{(*)}\quad\\[0.2cm]
	Soient $n \geq 1$ un entier et $a \in \C$. Calculer la somme et le
	produit des racines $n$-ièmes de $a$.
	
	
	\centering
	\rule{1\linewidth}{0.6pt}
\end{exo}



\end{minipage}	
\hfill\vrule\hfill
\begin{minipage}[t]{0.48\linewidth}
\raggedright

\begin{exo}\textbf{(*)}\quad\\[0.2cm]
	Soit $z$ un nombre complexe de module~$\rho$, d'argument~$\theta$. Calculer
	
	\centering$(z+\bar{z})(z^2+\bar{z}^2) \cdots (z^n+\bar{z}^n)
	$
	
	\raggedright en fonction de~$\rho$ et de~$\theta$.
	
	
	
	\centering
	\rule{1\linewidth}{0.6pt}
\end{exo}


\begin{exo}\textbf{(**)}\quad\\[0.2cm]
	%\begin{enumerate}
	%\item 
	Calculer le produit $\mathlarger\prod\limits_{k=1}^{n}\frac{2k-1}{2k}$.
	%\item Calculer le produit $\mathlarger\prod\limits_{k=1}^{n}\frac{1+k(k+1)+i}{1+k(k+1)-i}$.
	%\end{enumerate}	
	
	
	\centering
	\rule{1\linewidth}{0.6pt}
\end{exo}


\begin{exo}\textbf{(**)}\quad\\[0.2cm]
	\begin{enumerate}

		
		\item  Calculer les nombres suivants:
		
		 $\bullet \ \ 1,1111...=\lim_{n\rightarrow +\infty}1,\underbrace{111...1}_n$ 
		 
		$ \bullet \ \  0,9999...=\lim_{n\rightarrow +\infty}0,\underbrace{999...9}_n$
		
		\item Calculer $\underbrace{1-1+1-...+(-1)^{n-1}}_n$,
		$n\in\N^*$.
		

		\item Soient $n\in\N$ et $\theta\in\R$.
		
		 Calculer $\sum\limits_{k=0}^{n}\cos(k\theta)$ et
		$\sum\limits_{k=0}^{n}\sin(k\theta)$.
		
		\item  Pour $x\in[0,1]$ et $n\in\N^*$, on pose
		$$S_n=\sum_{k=1}^{n}(-1)^{k-1}\frac{x^k}{k}$$ Déterminer $\lim\limits_{n\rightarrow +\infty}S_n$.
		
		\item  On pose $u_0=1$ et, pour $n\in\N$, $u_{n+1}=2u_n-3$.
		\begin{enumerate}
			\item Calculer la suite $(u_n-3)_{n\in\N}$.
			\item Calculer $\sum_{k=0}^{n}u_k$.
		\end{enumerate}
	\end{enumerate}

	
	
	\centering
	\rule{1\linewidth}{0.6pt}
\end{exo}



\begin{exo}\textbf{(**)}\quad\\[0.2cm]
	 Soient $n \in \N^*$ et $x \in \ ]0 , \frac{\pi}{2}[$. Calculer la
	somme $\displaystyle \sum_{k=0}^{n-1} \cos^k (x) \cos(kx)$.
	
	
	\centering
	\rule{1\linewidth}{0.6pt}
\end{exo}



\end{minipage}
\end{minipage}

\raggedright

\begin{minipage}[t]{1\linewidth}
	\begin{minipage}[t]{0.48\linewidth}
		\raggedright
		
		\begin{exo}\textbf{(*)}\quad\\[0.2cm]
			Soit $n \in \mathbb{N}^{*}$,
			calculer $\displaystyle\sum_{k=0}^{n}{\sin^{3}(kx)}$.
			
			\centering
			\rule{1\linewidth}{0.6pt}
		\end{exo}
	
			\begin{exo}\textbf{(**)}\quad\\[0.2cm]
		Soit $n \in \mathbb{N}^{*}$, calculer $S_n=\sum\limits_{k=n}^{3n}\min(k,2n)$.
		
		\centering
		\rule{1\linewidth}{0.6pt}
	\end{exo}
		
		
		
		\begin{exo}\textbf{(**)}\quad\\[0.2cm]
			Soit $(a,b) \in \mathbb{R}^2$ avec $b \neq 0$. Calculer
			$$ C = \displaystyle\sum_{k=0}^{n}{\cosh(a+kb)} \text{ et } S = \displaystyle\sum_{k=0}^{n}{\sinh(a+kb)}$$
			
			\centering
			\rule{1\linewidth}{0.6pt}
		\end{exo}
	

			\begin{exo}\textbf{(***)}\quad\\[0.2cm]
		En utilisant la formule de la progression géométrique et la
		dérivation, calculer, pour $x$ réel et $n$ dans $\N^*$:
	
		\centering$\displaystyle\sum_{k=0}^{n}kx^k$
		
		\raggedright Pour $|x| < 1$, déterminer la limite de la somme
		précédente lorsque $n$ tend vers $+\infty$.
		
		\centering
		\rule{1\linewidth}{0.6pt}
	\end{exo}

		
		\subsection*{Coefficients binomiaux, binôme de Newton}	
		\begin{exo}\textbf{(*)}\quad\\[0.2cm]
			\begin{enumerate}
				\item Rappeler la formule du binôme de Newton et calculer $\sum_{k=0}^{n}\binom{n}{k}$
				\item Soit $n\geq1$ un entier. Exprimer les sommes suivantes $S_1 = \binom{n}{0}+\binom{n}{2}+\binom{n}{4}+...$ et $S_2 = \binom{n}{1}+\binom{n}{3}+\binom{n}{5}+...$ à l'aide du symbole $\sum$ et de la fonction partie entière.
				
				\noindent Vérifier que $S_1 = S_2 = 2^{n-1}$. 
			\end{enumerate}
			
			
			\centering
			\rule{1\linewidth}{0.6pt}
		\end{exo}
	
	\begin{exo}\textbf{(**)}\quad\\[0.2cm]
		
		Pour $n$ dans $\N$, $x$ dans $\R$, donner une expression simple
		de : $\displaystyle\sum\limits_{k=0}^{n}{ {n \choose k}\frac{x^{k+1}}{k+1}}$.
		
		
		\centering
		\rule{1\linewidth}{0.6pt}
	\end{exo}
	

		
		\begin{exo}\textbf{(***)}\quad\\[0.2cm]
			Calculer $\displaystyle\sum\limits_{k=0}^{n}{k^2 {n \choose k}}$.
			
			
			\centering
			\rule{1\linewidth}{0.6pt}
		\end{exo}
		
		
		
		
	\end{minipage}	
	\hfill\vrule\hfill
	\begin{minipage}[t]{0.48\linewidth}
	\raggedright
		
		
		
		\begin{exo}\textbf{(**)}\quad\\[0.2cm]
			Donner des expressions simples de:\\[0.2cm]
			
			\centering$\displaystyle\sum\limits_{k=0}^{n}{ {n \choose k}(-1)^k\cos(kx)}$, \quad$\displaystyle\sum\limits_{k=0}^{n}{ {n \choose k}(-1)^k\sin(kx)}$.
			
			
			\centering
			\rule{1\linewidth}{0.6pt}
		\end{exo}
		
		\begin{exo}\textbf{(***)}\quad\\[0.2cm]
			Calculer les sommes\\[0.4cm]
			
			\centering$\displaystyle\mathlarger\sum\limits_{\underset{k\equiv0[3]}{k=0}}^{n}{ {n \choose k}},\quad \mathlarger\sum\limits_{k=0}^{n}{ {n \choose 
					2k}(-1)^k},\quad \mathlarger\sum\limits_{k=0}^{n}{ {2n+1 \choose 2k+1}t^{2k+1}}.$
			
			
			\centering
			\rule{1\linewidth}{0.6pt}
		\end{exo}
		
		
		
		\begin{exo}\textbf{(**)}\quad\\[0.2cm]
			Montrer que $\binom{n}{0}^2 +\binom{n}{1}^2 + ... +\binom{n}{n}^2 =\binom{2n}{n}$
			
			 (utiliser le polynôme $(1+x)^{2n}$).
			
			\centering
			\rule{1\linewidth}{0.6pt}
		\end{exo}
	
		\begin{exo}\textbf{(**)}\quad\\[0.2cm]
			Calculer les sommes\quad  $\displaystyle\mathlarger\sum\limits_{\underset{k\equiv0[4]}{k=0}}^{n}{ {n \choose k}}$
			
			
			\centering
			\rule{1\linewidth}{0.6pt}
		\end{exo}
	
		\subsection*{Sommes et produits télescopiques}
		
	
		\begin{exo}\textbf{(*)}\quad\\[0.2cm]
		Calculer $\sum_{k=1}^n k\cdot k!$.
		
		
		
		\centering
		\rule{1\linewidth}{0.6pt}
	\end{exo}
	
		\begin{exo}\textbf{(**)}\quad\\[0.2cm]
				\begin{enumerate}
				\item (*) Calculer $\prod_{k=1}^{n}(1+\frac{1}{k})$, $n\in\N^*$.
				\item (***) Calculer $\prod_{k=1}^{n}\cos\frac{a}{2^k}$, $a\in]0,\pi[$, $n\in\N^*$.
			\end{enumerate}
			
			
			
			\centering
			\rule{1\linewidth}{0.6pt}
		\end{exo}
	

	
		
		\begin{exo}\textbf{(**)}\quad\\[0.2cm]
			
			Montrer que $\binom{p}{p}+\binom{p+1}{p}... +\binom{n}{p}=\binom{n+1}{p+1}$ où $0\leq p\leq n$. Donner un interprétation dans le triangle de \textsc{Pascal}~?
			
			
			\centering
			\rule{1\linewidth}{0.6pt}
		\end{exo}
	
	\begin{exo}\textbf{(**)}\quad\\[0.2cm]
		
		\begin{enumerate}
			\item Déterminer une suite $(u_k)$ telle que, pour tout $k\geq 0$, on ait
			
			\centering$u_{k+1}-u_k=(k+2) 2^k.$
			
			\raggedright
			\item En déduire $\sum_{k=0}^{n}(k+2)2^k.$
		\end{enumerate}
		
		
		\centering
		\rule{1\linewidth}{0.6pt}
	\end{exo}
		
	
		
	\end{minipage}
\end{minipage}






\begin{minipage}[c]{1\linewidth}
	\begin{minipage}[c]{0.48\linewidth}
		\raggedright
		
		
		
		
		\begin{exo}\textbf{(*)}\quad\\[0.2cm]
			\begin{enumerate}
				\item Si $n$ est dans $\N^*$, simplifier:
				
				\centering$\displaystyle\sum\limits_{k=1}^{n}{\ln\left(1+\dfrac{1}{k}\right)}$.
				
				\raggedright Quelle est la limite de cette expression lorsque $n$ tend vers $+\infty$?
				
				\item Si $n$ est un entier $n \geq 2$, simplifier :
				
				\centering$\displaystyle\sum\limits_{k=2}^{n}{\ln\left(1-\dfrac{1}{k^2}\right)}$.
				
				\raggedright Quelle est la limite de cette expression lorsque $n$ tend vers $+\infty$?
			\end{enumerate}
			
			
			\centering
			\rule{1\linewidth}{0.6pt}
		\end{exo}
		
		
		
		\begin{exo}\textbf{(*)}\quad\\[0.2cm]
			Déterminer trois réels $a, b, c$ tels que :\quad\\[0.25cm]
			
			\centering$\forall x\in\R\backslash\{0,-1,-2\},\ \dfrac{1}{x(x+1)(x+2)} = \dfrac{a}{x} + \dfrac{b}{x+1} + \dfrac{c}{x+2}$\quad\\[0.25cm]
			
			Donner pour $n$ dans $\N^*$, une expression simple de\quad\\[0.25cm]
			
			\centering$ U_n = \displaystyle\sum\limits_{k=1}^{n}{\dfrac{1}{k(k+1)(k+2)}}$.\quad\\[0.25cm]
			
			\raggedright Donner la limite de $(U_n)_{n\geq 1}$ lorsque $n$ tend vers $+\infty$?
			
			\centering
			\rule{1\linewidth}{0.6pt}
		\end{exo}
		
		\begin{exo}\textbf{(*)}\quad\\[0.2cm]
			
			\begin{enumerate}
				\item Pour $x\in\R$ étudier la quantité $x^3-(x-1)^3$ et retrouver l'expression simple de\quad\\[0.25cm]
				
				\centering$ \displaystyle\sum\limits_{k=1}^{n}{k^2}$.
				
				\raggedright
				\item Adapter cette méthode pour calculer :\quad\\[0.25cm]
				
				\centering$ \displaystyle\sum\limits_{k=1}^{n}{k^3}$.
				
				
			\end{enumerate}
			
			
			\centering
			\rule{1\linewidth}{0.6pt}
		\end{exo}
		
			\begin{exo}\textbf{(*)}\quad\\[0.2cm]
			
			Calculer les sommes suivantes 
			\begin{enumerate}
				\item $ \displaystyle\sum\limits_{k=0}^{n}{\dfrac{k}{(k+1)!}}$
				\item $ \displaystyle\sum\limits_{k=1}^{n}{\dfrac{1}{\sqrt{k+1} + \sqrt{k}}}$ 
			\end{enumerate}
		
			
			\centering
			\rule{1\linewidth}{0.6pt}
		\end{exo}
	
	
	\begin{exo}\textbf{(*)}\quad\\[0.2cm]
	
		\begin{enumerate}
			\item  Montrer pour $k\geq2$ que : $\dfrac{1}{k^2}\leq \dfrac{1}{k-1}- \dfrac{1}{k}$
			 \item En déduire que la suite  \ $ \left(\displaystyle\sum\limits_{k=1}^{n}{\dfrac{1}{k^2}}\right)_{n\geq1}$ converge.
		\end{enumerate}
		
		
		\centering
		\rule{1\linewidth}{0.6pt}
	\end{exo}
		
		
	\end{minipage}	
	\hfill\vrule\hfill
	\begin{minipage}[c]{0.48\linewidth}
		\raggedright
		
		\begin{exo}\textbf{(**)}\quad\\[0.2cm]
			
			Soit $x$ un nombre réel non multiple
			entier de $\pi$. En remarquant que :\quad\\[0.25cm]
			
			\centering
			
			$\forall y\in\R, \ \sin(2y)= 2\sin(y)\cos(y)$\quad\\[0.25cm]
			
		
			
			\raggedright simplifier, pour $n$ dans $\N^*$, le produit :\quad\\[0.25cm]
			
			\centering$ P_n(x) = \displaystyle\prod\limits_{k=1}^{n}{\cos\left(\dfrac{x}{2^k}\right)}$.\quad\\[0.25cm]
			
			\raggedright En utilisant, après l’avoir justifiée, la relation \quad\\[0.25cm]
			
			\centering$ \dfrac{\sin u}{u} \underset{u \rightarrow 0}{\longrightarrow} 1$\quad\\[0.25cm]
			
			\raggedright donner la limite de $P_n(x)$ lorsque $n$ tend vers $+\infty$.
			
			\centering
			\rule{1\linewidth}{0.6pt}
		\end{exo}
		
		
		
		
		
		\begin{exo}\textbf{(**)}\quad\\[0.2cm]
			Pour $n$ dans $\N^*$, soit:\quad\\[0.25cm]
			
			\centering
			
			$u_n= \sqrt{2+\sqrt{2+\sqrt{2+...+\sqrt{2}}}}$ \quad ($n$ radicaux)\quad\\[0.25cm]
			
			\begin{enumerate}
				\item Montrer : \quad $\forall n\in\N^*,$ \ $u_n = 2\cos\left(\dfrac{\pi}{2^{n+1}}\right)$
				\item Pour $n$ dans $\N^*$, on pose :\quad\\[0.25cm]
				
				\centering$ v_n = \displaystyle\prod\limits_{k=1}^{n}{u_k}$.\quad\\[0.25cm]
				
				\raggedright Montrer que: \quad\\[0.25cm]
				
				\centering$ \dfrac{v_n}{2^n} \underset{n \rightarrow +\infty}{\longrightarrow} \dfrac{2}{\pi}$\quad\\[0.35cm]
				
				\raggedright Cette formule a été découverte par Viète (1593), elle donne une expression de $\pi$ comme « produit infini ».
			\end{enumerate}
			
			\raggedright
			
	
			
			\centering
			\rule{1\linewidth}{0.6pt}
		\end{exo}
	
		\begin{exo}\textbf{(***)}\quad\\[0.2cm]
			Soient $(a_n)_{n\in\mathbb N}$ et $(B_n)_{n\in\mathbb N}$ deux suites de nombres complexes. On définit deux suites $(A_n)_{n\in\mathbb N}$ et $(b_n)_{n\in\mathbb N}$ en posant :
			$$A_n=\sum_{k=0}^n a_k,\quad\quad b_n=B_{n+1}-B_n.$$
			\begin{enumerate}
				\item Démontrer que $\sum_{k=0}^n a_kB_k=A_n B_n-\sum_{k=0}^{n-1}A_kb_k.$
				\item En déduire la valeur de $\sum_{k=0}^n 2^kk$.
			\end{enumerate}
			
			\centering
			\rule{1\linewidth}{0.6pt}
		\end{exo}
		
		
		
	\end{minipage}
\end{minipage}

\newpage
\section*{Sommes et produits doubles:}\hfill\\%[-0.25cm]
\begin{minipage}{1\linewidth}
	\begin{minipage}[c]{0.48\linewidth}
		\raggedright
		
				
		\begin{exo}\textbf{(*)}\quad\\[0.2cm]
			Calculer les sommes doubles suivantes :
			\begin{enumerate}
				\item $\sum_{1\leq i,j\leq n}ij$.
				\item $\sum_{1\leq i\leq j\leq n}\frac ij$.
			\end{enumerate}
			\centering
			\rule{1\linewidth}{0.6pt}
		\end{exo}
	
				
		\begin{exo}\textbf{(*)}\quad\\[0.2cm]
	Pour $n\in\mathbb N$, on note
	$$a_n=\sum_{k=1}^n k,\ b_n=\sum_{k=1}^n k^2\textrm{ et }c_n=\sum_{k=1}^n k^3.$$
	Pour cet exercice,
	
	 on admettra que $\displaystyle a_n=\frac{n(n+1)}2$,
	  que $\displaystyle b_n=\frac{n(n+1)(2n+1)}6$ et que $c_n=a_n^2$.\quad\\[0.2cm]
	\begin{enumerate}
		\item Calculer $\displaystyle \sum_{1\leq i\leq j\leq n} ij$.
		\item  Calculer $\displaystyle \sum_{i=1}^n\sum_{j=1}^n \min(i,j)$.
	\end{enumerate}
	
			\centering
			\rule{1\linewidth}{0.6pt}
		\end{exo}

				
		\begin{exo}\textbf{(**)}\quad\\[0.2cm]
			
			Calculer de deux manières différentes la somme $$\displaystyle\sum_{i=1}^{n}{\displaystyle\sum_{j = 1}^{i}{2^i}}$$
			En déduire la valeur de $\displaystyle\sum_{i=1}^{n}{i2^i}$
			
			\centering
			\rule{1\linewidth}{0.6pt}
		\end{exo}
		
		
		
		\begin{exo}\textbf{(**)}\quad\\[0.2cm]
			Calculer $$\displaystyle\sum_{1 \leq i,j \leq n}{\max(i,j)}$$
			
			\centering
			\rule{1\linewidth}{0.6pt}
		\end{exo}
		
		
		
		
		
	\end{minipage}	
	\hfill\vrule\hfill
	\begin{minipage}[c]{0.48\linewidth}
		\raggedright
		
		\begin{exo}\textbf{(**)}\quad\\[0.2cm]
			
			
			Calculer $$\displaystyle\sum_{k=0}^{n}{\displaystyle\sum_{l=0}^{n}{2^{2k-l}}}$$
			
			\centering
			\rule{1\linewidth}{0.6pt}
		\end{exo}
		
		\begin{exo}\textbf{(**)}\quad\\[0.2cm]
			Montrer, pour tout entier $n \geq 2$,
			$$\sum_{k=1}^{n-1}{H_k}=n H_n - n$$
			\centering
			\rule{1\linewidth}{0.6pt}
		\end{exo}

		
		
	\begin{exo}\textbf{(***)}\quad\\[0.2cm]
		
		Soit $x \in \mathbb{R}$ tel que $x \neq 0 [2 \pi]$ et $n \in \mathbb{N}^{*}$
		\begin{enumerate}
			\item Calculer et simplifier $D_n(x) = \displaystyle\sum_{k=-n}^{n}{e^{ikx}}$
			\item Calculer et simplifier $F_n(x) = \dfrac{1}{n} \displaystyle\sum_{k=0}^{n-1}{D_k(x)}$
		\end{enumerate}
		
		
		\centering
		\rule{1\linewidth}{0.6pt}
	\end{exo}

		\begin{exo}\textbf{(***)}\quad\\[0.2cm]
	\begin{enumerate}
		\item Soit $I_n=\int_{0}^{1}(1-x^2)^n\;dx$. Trouver une relation de récurrence liant $I_n$ et $I_{n+1}$ et en déduire
		$I_n$ en fonction de $n$ 
		
		(faire une intégration par parties dans $I_n-I_{n+1}$).
		\item Démontrer l'identité valable pour
		$n\geq1$~:~$1-\frac{\binom{n}{1}}{3}+\frac{\binom{n}{2}}{5}+...+(-1)^n\frac{\binom{n}{n}}{2n+1}
		=\frac{2.4.....(2n)}{1.3...(2n+1)}$.
	\end{enumerate}
	
	\centering
	\rule{1\linewidth}{0.6pt}
\end{exo}

	\begin{exo}\textbf{(****)}\quad\\[0.2cm]%exercice G du Mansuy
	
	%TODO : racines de l'unité, changement d'indice complexe
	Soit $n \in \mathbb{N}^{*}$, $\omega = e^{\frac{2 i \pi}{n}}$, on pose $Z = \displaystyle\sum_{k=0}^{n}{\omega^{k^2}}$. \\
	Calculer $|Z|^{2}$
	
	\centering
	\rule{1\linewidth}{0.6pt}
\end{exo}

		
		
		
	\end{minipage}
\end{minipage}

	

\end{document}