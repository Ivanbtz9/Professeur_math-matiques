\documentclass[11pt]{article}

 %Configuration de la feuille 
 
\usepackage{amsmath,amssymb,enumerate,graphicx,pgf,tikz,fancyhdr}
\usepackage{inputenc}
\usetikzlibrary{arrows}
\usepackage{geometry}
\usepackage{tabvar}
\geometry{hmargin=2.2cm,vmargin=1.5cm}\pagestyle{fancy}
\lfoot{\bfseries http://www.bibmath.net}
\rfoot{\bfseries\thepage}
\cfoot{}
\renewcommand{\footrulewidth}{0.5pt} %Filet en bas de page

 %Macros utilisées dans la base de données d'exercices 

\newcommand{\mtn}{\mathbb{N}}
\newcommand{\mtns}{\mathbb{N}^*}
\newcommand{\mtz}{\mathbb{Z}}
\newcommand{\mtr}{\mathbb{R}}
\newcommand{\mtk}{\mathbb{K}}
\newcommand{\mtq}{\mathbb{Q}}
\newcommand{\mtc}{\mathbb{C}}
\newcommand{\mch}{\mathcal{H}}
\newcommand{\mcp}{\mathcal{P}}
\newcommand{\mcb}{\mathcal{B}}
\newcommand{\mcl}{\mathcal{L}}
\newcommand{\mcm}{\mathcal{M}}
\newcommand{\mcc}{\mathcal{C}}
\newcommand{\mcmn}{\mathcal{M}}
\newcommand{\mcmnr}{\mathcal{M}_n(\mtr)}
\newcommand{\mcmnk}{\mathcal{M}_n(\mtk)}
\newcommand{\mcsn}{\mathcal{S}_n}
\newcommand{\mcs}{\mathcal{S}}
\newcommand{\mcd}{\mathcal{D}}
\newcommand{\mcsns}{\mathcal{S}_n^{++}}
\newcommand{\glnk}{GL_n(\mtk)}
\newcommand{\mnr}{\mathcal{M}_n(\mtr)}
\DeclareMathOperator{\ch}{ch}
\DeclareMathOperator{\sh}{sh}
\DeclareMathOperator{\vect}{vect}
\DeclareMathOperator{\card}{card}
\DeclareMathOperator{\comat}{comat}
\DeclareMathOperator{\imv}{Im}
\DeclareMathOperator{\rang}{rg}
\DeclareMathOperator{\Fr}{Fr}
\DeclareMathOperator{\diam}{diam}
\DeclareMathOperator{\supp}{supp}
\newcommand{\veps}{\varepsilon}
\newcommand{\mcu}{\mathcal{U}}
\newcommand{\mcun}{\mcu_n}
\newcommand{\dis}{\displaystyle}
\newcommand{\croouv}{[\![}
\newcommand{\crofer}{]\!]}
\newcommand{\rab}{\mathcal{R}(a,b)}
\newcommand{\pss}[2]{\langle #1,#2\rangle}
 %Document 

\begin{document} 

\begin{center}\textsc{{\huge sommes et produits}}\end{center}

% Exercice 3021


\vskip0.3cm\noindent\textsc{Exercice 1} - Somme télescopique et factorielle
\vskip0.2cm
En utilisant une somme télescopique, calculer $\sum_{k=1}^n k\cdot k!$.


% Exercice 3022


\vskip0.3cm\noindent\textsc{Exercice 2} - Transformer en somme télescopique
\vskip0.2cm
\begin{enumerate}
\item Déterminer une suite $(u_k)$ telle que, pour tout $k\geq 0$, on ait
$$u_{k+1}-u_k=(k+2) 2^k.$$
\item En déduire $\sum_{k=0}^{n}(k+2)2^k.$
\end{enumerate}


% Exercice 3043


\vskip0.3cm\noindent\textsc{Exercice 3} - Calcul de sommes par découpage
\vskip0.2cm
Soit $n\in\mathbb N$.
\begin{enumerate}
\item Calculer $A_n=\sum_{k=2n+1}^{3n}(2n)$.
\item Calculer $B_n=\sum_{k=n}^{2n}k$.
\item En déduire la valeur de $S_n=\sum_{k=n}^{3n}\min(k,2n)$.
\end{enumerate}


% Exercice 456


\vskip0.3cm\noindent\textsc{Exercice 4} - Sommation d'Abel 
\vskip0.2cm
Soient $(a_n)_{n\in\mathbb N}$ et $(B_n)_{n\in\mathbb N}$ deux suites de nombres complexes. On définit deux suites $(A_n)_{n\in\mathbb N}$ et $(b_n)_{n\in\mathbb N}$ en posant :
$$A_n=\sum_{k=0}^n a_k,\quad\quad b_n=B_{n+1}-B_n.$$
\begin{enumerate}
\item Démontrer que $\sum_{k=0}^n a_kB_k=A_n B_n-\sum_{k=0}^{n-1}A_kb_k.$
\item En déduire la valeur de $\sum_{k=0}^n 2^kk$.
\end{enumerate}


% Exercice 455


\vskip0.3cm\noindent\textsc{Exercice 5} - Somme de puissances
\vskip0.2cm
Soit $x\in\mathbb R$ et $n\in\mathbb N^*$.
\begin{enumerate}
\item Calculer $S_n(x)=\sum_{k=0}^n x^k.$
\item En déduire la valeur de $T_n(x)=\sum_{k=0}^n k x^k.$
\end{enumerate}



% Exercice 3024


\vskip0.3cm\noindent\textsc{Exercice 6} - Quelques sommes doubles
\vskip0.2cm
Calculer les sommes doubles suivantes :
\begin{enumerate}
\item $\sum_{1\leq i,j\leq n}ij$.
\item $\sum_{1\leq i\leq j\leq n}\frac ij$.
\end{enumerate}


% Exercice 454


\vskip0.3cm\noindent\textsc{Exercice 7} - Sommes doubles
\vskip0.2cm
Pour $n\in\mathbb N$, on note
$$a_n=\sum_{k=1}^n k,\ b_n=\sum_{k=1}^n k^2\textrm{ et }c_n=\sum_{k=1}^n k^3.$$
Pour cet exercice, on admettra que $\displaystyle a_n=\frac{n(n+1)}2$, que $\displaystyle b_n=\frac{n(n+1)(2n+1)}6$ et que $c_n=a_n^2$.
\begin{enumerate}
\item Calculer $\displaystyle \sum_{1\leq i\leq j\leq n} ij$.
\item  Calculer $\displaystyle \sum_{i=1}^n\sum_{j=1}^n \min(i,j)$.
\end{enumerate}


% Exercice 2603


\vskip0.3cm\noindent\textsc{Exercice 8} - Une somme à partir de la formule du binôme
\vskip0.2cm
L'objectif de l'exercice est de démontrer la (surprenante!) formule suivante :
$$\sum_{k=1}^n \binom nk\frac{(-1)^{k+1}}k=\sum_{k=1}^n\frac 1k.$$
\begin{enumerate}
\item Soit $x$ un réel non nul. Démontrer que 
$$\frac{1-(1-x)^n}{x}=\sum_{p=0}^{n-1}(1-x)^p.$$
\item On pose pour $x\in\mathbb R$, 
$$f(x)=\sum_{k=1}^n \binom nk \frac{(-1)^k}k x^k.$$
Démontrer que, pour $x\in\mathbb R$, on a  
$$f'(x)=-\sum_{p=0}^{n-1}(1-x)^p.$$
\item Conclure.
\end{enumerate}





\vskip0.5cm
\noindent{\small Cette feuille d'exercices a été conçue à l'aide du site \textsf{http://www.bibmath.net}}

%Vous avez accès aux corrigés de cette feuille par l'url : http://www.bibmath.net/ressources/justeunefeuille.php?id=24499
\end{document}