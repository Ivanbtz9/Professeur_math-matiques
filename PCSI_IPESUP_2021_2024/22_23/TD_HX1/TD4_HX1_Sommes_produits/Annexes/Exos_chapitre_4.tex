\documentclass[a4paper,10pt]{article}

 %Configuration de la feuille 
\usepackage[landscape]{geometry}
\usepackage{multicol}

\usepackage{amsmath,amssymb,enumerate,graphicx,pgf,tikz,fancyhdr}
\usepackage[utf8]{inputenc}
\usetikzlibrary{arrows}
\usepackage{geometry}
\usepackage{tabvar}
\geometry{hmargin=2.2cm,vmargin=1.5cm}\pagestyle{fancy}
\rfoot{\bfseries\thepage}
\cfoot{}
\renewcommand{\footrulewidth}{0.5pt} %Filet en bas de page

 %Macros utilisées dans la base de données d'exercices 

\newcommand{\mtn}{\mathbb{N}}
\newcommand{\mtns}{\mathbb{N}^*}
\newcommand{\mtz}{\mathbb{Z}}
\newcommand{\mtr}{\mathbb{R}}
\newcommand{\mtk}{\mathbb{K}}
\newcommand{\mtq}{\mathbb{Q}}
\newcommand{\mtc}{\mathbb{C}}
\newcommand{\mch}{\mathcal{H}}
\newcommand{\mcp}{\mathcal{P}}
\newcommand{\mcb}{\mathcal{B}}
\newcommand{\mcl}{\mathcal{L}}
\newcommand{\mcm}{\mathcal{M}}
\newcommand{\mcc}{\mathcal{C}}
\newcommand{\mcmn}{\mathcal{M}}
\newcommand{\mcmnr}{\mathcal{M}_n(\mtr)}
\newcommand{\mcmnk}{\mathcal{M}_n(\mtk)}
\newcommand{\mcsn}{\mathcal{S}_n}
\newcommand{\mcs}{\mathcal{S}}
\newcommand{\mcd}{\mathcal{D}}
\newcommand{\mcsns}{\mathcal{S}_n^{++}}
\newcommand{\glnk}{GL_n(\mtk)}
\newcommand{\mnr}{\mathcal{M}_n(\mtr)}
\DeclareMathOperator{\ch}{ch}
\DeclareMathOperator{\sh}{sh}
\DeclareMathOperator{\vect}{vect}
\DeclareMathOperator{\card}{card}
\DeclareMathOperator{\comat}{comat}
\DeclareMathOperator{\imv}{Im}
\DeclareMathOperator{\rang}{rg}
\DeclareMathOperator{\Fr}{Fr}
\DeclareMathOperator{\diam}{diam}
\DeclareMathOperator{\supp}{supp}
\newcommand{\veps}{\varepsilon}
\newcommand{\mcu}{\mathcal{U}}
\newcommand{\mcun}{\mcu_n}
\newcommand{\dis}{\displaystyle}
\newcommand{\croouv}{[\![}
\newcommand{\crofer}{]\!]}
\newcommand{\rab}{\mathcal{R}(a,b)}
\newcommand{\pss}[2]{\langle #1,#2\rangle}
 %Document 

\begin{document} 
\begin{multicols}{2}

\begin{center}\textsc{{\huge Chapitre 3 : somme et produits}}\end{center}

\lhead{Groupe IPESUP}\chead{}\rhead{Année~2022-2023}\lfoot{M.Botcazou \& M.Dupré}\cfoot{\thepage/3}\rfoot{\textbf{Tournez la page S.V.P.}}\renewcommand{\headrulewidth}{0.4pt}\renewcommand{\footrulewidth}{0.4pt}




\textcolor{blue}{\large{\bf Exercice 3.}} (**)
%TODO : système, ch,sh, équation deuxième degré
\begin{enumerate}
\item Soit $(a,b) \in \mathbb{R}^2$ avec $b \neq 0$. Calculer
$$ C = \displaystyle\sum_{k=0}^{n}{\cosh(a+kb)} \text{ et } S = \displaystyle\sum_{k=0}^{n}{\sinh(a+kb)}$$
\end{enumerate}





\subsection*{Coefficients binomiaux, binôme de Newton}

\textcolor{blue}{\large{\bf Exercice 3.}} (**)
%TODO : coeff binom
Calculer $\displaystyle\sum_{k=0}^{n}{k^2 {n \choose k}}$


\textcolor{blue}{\large{\bf Exercice 3.}} (**)
Déterminer l'abscisse du premier maximum local sur $\mathbb{R}_{+}^{*}$ de $f_n : x \longmapsto \displaystyle\sum_{k=1}^{n}{\dfrac{\sin(kx)}{k}}$


\subsection*{Changement d'indice, interversion de sommes}

\textcolor{blue}{\large{\bf Exercice 3.}} (*)
%TODO :
Soit $n \in \mathbb{N}^{*}$
Calculer $\displaystyle\sum_{k=0}^{n}{\sin^{3}(kx)}$

\textcolor{blue}{\large{\bf Exercice 3.}} (**)
%TODO : somme géométrique, interversion de somme
Calculer de deux manières différentes la somme $$\displaystyle\sum_{i=1}^{n}{\displaystyle\sum_{j = 1}^{i}{2^i}}$$
En déduire la valeur de $\displaystyle\sum_{i=1}^{n}{i2^i}$

\textcolor{blue}{\large{\bf Exercice 3.}} (**)
%TODO : 
Calculer $$\displaystyle\sum_{1 \leq i,j \leq n}{\max(i,j)}$$

\textcolor{blue}{\large{\bf Exercice 3.}} (***)
%TODO : complexes
Soit $x \in \mathbb{R}$ tel que $x \neq 0 [2 \pi]$ et $n \in \mathbb{N}^{*}$
\begin{enumerate}
\item Calculer et simplifier $D_n(x) = \displaystyle\sum_{k=-n}^{n}{e^{ikx}}$
\item Calculer et simplifier $F_n(x) = \dfrac{1}{n} \displaystyle\sum_{k=0}^{n-1}{D_k(x)}$
\end{enumerate}

\textcolor{blue}{\large{\bf Exercice 3.}} (**)
%TODO : somme géométrique, double somme
\begin{enumerate}
\item Calculer $$\displaystyle\sum_{k=0}^{n}{\displaystyle\sum_{l=0}^{n}{2^{2k-l}}}$$
\end{enumerate}

\textcolor{blue}{\large{\bf Exercice 3.}} (****)
%TODO : racines de l'unité, changement d'indice complexe
Soit $n \in \mathbb{N}^{*}$, $\omega = e^{\frac{2 i \pi}{n}}$, on pose $Z = \displaystyle\sum_{k=0}^{n}{\omega^{k^2}}$. \\
Calculer $|Z|^{2}$

\end{multicols}
\end{document}