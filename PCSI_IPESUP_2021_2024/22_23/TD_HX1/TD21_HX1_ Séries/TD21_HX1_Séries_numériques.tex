\documentclass[a4paper,11pt]{article}

\usepackage{inputenc}
\usepackage[T1]{fontenc}
\usepackage[frenchb]{babel}
\usepackage{fancyhdr,fancybox} % pour personnaliser les en-têtes
\usepackage{lastpage,setspace}
\usepackage{amsfonts,amssymb,amsmath,amsthm,mathrsfs}
\usepackage{mathdots}
\usepackage{relsize,exscale,bbold}
\usepackage{paralist}
\usepackage{xspace,multicol,diagbox,array}
\usepackage{xcolor}
\usepackage{variations}
\usepackage{xypic}
\usepackage{eurosym,stmaryrd}
\usepackage{graphicx}
\usepackage[np]{numprint}
\usepackage{hyperref} 
\usepackage{tikz}
\usepackage{colortbl}
\usepackage{multirow}
\usepackage{MnSymbol,wasysym}
\usepackage[top=1.5cm,bottom=1.5cm,right=1.2cm,left=1.5cm]{geometry}
\usetikzlibrary{calc, arrows, plotmarks, babel,decorations.pathreplacing}
\setstretch{1.25}
%\usepackage{lipsum} %\usepackage{enumitem} %\setlist[enumerate]{itemsep=1mm} bug avec enumerate



\newtheorem{thm}{Théorème}
\newtheorem{rmq}{Remarque}
\newtheorem{prop}{Propriété}
\newtheorem{cor}{Corollaire}
\newtheorem{lem}{Lemme}
\newtheorem{prop-def}{Propriété-définition}

\theoremstyle{definition}

\newtheorem{defi}{Définition}
\newtheorem{ex}{Exemple}
\newtheorem*{rap}{Rappel}
\newtheorem{cex}{Contre-exemple}
\newtheorem{exo}{Exercice} % \large {\fontfamily{ptm}\selectfont EXERCICE}
\newtheorem{nota}{Notation}
\newtheorem{ax}{Axiome}
\newtheorem{appl}{Application}
\newtheorem{csq}{Conséquence}
\def\di{\displaystyle}



\renewcommand{\thesection}{\Roman{section}}\renewcommand{\thesubsection}{\arabic{subsection} }\renewcommand{\thesubsubsection}{\alph{subsubsection} }


\newcommand{\bas}{~\backslash}\newcommand{\ba}{\backslash}
\newcommand{\C}{\mathbb{C}}\newcommand{\K}{\mathbb{K}}\newcommand{\R}{\mathbb{R}}\newcommand{\Q}{\mathbb{Q}}\newcommand{\Z}{\mathbb{Z}}\newcommand{\N}{\mathbb{N}}\newcommand{\V}{\overrightarrow}\newcommand{\Cs}{\mathscr{C}}\newcommand{\Ps}{\mathscr{P}}\newcommand{\Rs}{\mathscr{R}}\newcommand{\Gs}{\mathscr{G}}\newcommand{\Ds}{\mathscr{D}}\newcommand{\happy}{\huge\smiley}\newcommand{\sad}{\huge\frownie}\newcommand{\danger}{\begin{tikzpicture}[x=1.5pt,y=1.5pt,rotate=-14.2]
	\definecolor{myred}{rgb}{1,0.215686,0}
	\draw[line width=0.1pt,fill=myred] (13.074200,4.937500)--(5.085940,14.085900)..controls (5.085940,14.085900) and (4.070310,15.429700)..(3.636720,13.773400)
	..controls (3.203130,12.113300) and (0.917969,2.382810)..(0.917969,2.382810)
	..controls (0.917969,2.382810) and (0.621094,0.992188)..(2.097660,1.359380)
	..controls (3.574220,1.726560) and (12.468800,3.984380)..(12.468800,3.984380)
	..controls (12.468800,3.984380) and (13.437500,4.132810)..(13.074200,4.937500)
	--cycle;
	\draw[line width=0.1pt,fill=white] (11.078100,5.511720)--(5.406250,11.875000)..controls (5.406250,11.875000) and (4.683590,12.812500)..(4.367190,11.648400)
	..controls (4.050780,10.488300) and (2.375000,3.675780)..(2.375000,3.675780)
	..controls (2.375000,3.675780) and (2.156250,2.703130)..(3.214840,2.964840)
	..controls (4.273440,3.230470) and (10.640600,4.847660)..(10.640600,4.847660)
	..controls (10.640600,4.847660) and (11.332000,4.953130)..(11.078100,5.511720)
	--cycle;
	\fill (6.144520,8.839900)..controls (6.460940,7.558590) and (6.464840,6.457090)..(6.152340,6.378910)
	..controls (5.835930,6.300840) and (5.320300,7.277400)..(5.003900,8.554750)
	..controls (4.683590,9.835940) and (4.679690,10.941400)..(4.996090,11.019600)
	..controls (5.312490,11.097700) and (5.824210,10.121100)..(6.144520,8.839900)
	--cycle;
	\fill (7.292960,5.261780)..controls (7.382800,4.898500) and (7.128900,4.523500)..(6.730460,4.421880)
	..controls (6.328120,4.324220) and (5.929680,4.535220)..(5.835930,4.898500)
	..controls (5.746080,5.261780) and (5.999990,5.640630)..(6.402340,5.738340)
	..controls (6.804690,5.839840) and (7.203110,5.625060)..(7.292960,5.261780)
	--cycle;
	\end{tikzpicture}}\newcommand{\alors}{\Large\Rightarrow}\newcommand{\equi}{\Leftrightarrow}
\newcommand{\fonction}[5]{\begin{array}{l|rcl}
		#1: & #2 & \longrightarrow & #3 \\
		& #4 & \longmapsto & #5 \end{array}}
\newcommand{\disp}{\displaystyle}

\definecolor{vert}{RGB}{11,160,78}
\definecolor{rouge}{RGB}{255,120,120}
\definecolor{bleu}{RGB}{15,5,107}



\pagestyle{fancy}
\lhead{Groupe IPESUP}\chead{}\rhead{Année~2022-2023}\lfoot{M. Botcazou \& M.Dupré}\cfoot{\thepage/3}\rfoot{PCSI }\renewcommand{\headrulewidth}{0.4pt}\renewcommand{\footrulewidth}{0.4pt}


\begin{document}
 %%%%BIBMATH%%%%
 
%(1) https://www.bibmath.net/ressources/index.php?action=affiche&quoi=bde/analyse/suitesseries/serienum_prat&type=fexo
 
%(2) 
 
%(3)
 
  

\noindent\shadowbox{
	\begin{minipage}{1\linewidth}
		\centering
		\huge{\textbf{ TD 21 : Séries numériques }}
	\end{minipage}}

\smallskip
\section*{Connaître son cours:}
\noindent Soit $(u_n )_n \in \C^{\N}$, montrer les propriétés suivantes:
\begin{itemize}[$\bullet$]
	\item La suite $(u_n )_n$ et la série de terme général $(u_n - u_{n-1})_n$ sont de même nature.
	\item  Si la série de terme général $u_n$ converge, alors la suite $(u_n )_n$ tend vers $0$. La réciproque est-elle vraie ? Donner un exemple d'une série qui diverge grossièrement.
	\item Si la série de terme général $u_n$ converge absolument, alors elle converge.
	\item Énoncer le critère de comparaison série-intégrale et donner la preuve de celui-ci. 
	\item Rappeler le critère des séries de Riemann et donner la preuve de celui-ci. 
	\item Soit $(v_n)_{n\in\N}$ une suite d’éléments de $\R^+$ telle que $\ u_n \underset{\sim_{+\infty}}{=} o(v_n)$. Si la série de terme général $(v_n)_{n\in\N}$ est convergente, montrer que $(u_n)_{n\in\N}$ est absolument convergente donc convergente. La réciproque est-elle vraie ? 
	\item Soit $(u_n)_{n\in\N}$ et $(v_n)_{n\in\N}$ deux suites d’éléments de $\R^+$ telles que $u_n \sim v_n$. Montrer que la série de terme général $(u_n)_{n\in\N}$ converge si, et seulement si, la série de terme général $(v_n)_{n\in\N}$ converge.

\end{itemize}
\raggedright

\section*{Séries à termes positifs:}\hfill\\%[-0.25cm]

   
\begin{minipage}{1\linewidth}\begin{minipage}[t]{0.48\linewidth}\raggedright

\subsection*{Relations de comparaison}

\begin{exo}\textbf{(*)}\quad\\[0.2cm]
Donner la nature de la série de terme général 

\begin{center}
	\begin{tabular}{ll}
		\textbf{1) }$\disp \ln\left(\frac{n^2+n+1}{n^2+n-1}\right)$&\textbf{2) }  $\disp\frac{1}{n+(-1)^n\sqrt{n}}$\\[0.5cm]
		\textbf{3) } $\disp\left(\frac{n+3}{2n+1}\right)^{\ln n}$  &\textbf{4) } $\disp\left(\cos\frac{1}{\sqrt{n}}\right)^n-\frac{1}{\sqrt{e}}$\\[0.35cm]
		\textbf{5) } $\disp\ln\left(\frac{2}{\pi}\arctan\left(\frac{n^2+1}{n}\right)\right)$&
	\end{tabular}
\end{center}


	
\centering\rule{1\linewidth}{0.6pt}\end{exo}



\begin{exo}\textbf{(*)}\quad\\[0.2cm]
On considère la suite $(u_n)_n$ où $\forall n\in\N^*$, $\disp u_n=\frac{1}{n}e^{-u_{n-1}}$ avec $u_0\in\R$. Donner la nature de la série de terme général $(u_n)_n$.

	\centering\rule{1\linewidth}{0.6pt}\end{exo}

\begin{exo}\textbf{(**)}\quad\\[0.2cm]
Calculer les sommes des séries suivantes après avoir vérifié leur convergence.
\begin{center}
	\begin{tabular}{ll}
		\textbf{1) } $\disp\sum_{n=0}^{+\infty}\frac{n+1}{3^n}$&\textbf{2) } $\disp\sum_{n=3}^{+\infty}\frac{2n-1}{n^3-4n}$
	\end{tabular}
\end{center}	
	\centering\rule{1\linewidth}{0.6pt}\end{exo}




%%%%%%%%%%%%%%%%%%%%%%%%%%%%%%%%%%%%%%%%%%%%%%%%%%%%%%%%%%%%%%%%%%%%%%%%%%%%%%%%%%%%%%%%%%
\end{minipage}\hfill\vrule\hfill\begin{minipage}[t]{0.48\linewidth}\raggedright
%%%%%%%%%%%%%%%%%%%%%%%%%%%%%%%%%%%%%%%%%%%%%%%%%%%%%%%%%%%%%%%%%%%%%%%%%%%%%%%%%%%%%%%%%%

\begin{exo}\textbf{(*)}\quad\\[0.2cm]
Soit $(u_n)_{n\in\N}$ une suite positive telle que la série de terme général $u_n$ converge. Étudier la nature de la série de terme général $\frac{\sqrt{u_n}}{n}$.

\centering\rule{1\linewidth}{0.6pt}\end{exo}


\begin{exo}\textbf{(**)}\quad\\[0.2cm]
Soit $(u_n)_{n\in\N}$ une suite décroissante de nombres réels strictement positifs telle que la série de terme général $u_n$ converge. Montrer que $\disp u_n\underset{n\rightarrow+\infty}{=}o\left(\frac{1}{n}\right)$. Trouver un exemple de suite $(u_n)_{n\in\N}$ de réels strictement positifs telle que la série de terme général $u_n$ converge mais telle que la suite de terme général $nu_n$ ne tende pas vers $0$.

\centering\rule{1\linewidth}{0.6pt}\end{exo}

	\begin{exo}\textbf{(*)}\quad\\[0.2cm]
	\'Etudier la convergence des séries $\disp\sum u_n$ suivantes :
	$$\begin{array}{rcl}
	\displaystyle \mathbf 1)\ u_n=\left(\frac{1}{2}\right)^{\sqrt{n}}&&\displaystyle \mathbf 2) \ u_n=ne^{-\sqrt n}\\[0.5cm]
	&\displaystyle \mathbf 3)\ u_n=\frac{(n!)^3}{(3n)!}&
	\end{array}$$
	
	\centering\rule{1\linewidth}{0.6pt}\end{exo}

\end{minipage}\end{minipage} \newpage


\begin{minipage}{1\linewidth}\begin{minipage}[t]{0.48\linewidth}\raggedright
		\subsection*{Comparaison série-intégrale}
		\begin{exo}\textbf{(**)}\quad\\[0.2cm]
			\begin{enumerate}
				\item Donner un développement limité à l'ordre $2$ de la suite $\disp R_n = \sum_{k=n+1}^{+\infty}\frac{1}{k^2}$ quand $n$ tend vers l'infini.
				\item Déterminer la nature de la série de terme général $(R_n)_n$. 
			\end{enumerate}
			
			
			\centering\rule{1\linewidth}{0.6pt}\end{exo}
		
		
		
		\begin{exo}\textbf{(**)}\quad\text{(Séries de Bertrand)}\\[0.2cm]
			On souhaite étudier, suivant la valeur de $\alpha,\beta\in\mathbb R$, la convergence de la série de terme général
			$$u_n=\frac{1}{n^\alpha(\ln n)^\beta}.$$
			\begin{enumerate}
				\item Démontrer que la série converge si $\alpha>1$.
				\item Traiter le cas $\alpha<1$.
				\item On suppose que $\alpha=1$. 
				On pose $$\disp T_n=\int_2^n \frac{dx}{x(\ln x)^\beta}$$
				\begin{enumerate}
					\item Montrer si $\beta\leq 0$, alors la série de terme général $u_n$ est divergente.
					\item Montrer que si $\beta>1$, alors la suite $(T_n)$ est bornée, alors que si $\beta\leq 1$, la suite $(T_n)$ tend vers $+\infty$.
					\item Conclure pour la série de terme général $u_n$, lorsque $\alpha=1$.
				\end{enumerate}
			\end{enumerate}
			
			\centering\rule{1\linewidth}{0.6pt}\end{exo}
		
		
		
		
		%%%%%%%%%%%%%%%%%%%%%%%%%%%%%%%%%%%%%%%%%%%%%%%%%%%%%%%%%%%%%%%%%%%%%%%%%%%%%%%%%%%%%%%%%%
	\end{minipage}\hfill\vrule\hfill\begin{minipage}[t]{0.48\linewidth}\raggedright
		%%%%%%%%%%%%%%%%%%%%%%%%%%%%%%%%%%%%%%%%%%%%%%%%%%%%%%%%%%%%%%%%%%%%%%%%%%%%%%%%%%%%%%%%%%
		
		\subsection*{Quelques classiques}
		\begin{exo}\textbf{(***)}\quad\textit{(La série harmonique)}\\[0.2cm]
			On pose $\disp H_n=1+\frac12+\dots+\frac1n$. 
			\begin{enumerate}
				\item Prouver que $\disp H_n\sim_{+\infty}\ln n$.
				\item On pose $\disp u_n=H_n-\ln n$, et $v_n=u_{n+1}-u_n$.
				\'Etudier la nature de la série $\disp \sum_n v_n$. En déduire que la suite $(u_n)$ est convergente. On notera $\gamma$ sa limite.
				\item Soit $\disp R_n=\sum_{k=n}^{+\infty} \frac{1}{k^2}$. Donner un équivalent de $R_n$.
				\item Soit $w_n$ tel que $\disp H_n=\ln n+\gamma+w_n$, et soit 
				$\disp t_n=w_{n+1}-w_n$. Donner un équivalent du reste $\disp \sum_{k\geq n}t_k$.
				En déduire que\\[-0.2cm] $$\disp H_n=\ln n+\gamma+\frac{1}{2n}+o\left(\frac1n\right)$$
			\end{enumerate}
			
			
			\centering\rule{1\linewidth}{0.6pt}\end{exo}
		
			\begin{exo}\textbf{(**)}\quad\textit{(Formule de Stirling)}\\[0.2cm]
			\begin{enumerate}
				\item On pose  $(u_n)$ la suite définie par $$\disp u_n=\frac{n^ne^{-n}\sqrt{n}}{n!}$$ 
				Donner la nature de la série de terme général $$\disp v_n=\ln\left(\frac{u_{n+1}}{u_n}\right)$$ 
				\item En déduire l'existence d'une constante $C>0$ telle que :\\[-0.2cm]
				$$n!\sim_{+\infty} C\sqrt{n}n^ne^{-n}.$$
			\end{enumerate}
			
			\centering\rule{1\linewidth}{0.6pt}\end{exo}
		
\end{minipage}\end{minipage}

\section*{Séries générales:}\hfill\\%[-0.25cm]

\begin{minipage}{1\linewidth}\begin{minipage}[t]{0.48\linewidth}\raggedright

		
		
	\begin{exo}\textbf{(*)}\quad\\[0.2cm]
		On donne $\disp \sum_{k=1}^{+\infty} \frac{1}{k^{2}}=\frac{\pi^{2}}{6}$.\\
	 Calculer
		
		$$
		\sum_{k=1}^{+\infty} \frac{1}{k^{2}(k+1)^{2}}
		$$
		
		après en avoir justifié l'existence.
		
		\centering\rule{1\linewidth}{0.6pt}\end{exo}
		
		
		
		%%%%%%%%%%%%%%%%%%%%%%%%%%%%%%%%%%%%%%%%%%%%%%%%%%%%%%%%%%%%%%%%%%%%%%%%%%%%%%%%%%%%%%%%%%
	\end{minipage}\hfill\vrule\hfill\begin{minipage}[t]{0.48\linewidth}\raggedright
		%%%%%%%%%%%%%%%%%%%%%%%%%%%%%%%%%%%%%%%%%%%%%%%%%%%%%%%%%%%%%%%%%%%%%%%%%%%%%%%%%%%%%%%%%%
				
		\begin{exo}\textbf{(**)}\quad\\[0.2cm]
			
			Étudier la limite $+\infty$ de $\disp\sum_{k=1}^{n}\left(\frac{k}{n}\right)^{n} .
			$
			
			\centering\rule{1\linewidth}{0.6pt}\end{exo}
		
			\begin{exo}\textbf{(***)}\quad\\[0.2cm]
			Soit $a \in \mathbb{N} \backslash\{0\}$. Déterminer la somme de la série de terme général $\disp \dfrac{n-a\left\lfloor\frac{n}{a}\right\rfloor}{n(n+1)}$.
		
			\centering\rule{1\linewidth}{0.6pt}\end{exo}
		
		
		

		
		
\end{minipage}\end{minipage} \newpage


\begin{minipage}{1\linewidth}\begin{minipage}[t]{0.48\linewidth}\raggedright
		\subsection*{Séries alternées}
		
		\begin{exo}\textbf{(**)}\quad\\[0.2cm]%https://www.bibmath.net/dico/index.php?action=affiche&quoi=./s/seriealt.html
		Soit $(a_n)$ une suite de réels positifs, décroissante, et tendant vers 0.
		
		\begin{enumerate}
			\item Montrer que la série $\sum (-1)^na_n$ converge.
			\item Exprimer la suite des restes $(R_n)_n$ en valeur absolue et trouver une majoration de celle-ci. 
		\end{enumerate}
			
			
			
			\centering\rule{1\linewidth}{0.6pt}\end{exo}
		
		\begin{exo}\textbf{(**)}\quad\\[0.2cm]
			Soit $\alpha\in\R$. Discuter de la nature de la série de terme général $\disp u_n=\frac{1+(-1)^nn^\alpha}{n^{2\alpha}}$, $n\geqslant1$ selon la valeur de $\alpha$.
			
			\centering\rule{1\linewidth}{0.6pt}\end{exo}
		
	\begin{exo}\textbf{(*)}\quad\\[0.2cm]
		\begin{enumerate}
			\item Justifier que la série $\disp \sum \frac{(-1)^n}{\sqrt n}$ converge.
			\item Démontrer que $$\displaystyle u_n = \frac{(-1)^n}{\sqrt n+(-1)^n}=\frac{(-1)^n}{\sqrt n}-\frac1n+\frac{(-1)^n}{n\sqrt
				n}+o\left(\frac 1{n\sqrt n}\right)$$
			\item \'Etudier la convergence de la série $$\displaystyle \sum \frac{(-1)^n}{\sqrt n+(-1)^n}$$
			\item Donner un équivalent de la suite $(u_n)$ en $+\infty$, que remarquez-vous ? 
		\end{enumerate}
		
		\centering\rule{1\linewidth}{0.6pt}\end{exo}


\begin{exo}\textbf{(**)}\quad\\[0.2cm]
	Donner la nature de la série de terme général 
	
		$$\begin{array}{ccc}
	\displaystyle \mathbf 1)\ \ln\left(1+\frac{(-1)^n}{\sqrt{n}}\right)&&\displaystyle \mathbf 2) \ \sin\left(\frac{\pi n^2}{n+1}\right)\\[0.5cm]
	&\displaystyle \mathbf 3)\ \frac{(-1)^n}{n+(-1)^{n-1}}&
	\end{array}$$
	
	
	\centering\rule{1\linewidth}{0.6pt}\end{exo}






%%%%%%%%%%%%%%%%%%%%%%%%%%%%%%%%%%%%%%%%%%%%%%%%%%%%%%%%%%%%%%%%%%%%%%%%%%%%%%%%%%%%%%%%%%
\end{minipage}\hfill\vrule\hfill\begin{minipage}[t]{0.48\linewidth}\raggedright
%%%%%%%%%%%%%%%%%%%%%%%%%%%%%%%%%%%%%%%%%%%%%%%%%%%%%%%%%%%%%%%%%%%%%%%%%%%%%%%%%%%%%%%%%%


\subsection*{Propriétés complémentaires}

\begin{exo}\textbf{(**)}\quad\\[0.2cm]
	Soit $(u_n)$ la suite définie par $u_0\in[0;\pi]$ et pour tout $n\in\N$
	$$u_{n+1}= 1-\cos(u_n)$$
	
	Donner la limite de la suite $(u_n)$ et déterminer la nature de la série de terme général $u_n$.
	
	\centering\rule{1\linewidth}{0.6pt}\end{exo}


\begin{exo}\textbf{(**)}\quad \textit{(Règle de Raabe-Duhamel)}\\[0.2cm]
	Soient $\left(u_{n}\right)_{n \in \mathbb{N}}$ et $\left(v_{n}\right)_{n \in \mathbb{N}}$ deux suites de réels strictement positifs.
	\begin{enumerate}
		 
		
		\item On suppose qu'à partir d'un certain rang
		
		$$
		\frac{u_{n+1}}{u_{n}} \leq \frac{v_{n+1}}{v_{n}} .
		$$
		
		Montrer que $u_{n} \underset{n \rightarrow+\infty}{=} \mathrm{O}\left(v_{n}\right)$.
		
		\item On suppose que
		
		$$
		\dfrac{u_{n+1}}{u_{n}} \underset{n \rightarrow+\infty}{=} 1-\dfrac{\alpha}{n}+o\left(\frac{1}{n}\right) \text { avec } \alpha>1 .
		$$
		
		Montrer, à l'aide d'une comparaison avec une série de Riemann, que la série $\sum u_{n}$ converge.
		
		\item On suppose cette fois-ci que
		
		$$
		\frac{u_{n+1}}{u_{n}} \underset{n \rightarrow+\infty}{=} 1-\frac{\alpha}{n}+o\left(\frac{1}{n}\right) \text { avec } \alpha<1 .
		$$
		
		Montrer que la série $\sum u_{n}$ diverge
		
	\end{enumerate}
	
	\centering\rule{1\linewidth}{0.6pt}\end{exo}



\begin{exo}\textbf{(**)}\quad\textit{(Sommation des équivalents)}\\[0.2cm]

Soit $\left(u_{n}\right)_{n}$ et $\left(v_{n}\right)_{n}$ deux suites positives équivalentes. Montrer que
\begin{itemize}[$\bullet$]
	\item Si les séries de termes généraux $u_{n}$ et $v_{n}$ divergent, alors $\disp\sum_{k=0}^{n} u_{k} \sim \sum_{k=0}^{n} v_{k}$;
	\item Si les séries de termes généraux $u_{n}$ et $v_{n}$ convergent, alors $\disp \sum_{k=n}^{\infty} u_{k} \sim \sum_{k=n}^{\infty} v_{k}$.
\end{itemize}

	\centering\rule{1\linewidth}{0.6pt}\end{exo}


\end{minipage}\end{minipage} \newpage

\end{document}