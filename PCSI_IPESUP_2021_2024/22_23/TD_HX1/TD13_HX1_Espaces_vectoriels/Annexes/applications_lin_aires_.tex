
 %Macros utilisées dans la base de données d'exercices 

\newcommand{\mtn}{\mathbb{N}}
\newcommand{\mtns}{\mathbb{N}^*}
\newcommand{\mtz}{\mathbb{Z}}
\newcommand{\mtr}{\mathbb{R}}
\newcommand{\mtk}{\mathbb{K}}
\newcommand{\mtq}{\mathbb{Q}}
\newcommand{\mtc}{\mathbb{C}}
\newcommand{\mch}{\mathcal{H}}
\newcommand{\mcp}{\mathcal{P}}
\newcommand{\mcb}{\mathcal{B}}
\newcommand{\mcl}{\mathcal{L}}
\newcommand{\mcm}{\mathcal{M}}
\newcommand{\mcc}{\mathcal{C}}
\newcommand{\mcmn}{\mathcal{M}}
\newcommand{\mcmnr}{\mathcal{M}_n(\mtr)}
\newcommand{\mcmnk}{\mathcal{M}_n(\mtk)}
\newcommand{\mcsn}{\mathcal{S}_n}
\newcommand{\mcs}{\mathcal{S}}
\newcommand{\mcd}{\mathcal{D}}
\newcommand{\mcsns}{\mathcal{S}_n^{++}}
\newcommand{\glnk}{GL_n(\mtk)}
\newcommand{\mnr}{\mathcal{M}_n(\mtr)}
\DeclareMathOperator{\ch}{ch}
\DeclareMathOperator{\sh}{sh}
\DeclareMathOperator{\vect}{vect}
\DeclareMathOperator{\card}{card}
\DeclareMathOperator{\comat}{comat}
\DeclareMathOperator{\imv}{Im}
\DeclareMathOperator{\rang}{rg}
\DeclareMathOperator{\Fr}{Fr}
\DeclareMathOperator{\diam}{diam}
\DeclareMathOperator{\supp}{supp}
\newcommand{\veps}{\varepsilon}
\newcommand{\mcu}{\mathcal{U}}
\newcommand{\mcun}{\mcu_n}
\newcommand{\dis}{\displaystyle}
\newcommand{\croouv}{[\![}
\newcommand{\crofer}{]\!]}
\newcommand{\rab}{\mathcal{R}(a,b)}
\newcommand{\pss}[2]{\langle #1,#2\rangle}
 %Document 

\begin{document} 

\begin{center}\textsc{{\huge }}\end{center}

% Exercice 839


\vskip0.3cm\noindent\textsc{Exercice 1} - Applications linéaires ou non (sur $\mathbb R^n$)?
\vskip0.2cm
Dire si les applications suivantes sont des applications linéaires :
\begin{enumerate}
\item $f:\mathbb R^2\to\mathbb R^3,\ (x,y)\mapsto (x+y,x-2y,0)$;
\item $f:\mathbb R^2\to\mathbb R^3,\ (x,y)\mapsto (x+y,x-2y,1)$;
\item $f:\mathbb R^2\to\mathbb R,\ (x,y)\mapsto x^2-y^2$.
\end{enumerate}


% Exercice 2944


\vskip0.3cm\noindent\textsc{Exercice 2} - Applications linéaires ou non (sur les polynômes)?
\vskip0.2cm
Dire si les applications suivantes sont des applications linéaires :
\begin{enumerate}
\item $f:\mathbb R[X]\to \mathbb R^2,\ P\mapsto \big(P(0),P'(1)\big)$;
\item $f:\mathbb R[X]\to \mathbb R[X],\ P\mapsto AP$, où $A\in\mathbb R[X]$ est un polynôme fixé;
\item $f:\mathbb R[X]\to\mathbb R[X],\ P\mapsto P^2$.
\end{enumerate}


% Exercice 874


\vskip0.3cm\noindent\textsc{Exercice 3} - Noyau et image
\vskip0.2cm
On considère l'application linéaire $f$ de $\mathbb R^3$
dans $\mathbb R^4$ définie par
$$f(x,y,z)=(x+z,y-x,z+y,x+y+2z).$$
\begin{enumerate}
\item Déterminer  une base de $\textrm{Im}(f)$.
\item Déterminer une base de $\ker(f)$.
\item L'application $f$ est-elle injective? surjective?
\end{enumerate}


% Exercice 841


\vskip0.3cm\noindent\textsc{Exercice 4} - Dérivation
\vskip0.2cm
Soit $E=\mathcal C^{\infty}(\mathbb R)$ et $\phi\in\mathcal L(E)$ définie par $\phi(f)=f'$. Quel est le noyau de $\phi$? Quelle est son image? $\phi$ est-elle injective? surjective?


% Exercice 2947


\vskip0.3cm\noindent\textsc{Exercice 5} - 
\vskip0.2cm
Soit $E$ l’espace vectoriel des applications de $\mathbb R$ dans $\mathbb R$. On note $L:E\to E$ l’application qui à $f\in E$ associe $L(f)$ définie par $L(f):x\mapsto f(x)−f(−x)$.
\begin{enumerate}
\item Montrer que $L$ est un endomorphisme de E.
\item Préciser le noyau et l’image de L.
\item L’application $L$ est-elle injective ? surjective ?
\end{enumerate}


% Exercice 838


\vskip0.3cm\noindent\textsc{Exercice 6} - Application linéaire définie sur un espace de polynôme
\vskip0.2cm
Soit $E=\mathbb C[X]$, $p$ un entier naturel et $f$ l'application de $E$ dans $E$ définie par
$f(P)=(1-pX)P+X^2P'$. $f$ est-elle injective? surjective?



% Exercice 887


\vskip0.3cm\noindent\textsc{Exercice 7} - Applications linéaires dans un espace de polynômes
\vskip0.2cm
Soit $E=\mathbb R_3[X]$ l'espace vectoriel des polynômes à coefficients réels de degré inférieur ou égal à 3.
On définit $u$ l'application de $E$ dans lui-même par
$$u(P)=P+(1-X)P'.$$
\begin{enumerate}
\item Montrer que $u$ est un endomorphisme de $E$.
\item Déterminer une base de $\textrm{Im}(u)$.
\item Déterminer une base de $\ker(u)$.
\item Montrer que $\ker(u)$ et $\textrm{Im}(u)$ sont deux sous-espaces
vectoriels supplémentaires de $E$.
\end{enumerate}



% Exercice 3109


\vskip0.3cm\noindent\textsc{Exercice 8} - Une projection dans $\mathbb R[X]$
\vskip0.2cm
Soit $A\in\mathbb R[X]$ non nul, et $\phi:\mathbb R[X]\to\mathbb R[X]$ l'application qui à un polynôme $P$ associe son reste dans la division euclidienne par $A$. Démontrer que $\phi$ est un projecteur et préciser ses éléments caractéristiques.


% Exercice 853


\vskip0.3cm\noindent\textsc{Exercice 9} - Somme de deux projecteurs
\vskip0.2cm
Soit $E$ un $\mathbb R$-espace vectoriel. Soient $p$ et $q$ deux projecteurs de $E$.
\begin{enumerate}
\item Montrer que $p+q$ est un projecteur si et seulement si $p\circ q=q\circ p=0$.
\item Montrer que, dans ce cas, on a $\textrm{Im}(p+q)=\textrm{Im}(p)\oplus \textrm{Im}(q)$
et $\ker(p+q)=\ker p\cap \ker q$.
\end{enumerate}


% Exercice 849


\vskip0.3cm\noindent\textsc{Exercice 10} - Endomorphismes qui commutent, noyaux et images
\vskip0.2cm
Soit $E$ un espace vectoriel et $u,v\in\mathcal L(E)$. On suppose que $u\circ v=v\circ u$. Démontrer que $\textrm{ker}(u)$ et $\textrm{Im}(u)$ sont stables par $v$, c'est-à-dire que
$$v(\ker (u))\subset \ker (u)\textrm{ et }v(\textrm{Im}(u))\subset \textrm{Im}(u).$$


% Exercice 3107


\vskip0.3cm\noindent\textsc{Exercice 11} - Image de la composée et somme de l'image et du noyau
\vskip0.2cm
Soit $E$ un espace vectoriel et $f,g\in\mathcal L(E)$. Démontrer que 
$$E=\textrm{Im}(f)+\ker(g)\iff \textrm{Im}(g\circ f)=\textrm{Im}(g).$$


% Exercice 901


\vskip0.3cm\noindent\textsc{Exercice 12} - 
\vskip0.2cm
Soit $E$ un espace vectoriel et $f\in\mathcal L(E)$.
\begin{enumerate}
\item Montrer que
$$\ker(f)=\ker(f^2)\iff \textrm{Im}f\cap\ker(f)=\{0\}.$$
\item On suppose que $E$ est de dimension finie. Montrer que
$$\ker(f)=\ker(f^2)\iff \textrm{Im}f\oplus \ker(f)=E\iff \textrm{Im}(f)=\textrm{Im}(f^2).$$
\end{enumerate}




\vskip0.5cm
\noindent{\small Cette feuille d'exercices a été conçue à l'aide du site \textsf{https://www.bibmath.net}}

%Vous avez accès aux corrigés de cette feuille par l'url : https://www.bibmath.net/ressources/justeunefeuille.php?id=27129
\end{document}