
 %Macros utilisées dans la base de données d'exercices 

\newcommand{\mtn}{\mathbb{N}}
\newcommand{\mtns}{\mathbb{N}^*}
\newcommand{\mtz}{\mathbb{Z}}
\newcommand{\mtr}{\mathbb{R}}
\newcommand{\mtk}{\mathbb{K}}
\newcommand{\mtq}{\mathbb{Q}}
\newcommand{\mtc}{\mathbb{C}}
\newcommand{\mch}{\mathcal{H}}
\newcommand{\mcp}{\mathcal{P}}
\newcommand{\mcb}{\mathcal{B}}
\newcommand{\mcl}{\mathcal{L}}
\newcommand{\mcm}{\mathcal{M}}
\newcommand{\mcc}{\mathcal{C}}
\newcommand{\mcmn}{\mathcal{M}}
\newcommand{\mcmnr}{\mathcal{M}_n(\mtr)}
\newcommand{\mcmnk}{\mathcal{M}_n(\mtk)}
\newcommand{\mcsn}{\mathcal{S}_n}
\newcommand{\mcs}{\mathcal{S}}
\newcommand{\mcd}{\mathcal{D}}
\newcommand{\mcsns}{\mathcal{S}_n^{++}}
\newcommand{\glnk}{GL_n(\mtk)}
\newcommand{\mnr}{\mathcal{M}_n(\mtr)}
\DeclareMathOperator{\ch}{ch}
\DeclareMathOperator{\sh}{sh}
\DeclareMathOperator{\vect}{vect}
\DeclareMathOperator{\card}{card}
\DeclareMathOperator{\comat}{comat}
\DeclareMathOperator{\imv}{Im}
\DeclareMathOperator{\rang}{rg}
\DeclareMathOperator{\Fr}{Fr}
\DeclareMathOperator{\diam}{diam}
\DeclareMathOperator{\supp}{supp}
\newcommand{\veps}{\varepsilon}
\newcommand{\mcu}{\mathcal{U}}
\newcommand{\mcun}{\mcu_n}
\newcommand{\dis}{\displaystyle}
\newcommand{\croouv}{[\![}
\newcommand{\crofer}{]\!]}
\newcommand{\rab}{\mathcal{R}(a,b)}
\newcommand{\pss}[2]{\langle #1,#2\rangle}
 %Document 

\begin{document} 

\begin{center}\textsc{{\huge }}\end{center}

% Exercice 2943


\vskip0.3cm\noindent\textsc{Exercice 1} - Est-ce un sous-espace vectoriel (matrices)?
\vskip0.2cm
Déterminer si les parties suivantes sont des sous-espaces vectoriels de $M_2(\mathbb R)$ : 
\begin{enumerate}
\item $E_1=\left\{\begin{pmatrix}
a & b \\
c & d \\
\end{pmatrix}\in M_2(\mathbb R):\ ad-bc=1\right\}$;
\item $E_2=\left\{\begin{pmatrix}
x_1 & x_2 \\
x_3 & x_4 \\
\end{pmatrix}\in M_2(\mathbb R):\ x_1 + x_2 = x_4\right\}$~;
\item $E_3=\left\{A\in M_2(\mathbb R):\ {}^tA=A\right\}$.
\end{enumerate}


% Exercice 2621


\vskip0.3cm\noindent\textsc{Exercice 2} - Combinaisons linéaires?
\vskip0.2cm
\begin{enumerate}
\item Dans l'espace vectoriel $\mathbb R[X]$, le polynôme $P(X)=16X^3-7X^2+21X-4$ est-il combinaison linéaire de $P_1(X)=8X^3-5X^2+1$ et $P_2(X)=X^2+7X-2$?
\item Dans l'espace vectoriel $\mathcal F(\mathbb R,\mathbb R)$ des fonctions de $\mathbb R$ dans $\mathbb R$, la fonction $x\mapsto \sin(2x)$ est-elle combinaison linéaire des fonctions $\sin$ et $\cos$?
\end{enumerate}


% Exercice 2940


\vskip0.3cm\noindent\textsc{Exercice 3} - En somme directe?
\vskip0.2cm
Pour chacun des sous-espaces vectoriels $F$ et $G$ de $\mathbb R^3$ suivants, déterminer s'ils sont en somme directe.
\begin{enumerate}
\item $F=\left\{(x,y,z)\in \mathbb R^3\mid x+2y+z=0\right\}$ et $G=\left\{(x,y,z)\in \mathbb R^3\mid \left\{\begin{array}{l}
2x + y + 3z = 0 \\
x - 2y - z = 0 \\
\end{array}\right.\right\}$; 
\item $F=\left\{(x,y,z)\in \mathbb R^3\mid x+y+2z=0\right\}$ et $G=\left\{(x,y,z)\in \mathbb R^3\mid \left\{\begin{array}{l}
2x + y + 3z = 0 \\
x - 2y - z = 0 \\
\end{array}\right.\right\}$.
\end{enumerate}


% Exercice 832


\vskip0.3cm\noindent\textsc{Exercice 4} - Périodiques et tend vers 0 à l'infini
\vskip0.2cm
Soit $E$ l'espace vectoriel des fonctions de $\mathbb R$ dans $\mathbb R$, $F$ le sous-espace vectoriel des fonctions périodiques de période 1 et $G$ le sous-espace vectoriel des fonctions $f$ telles que $\lim_{+\infty}f=0$. Démontrer que $F\cap G=\{0\}$. Est-ce que $F$ et $G$ sont supplémentaires?


% Exercice 831


\vskip0.3cm\noindent\textsc{Exercice 5} - Transformer une somme en somme directe
\vskip0.2cm
Soient $F$ et $G$ deux sous-espaces vectoriels d'un espace vectoriel $E$ tels que
$F+G=E$. Soit $F'$ un supplémentaire de $F\cap G$ dans $F$. Montrer que
$F'\oplus G=E$.


% Exercice 834


\vskip0.3cm\noindent\textsc{Exercice 6} - Un supplémentaire n'est jamais unique
\vskip0.2cm
Soit $E$ un espace vectoriel dans lequel tout sous-espace vectoriel admet un supplémentaire. Soit $F$ un sous-espace vectoriel propre de $E$ (c'est-à-dire que $F\neq \{0\}$ et que $F\neq E$). Démontrer que $F$ admet au moins deux supplémentaires distincts.




\vskip0.5cm
\noindent{\small Cette feuille d'exercices a été conçue à l'aide du site \textsf{https://www.bibmath.net}}

%Vous avez accès aux corrigés de cette feuille par l'url : https://www.bibmath.net/ressources/justeunefeuille.php?id=27108
\end{document}