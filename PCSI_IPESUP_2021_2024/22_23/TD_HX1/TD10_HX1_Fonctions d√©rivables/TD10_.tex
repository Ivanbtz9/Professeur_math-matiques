\documentclass[a4paper,11pt]{article}

\usepackage{inputenc}
\usepackage[T1]{fontenc}
\usepackage[frenchb]{babel}
\usepackage{fancyhdr,fancybox} % pour personnaliser les en-têtes
\usepackage{lastpage,setspace}
\usepackage{amsfonts,amssymb,amsmath,amsthm,mathrsfs}
\usepackage{relsize,exscale,bbold}
\usepackage{paralist}
\usepackage{xspace,multicol,diagbox,array}
\usepackage{xcolor}
\usepackage{variations}
\usepackage{xypic}
\usepackage{eurosym,stmaryrd}
\usepackage{graphicx}
\usepackage[np]{numprint}
\usepackage{hyperref} 
\usepackage{tikz}
\usepackage{colortbl}
\usepackage{multirow}
\usepackage{MnSymbol,wasysym}
\usepackage[top=1.5cm,bottom=1.5cm,right=1.2cm,left=1.5cm]{geometry}
\usetikzlibrary{calc, arrows, plotmarks, babel,decorations.pathreplacing}
\setstretch{1.25}
%\usepackage{lipsum} %\usepackage{enumitem} %\setlist[enumerate]{itemsep=1mm} bug avec enumerate



\newtheorem{thm}{Théorème}
\newtheorem{rmq}{Remarque}
\newtheorem{prop}{Propriété}
\newtheorem{cor}{Corollaire}
\newtheorem{lem}{Lemme}
\newtheorem{prop-def}{Propriété-définition}

\theoremstyle{definition}

\newtheorem{defi}{Définition}
\newtheorem{ex}{Exemple}
\newtheorem*{rap}{Rappel}
\newtheorem{cex}{Contre-exemple}
\newtheorem{exo}{Exercice} % \large {\fontfamily{ptm}\selectfont EXERCICE}
\newtheorem{nota}{Notation}
\newtheorem{ax}{Axiome}
\newtheorem{appl}{Application}
\newtheorem{csq}{Conséquence}
\def\di{\displaystyle}



\renewcommand{\thesection}{\Roman{section}}\renewcommand{\thesubsection}{\arabic{subsection} }\renewcommand{\thesubsubsection}{\alph{subsubsection} }


\newcommand{\bas}{~\backslash}\newcommand{\ba}{\backslash}
\newcommand{\C}{\mathbb{C}}\newcommand{\R}{\mathbb{R}}\newcommand{\Q}{\mathbb{Q}}\newcommand{\Z}{\mathbb{Z}}\newcommand{\N}{\mathbb{N}}\newcommand{\V}{\overrightarrow}\newcommand{\Cs}{\mathscr{C}}\newcommand{\Ps}{\mathscr{P}}\newcommand{\Rs}{\mathscr{R}}\newcommand{\Gs}{\mathscr{G}}\newcommand{\Ds}{\mathscr{D}}\newcommand{\happy}{\huge\smiley}\newcommand{\sad}{\huge\frownie}\newcommand{\danger}{\begin{tikzpicture}[x=1.5pt,y=1.5pt,rotate=-14.2]
	\definecolor{myred}{rgb}{1,0.215686,0}
	\draw[line width=0.1pt,fill=myred] (13.074200,4.937500)--(5.085940,14.085900)..controls (5.085940,14.085900) and (4.070310,15.429700)..(3.636720,13.773400)
	..controls (3.203130,12.113300) and (0.917969,2.382810)..(0.917969,2.382810)
	..controls (0.917969,2.382810) and (0.621094,0.992188)..(2.097660,1.359380)
	..controls (3.574220,1.726560) and (12.468800,3.984380)..(12.468800,3.984380)
	..controls (12.468800,3.984380) and (13.437500,4.132810)..(13.074200,4.937500)
	--cycle;
	\draw[line width=0.1pt,fill=white] (11.078100,5.511720)--(5.406250,11.875000)..controls (5.406250,11.875000) and (4.683590,12.812500)..(4.367190,11.648400)
	..controls (4.050780,10.488300) and (2.375000,3.675780)..(2.375000,3.675780)
	..controls (2.375000,3.675780) and (2.156250,2.703130)..(3.214840,2.964840)
	..controls (4.273440,3.230470) and (10.640600,4.847660)..(10.640600,4.847660)
	..controls (10.640600,4.847660) and (11.332000,4.953130)..(11.078100,5.511720)
	--cycle;
	\fill (6.144520,8.839900)..controls (6.460940,7.558590) and (6.464840,6.457090)..(6.152340,6.378910)
	..controls (5.835930,6.300840) and (5.320300,7.277400)..(5.003900,8.554750)
	..controls (4.683590,9.835940) and (4.679690,10.941400)..(4.996090,11.019600)
	..controls (5.312490,11.097700) and (5.824210,10.121100)..(6.144520,8.839900)
	--cycle;
	\fill (7.292960,5.261780)..controls (7.382800,4.898500) and (7.128900,4.523500)..(6.730460,4.421880)
	..controls (6.328120,4.324220) and (5.929680,4.535220)..(5.835930,4.898500)
	..controls (5.746080,5.261780) and (5.999990,5.640630)..(6.402340,5.738340)
	..controls (6.804690,5.839840) and (7.203110,5.625060)..(7.292960,5.261780)
	--cycle;
	\end{tikzpicture}}\newcommand{\alors}{\Large\Rightarrow}\newcommand{\equi}{\Leftrightarrow}
\newcommand{\fonction}[5]{\begin{array}{l|rcl}
		#1: & #2 & \longrightarrow & #3 \\
		& #4 & \longmapsto & #5 \end{array}}


\definecolor{vert}{RGB}{11,160,78}
\definecolor{rouge}{RGB}{255,120,120}
\definecolor{bleu}{RGB}{15,5,107}



\pagestyle{fancy}
\lhead{Groupe IPESUP}\chead{}\rhead{Année~2022-2023}\lfoot{M. Botcazou \& M.Dupré}\cfoot{\thepage/5}\rfoot{MPSI }\renewcommand{\headrulewidth}{0.4pt}\renewcommand{\footrulewidth}{0.4pt}


\begin{document}
	
%https://www.bibmath.net/ressources/index.php?action=affiche&quoi=capes/feuillesexo/derivee&type=fexo 	(1)

%https://www.bibmath.net/ressources/index.php?action=affiche&quoi=mpsi/feuillesexo/derivee&type=fexo 	(2)

%https://www.bibmath.net/ressources/index.php?action=affiche&quoi=bde/analyse/unevariable/derivee&type=fexo	(3)

%https://www.bibmath.net/ressources/index.php?action=affiche&quoi=mpsi/feuillesexo/convexe&type=fexo ""CONVEXITE"" (4)
	

\noindent\shadowbox{
	\begin{minipage}{1\linewidth}
		\centering
		\huge{\textbf{ TD 10 : Fonctions dérivables et convexité }}
	\end{minipage}
}
\smallskip
\section*{Connaître son cours:}
\noindent Démontrer les assertions suivantes:
\begin{itemize}[$\bullet$]
\item Soient $D\subset\R$, $f:D\rightarrow\C$ une fonction et $a\in D$. Si $f$ est dérivable en $a$, alors $f$ est continue en $a$.

\item Soit $n\in\N$, la fonction $x \in\R \mapsto x^n$ est dérivable en tout point de $\R$ et donner sa fonction dérivée associée. 
\item La fonction $\begin{array}{ccl}
%\R & \rightarrow & \R\\
x & \longmapsto & \left\{\begin{array}{cl}
x\sin\left(\frac{1}{x}\right) & \text{si } x\neq 0\\
0& \text{sinon}
\end{array}\right.
\end{array}$
\noindent est continue en $0$ mais non dérivable en $0$.
	
	\item Une application dérivable de $\R$ dans $\R$ est lipschitzienne si et seulement si sa dérivée est bornée sur $\R$. En déduire qu'une fonction de classe $\Cs^1$ sur un segment $I$ est lipschitzienne. 
	\item La fonction $\begin{array}{ccl}
	%\R & \rightarrow & \R\\
	x & \longmapsto & \left\{\begin{array}{cl}
	e^{-\frac{1}{x}} & \text{si } x>0\\
	0& \text{sinon}
	\end{array}\right.
	\end{array}$
	\noindent est de classe $\Cs^{\infty}$ sur $\R$.
	\item  Si $f$ est dérivable sur $I$, $f$ est convexe si et seulement si $f'$ est croissante.
	\item  Si $f$ est deux fois dérivable sur $I$, $f$ est convexe si et seulement si $f''\geq0$. 
\end{itemize}


\section*{Fonctions régulières:}\hfill\\%[-0.25cm]
\begin{minipage}{1\linewidth}
	\begin{minipage}[t]{0.48\linewidth}
		\raggedright
	

\begin{exo}\textbf{(*)}\quad\\[0.2cm]
	\'Etudier la d\'erivabilit\'e des fonctions suivantes sur $\R$:
	
	$f_1(x)=x^2\cos \frac{1}{x}, \text{\ \  si }x\not=0 \qquad ; \quad f_1(0)=0 ;$
	
	$f_2(x)= \sin x \cdot \sin \frac{1}{x}, \text{\ \  si }x\not=0 \quad ; \quad f_2(0)=0 ;$
	
	$f_3(x) = \frac{|x|\sqrt{x^2-2x+1}}{x-1}, \text{\ \  si } x\not= 1 \quad ; \quad f_3(1)=1.$
	
	\centering
	\rule{1\linewidth}{0.6pt}
\end{exo}




\begin{exo}\textbf{(*)}\quad\\[0.2cm]
 	 Soit $n \geq 2$ un entier fix\'e et $f:\R^{+} = [0,
 	+ \infty[ \longrightarrow \R$ la fonction d\'efinie par la formule
 	suivante:\quad\\[-0.3cm]
 	$$ f (x) = \frac{1 + x^n}{(1 + x)^n}, \ \ x \geq 0.$$
 	\begin{enumerate}
 		\item
 		\begin{enumerate}
 			\item Montrer que $f$ est d\'erivable sur $\R^{+}$ et calculer $f' (x)$
 			pour $x \geq 0.$
 			\item En \'etudiant le signe de $f' (x)$ sur $\R^{+},$ montrer que $f$
 			atteint un minimum sur $\R^{+}$ que l'on d\'eterminera.
 		\end{enumerate}
 		\item
 		\begin{enumerate}
 			\item En d\'eduire l'in\'egalit\'e suivante:
 			
 			$ (1+ x)^n \leq 2^{n - 1} (1+ x^n), \ \ \forall x \in \R^{+}.$
 			\item Montrer que si  $x \in \R^{+}$ et $y \in \R^{+}$ alors
 			
 			 on a: $ (x + y)^n \leq 2^{n - 1} (x^n + y^n).$
 		\end{enumerate}
 	\end{enumerate}
 	
	\centering
	\rule{1\linewidth}{0.6pt}
\end{exo}

\end{minipage}	
\hfill\vrule\hfill
\begin{minipage}[t]{0.48\linewidth}
\raggedright
\begin{exo}\textbf{(*)}\quad\\[0.2cm]
	Etudier la dérivabilité en $0$ des fonctions:
	\begin{enumerate}
		\item  $f~:~x\mapsto\cos\sqrt{x} \ , \qquad x\geq0$.
		\item  $g~:~x\mapsto x^2\tan\frac{1}{x}\sin\frac{2}{x} \ \text{ si } x\neq 0 \text{ et } g(0)=0$.
	\end{enumerate}
	
	
	
	
	\centering
	\rule{1\linewidth}{0.6pt}
\end{exo}





\begin{exo}\textbf{(*)}\quad\\[0.2cm]
	Soit $f$ de classe $C^1$ sur $\R_+^*$ telle que $\lim\limits_{x\rightarrow +\infty}xf'(x)=1$. 
	Montrer que $\lim\limits_{x\rightarrow +\infty}f(x)=+\infty$.

	
	
	\centering
	\rule{1\linewidth}{0.6pt}
\end{exo}

\begin{exo}\textbf{(**)}\quad\\[0.2cm]
	Soit $f\in C^1([a,b],\R)$ telle que $$\frac{f(b)-f(a)}{b-a}=\mbox{sup}\{f'(x),\;x\in[a,b]\}$$
	Montrer que $f$ est affine.
	
	\centering
	\rule{1\linewidth}{0.6pt}
\end{exo}

\begin{exo}\textbf{(**)}\quad\\[0.2cm]
	
	Montrer que le polyn\^ome $  X^n+aX+b  $,  ($  a  $ et $  b  $ r\' eels) admet au plus trois racines r\' eelles.
	
	\centering
	\rule{1\linewidth}{0.6pt}
\end{exo}


\end{minipage}
\end{minipage}

\newpage


\begin{minipage}{1\linewidth}
	\begin{minipage}[t]{0.48\linewidth}
		\raggedright
		
				\begin{exo}\textbf{(*)}\quad\\[0.2cm]
			Démontrer que les courbes d'équation $y=x^2$ et $y=1/x$ admettent une unique tangente commune.
			
			\centering
			\rule{1\linewidth}{0.6pt}
		\end{exo}
		
			\begin{exo}\textbf{(**)}\quad\\[0.2cm]
			Soit $P$ un polynôme réel de degré supèrieur ou égal à $2$.
			\begin{enumerate}
				\item  Montrer que si $P$ n'a que des racines simples et réelles, il en est de même de $P'$.
				\item  Montrer que si $P$ est scindé sur $\R$, il en est de même de $P'$.
			\end{enumerate}
			
			\centering
			\rule{1\linewidth}{0.6pt}
		\end{exo}
		
		
		\begin{exo}\textbf{(***)} \ \textit{"Polynômes de \textsc{Legendre}"}\quad\\[0.2cm]
			Pour $n$ entier naturel non nul donné, on pose $L_n=((X^2-1)^n)^{(n)}$.
			\begin{enumerate}
				\item  Déterminer le degré et le coefficient dominant de $L_n$.
				\item  En étudiant le polynôme $A_n=(X^2-1)^n$, montrer que $L_n$ admet $n$ racines réelles simples et toutes dans $]-1;1[$.
			\end{enumerate}
			
			\centering
			\rule{1\linewidth}{0.6pt}
		\end{exo}
		
		

		
		
		
	\end{minipage}	
	\hfill\vrule\hfill
	\begin{minipage}[t]{0.48\linewidth}
		\raggedright
		

		
		
		
		\begin{exo}\textbf{(**)}\quad\\[0.2cm]
			Soit $f$ une fonction dérivable en un point $x_0$. Montrer que
			$$\lim_{h\to 0}\frac{f(x_0+h)-f(x_0-h)}{2h}=f'(x_0).$$
			Réciproquement, si la limite précédente existe, peut-on dire que $f$ est dérivable en $x_0$?
			
			\centering
			\rule{1\linewidth}{0.6pt}
		\end{exo}
	
	
	
	\begin{exo}\textbf{(**)}\quad \textit{"\textsc{Rolle} à l'infini"}\\[0.2cm] 
	Soit $f:[0,+\infty[\to\mathbb R$ une fonction continue, dérivable sur $]0,+\infty[$ et telle que $f(0)=\lim\limits_{+\infty}f=0$.\\[0.2cm]
	
	
	On souhaite démontrer qu'il existe $d\in]0,+\infty[$ tel que $f'(d)=0$. Le résultat étant direct si $f$ est identiquement nulle, on suppose que ce n'est pas le cas et qu'il existe $c\in]0,+\infty[$ tel que $f(c)>0$ 
	
	(le cas où $f(c)<0$ étant similaire).
	\begin{enumerate}
		\item Démontrer qu'il existe $a\in]0,c[$ et $b\in]c,+\infty[$ tel que $f(a)=f(b)$.
		\item Conclure.
	\end{enumerate}

		\centering
	\rule{1\linewidth}{0.6pt}
	\end{exo}

		
		
	\end{minipage}
\end{minipage}
\hfill\\
\section*{Fonctions de régularité supérieure :}\hfill\\
\begin{minipage}{1\linewidth}
	\begin{minipage}[c]{0.48\linewidth}
		\raggedright
		
			\begin{exo}\textbf{(*)}\quad\\[0.2cm]
			Soit $n\geq 1$ et $1\leq k\leq n$. 
			\begin{enumerate}
				\item Calculer la dérivée $k$-ème de $x\mapsto x^{n-1}$ et $x\mapsto \ln(1+x)$.
				\item En déduire la dérivée $n$-ième de la fonction suivante :
				$x\mapsto x^{n-1}\ln(1+x).$
			\end{enumerate}
			
			\centering
			\rule{1\linewidth}{0.6pt}
		\end{exo}
	

		
	
		
	\end{minipage}	
	\hfill\vrule\hfill
	\begin{minipage}[c]{0.48\linewidth}
		\raggedright
		
			\begin{exo}\textbf{(**)}\quad\\[0.2cm]
			Soit $f:[a,b]\to\mathbb R$ $n$ fois dérivable.
			\begin{enumerate}
				\item On suppose que $f$ s'annule en $(n+1)$ points distincts de $[a,b]$. Démontrer qu'il existe $c\in ]a,b[$ tel que $f^{(n)}(c)=0$.
				\item On suppose que $f(a)=f'(a)=\dots=f^{(n-1)}(a)=f(b)=0$. Démontrer qu'il existe $c\in ]a,b[$ tel que $f^{(n)}(c)=0$.
			\end{enumerate}
			
			\centering
			\rule{1\linewidth}{0.6pt}
		\end{exo}
		
		
	\end{minipage}
\end{minipage}

\newpage

\begin{minipage}{1\linewidth}
	\begin{minipage}[t]{0.48\linewidth}
		\raggedright
		
		

				
		\begin{exo}\textbf{(**)}\quad\\[0.2cm]
			On consid\`ere la fonction $f : \R \to \R$ d\'efinie par
			\begin{equation*}
			f(t) =
			\begin{cases}
			e^{1/t} & \mathrm{\ si\ } t<0\\
			0  & \mathrm{\ si\ } t \geq 0
			\end{cases}
			\end{equation*}
			
			\begin{enumerate}
				\item D\'emontrer  que $f$ est d\'erivable sur $\R$, en particulier en $t=0$.
				\item Etudier l'existence de $f''(0)$.
				\item On veut montrer que pour $t<0$, la d\'eriv\'ee $n$-i\`eme de $f$ s'\'ecrit
				$$f^{(n)}(t)=\frac{P_n(t)}{t^{2n}}e^{1/t}$$
				o\`u $P_n$ est un polyn\^ome.
				\begin{enumerate}
					\item Trouver $P_1$ et $P_2$.
					\item Trouver une relation de r\'ecurrence entre $P_{n+1}, P_n$ et $P'_n$ pour
					$n\in \N^*$.
				\end{enumerate}
				\item Montrer que $f$ est de classe $C^{\infty}$.
			\end{enumerate}
			\centering
			\rule{1\linewidth}{0.6pt}
		\end{exo}
		
		

		
		
		
	\end{minipage}	
	\hfill\vrule\hfill
	\begin{minipage}[t]{0.48\linewidth}
		\raggedright
		
				\begin{exo}\textbf{(**)}\quad\\[0.2cm]
			Déterminer dans chacun des cas suivants la dérivée $n$-ème de la fonction proposée~:
			\begin{multicols}{2}
				\begin{enumerate}
				\item $x\mapsto x^{n-1}\ln(1+x) $
				\item $x\mapsto\cos^3x\sin(2x) $
				\item $x\mapsto\frac{x^2+1}{(x-1)^3}$
				\item $x\mapsto(x^3+2x-7)e^x$
			\end{enumerate}
			\end{multicols}
	
			\centering
			\rule{1\linewidth}{0.6pt}
			\end{exo}
		
				\begin{exo}\textbf{(**)}\quad\\[0.2cm]
			Montrer que la fonction définie sur $\R$ par $f(x)=e^{-1/x^2}$ si $x\neq0$ et $0$ si $x=0$ est de classe $C^\infty$ sur $\R$.
			
			
			\centering
			\rule{1\linewidth}{0.6pt}
		\end{exo}
	
		\begin{exo}\textbf{(***)}\quad\\[0.2cm]
		
		Soit $f:\mathbb R\to\mathbb R$ dérivable telle que $f(0)=0$. Montrer que
		$\sum_{k=1}^n f\left(\frac k{n^2}\right)$ admet une limite lorsque $n\to+\infty$ et la déterminer.
		
		
		\centering
		\rule{1\linewidth}{0.6pt}
	\end{exo}
	

		
		
	\end{minipage}
\end{minipage}

\section*{Propriété des accroissements finis :}\hfill\\%[-0.25cm]
\begin{minipage}{1\linewidth}
	\begin{minipage}[t]{0.48\linewidth}
		\raggedright
		
		\begin{exo}\textbf{(*)}\quad\\[0.2cm]
		Démontrer les inégalités suivantes : 
		\begin{enumerate}
			\item $\forall x,y\in\mathbb R,\ |\arctan(x)-\arctan(y)|\leq |x-y|$.
			\item $\forall x\geq 0$, $x\leq e^x-1\leq xe^x$.
		\end{enumerate}
	
		\centering
	\rule{1\linewidth}{0.6pt}
\end{exo}
		
		
		\begin{exo}\textbf{(*)}\quad\\[0.2cm]
	Dans l'application du th\'eor\`eme des accroissements finis
	\`a la fonction
	$$ f(x) =\alpha x^2+\beta x+\gamma$$
	sur l'intervalle $[a,b]$
	pr\'eciser le nombre ``$c$'' de $]a,b[$.
	Donner une interpr\'etation g\'eom\'etrique.
	
	\centering
	\rule{1\linewidth}{0.6pt}
\end{exo}
		
		
	\end{minipage}	
	\hfill\vrule\hfill
	\begin{minipage}[t]{0.48\linewidth}
		\raggedright
		
		\begin{exo}\textbf{(***)} \quad\\[0.2cm]
			\textit{"Formule de \textsc{Taylor-Lagrange}"}\\[0.4cm] 
			
			Soient $a$ et $b$ deux réels tels que $a<b$ et $n$ un entier naturel. Soit $f$ une fonction élément de $C^n([a,b],\R)\cap D^{n+1}(]a,b[,\R)$. Montrer qu'il existe $c\in]a,b[$ tel que 
			
			$$f(b)=\sum_{k=0}^{n}\frac{f^{(k)}(a)}{k!}(b-a)^k+\frac{(b-a)^{n+1}f^{(n+1)}(c)}{(n+1)!}.$$
			
			Indication. Appliquer le théorème de \textsc{Rolle} à la fonction $g(x)=f(b)-\sum_{k=0}^{n}\frac{f^{(k)}(x)}{k!}(b-x)^k-A\frac{(b-x)^{n+1}}{(n+1)!}$ où $A$ est intelligemment choisi.
			
			\centering
			\rule{1\linewidth}{0.6pt}
		\end{exo}

		
		
	\end{minipage}
\end{minipage}
\newpage

\begin{minipage}{1\linewidth}
	\begin{minipage}[t]{0.48\linewidth}
		\raggedright
		
		\begin{exo}\textbf{(**)} \textit{"Règle de l'Hospital"}\quad\\[0.2cm]
			Soient $f,g :[a , b] \longrightarrow \R$ deux
			fonctions continues
			sur $[a, b]$ ($a < b$) et d\'erivables sur $]a , b[.$ On suppose que
			$g' (x) \neq 0$ pour tout $x \in ]a , b[.$
			
			\begin{enumerate}
				\item
				Montrer que $g (x) \neq g (a)$ pour tout $x \in ]a , b[.$
				
				\item Posons $p = \frac{f (b) - f (a)}{g (b) - g (a)}$ et
				consid\'erons  la fonction $h (x) = f
				(x) - p g (x)$ pour $x \in [a , b].$
				Montrer que $h$ v\'erifie les hypoth\`eses du th\'eor\`eme de Rolle
				et en d\'eduire qu'il existe un nombre r\'eel $c \in ]a , b[$ tel que
				$$ \frac{f (a) - f (b)}{g (a) - g (b)} = \frac{f' (c)}{g' (c)}.$$
				\item On suppose que $\lim_{x \to b^{-}} \frac{f' (x)}{g' (x)} =
				\ell,$ o\`u $\ell $ est un nombre r\'eel.
				Montrer que  $$ \lim_{x \to b^{-}} \frac{f (x) - f (b)}{g (x) - g (b)} = \ell.$$
				\item  \emph{Application.} Calculer la limite suivante:
				$$ \lim_{x \to 1^{-}} \frac{\arccos x}{\sqrt{1- x^{2}}}.$$
			\end{enumerate}
			\centering
			\rule{1\linewidth}{0.6pt}
		\end{exo}
		
		\begin{exo}\textbf{(**)}\quad\\[0.2cm]
			Soit $f:[0,1]\to\mathbb R$ une fonction de classe $C^1$ vérifiant $f(0)=0$ et $f(1)=1$.
			Démontrer que, pour tout $n\geq 1$, il existe $0<x_1<\dots<x_n<1$ vérifiant $f'(x_1)+\dots+f'(x_n)=n$.
			
			\centering
			\rule{1\linewidth}{0.6pt}
		\end{exo}
	
	\begin{exo}\textbf{(**)}\quad\\[0.2cm]
	On considère la suite récurrente définie par $u_0\in \mathbb R^*$ et $u_{n+1}=f(u_n)$ pour tout $n\in\mathbb N$,
	où $f$ la fonction définie par $f(x)=1+\frac 14\sin\frac 1x$.
	\begin{enumerate}
		\item Déterminer $I=f(\mathbb R^*)$, et montrer que $I$ est stable par $f$.
		\item Démontrer qu'il existe $\gamma\in I$ tel que $f(\gamma)=\gamma$.
		\item Démontrer que, pour tout $x\in I$, 
		$$|f'(x)|\leq\frac 49.$$
		\item Démontrer que $(u_n)$ converge vers $\gamma$.
	\end{enumerate}

	\centering
	\rule{1\linewidth}{0.6pt}
	\end{exo}

		
		
	\end{minipage}	
	\hfill\vrule\hfill
	\begin{minipage}[t]{0.48\linewidth}
		\raggedright
		
		
		
		
		
		\begin{exo}\textbf{(***)} \ \textit{"Théorème de \textsc{Darboux}"}\quad\\[0.2cm]
			Soit $I$ un intervalle ouvert de $\R$, et $f$ une fonction dérivable sur $I$. On veut prouver que $f'$ vérifie le théorème des valeurs intermédiaires.
			\begin{enumerate}
				\item Pourquoi n'est-ce pas un résultat direct?
				\item Soit $(a,b)\in I^2$, tel que $f'(a)<f'(b)$, et soit $z\in]f'(a),f'(b)[$. Montrer qu'il existe $\alpha>0$ tel que, pour tout réel $h\in]0,\alpha]$, on ait :
				$$\frac{1}{h}\left(f(a+h)-f(a)\right)<z<\frac{1}{h}\left(f(b+h)-f(b)\right).$$
				\item En déduire l'existence d'un réel $h>0$ et d'un point $y$ de $I$ tels que :
				$$y+h\in I \textrm{ et }\frac{1}{h}\left(f(y+h)-f(y)\right)=z.$$
				\item Montrer qu'il existe un point $x$ de $I$ tel que $z=f'(x)$.
				\item En déduire que $f'(I)$ est un intervalle.
				\item Soit $f(x)=x^2\sin\left(\frac{1}{x^2}\right)$ sur $]0,1]$ et $0$ en $0$. Montrer que $f$ est dérivable sur $[0,1]$.
				
				 $f'$ est-elle continue sur $[0,1]$? 
				 
				 Déterminer $f'([0,1])$. Qu'en concluez-vous?
			\end{enumerate}
			
			\centering
			\rule{1\linewidth}{0.6pt}
		\end{exo}
		
		
		
		\begin{exo}\textbf{(**)} \textit{"Une approximation de $e$"}\quad\\[0.2cm]
			
			On note $f$ la fonction définie sur $[1,e]$ par $f(x)=\frac{2x}{\ln (x)+1}$ et $g$ la fonction définie sur $[0,1]$
			
			 par $g(y)=\frac{2y}{(1+y)^2}$. 
			\begin{enumerate}
				\item Démontrer que, pour tout $y\in [0,1]$, $0\leq g(y)\leq \frac{1}2$. 
				\item \'Etudier $f$ et démontrer que l'intervalle $[1,e]$ est stable par $f$. 
				\item Démontrer que, pour tous $x,y\in [1,e]$, $|f(x)-f(y)|\leq \frac 12|x-y|$ (on pourra utiliser le résultat de la première question).
				\item On définit une suite $(u_n)$ par $u_0=1$ et $u_{n+1}=f(u_n)$. Démontrer que, pour tout 
				$n\geq 0$, $|u_n-e|\leq\frac{e-1}{2^n}$. Que peut-on en déduire sur $(u_n)$?
				\item Déterminer un rang $n$ pour lequel $u_n$ est une approximation de $e$ à $10^{-3}$ près.
			\end{enumerate}
			
			
			\centering
			\rule{1\linewidth}{0.6pt}
		\end{exo}

		
		
	\end{minipage}
\end{minipage}

\section*{Convexité :}\hfill\\%[-0.25cm]
\begin{minipage}{1\linewidth}
	\begin{minipage}[t]{0.48\linewidth}
		\raggedright
		
		
		\begin{exo}\textbf{(*)}\quad\\[0.2cm]
			Soit $n\geq 2$. 
			\begin{enumerate}
				\item \'Etudier la convexité de la fonction $f$ définie sur $[-1;+\infty[$ par $f(x)=(1+x)^n$. 
				\item En déduire que, pour tout $x\geq -1$, $(1+x)^n\geq 1+nx$.
			\end{enumerate}
			
			\centering
			\rule{1\linewidth}{0.6pt}
		\end{exo}
		
		
		\begin{exo}\textbf{(*)}\quad\\[0.2cm]
			Soit $f,g:I\to\mathbb R$ deux fonctions convexes, avec $I\subset \mathbb R$ un intervalle.
			\begin{enumerate}
				\item Est-ce que $\max(f,g)$ est toujours convexe?
				\item Est-ce que $\min(f,g)$ est toujours convexe?
			\end{enumerate}
			
			
			\centering
			\rule{1\linewidth}{0.6pt}
		\end{exo}
		
		\begin{exo}\textbf{(**)}\quad\\[0.2cm]
			Soit $f:\mathbb R\to\mathbb R$ une fonction convexe dérivable possédant une limite finie en $+\infty$.
			\begin{enumerate}
				\item Démontrer que $f$ est décroissante sur $\mathbb R$.
				\item Démontrer que $f'$ tend vers $0$ en $+\infty$.
				\item Le résultat de la question précédente reste-t-il vrai si on ne suppose pas que $f$ est convexe?
			\end{enumerate}
			
			
			\centering
			\rule{1\linewidth}{0.6pt}
		\end{exo}
	
		\begin{exo}\textbf{(**)}\quad\\[0.2cm]
		Soit $f:\mathbb R\to\mathbb R$ une fonction convexe. 
		\begin{enumerate}
			\item On suppose que $\lim\limits_{+\infty}f=0$. Montrer que $f\geq 0$.
			\item Montrer que la somme d'une fonction convexe et d'une fonction affine est convexe.
			\item On suppose que la courbe représentative de $f$ admet une asymptote. Montrer que la courbe est (toujours) au-dessus
			de l'asymptote.
		\end{enumerate}
		
		\centering
		\rule{1\linewidth}{0.6pt}
	\end{exo}
	
		
		
		
	\end{minipage}	
	\hfill\vrule\hfill
	\begin{minipage}[t]{0.48\linewidth}
		\raggedright
		
		\begin{exo}\textbf{(**)}\quad\\[0.2cm]
			Soit $f$ une fonction convexe sur un intervalle ouvert $I$ de $\R$. Montrer que $f$ est continue sur $I$ et même dérivable à droite et à gauche en tout point de $I$.
			
			\centering
			\rule{1\linewidth}{0.6pt}
		\end{exo}
		
		
		
		\begin{exo}\textbf{(***)}\quad\\[0.2cm]
			\begin{enumerate}
				\item  Soient $x_1$, $x_2$,..., $x_n$, $n$ réels positifs ou nuls et $\alpha_1$,..., $\alpha_n$, $n$ réels strictement positifs tels que $\alpha_1+...+\alpha_n=1$.
				
				 Montrer que $x_1^{\alpha_1}..x_n^{\alpha_n}\leq\alpha_1x_1+...+\alpha_nx_n$.
				 
				  En déduire que
				  
				  $(\mbox{Inégalité \textsc{arithmético-géométrique}}).$\quad\\[-0.2cm]
				  
				   $$\sqrt[n]{x_1...x_n} \ \leq \ \dfrac{x_1+...+x_n}{n}$$
				\item  Soient $p$ et $q$ deux réels strictement positifs tels que $\frac{1}{p}+\frac{1}{q}=1$.
				\begin{enumerate}
					\item Montrer que, pour tous réels $a$ et $b$ positifs ou nuls, $ab\leq\frac{a^p}{p}+\frac{b^q}{q}$ avec égalité si et seulement si $a^p=b^q$. 
					\item Soient $a_1$,..., $a_n$ et $b_1$,..., $b_n$, $2n$ nombres complexes. Montrer que~:
					
					$(\mbox{Inégalité de \textsc{Hölder}}).$\quad\\[-0.8cm]
					
					$$\left|\sum_{k=1}^{n}a_kb_k\right|\leq\sum_{k=1}^{n}|a_kb_k|\leq\left(\sum_{k=1}^{n}|a_k|^p\right)^{1/p}\left(\sum_{k=1}^{n}|b_k|^q\right)^{1/q}$$
					
					\item Soit $p\geq1$, montrer que la fonction $x\mapsto x^p$ est convexe sur $\R^+$ et retrouver ainsi l'inégalité de \textsc{Hölder}.
					\item Trouver une démonstration dans le cas $p=q=2$ à l'aide d'une fonction polynomiale du second degré
					$(\mbox{Inégalité de \textsc{Cauchy}-\textsc{Schwarz}}).$ 
			
				\end{enumerate}
			\end{enumerate}
		
			\centering
			\rule{1\linewidth}{0.6pt}
		\end{exo}

		
		
	\end{minipage}
\end{minipage}
	

\end{document}



