
%%%%%%%%%%%%%%%%%% PREAMBULE %%%%%%%%%%%%%%%%%%

\documentclass[11pt,a4paper]{article}

\usepackage{amsfonts,amsmath,amssymb,amsthm}
\usepackage[utf8]{inputenc}
\usepackage[T1]{fontenc}
\usepackage[francais]{babel}
\usepackage{mathptmx}
\usepackage{fancybox}
\usepackage{graphicx}
\usepackage{ifthen}

\usepackage{tikz}   

\usepackage{hyperref}
\hypersetup{colorlinks=true, linkcolor=blue, urlcolor=blue,
pdftitle={Exo7 - Exercices de mathématiques}, pdfauthor={Exo7}}

\usepackage{geometry}
\geometry{top=2cm, bottom=2cm, left=2cm, right=2cm}

%----- Ensembles : entiers, reels, complexes -----
\newcommand{\Nn}{\mathbb{N}} \newcommand{\N}{\mathbb{N}}
\newcommand{\Zz}{\mathbb{Z}} \newcommand{\Z}{\mathbb{Z}}
\newcommand{\Qq}{\mathbb{Q}} \newcommand{\Q}{\mathbb{Q}}
\newcommand{\Rr}{\mathbb{R}} \newcommand{\R}{\mathbb{R}}
\newcommand{\Cc}{\mathbb{C}} \newcommand{\C}{\mathbb{C}}
\newcommand{\Kk}{\mathbb{K}} \newcommand{\K}{\mathbb{K}}

%----- Modifications de symboles -----
\renewcommand{\epsilon}{\varepsilon}
\renewcommand{\Re}{\mathop{\mathrm{Re}}\nolimits}
\renewcommand{\Im}{\mathop{\mathrm{Im}}\nolimits}
\newcommand{\llbracket}{\left[\kern-0.15em\left[}
\newcommand{\rrbracket}{\right]\kern-0.15em\right]}
\renewcommand{\ge}{\geqslant} \renewcommand{\geq}{\geqslant}
\renewcommand{\le}{\leqslant} \renewcommand{\leq}{\leqslant}

%----- Fonctions usuelles -----
\newcommand{\ch}{\mathop{\mathrm{ch}}\nolimits}
\newcommand{\sh}{\mathop{\mathrm{sh}}\nolimits}
\renewcommand{\tanh}{\mathop{\mathrm{th}}\nolimits}
\newcommand{\cotan}{\mathop{\mathrm{cotan}}\nolimits}
\newcommand{\Arcsin}{\mathop{\mathrm{arcsin}}\nolimits}
\newcommand{\Arccos}{\mathop{\mathrm{arccos}}\nolimits}
\newcommand{\Arctan}{\mathop{\mathrm{arctan}}\nolimits}
\newcommand{\Argsh}{\mathop{\mathrm{argsh}}\nolimits}
\newcommand{\Argch}{\mathop{\mathrm{argch}}\nolimits}
\newcommand{\Argth}{\mathop{\mathrm{argth}}\nolimits}
\newcommand{\pgcd}{\mathop{\mathrm{pgcd}}\nolimits} 

%----- Structure des exercices ------

\newcommand{\exercice}[1]{\video{0}}
\newcommand{\finexercice}{}
\newcommand{\noindication}{}
\newcommand{\nocorrection}{}

\newcounter{exo}
\newcommand{\enonce}[2]{\refstepcounter{exo}\hypertarget{exo7:#1}{}\label{exo7:#1}{\bf Exercice \arabic{exo}}\ \  #2\vspace{1mm}\hrule\vspace{1mm}}

\newcommand{\finenonce}[1]{
\ifthenelse{\equal{\ref{ind7:#1}}{\ref{bidon}}\and\equal{\ref{cor7:#1}}{\ref{bidon}}}{}{\par{\footnotesize
\ifthenelse{\equal{\ref{ind7:#1}}{\ref{bidon}}}{}{\hyperlink{ind7:#1}{\texttt{Indication} $\blacktriangledown$}\qquad}
\ifthenelse{\equal{\ref{cor7:#1}}{\ref{bidon}}}{}{\hyperlink{cor7:#1}{\texttt{Correction} $\blacktriangledown$}}}}
\ifthenelse{\equal{\myvideo}{0}}{}{{\footnotesize\qquad\texttt{\href{http://www.youtube.com/watch?v=\myvideo}{Vidéo $\blacksquare$}}}}
\hfill{\scriptsize\texttt{[#1]}}\vspace{1mm}\hrule\vspace*{7mm}}

\newcommand{\indication}[1]{\hypertarget{ind7:#1}{}\label{ind7:#1}{\bf Indication pour \hyperlink{exo7:#1}{l'exercice \ref{exo7:#1} $\blacktriangle$}}\vspace{1mm}\hrule\vspace{1mm}}
\newcommand{\finindication}{\vspace{1mm}\hrule\vspace*{7mm}}
\newcommand{\correction}[1]{\hypertarget{cor7:#1}{}\label{cor7:#1}{\bf Correction de \hyperlink{exo7:#1}{l'exercice \ref{exo7:#1} $\blacktriangle$}}\vspace{1mm}\hrule\vspace{1mm}}
\newcommand{\fincorrection}{\vspace{1mm}\hrule\vspace*{7mm}}

\newcommand{\finenonces}{\newpage}
\newcommand{\finindications}{\newpage}


\newcommand{\fiche}[1]{} \newcommand{\finfiche}{}
%\newcommand{\titre}[1]{\centerline{\large \bf #1}}
\newcommand{\addcommand}[1]{}

% variable myvideo : 0 no video, otherwise youtube reference
\newcommand{\video}[1]{\def\myvideo{#1}}

%----- Presentation ------

\setlength{\parindent}{0cm}

\definecolor{myred}{rgb}{0.93,0.26,0}
\definecolor{myorange}{rgb}{0.97,0.58,0}
\definecolor{myyellow}{rgb}{1,0.86,0}

\newcommand{\LogoExoSept}[1]{  % input : echelle       %% NEW
{\usefont{U}{cmss}{bx}{n}
\begin{tikzpicture}[scale=0.1*#1,transform shape]
  \fill[color=myorange] (0,0)--(4,0)--(4,-4)--(0,-4)--cycle;
  \fill[color=myred] (0,0)--(0,3)--(-3,3)--(-3,0)--cycle;
  \fill[color=myyellow] (4,0)--(7,4)--(3,7)--(0,3)--cycle;
  \node[scale=5] at (3.5,3.5) {Exo7};
\end{tikzpicture}}
}


% titre
\newcommand{\titre}[1]{%
\vspace*{-4ex} \hfill \hspace*{1.5cm} \hypersetup{linkcolor=black, urlcolor=black} 
\href{http://exo7.emath.fr}{\LogoExoSept{3}} 
 \vspace*{-5.7ex}\newline 
\hypersetup{linkcolor=blue, urlcolor=blue}  {\Large \bf #1} \newline 
 \rule{12cm}{1mm} \vspace*{3ex}}

%----- Commandes supplementaires ------



\begin{document}

%%%%%%%%%%%%%%%%%% EXERCICES %%%%%%%%%%%%%%%%%%
\fiche{f00013, bodin, 2007/09/01} 

\titre{Fonctions dérivables}

\section{Calculs}
\exercice{699, bodin, 1998/09/01}
\video{XPnRztcUKK4}
\enonce{000699}{}
D\'eterminer $a,b \in \Rr$ de mani\`ere \`a ce que la fonction $f$ d\'efinie sur $\Rr_+$ par :
$$ f(x)=\sqrt{x} \quad \text{ si } 0\leqslant x \leqslant 1 \quad \text{\ \  et\ \  } \quad f(x) = ax^2+bx+1 \quad \text{ si } x>1$$
soit d\'erivable sur $\Rr_+^*$.
\finenonce{000699} 


\finexercice
\exercice{700, bodin, 1998/09/01}
\video{K9yZt1R0xW8}
\enonce{000700}{}
Soit $f : \Rr^* \longrightarrow \Rr$ d\'efinie par
$\displaystyle{f(x)= x^2\sin \frac{1}{x} }$. Montrer que
$f$ est prolongeable par continuit\'e en $0$ ; on note encore
$f$ la fonction prolong\'ee. Montrer que $f$ est
d\'erivable sur $\Rr$ mais que $f'$ n'est pas continue en $0$.
\finenonce{000700} 


\finexercice
\exercice{698, bodin, 1998/09/01}
\video{yNbsJ0UALww}
\enonce{000698}{}
\'Etudier  la d\'erivabilit\'e des fonctions suivantes :
$$f_1(x)=x^2\cos \frac{1}{x}, \text{\ \  si }x\not=0 \qquad ; \qquad f_1(0)=0 ;$$

$$f_2(x)= \sin x \cdot \sin \frac{1}{x}, \text{\ \  si }x\not=0 \qquad ; \qquad f_2(0)=0 ;$$

$$f_3(x) = \frac{|x|\sqrt{x^2-2x+1}}{x-1}, \text{\ \  si } x\not= 1 \qquad ; \qquad f_3(1)=1.$$
\finenonce{000698}
 

\finexercice
\exercice{739, bodin, 2001/11/01}
\video{OFIk3Tj3QIA}
\enonce{000739}{}
 Soit $n \geq 2$ un entier fix\'e et $f:\R^{+} = [0,
+ \infty[ \longrightarrow \R$ la fonction d\'efinie par la formule
suivante:
$$ f (x) = \frac{1 + x^n}{(1 + x)^n}, \ \ x \geq 0.$$
\begin{enumerate}
  \item
  \begin{enumerate}
     \item Montrer que $f$ est d\'erivable sur $\R^{+}$ et calculer $f' (x)$
pour $x \geq 0.$
     \item En \'etudiant le signe de $f' (x)$ sur $\R^{+},$ montrer que $f$
atteint un minimum sur $\R^{+}$ que l'on d\'eterminera.
  \end{enumerate}
  \item
  \begin{enumerate}
     \item En d\'eduire l'in\'egalit\'e suivante:
$$ (1+ x)^n \leq 2^{n - 1} (1+ x^n), \ \ \forall x \in \R^{+}.$$
     \item Montrer que si  $x \in \R^{+}$ et $y \in \R^{+}$ alors on a
$$ (x + y)^n \leq 2^{n - 1} (x^n + y^n).$$
   \end{enumerate}
 \end{enumerate}
\finenonce{000739} 


\finexercice
\section{Théorème de Rolle et accroissements finis}
\exercice{717, legall, 1998/09/01}
\video{w_zY_6IuJnE}
\enonce{000717}{}
Montrer que le polyn\^ome $  X^n+aX+b  $,  ($  a  $ et $  b  $ r\' eels) admet au plus trois racines r\' eelles.
\finenonce{000717} 


\finexercice\exercice{715, bodin, 1998/09/01}
\video{J-dlr6vwO8A}
\enonce{000715}{}
Montrer que le polyn\^ome $P_n$ d\'efini par
$$P_n(t)=\left[ \left( 1-t^2 \right)^n \right]^{(n)}$$
est un polyn\^ome de degr\'e $n$ dont les racines sont
 r\'eelles, simples, et appartiennent \`a $[-1,1]$.
\finenonce{000715} 


\finexercice
\exercice{721, bodin, 1998/09/01}
\video{wAM4vww5WSo}
\enonce{000721}{}
Dans l'application du th\'eor\`eme des accroissements finis
\`a la fonction
$$ f(x) =\alpha x^2+\beta x+\gamma$$
sur l'intervalle $[a,b]$
pr\'eciser le nombre ``$c$'' de $]a,b[$.
Donner une interpr\'etation g\'eom\'etrique.
\finenonce{000721} 


\finexercice\exercice{724, bodin, 1998/09/01}
\video{CfqCgc6zKgg}
\enonce{000724}{}
Soient $x$ et $y$ r\'eels avec $0<x<y$.
\begin{enumerate}
    \item Montrer que
$$ x < \frac{y-x}{\ln y - \ln x} < y.$$
    \item On consid\`ere la fonction $f$ d\'efinie sur
$[0,1]$ par
$$\alpha \mapsto f(\alpha) = \ln (\alpha x +(1-\alpha)y)-\alpha
\ln x -(1-\alpha)\ln y.$$
De l'\'etude de $f$ d\'eduire que pour tout $\alpha$ de $]0,1[$
$$ \alpha
\ln x +(1-\alpha)\ln y < \ln (\alpha x +(1-\alpha)y) .$$
Interpr\'etation g\'eom\'etrique ?
\end{enumerate}
\finenonce{000724} 


\finexercice

\section{Divers}
\exercice{733, ridde, 1999/11/01}
\video{_YeLn2_0Tso}
\enonce{000733}{}
D\'eterminer les extremums de $f (x) = x^4-x^3 + 1$ sur $\Rr$.
\finenonce{000733} 


\finexercice\exercice{738, bodin, 2001/11/01}
\video{MqjLEkOQD3w}
\enonce{000738}{}
 Soient $f,g :[a , b] \longrightarrow \R$ deux
fonctions continues
 sur $[a, b]$ ($a < b$) et d\'erivables sur $]a , b[.$ On suppose que
 $g' (x) \neq 0$ pour tout $x \in ]a , b[.$

\begin{enumerate}
\item
 Montrer que $g (x) \neq g (a)$ pour tout $x \in ]a , b[.$

\item Posons $p = \frac{f (b) - f (a)}{g (b) - g (a)}$ et
consid\'erons  la fonction $h (x) = f
 (x) - p g (x)$ pour $x \in [a , b].$
 Montrer que $h$ v\'erifie les hypoth\`eses du th\'eor\`eme de Rolle
 et en d\'eduire qu'il existe un nombre r\'eel $c \in ]a , b[$ tel que
 $$ \frac{f (a) - f (b)}{g (a) - g (b)} = \frac{f' (c)}{g' (c)}.$$
\item On suppose que $\lim_{x \to b^{-}} \frac{f' (x)}{g' (x)} =
\ell,$ o\`u $\ell $ est un nombre r\'eel.
 Montrer que  $$ \lim_{x \to b^{-}} \frac{f (x) - f (b)}{g (x) - g (b)} = \ell.$$
\item  \emph{Application.} Calculer la limite suivante:
 $$ \lim_{x \to 1^{-}} \frac{\Arccos x}{\sqrt{1- x^{2}}}.$$
\end{enumerate}
\finenonce{000738} 


\finexercice\exercice{740, monthub, 2001/11/01}
\video{EVKKvovB3ps}
\enonce{000740}{}
On consid\`ere la fonction $f : \R \to \R$ d\'efinie par
\begin{equation*}
f(t) =
\begin{cases}
  e^{1/t} & \mathrm{\ si\ } t<0\\
0  & \mathrm{\ si\ } t \geq 0
\end{cases}
\end{equation*}

\begin{enumerate}
\item D\'emontrer  que $f$ est d\'erivable sur $\R$, en particulier en $t=0$.
\item Etudier l'existence de $f''(0)$.
\item On veut montrer que pour $t<0$, la d\'eriv\'ee $n$-i\`eme de $f$ s'\'ecrit
$$f^{(n)}(t)=\frac{P_n(t)}{t^{2n}}e^{1/t}$$
o\`u $P_n$ est un polyn\^ome.
\begin{enumerate}
\item Trouver $P_1$ et $P_2$.
\item Trouver une relation de r\'ecurrence entre $P_{n+1}, P_n$ et $P'_n$ pour
  $n\in \N^*$.
\end{enumerate}
\item Montrer que $f$ est de classe $C^{\infty}$.
\end{enumerate}
\finenonce{000740}


\finexercice
\finfiche

 \finenonces 



 \finindications 

\indication{000699}
Vous avez deux conditions : il faut que la fonction soit continue (car on veut qu'elle soit d\'erivable donc elle doit \^etre continue) et ensuite la condition de d\'erivabilit\'e proprement dite.
\finindication
\indication{000700}
$f$ est continue en $0$ en la prolongeant par $f(0)=0$.
$f$ est alors d\'erivable en $0$ et $f'(0)=0$.
\finindication
\indication{000698}
Les probl\`emes sont seulement en $0$ ou $1$. 
$f_1$ est d\'erivable en $0$ mais pas $f_2$.  $f_3$ n'est dérivable ni en $0$, ni en $1$.
\finindication
\noindication
\indication{000717}
On peut appliquer le th\'eor\`eme de Rolle plusieurs fois.
\finindication
\indication{000715}
Il faut appliquer le th\'eor\`eme de Rolle une fois au polyn\^ome $(1-t^2)^n$,
puis deux fois \`a sa d\'eriv\'ee premi\`ere, puis trois fois \`a sa d\'eriv\'ee seconde,...
\finindication
\noindication
\indication{000724}
\begin{enumerate}
    \item Utiliser le th\'eor\`eme des accroissements finis avec la fonction $t \mapsto \ln t$
    \item Montrer d'abord que $f''$ est n\'egative. Se servir du th\'eor\`eme des valeurs interm\'ediaires pour $f'$.
\end{enumerate}
\finindication
\noindication
\indication{000738}
\begin{enumerate}
 \item Raisonner par l'absurde et appliquer le th\'eor\`eme de Rolle.
 \item Calculer $h(a)$ et $h(b)$.
 \item Appliquer la question 2. sur l'intervalle $[x,b]$.
 \item Calculer $f'$ et $g'$.
\end{enumerate}
\finindication
\noindication


\newpage

\correction{000699}
La fonction $f$ est continue et dérivable sur $]0,1[$ et sur
$]1,+\infty[$. Le seul problème est en $x=1$.

Il faut d'abord que la fonction soit continue en $x=1$.
La limite \`a gauche est
$\lim_{x\rightarrow 1^-} \sqrt x = +1$
et \`a droite 
$\lim_{x\rightarrow 1^+} ax^2+bx+1 = a+b+1$.
Donc  $a+b+1=1$. Autrement dit $b = -a$.

Il faut maintenant que les d\'eriv\'ees \`a droite et \`a gauche soient
\'egales.
Comme la fonction $f$ restreinte à $]0,1]$ est définie par $x \mapsto \sqrt{x}$ alors elle est dérivable
à gauche et la dérivée à gauche s'obtient en évaluant la fonction dérivée $x \mapsto \frac{1}{2\sqrt x}$
en $x=1$. Donc $f'_g(1)=\frac 12$.

Pour la dérivée à droite il s'agit de calculer la limite du taux d'accroissement
$\frac{f(x)-f(1)}{x-1}$, lorsque $x \to 1$ avec $x>1$.
Or 
$$
\frac{f(x)-f(1)}{x-1} = \frac{ax^2+bx+1 - 1}{x-1} = \frac{ax^2-ax}{x-1} = \frac{ax(x-1)}{x-1} = ax.
$$
Donc $f$ est dérivable à droite et $f'_d(1) = a$.
Afin que $f$ soit dérivable, il faut et il suffit que les dérivées à droite et à gauche
existent et soient égales, donc ici la condition est $a=\frac 12$.


Le seul couple $(a,b)$ que rend $f$ dérivable sur $]0,+\infty[$ 
est $(a=\frac 12, b= -\frac 12)$.
\fincorrection
\correction{000700}
$f$ est $C^\infty$ sur $\Rr^*$.
\begin{enumerate}
  \item Comme $|\sin (1/x)| \leq 1$ alors
$f$ tend vers $0$ quand $x\rightarrow 0$. Donc en prolongeant $f$ par $f(0) =0$,
la fonction  $f$ prolongée est continue sur $\Rr$.
  \item Le taux d'accroissement est
$$\frac{f(x)-f(0)}{x-0}= x \sin\frac{1}{x}.$$
Comme ci-dessus il y a une limite (qui vaut $0$) en $x=0$.
Donc $f$ est d\'erivable en $0$ et $f'(0)=0$.
  \item Sur $\Rr^*$, $f'(x) = 2x\sin (1/x) -\cos(1/x)$,
Donc $f'(x)$ n'a pas de limite quand $x\rightarrow 0$.
Donc $f'$ n'est pas continue en $0$.
\end{enumerate}
\fincorrection
\correction{000698}
\begin{enumerate}
  \item La fonction $f_1$ est d\'erivable en dehors de $x=0$.
En effet $x \mapsto \frac 1x$ est dérivable sur $\Rr^*$ et $x \mapsto \cos x$ est dérivable sur $\Rr$,
donc par composition $x \mapsto  \cos \frac 1x$ est dérivable sur $\Rr^*$. Puis par multiplication par la fonction dérivable 
$x \mapsto x^2$, la fonction $f_1$ est dérivable sur $\Rr^*$. 
Par la suite on omet souvent ce genre de discussion ou on l'abrège sous la forme ``$f$ est dérivable sur $I$ comme somme, produit,
composition de fonctions dérivables sur $I$''.

Pour savoir si $f_1$ est d\'erivable en $0$ regardons le taux d'accroissement:
$$ \frac{f_1(x)-f_1(0)}{x-0}= x\cos \frac 1 x.$$
Mais $x \cos (1/x)$ tend vers $0$ (si $x\rightarrow 0$) car
$|\cos (1/x)| \leq 1$.
Donc le taux d'accroissement tend vers $0$. Donc $f_1$ est d\'erivable en $0$ et $f_1'(0)=0$.
  \item Encore une fois $f_2$ est d\'erivable en dehors de $0$.
Le taux d'accroissement en $x=0$ est :
$$ \frac{f_2(x)-f_2(0)}{x-0}= \frac{\sin x}{x} \sin \frac 1 x$$
Nous savons que $\frac{\sin x}{x} \rightarrow 1$ et que
$\sin 1/x$ n'a pas de limite quand $x\rightarrow 0$. Donc le taux d'accroissement n'a pas de limite, donc $f_2$ n'est pas d\'erivable en $0$.
  \item 
La fonction $f_3$ s'\'ecrit :
$$f_3(x) = \frac{|x||x-1|}{x-1}.$$
\begin{itemize}
  \item
  Donc pour $x \geq 1$ on a $f_3(x) = x$ ;
pour $0 \leq x < 1$ on a $f_3(x) = -x$ ;
pour $x <0$ on a $f_3(x) = x$.

\item La fonction $f_3$ est d\'efinie, continue et d\'erivable sur 
$\Rr \setminus \{0,1\}$. Attention ! La fonction $x \mapsto |x|$ n'est pas dérivable en $0$.

\item 
La fonction $f_3$ n'est pas continue en $1$, en effet
$\lim_{x \rightarrow 1+} f_3(x) = +1$ et  $\lim_{x \rightarrow 1-} f_3(x) = -1$. Donc la fonction n'est pas d\'erivable en $1$.

\item
La fonction $f_3$ est continue en $0$. 
Le taux d'accroissement pour $x>0$ est 
$$\frac{f_3(x)-f_3(0)}{x-0}= \frac{-x}{x} = -1$$
et pour $x <0$, 
$$\frac{f_3(x)-f_3(0)}{x-0}= \frac{x}{x} = +1.$$
Donc le taux d'accroissement n'a pas de limite en $0$ et donc $f_3$ n'est pas d\'erivable en $0$.
\end{itemize}
\end{enumerate}
\fincorrection
\correction{000739}
\begin{enumerate}
\item
  \begin{enumerate}
  \item
 Il est clair que la fonction $f$ est d\'erivable sur $\R^{+}$
puisque c'est une fonction rationnelle sans p\^ole dans cet
intervalle. De plus d'apr\`es la formule de la d\'eriv\'ee d'un
quotient, on obtient pour $x \geq 0$ :
$$ f' (x) = \frac{n (x^{n-1} - 1)}{(1 + x)^{n + 1}}.$$
  \item  Par l'expression pr\'ec\'edente $f' (x)$ est du signe de $x^{n - 1} - 1$ sur $\R^{+}.$ Par
cons\'equent on obtient: $ f' (x) \leq 0$ pour $0 \leq x \leq 1$
et $ f' (x) \geq 0$ pour $x \geq 1.$ Il en r\'esulte que $f$ est
d\'ecroissante sur $[0 , 1]$ et croissante sur $[1, + \infty[$ et
par suite $f$ atteint son minimum sur $\R^{+}$ au point $1$ et ce
minimum vaut $f (1) = 2^{1 - n}.$
  \end{enumerate}
\item
  \begin{enumerate}
  \item
 Il   r\'esulte de la
question 1.b que $f (x) \geq f (1)$ pour tout $x \in \R^{+}$ et
donc
$$ (1 + x)^n \leq 2^{n - 1} (1 + x^n).$$
 \item En appliquant l'in\'egalit\'e pr\'ec\'edente avec $ x = b \slash
a,$ on en d\'eduit imm\'ediatement l'in\'egalit\'e requise (le cas du couple $(0,0)$ étant trivial).
  \end{enumerate}
\end{enumerate}
\fincorrection
\correction{000717}
\begin{enumerate}
\item Par l'absurde on suppose qu'il y a (au moins) quatre racines distinctes  pour 
$  P_n(X) =  X^n+aX+b  $. Notons les $x_1 <x_2<x_3<x_4$.
Par le th\'eor\`eme de Rolle appliqu\'e trois fois (entre $x_1$ et $x_2$, entre $x_2$ et $x_3$,...) il existe $x'_1<x'_2<x'_3$ des racines de $P'_n$. On applique deux fois le th\'eor\`eme Rolle entre $x'_1$ et $x'_2$ et entre $x'_2$ et $x'_3$. On obtient deux racines distinctes pour $P''_n$. Or 
$P''_n = n(n-1)X^{n-2}$ ne peut avoir que $0$ comme racines. Donc nous avons 
obtenu une contradiction. 
\item \emph{Autre m\'ethode :} 
Le r\'esultat est \'evident si $  n\leq 3  .$ On suppose donc $
n\geq 3  .$ Soit $  P_n  $ l'application $  X\mapsto X^n+aX+b  $
de $  \R   $ dans lui-m\^eme. Alors $  P'_n(X)=nX^{n-1}+a  $
s'annule en au plus deux valeurs. Donc $P_n$ est successivement
croissante-d\'ecroissante-croissante ou bien
d\'ecroissante-croissante-d\'ecroissante. Et donc $P_n$ s'annule au plus 
trois fois.
\end{enumerate}
\fincorrection
\correction{000715}
$Q_n(t) = (1-t^2)^n$ est un polyn\^ome de degr\'e $2n$, on le d\'erive $n$ fois, on obtient un polyn\^ome de degr\'e $n$.
Les valeurs $-1$ et $+1$ sont des racines d'ordre $n$ de $Q_n$, donc
$Q_n(1)=Q_n'(1)=\ldots = Q_n^{(n-1)}(1)=0$. M\^eme chose en $-1$.
Enfin $Q(-1)=0=Q(+1)$ donc d'apr\`es le th\'eor\`eme de Rolle il existe $c \in ]-1,1[$ telle que $Q_n'(c)=0$. 

Donc $Q_n'(-1)=0$, $Q_n'(c)=0$, $Q_n'(-1)=0$. En appliquant le th\'eor\`eme de Rolle deux fois 
(sur $[-1,c]$ et sur $[c,+1]$), on obtient l'existence
de racines $d_1,d_2$ pour $Q_n''$, qui s'ajoutent aux racines $-1$ et $+1$.


On continue ainsi par r\'ecurrence. On obtient pour $Q_n^{(n-1)}$,
$n+1$ racines: $-1, e_1,\ldots, e_{n-1}, +1$. Nous appliquons le th\'eor\`eme de Rolle $n$ fois. Nous obtenons $n$ racines pour 
$P_n = Q_n^{(n)}$.  Comme un polyn\^ome de degr\'e $n$ a au plus
$n$ racines, nous avons obtenu toutes les racines. Par constructions ces racines sont r\'eelles distinctes, donc simples.
\fincorrection
\correction{000721}
La fonction $f$ est continue et dérivable sur $\Rr$ donc en particulier sur $[a,b]$.
Le théorème des accroissement finis assure l'existence d'un nombre $c \in ]a,b[$ tel que
$f(b)-f(a) = f'(c) (b-a)$.

Mais pour la fonction particulière de cet exercice nous pouvons expliciter ce $c$.
En effet $f(b)-f(a) = f'(c) (b-a)$ implique $\alpha(b^2-a^2)+\beta(b-a)= (2\alpha c+\beta)(b-a)$.
Donc $c = \frac{a+b}{2}$. 

G\'eom\'etriquement, le graphe $\mathcal{P}$ de $f$ est une parabole.
Si l'on prend deux points $A = (a,f(a))$ et $B = (b,f(b))$ appartenant 
à cette parabole, alors la droite $(AB)$
est parall\`ele \`a la tangente en $\mathcal{P}$ qui passe en $M=(\frac{a+b}{2}, f(\frac{a+b}{2}))$.
L'abscisse de $M$ étant le milieu des abscisses de $A$ et $B$.
\fincorrection
\correction{000724}
\begin{enumerate}
  \item Soit $g(t) = \ln t$. Appliquons le th\'eor\`eme des accroissements finis sur $[x,y]$. Il existe $c \in ]x,y[$, 
$g(y)-g(x) = g'(c)(y-x)$. Soit $\ln y - \ln x = \frac 1c (y-x)$.
Donc $\frac{\ln y - \ln x}{y-x} = \frac 1c$.
Or $x <c<y$ donc $\frac 1y < \frac 1c < \frac 1x$.
Ce qui donne les in\'egalit\'es recherch\'ees. 
  \item $f'(\alpha)= \frac{x-y}{\alpha x + (1-\alpha)y} - \ln x + \ln y$. Et $f''(\alpha) = -\frac{(x-y)^2}{(\alpha x + (1-\alpha)y)^2}$.
Comme $f''$ est n\'egative alors $f'$ est d\'ecroissante sur $[0,1]$.
Or $f'(0) = \frac{x-y - y(\ln x - \ln y)}{y} >0$ d'apr\`es la premi\`ere question et de m\^eme $f'(1) < 0$. Par le th\'eor\`eme des valeurs interm\'ediaires, il existe $c \in [x,y]$ tel que $f'(c) = 0$.
Maintenant $f'$ est positive sur $[0,c]$ et n\'egative sur $[c,1]$.
Donc $f$ est croissante sur $[0,c]$ et d\'ecroissante sur $[c,1]$.
Or $f(0)=0$ et $f(1)=0$ donc pour tout $x\in[0,1]$, $f(x) \geq 0$.
Cela prouve l'in\'egalit\'e demand\'ee.
  \item G\'eom\'etriquement nous avons prouv\'e que la fonction $\ln$
est concave, c'est-\`a-dire que la corde
(le segment qui va de $(x,f(x))$ \`a $(y,f(y)$) est sous la courbe d'\'equation $y=f(x)$.
\end{enumerate}
\fincorrection
\correction{000733}
$f'(x) = 4x^3-3x^2 = x^2(4x-3)$ donc les extremums
appartiennent à  $\{0,\frac 34\}$. Comme $f''(x) = 12x^2-6x
= 6x(2x-1)$. Alors $f''$ ne s'annule pas en $\frac 34$, donc
$\frac 34$ donne un extremum local (qui est même un minimum global).
Par contre $f''(0) = 0$ et $f'''(0)\not=0$ donc $0$ est
un point d'inflexion qui n'est pas un extremum (m\^eme pas local,
pensez \`a un fonction du type  $x \mapsto x^3$). 
\fincorrection
\correction{000738}

 Le th\'eor\`eme de Rolle dit que si $h :[a,b]
\longrightarrow \R$ est une fonction continue sur l'intervalle
ferm\'e $[a,b]$ et d\'erivable sur l'ouvert $]a,b[$ alors il
existe $c \in ]a,b[$ tel que $h' (c) = 0.$

  \begin{enumerate}
  \item Supposons par l'absurde, qu'il existe $x_{0}\in ]a,b]$  tel
que $g (x_{0}) = g (a).$ Alors en appliquant le th\'eor\`eme de
Rolle \`a la restriction de $g$ \`a l'intervalle $[a,x_{0}]$ (les
hypoth\`eses \'etant clairement v\'erifi\'ees), on en d\'eduit
qu'il existe $c \in ]a,x_{0}[$ tel que $g' (c) = 0,$ ce qui
contredit les hypoth\`eses faites sur $g.$ Par cons\'equent on a
d\'emontr\'e que $g (x) \neq g (a)$ pour tout $x \in ]a,b].$

  \item D'apr\`es la question pr\'ec\'edente, on a en particulier  $g (b)
\neq g (a)$ et donc $p$ est un nombre r\'eel bien d\'efini et $h =
f - p \cdot g$ est alors une fonction continue sur $[a,b]$ et
d\'erivable sur $]a,b[.$ Un calcul simple montre que $h (a) = h
(b).$ D'apr\`es le th\'eor\`eme de Rolle il en r\'esulte  qu'il
existe $c \in ]a,b[$ tel que $h' (c) = 0.$ Ce qui implique la
relation requise.

  \item  Pour chaque $x \in ]a,b[,$ on peut appliquer
la question 2. aux
 restrictions de $f$ et $g$ \`a l'intervalle $[x,b],$ on en d\'eduit qu'il
 existe un point $c (x) \in ]x,b[,$ d\'ependant de $x$ tel que
 $$\frac{f (x) - f (b)}{g (x) - g (b)} = \frac{f' (c (x))}{g' (c (x))} .
 \leqno (*)$$
 Alors, comme $\lim_{x \to b^{-}} \frac{f' (t)}{g' (t)} = \ell$ et
 $\lim_{x \to b^{-}} c (x) = b,$ (car $c(x) \in ]x,b[$) on en d\'eduit en passant \`a la
 limite dans $(*)$ que
 $$ \lim_{x \to b^{-}} \frac{f (x) - f (b)}{g (x) - g (b)} = \ell.$$
 Ce r\'esultat est connu sous le nom de ``r\`egle de
 l'H\^opital''.

 \item Consid\'erons les deux fonctions $f
 (x) = \Arccos x$ et $g (x) = \sqrt{1-x^2}$ pour $x \in [0,1].$ Ces fonctions sont continues sur  $[0,1]$ et
 d\'erivables sur $]0,1[$ et $f' (x) = - 1 \slash \sqrt{1-x^2}$,
 $g' (x) = - x \slash \sqrt{1-x^2} \neq 0$ pour tout $ x
 \in ]0,1[.$ En appliquant les r\'esultats de la question 3., on en
 d\'eduit que
 $$ \lim_{x \to 1^{-}} \frac{\Arccos x}{\sqrt{1-x^2}} = \lim_{x \to 1^{-}} \frac{\frac{-1}{\sqrt{1-x^2}}} {\frac{-x}{\sqrt{1-x^2}}}= \lim_{x \to 1^{-}} \frac{1}{x} = 1.$$
\end{enumerate}
\fincorrection
\correction{000740}
\begin{enumerate}
\item $f$ est d\'erivable sur $\R_-^*$ en tant que compos\'ee de fonctions
  d\'erivables, et sur $\R_+^*$ car elle est nulle sur cet intervalle ; \'etudions donc la
  d\'erivabilit\'e en $0$.

On a
$$\frac{f(t)-f(0)}{t}=\begin{cases}
  e^{1/t}/t & \mathrm{\ si\ } t<0\\
0  & \mathrm{\ si\ } t > 0
\end{cases}$$
or $e^{1/t}/t$ tend vers $0$ quand $t$ tend vers $0$ par valeurs
n\'egatives. Donc $f$ est d\'erivable \`a gauche et \`a droite en 0 et ces
d\'eriv\'ees sont identiques, donc $f$ est d\'erivable et $f'(0)=0$.
\item On a
$$f'(t)=\begin{cases}
  -e^{1/t}/t^2 & \mathrm{\ si\ } t<0\\
0  & \mathrm{\ si\ } t \geq 0
\end{cases}$$
donc le taux d'accroissement de $f'$ au voisinage de 0 est
$$\frac{f'(t)-f'(0)}{t}=\begin{cases}
  -e^{1/t}/t^3& \mathrm{\ si\ } t<0\\
0  & \mathrm{\ si\ } t > 0
\end{cases}$$
et il tend vers $0$ quand $t$ tend vers $0$ par valeurs sup\'erieures
comme inf\'erieures. Donc $f$ admet une d\'eriv\'ee seconde en $0$, et
$f''(0)=0$.

\item
  \begin{enumerate}
  \item On a d\'ej\`a trouv\'e que $f'(t)=-e^{1/t}/t^2$, donc  $f'(t)=P_1(t)/t^2
    e^{1/t}$ si on pose $P_1(t)=-1$.

Par ailleurs, $f''(t)=e^{1/t}/t^4+e^{1/t}
(2/t^3)=\frac{1+2t}{t^4}e^{1/t}$ donc la formule est vraie pour
$n=2$ en posant $P_2(t)=1+2t$.
\item Supposons que la formule est vraie au rang $n$.
Alors $f^{(n)}(t)=\frac{P_n(t)}{t^{2n}}e^{1/t}$ d'o\`u
\begin{equation*}
\begin{split}
  f^{(n+1)}(t)&=\frac{P'_n(t)t^{2n}-P_n(t)(2n)t^{2n-1}}{t^{4n}}e^{1/t}+\frac{P_n(t)}{t^{2n}}e^{1/t}(-1/t^2)\\
&= \frac{P'_n(t)t^{2}-(2n t+1) P_n(t)}{t^{2(n+1)}}e^{1/t}
\end{split}
\end{equation*}
donc la formule est vraie au rang $n+1$ avec
$$P_{n+1}(t)=P'_n(t)t^{2}-(2n t+1) P_n(t).$$
  \end{enumerate}
\item Sur $\R_-^*$ et sur $\R_+^*$, $f$ est ind\'efiniment d\'erivable, donc il
  suffit d'\'etudier ce qui se passe en $0$.

Montrons  par r\'ecurrence que $f$ est ind\'efiniment d\'erivable en $0$,
et que pour tout   $n \in \N,   f^{(n)}(0)=0$. On sait que
  c'est vrai au rang 1. Supposons que $f$ est $n$-fois d\'erivable, et que
  $f^{(n)}(0)=0$. Alors le taux d'accroissement de $f^{(n)}$ en $0$ est :
$$\frac{f^{(n)}(t)-f^{(n)}(0)}{t}=\begin{cases}
  P_n(t)e^{1/t}/t^{2n+1} & \mathrm{\ si\ } t<0\\
0  & \mathrm{\ si\ } t > 0
\end{cases}$$
et sa limite est $0$ quand  $t$ tend vers $0$ par valeurs sup\'erieures
comme inf\'erieures. Donc $f^{(n)}$ est d\'erivable  en $0$, et
$f^{(n+1)}(0)=0$. Donc l'hypoth\`ese de r\'ecurrence est v\'erifi\'ee  au
rang $n+1$.

Par cons\'equent, $f$ est de classe $C^\infty$.
\end{enumerate}
\fincorrection


\end{document}

