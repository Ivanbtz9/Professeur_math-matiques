
%%%%%%%%%%%%%%%%%% PREAMBULE %%%%%%%%%%%%%%%%%%

\documentclass[11pt,a4paper]{article}

\usepackage{amsfonts,amsmath,amssymb,amsthm}
\usepackage[utf8]{inputenc}
\usepackage[T1]{fontenc}
\usepackage[francais]{babel}
\usepackage{mathptmx}
\usepackage{fancybox}
\usepackage{graphicx}
\usepackage{ifthen}

\usepackage{tikz}   

\usepackage{hyperref}
\hypersetup{colorlinks=true, linkcolor=blue, urlcolor=blue,
pdftitle={Exo7 - Exercices de mathématiques}, pdfauthor={Exo7}}

\usepackage{geometry}
\geometry{top=2cm, bottom=2cm, left=2cm, right=2cm}

%----- Ensembles : entiers, reels, complexes -----
\newcommand{\Nn}{\mathbb{N}} \newcommand{\N}{\mathbb{N}}
\newcommand{\Zz}{\mathbb{Z}} \newcommand{\Z}{\mathbb{Z}}
\newcommand{\Qq}{\mathbb{Q}} \newcommand{\Q}{\mathbb{Q}}
\newcommand{\Rr}{\mathbb{R}} \newcommand{\R}{\mathbb{R}}
\newcommand{\Cc}{\mathbb{C}} \newcommand{\C}{\mathbb{C}}
\newcommand{\Kk}{\mathbb{K}} \newcommand{\K}{\mathbb{K}}

%----- Modifications de symboles -----
\renewcommand{\epsilon}{\varepsilon}
\renewcommand{\Re}{\mathop{\mathrm{Re}}\nolimits}
\renewcommand{\Im}{\mathop{\mathrm{Im}}\nolimits}
\newcommand{\llbracket}{\left[\kern-0.15em\left[}
\newcommand{\rrbracket}{\right]\kern-0.15em\right]}
\renewcommand{\ge}{\geqslant} \renewcommand{\geq}{\geqslant}
\renewcommand{\le}{\leqslant} \renewcommand{\leq}{\leqslant}

%----- Fonctions usuelles -----
\newcommand{\ch}{\mathop{\mathrm{ch}}\nolimits}
\newcommand{\sh}{\mathop{\mathrm{sh}}\nolimits}
\renewcommand{\tanh}{\mathop{\mathrm{th}}\nolimits}
\newcommand{\cotan}{\mathop{\mathrm{cotan}}\nolimits}
\newcommand{\Arcsin}{\mathop{\mathrm{arcsin}}\nolimits}
\newcommand{\Arccos}{\mathop{\mathrm{arccos}}\nolimits}
\newcommand{\Arctan}{\mathop{\mathrm{arctan}}\nolimits}
\newcommand{\Argsh}{\mathop{\mathrm{argsh}}\nolimits}
\newcommand{\Argch}{\mathop{\mathrm{argch}}\nolimits}
\newcommand{\Argth}{\mathop{\mathrm{argth}}\nolimits}
\newcommand{\pgcd}{\mathop{\mathrm{pgcd}}\nolimits} 

%----- Structure des exercices ------

\newcommand{\exercice}[1]{\video{0}}
\newcommand{\finexercice}{}
\newcommand{\noindication}{}
\newcommand{\nocorrection}{}

\newcounter{exo}
\newcommand{\enonce}[2]{\refstepcounter{exo}\hypertarget{exo7:#1}{}\label{exo7:#1}{\bf Exercice \arabic{exo}}\ \  #2\vspace{1mm}\hrule\vspace{1mm}}

\newcommand{\finenonce}[1]{
\ifthenelse{\equal{\ref{ind7:#1}}{\ref{bidon}}\and\equal{\ref{cor7:#1}}{\ref{bidon}}}{}{\par{\footnotesize
\ifthenelse{\equal{\ref{ind7:#1}}{\ref{bidon}}}{}{\hyperlink{ind7:#1}{\texttt{Indication} $\blacktriangledown$}\qquad}
\ifthenelse{\equal{\ref{cor7:#1}}{\ref{bidon}}}{}{\hyperlink{cor7:#1}{\texttt{Correction} $\blacktriangledown$}}}}
\ifthenelse{\equal{\myvideo}{0}}{}{{\footnotesize\qquad\texttt{\href{http://www.youtube.com/watch?v=\myvideo}{Vidéo $\blacksquare$}}}}
\hfill{\scriptsize\texttt{[#1]}}\vspace{1mm}\hrule\vspace*{7mm}}

\newcommand{\indication}[1]{\hypertarget{ind7:#1}{}\label{ind7:#1}{\bf Indication pour \hyperlink{exo7:#1}{l'exercice \ref{exo7:#1} $\blacktriangle$}}\vspace{1mm}\hrule\vspace{1mm}}
\newcommand{\finindication}{\vspace{1mm}\hrule\vspace*{7mm}}
\newcommand{\correction}[1]{\hypertarget{cor7:#1}{}\label{cor7:#1}{\bf Correction de \hyperlink{exo7:#1}{l'exercice \ref{exo7:#1} $\blacktriangle$}}\vspace{1mm}\hrule\vspace{1mm}}
\newcommand{\fincorrection}{\vspace{1mm}\hrule\vspace*{7mm}}

\newcommand{\finenonces}{\newpage}
\newcommand{\finindications}{\newpage}


\newcommand{\fiche}[1]{} \newcommand{\finfiche}{}
%\newcommand{\titre}[1]{\centerline{\large \bf #1}}
\newcommand{\addcommand}[1]{}

% variable myvideo : 0 no video, otherwise youtube reference
\newcommand{\video}[1]{\def\myvideo{#1}}

%----- Presentation ------

\setlength{\parindent}{0cm}

\definecolor{myred}{rgb}{0.93,0.26,0}
\definecolor{myorange}{rgb}{0.97,0.58,0}
\definecolor{myyellow}{rgb}{1,0.86,0}

\newcommand{\LogoExoSept}[1]{  % input : echelle       %% NEW
{\usefont{U}{cmss}{bx}{n}
\begin{tikzpicture}[scale=0.1*#1,transform shape]
  \fill[color=myorange] (0,0)--(4,0)--(4,-4)--(0,-4)--cycle;
  \fill[color=myred] (0,0)--(0,3)--(-3,3)--(-3,0)--cycle;
  \fill[color=myyellow] (4,0)--(7,4)--(3,7)--(0,3)--cycle;
  \node[scale=5] at (3.5,3.5) {Exo7};
\end{tikzpicture}}
}


% titre
\newcommand{\titre}[1]{%
\vspace*{-4ex} \hfill \hspace*{1.5cm} \hypersetup{linkcolor=black, urlcolor=black} 
\href{http://exo7.emath.fr}{\LogoExoSept{3}} 
 \vspace*{-5.7ex}\newline 
\hypersetup{linkcolor=blue, urlcolor=blue}  {\Large \bf #1} \newline 
 \rule{12cm}{1mm} \vspace*{3ex}}

%----- Commandes supplementaires ------



\begin{document}

%%%%%%%%%%%%%%%%%% EXERCICES %%%%%%%%%%%%%%%%%%
\fiche{f00014, bodin, 2007/09/01} 

\titre{Fonctions circulaires et hyperboliques inverses}

Corrections de Léa Blanc-Centi. 

\section{Fonctions circulaires inverses}
\exercice{752, bodin, 1998/09/01}
\video{AjgAXbaAu3E}
\enonce{000752}{}
Vérifier
$$
\Arcsin x + \Arccos x = \frac{\pi}{2}\qquad\text{ et } \quad
\Arctan x + \Arctan\frac{1}{x} = \text{sgn}(x)\frac{\pi}{2}.
$$

\finenonce{000752} 

\finexercice\exercice{745, bodin, 1998/09/01}
\video{_HL7BV1u578}
\enonce{000745}{}
 Une statue de hauteur $s$ est placée sur un 
piédestal de hauteur $p$. 
\begin{enumerate}
\item \`A quelle distance $x_0$ doit se placer un observateur
(dont la taille est supposée négligeable) pour voir la statue sous un 
angle maximal $\alpha_0$? 
\item Vérifier que $\alpha_0=\Arctan \frac{s}{2\sqrt{p(p+s)}}$.
\item Application à la statue de la liberté : haute de $46$ mètres avec un piédestal de 
$47$ mètres.
\end{enumerate}
\finenonce{000745}
 

\finexercice\exercice{747, bodin, 1998/09/01}
\video{elLpf1K7wu4}
\enonce{000747}{}
\'Ecrire sous forme d'expression algébrique
\begin{enumerate}
\item $ \sin(\Arccos x),\quad \cos(\Arcsin x),\quad \cos(2 \Arcsin x)$.
\item $ \sin(\Arctan x),\quad \cos(\Arctan x),\quad \sin(3 \Arctan x)$.
\end{enumerate}
\finenonce{000747} 


\finexercice\exercice{749, bodin, 1998/09/01}
\video{s66EQAvtk84}
\enonce{000749}{}
Résoudre les équations suivantes:
\begin{enumerate}
\item $\Arccos x = 2\Arccos \frac{3}{4}$.
\item $\Arcsin x = \Arcsin \frac{2}{5} + \Arcsin \frac{3}{5}$.
\item $\Arctan {2x}+\Arctan x=\frac{\pi}{4}$.
\end{enumerate}
\finenonce{000749} 


\finexercice\exercice{6973, blanc-centi, 2014/05/06}
\video{70pEydvJEw4}
\enonce{006973}{}
Montrer que pour tout $x>0$, on a
$$\Arctan\left(\frac{1}{2x^2}\right)=\Arctan\left(\frac{x}{x+1}\right)-\Arctan\left(\frac{x-1}{x}\right).$$
En déduire une expression de $\displaystyle S_n=\sum_{k=1}^n\Arctan\left(\frac{1}{2k^2}\right)$ 
et calculer $\displaystyle\lim_{n\to +\infty}S_n$.
\finenonce{006973} 


\finexercice
\exercice{6974, blanc-centi, 2014/05/06}
\video{pT2xX0g3eDA}
\enonce{006974}{}
Soit $z=x+iy$ un nombre complexe, où $x=\Re z$ et $y=\Im z$. 
On sait que si $z$ est non nul, on peut l'écrire de façon unique sous la forme 
$z=x+iy=re^{i\theta}$, où $\theta\in]-\pi,\pi]$ et $r=\sqrt{x^2+y^2}$. 
\begin{center}
\begin{tikzpicture}[scale=1]
      \draw[->,>=latex, gray] (-0.5,0)--(4.5,0);
       \draw[->,>=latex, gray] (0,-0.5)--(0,3);

      \coordinate (A) at (3,2);
      \coordinate (Ax) at (3,0);
      \coordinate (Ay) at (0,2);

       \draw[thin] (0,0)--(A) node[midway,above] {$r$};
       \draw[dashed,gray] (A)--(Ax);
       \draw[dashed,gray] (A)--(Ay);

       \fill (A) circle (2pt);
       \fill (0,0) circle (2pt);

       \node at (0,0) [below left] {$0$}; 
       \node at (A) [above right] {$z=x+iy$}; 
       \node at (Ax) [below] {$x$}; 
       \node at (Ay) [left] {$y$};
       
       \draw[->,>=latex] (0:1.6) arc (0:33.69:1.6);
       \node[right] at (16.5:1.6) {$\theta$};
\end{tikzpicture}  
\end{center}

\begin{enumerate}
\item Montrer que si $x>0$, alors $\theta=\Arctan\frac{y}{x}$.
\item Montrer que si $\theta\in]-\pi,\pi[$, alors 
$\theta=2\Arctan\left(\frac{\sin\theta}{1+\cos\theta}\right)$. 
\item En déduire que si $z$ n'est pas réel négatif ou nul, on a l'égalité
$$\theta=2\Arctan\left(\frac{y}{x+\sqrt{x^2+y^2}}\right).$$
\end{enumerate}
\finenonce{006974} 


\finexercice
\section{Fonctions hyperboliques}
\exercice{6975, blanc-centi, 2014/05/06}
\video{g13-StbHolQ}
\enonce{006975}{} 
Simplifier l'expression $\displaystyle\frac{2\ch^2(x)-\sh(2x)}{x-\ln(\ch x)-\ln 2}$ 
et donner ses limites en $-\infty$ et $+\infty$.
\finenonce{006975} 


\finexercice
\exercice{764, bodin, 1998/09/01}
\video{eW51oHrCmVU}
\enonce{000764}{} 
Soit $x\in\R$. On pose $t=\Arctan(\sh x)$.
\begin{enumerate}
\item \'Etablir les relations  
$$\tan t=\sh x \qquad\qquad \frac{1}{\cos t}=\ch x \qquad\qquad \sin t=\tanh x$$
\item Montrer que $x = \ln \big(\tan\big(\frac{t}{2}+\frac{\pi}{4}\big)\big)$.
\end{enumerate}
\finenonce{000764} 


\finexercice\exercice{6976, blanc-centi, 2014/05/06}
\video{pMuDpGtaNAM}
\enonce{006976}{} 
Soit $x$ un réel fixé. Pour $n\in\Nn^*$, on pose
$$C_n=\sum_{k=1}^n\ch(kx)\qquad\text{ et }\qquad S_n=\sum_{k=1}^n\sh(kx).$$
Calculer $C_n$ et $S_n$.
\finenonce{006976} 


\finexercice
\exercice{6977, blanc-centi, 2014/05/06}
\video{GEaDEYtHv7U}
\enonce{006977}{} 
Soit $a$ et $b$ deux réels positifs tels que $a^2-b^2=1$. Résoudre le système
$$\left\{\begin{array}{l}
\ch(x)+\ch(y)=2a\\
\sh(x)+\sh(y)=2b
\end{array}\right.$$
\finenonce{006977} 


\finexercice

\section{Fonctions hyperboliques inverses}
\exercice{6978, blanc-centi, 2014/05/06}
\video{UkRPGUQfPP8}
\enonce{006978}{} 
Simplifier les expressions suivantes:
\begin{enumerate}
\item $\ch(\Argsh x),\quad \tanh(\Argsh x),\quad \sh(2\Argsh x)$.
\item $ \sh(\Argch x),\quad \tanh(\Argch x),\quad \ch(3\Argch x)$.
\end{enumerate}
\finenonce{006978} 


\finexercice
\exercice{6979, blanc-centi, 2014/05/06}
\video{bRKLAqnlbBM}
\enonce{006979}{} 
\'Etudier le domaine de définition de la fonction $f$ définie par 
$$f(x)=\Argch\left[\frac{1}{2}\left(x+\frac{1}{x}\right)\right]$$
et simplifier son expression lorsqu'elle a un sens.
\finenonce{006979} 


\finexercice
\exercice{6980, blanc-centi, 2014/05/06}
\video{QkvgaXwOwyU}
\enonce{006980}{} 
Montrer que l'équation $\Argsh x+\Argch x=1$ admet une unique solution, puis la déterminer.
\finenonce{006980} 


\finexercice

\finfiche

 \finenonces 



 \finindications 

\indication{000752}
Faire une étude de fonction.
La fonction $\text{sgn}(x)$ est la \emph{fonction signe} : elle vaut $+1$ si $x> 0$,
$-1$ si $x < 0$ (et $0$ si $x=0$).
\finindication
\indication{000745}
Faire un dessin. Calculer l'angle d'observation $\alpha$ en fonction de la distance $x$
et étudier cette fonction. Pour simplifier l'expression de $\alpha_0$, 
calculer $\tan\alpha_0$ à l'aide de la formule donnant $\tan(a-b)$.
\finindication
\indication{000747}
Il faut utiliser les identités trigonométriques classiques.
\finindication
\indication{000749}
On compose les équations par la bonne fonction (sur le bon domaine de définition), 
par exemple cosinus pour la premi\`ere. Pour la dernière, commencer par 
étudier la fonction pour montrer qu'il existe une unique solution.
\finindication
\indication{006973}
Dériver la différence des deux expressions.
\finindication
\noindication
\indication{006975}
On trouve $-\frac{1+e^{-2x}}{\ln(1+e^{-2x})}$.
\finindication
\indication{000764}
Pour la première question calculer $\frac{1}{\cos^2t}$.
Pour la seconde question, vérifier que 
$y=\ln \left(\tan\left(\frac{t}{2}+\frac{\pi}{4}\right)\right)$ 
est bien défini et calculer $\sh y$.
\finindication
\indication{006976}
Commencer par calculer $C_n+S_n$ et $C_n-S_n$ à l'aide des fonctions $\ch$ et $\sh$.
\finindication
\indication{006977}
Poser $X=e^x$ et $Y=e^y$ et se ramener à un système d'équations du type somme-produit.
\finindication
\noindication
\indication{006979}
On trouve $f(x)=|\ln x|$ pour tout $x>0$.
\finindication
\indication{006980}
Faire le tableau de variations de $f:x\mapsto\Argsh x+\Argch x$.
\finindication


\newpage

\correction{000752}\ 
\begin{enumerate}
    \item Soit $f$ la fonction définie sur $[-1,1]$ par 
$f(x) = \Arcsin x +\Arccos x$: $f$ est continue sur l'intervalle $[-1,1]$, 
et dérivable sur $]-1,1[$. Pour tout $x\in]-1,1[$, 
$f'(x)= \frac{1}{\sqrt{1-x^2}}+\frac{-1}{\sqrt{1-x^2}}  = 0$. 
Ainsi $f$ est constante sur $]-1,1[$, donc sur $[-1,1]$ (car continue aux extrémités). 
Or $f(0) = \Arcsin 0 +\Arccos 0 = \frac \pi 2$
donc pour tout $x\in[-1,1]$, $f(x) = \frac \pi 2$.

   \item Soit $g(x) = \Arctan x + \Arctan \frac 1x$.
Cette fonction est définie sur $]-\infty,0[$ et sur $]0,+\infty[$ (mais pas en $0$).
On a 
$$g'(x)= \frac{1}{1+x^2} + \frac{-1}{x^2} \cdot \frac{1}{1+\frac{1}{x^2}} = 0,$$
donc $g$ est constante sur chacun de ses intervalles de définition:
$g(x) = c_1$ sur $]-\infty,0[$ et $g(x) = c_2$ sur $]0,+\infty[$.
Sachant $\Arctan 1 = \frac\pi4$, on calcule $g(1)$ et $g(-1)$ on obtient $c_1 = -\frac \pi 2$
et $c_2 = +\frac \pi 2$.
\end{enumerate}
\fincorrection
\correction{000745}\ 
\begin{enumerate}
  \item 

On note $x$ la distance de l'observateur au pied de la statue.
On note $\alpha$ l'angle d'observation de la statue seule, et $\beta$
l'angle d'observation du piédestal seul.

\begin{center}
\begin{tikzpicture}[scale=1]

\filldraw[fill=gray!50, draw=gray] (5,0)--(6,0)--(6,4)--(5,4)--cycle;
\filldraw[fill=gray, draw=black] (5,4)--(5.5,4)--(5.5,6)--(5,6)--cycle;
\draw[thick] (0,0)--(5,0);
\draw[thick] (0,0)--(5,4);
\draw[thick] (0,0)--(5,6);
 
 \node at (5,5) [left] {$s$};    
 \node at (5,2) [left] {$p$};  
 \node at (2.5,0) [below] {$x$};  

\draw (2,0) arc(0:39:2) ;
\draw (39:2.5) arc(39:50:2.5) ;
\draw (39:2.6) arc(39:50:2.6) ;

\node at (45:2.9)  {$\alpha$};  
\node at (20:2.3)  {$\beta$};  
\end{tikzpicture}
\end{center}

Nous avons les relations trigonométriques dans les triangles rectangles :
$$\tan(\alpha+\beta) = \frac{p+s}{x} \qquad\text{ et }\qquad \tan\beta = \frac{p}{x}$$
On en déduit les deux identités :
$$\alpha+\beta = \Arctan\left(\frac{p+s}{x}\right) \qquad\text{ et }\qquad \beta = \Arctan\left(\frac{p}{x}\right)$$
à partir desquelles on obtient $\alpha=\alpha(x)=\Arctan\left(\frac{p+s}{x}\right)-\Arctan\left(\frac{p}{x}\right)$.

\'Etudions cette fonction sur $]0,+\infty[$ : elle est dérivable et 
$$\alpha'(x)=\frac{-\frac{s+p}{x^2}}{1+\left(\frac{s+p}{x}\right)^2}-\frac{-\frac{p}{x^2}}{1+\left(\frac{p}{x}\right)^2}=\frac{s}{(x^2+p^2)(x^2+(s+p)^2)}\big(p(p+s)-x^2\big)$$
Ainsi $\alpha'$ ne s'annule sur $]0,+\infty[$ qu'en $x_0 = \sqrt{p(p+s)}$.
Par des considérations physiques, à la limite en $0$ et en $+\infty$, 
l'angle $\alpha$ est nul,
alors en $x_0$ nous obtenons un angle $\alpha$ maximum.
Donc la distance optimale de vision est  $x_0 = \sqrt{p(p+s)}$.

\item
Pour calculer l'angle maximum $\alpha_0$ correspondant, on pourrait 
calculer $\alpha_0 = \alpha(x_0)$ à partir de la définition de la fonction
$\alpha(x)$. Pour obtenir une formule plus simple nous 
utilisons la formule trigonométrique suivante : si
$a$, $b$ et $a-b$ sont dans l'intervalle de définition de la fonction $\tan$, alors
$\tan(a-b)=\frac{\tan a-\tan b}{1+\tan a \tan b}$, ce qui donne ici
$$\tan\alpha_0 = \tan\big( (\alpha_0+\beta_0) - \beta_0 \big)
=\frac{\frac{p+s}{x_0}-\frac{p}{x_0}}{1+\frac{p+s}{x_0}\cdot\frac{p}{x_0}}
= \frac{s}{2x_0}=\frac{s}{2\sqrt{p(p+s)}}$$
Comme $\alpha_0\in]-\frac{\pi}{2},\frac{\pi}{2}[$, 
on en déduit $\alpha_0 = \Arctan \frac{s}{2x_0} = \Arctan \frac{s}{2\sqrt{p(p+s)}}$.

\item Pour la statue de la liberté, on a la hauteur de la statue $s=46$ mètres
et la hauteur du piédestal $p=47$ mètres.
On trouve donc 
$$x_0 = \sqrt{p(p+s)} \simeq 65,40 \text{mètres} \qquad 
\alpha_0 = \Arctan \frac{s}{2\sqrt{p(p+s)}} \simeq 19^\circ.$$

Voici les représentations de la statue et de la fonction $\alpha(x)$ pour ces valeurs de $s$
et $p$.
\end{enumerate}

\begin{center}
\begin{tikzpicture}[scale=0.5]

\filldraw[fill=gray!50, draw=gray] (6.54,0)--(7.54,0)--(7.54,4.7)--(6.54,4.7)--cycle;
\filldraw[fill=gray, draw=black] (6.54,4.7)--(7.04,4.7)--(7.04,9.3)--(6.54,9.3)--cycle;
\draw[thick] (0,0)--(6.54,0);
\draw[thick] (0,0)--(6.54,4.7);
\draw[thick] (0,0)--(6.54,9.3);
 
 \node at (6.5,6.5) [left] {$s$};    
 \node at (6.5,2) [left] {$p$};  
 \node at (3.5,0) [below] {$x_0$};  

\draw (2,0) arc(0:36:2) ;
\draw (36:2.5) arc(36:55:2.5) ;
\draw (36:2.6) arc(36:55:2.6) ;

\node at (45:3)  {$\alpha_0$};  
\node at (18:2.5)  {$\beta_0$};  
\end{tikzpicture} 
\qquad
\begin{tikzpicture}[scale=0.2]
      \draw[->,>=latex, gray] (-0.5,0)--(30,0) node[below,black] {$x$};
      \draw[->,>=latex, gray] (0,-0.5)--(0,23) node[left,black] {$\alpha(x)$};  
 
          \draw[ultra thick, color=red,domain=0.1:30,samples=200,smooth] plot (\x,{(atan(93/(\x*10))-atan(47/(\x*10)))}) node[above left] {$\alpha(x)$}; 

   \draw[dashed] (0,19.3)--(6.54,19.3);
   \draw[dashed] (6.54,0)--(6.54,19.3);

     \fill (0,0) circle (7pt);     
     \fill (6.54,0) circle (7pt);
     \fill (0,19.3) circle (7pt);
    
     \node at (0,19.3)[left] {$\alpha_0$};
     \node at (6.54,0)[below] {$x_0$};
     \node at (0,0)[below right] {$0$};
\end{tikzpicture}
\end{center}

\fincorrection
\correction{000747}\
\begin{enumerate}
    \item $\sin^2 y = 1-\cos^2 y$, donc
$\sin y = \pm\sqrt{1-\cos^2 y}$.
Avec $y=\Arccos x$, il vient $\sin(\Arccos x)= \pm\sqrt{1-x^2}$. Or $\Arccos x \in [0,\pi]$, 
donc $\sin(\Arccos x)$ est positif et finalement $\sin(\Arccos x)=  +\sqrt{1-x^2}$.
De la m\^eme mani\`ere on trouve $\cos(\Arcsin x) = \pm\sqrt{1-x^2}$. 
Or $\Arcsin x \in [-\frac{\pi}{2},\frac{\pi}{2}]$, donc $\cos(\Arcsin x)$ est positif 
et finalement $\cos(\Arcsin x)=+\sqrt{1-x^2}$.

Ces deux égalités sont à connaître ou à savoir retrouver très rapidement :
$$\sin(\Arccos x)  = \sqrt{1-x^2} = \cos(\Arcsin x).$$


Enfin, puisque $\cos(2y) = \cos^2 y - \sin^2 y$, on obtient avec $y=\Arcsin x$,
$$\cos(2\Arcsin x)=(\sqrt{1-x^2})^2-x^2=1-2x^2.$$

\item Commen\c{c}ons par calculer $\sin(\Arctan x)$, $\cos(\Arctan x)$.
On utilise l'identité $1+\tan^2y=\frac 1{\cos^2 y}$ avec $y = \Arctan x$, ce qui donne 
$\cos^2 y = \frac{1}{1+x^2}$ et
 $\sin^2 y = 1-\cos^2 y=\frac{x^2}{1+x^2}$.
 Il reste à déterminer les signes de $\cos(\Arctan x)= \pm\frac{1}{\sqrt{1+x^2}}$ et $\sin(\Arctan x) = \pm\frac{x}{\sqrt{1+x^2}}$
Or $y=\Arctan x$ donc $y\in]-\frac{\pi}{2},\frac{\pi}{2}[$ et $y$ a le même signe que $x$ : 
ainsi $\cos y>0$, et $\sin y$ a le même signe que $y$ et donc que $x$. Finalement, on a 
$\cos(\Arctan x)= \frac{1}{\sqrt{1+x^2}}$ et $\sin(\Arctan x) = \frac{x}{\sqrt{1+x^2}}$.

\medskip


Il ne reste plus qu'à linéariser $\sin(3y)$ :
\begin{eqnarray*}
\sin(3y) &=& \sin(2y+y)=\cos(2y)\sin(y)+\cos(y)\sin(2y)\\
 &=&(2\cos^2 y-1)\sin y +2\sin y\cos^2y\\
 &=&4 \sin y \cos^2 y - \sin y
\end{eqnarray*}



Maintenant
\begin{eqnarray*}
\sin(3\Arctan x) &=& \sin(3y)= 4 \sin y \cos^2 y - \sin y\\
 &=& 4{\frac {x}{\left (1+{x}^{2}\right )^{3/2}}}-{\frac {x}{\sqrt {1+{x}^{2}}}}={\frac {x(3-x^2)}{\left (1+{x}^{2}\right )^{3/2}}}
\end{eqnarray*}

\bigskip

\emph{Remarque :} la méthode générale pour obtenir la formule de linéarisation de $\sin(3y)$ est 
d'utiliser les nombres complexes
et la formule de Moivre. On développe 
$$\cos(3y) + i \sin(3y) = (\cos y + i \sin y)^3 = \cos^3 y + 3i\cos^2y \sin y + \cdots$$ 
puis on identifie les parties imaginaires pour avoir $\sin(3y)$, 
ou les parties réelles pour avoir $\cos(3y)$.

\end{enumerate}
\fincorrection
\correction{000749}\
\begin{enumerate}
\item On vérifie d'abord que $2\Arccos \frac{3}{4}\in[0,\pi]$ (sinon, l'équation 
n'aurait aucune solution). En effet, par définition, la fonction $\Arccos$ est décroissante 
sur $[-1,1]$ à valeurs dans $[0,\pi]$, donc puisque $\frac{1}{2}\le\frac{3}{4}\le 1$ 
on a $\frac{\pi}{3}\ge\cos\left(\frac{3}{4}\right)\ge 0$.
Puisque par définition $\Arccos x\in[0,\pi]$, on obtient en prenant le cosinus:
$$\Arccos x = 2\Arccos\left(\frac{3}{4}\right)\Longleftrightarrow x 
= \cos\left(2\Arccos \frac{3}{4}\right)$$
En appliquant la formule $\cos 2u = 2\cos^2u-1$, on arrive  donc \`a une unique solution
$x = 2(\frac{3}{4})^2-1 = \frac{1}{8}$.
\item Vérifions d'abord que $-\frac{\pi}{2}\le\Arcsin \frac{2}{5} + \Arcsin \frac{3}{5}\le\frac{\pi}{2}$. 
En effet, la fonction $\Arcsin$ est strictement croissante et 
$0<\frac{2}{5}<\frac{1}{2}<\frac{3}{5}<\frac{\sqrt{2}}{2}$, ce qui donne
$0<\Arcsin\left(\frac{2}{5}\right)<\frac{\pi}{6}<\Arcsin\left(\frac{3}{5}\right)<\frac{\pi}{4}$, 
d'où l'encadrement
$0+\frac{\pi}{6}<\Arcsin \frac{2}{5} + \Arcsin \frac{3}{5}\le\frac{\pi}{6}+\frac{\pi}{4}$.

Puisque par définition on aussi $\Arcsin x\in [-\frac{\pi}{2},\frac{\pi}{2}]$, 
il vient en prenant le sinus:
\begin{eqnarray*}
\lefteqn{\Arcsin x=\Arcsin \frac{2}{5} + \Arcsin \frac{3}{5}}\\
 &\Longleftrightarrow& x=\sin\left(\Arcsin \frac{2}{5} + \Arcsin \frac{3}{5}\right)\\
 &\Longleftrightarrow& x = \frac{3}{5}\cos\left(\Arcsin \frac{2}{5}\right)+\frac{2}{5} \cos\left(\Arcsin \frac{3}{5}\right)
\end{eqnarray*}
La dernière équivalence vient de la formule de $\sin(a+b)= \cos a \sin b + \cos b \sin a$
et de l'identité $\sin\big( \Arcsin u\big) = u$.

En utilisant la formule $\cos\left(\Arcsin x\right) = \sqrt{1-x^2}$, 
on obtient une unique solution: $x = \frac35\sqrt{\frac{21}{25}}+\frac25\frac 45 
=\frac{3\sqrt{21}+8}{25}$.

\item Supposons d'abord que $x$ est solution. Remarquons d'abord que $x$ est 
nécessairement positif, puisque $\Arctan x$ a le même signe que $x$. 
Alors, en prenant la tangente des deux membres, on obtient $\tan\big(\Arctan(2x)+\Arctan(x)\big)=1$.

En utilisant la formule donnant la tangente d'une somme :
$\tan(a+b)=\frac{\tan a +\tan b}{1-\tan a \tan b}$, on obtient $\frac{2x+x}{1-2x\cdot x}=1$, 
et finalement $2x^2+3x-1=0$ qui admet une unique solution positive $x_0=\frac{-3+\sqrt{17}}{4}$. 
Ainsi, {\it si} l'équation de départ admet une solution, c'est nécessairement $x_0$. 

Or, en posant $f(x)=\Arctan(2x)+\Arctan(x)$, la fonction $f$ est continue sur $\Rr$. 
Comme $f(x)\xrightarrow[x\to -\infty]{}-\pi$ et $f(x)\xrightarrow[x\to +\infty]{}+\pi$, 
on sait d'après le théorème des valeurs intermédiaires que $f$ prend la valeur 
$\frac{\pi}{4}$ au moins une fois (et en fait une seule fois, puisque $f$ est 
strictement croissante comme somme de deux fonctions strictement croissantes). 
Ainsi l'équation de départ admet bien une solution, qui est $x_0$.
\end{enumerate}
\fincorrection
\correction{006973}
Posons $f(x)=\Arctan\left(\frac{1}{2x^2}\right)-\Arctan\left(\frac{x}{x+1}\right)+\Arctan\left(\frac{x-1}{x}\right)$ 
pour tout $x>0$. La fonction $f$ est dérivable, et
\begin{eqnarray*}
f'(x)
 &=& \frac{-\frac{2}{2x^3}}{1+\left(\frac{1}{2x^2}\right)^2}-
\frac{\frac{1}{(1+x)^2}}{1+\left( \frac{x}{x+1} \right)^2}
+\frac{\frac{1}{x^2}}{1+\left( \frac{x-1}{x} \right)^2}\\
 &=& \frac{-4x}{4x^4+1}-\frac{1}{(1+x)^2+x^2}+\frac{1}{x^2+(x-1)^2}\\
 &=& \frac{-4x}{4x^4+1}+\frac{-\big(x^2+(x-1)^2\big)+\big((1+x)^2+x^2\big)}{\big((1+x)^2+x^2\big)\big(x^2+(x-1)^2\big)}\\
 &=& 0
\end{eqnarray*}
Ainsi $f$ est une fonction constante. 
Or $f(x)\xrightarrow[x\to +\infty]{}\Arctan 0-\Arctan 1+\Arctan 1=0$. Donc la constante vaut $0$, 
d'où l'égalité cherchée. 

Alors :
\begin{eqnarray*}
S_n
 &=&\sum_{k=1}^n\Arctan\left(\frac{1}{2k^2}\right)\\
 &=&\sum_{k=1}^n\Arctan\left(\frac{k}{k+1}\right)-\sum_{k=1}^n\Arctan\left(\frac{k-1}{k}\right)
 \quad \text{(par l'identité prouvée)}\\
 &=&\sum_{k=1}^n\Arctan\left(\frac{k}{k+1}\right)-\sum_{k'=0}^{n-1}\Arctan\left(\frac{k'}{k'+1}\right)
 \quad \text{(en posant $k'=k-1$)}\\
 &=&\Arctan\left(\frac{n}{n+1}\right)-\Arctan\left(\frac{0}{0+1}\right)
 \quad \text{(les sommes se simplifient)}\\
 &=&\Arctan\left(1-\frac{1}{n+1}\right)
 \quad \text{(car $\tfrac{n}{n+1} = 1-\tfrac{1}{n+1}$)} \\
\end{eqnarray*} 
Ainsi $S_n\xrightarrow[n\to +\infty]{}\Arctan 1=\frac{\pi}{4}$.
\fincorrection
\correction{006974}\
\begin{enumerate}
\item Si $x>0$, alors $\frac{y}{x}$ est bien défini et $\Arctan\frac{y}{x}$ aussi. 
Comme $x=r\cos\theta$ et $y=r\sin\theta$, on a bien $\frac{y}{x}=\tan\theta$.
Puisque par hypothèse $\theta\in]-\pi,\pi]$ et que l'on a supposé $x>0$,
alors $\cos\theta>0$.
Cela implique $\theta\in]-\frac{\pi}{2},\frac{\pi}{2}[$. 
Donc $\theta=\Arctan(\tan\theta)=\Arctan\frac{y}{x}$. (Attention ! Il est important
d'avoir $\theta\in]-\frac{\pi}{2},\frac{\pi}{2}[$ pour considérer l'identité 
$\Arctan(\tan\theta) = \theta$.)


\item Si $\theta\in]-\pi,\pi[$ alors $\frac{\theta}{2}\in]-\frac{\pi}{2},\frac{\pi}{2}[$, 
donc $\frac{\theta}{2}=\Arctan\left(\tan\frac{\theta}{2}\right)$. Or 
$$\frac{\sin\theta}{1+\cos\theta}=
\frac{2\cos\frac{\theta}{2}\sin\frac{\theta}{2}}{1+\big(2\cos^2\left(\frac{\theta}{2}\right)-1\big)}
=\frac{\sin\frac{\theta}{2}}{\cos\frac{\theta}{2}}=\tan\frac{\theta}{2}$$
d'où $\frac\theta2=\Arctan\left(\tan\frac{\theta}{2}\right)=\Arctan\left(\frac{\sin\theta}{1+\cos\theta}\right)$.

\item Remarquons que $z=x+iy$, supposé non nul, est un nombre réel négatif si et seulement si 
($x=r\cos\theta< 0$ et $y=r\sin\theta=0$), c'est-à-dire $\theta=\pi$. 
Par conséquent, dire que $z$ n'est pas réel négatif ou nul signifie que 
$\theta\in]-\pi,\pi[$. On a alors $x+\sqrt{x^2+y^2}\not=0$ (sinon, on aurait $\sqrt{x^2+y^2}=-x$ 
et donc $y=0$ et $x\le 0$) et 
$$\frac{y}{x+\sqrt{x^2+y^2}}=\frac{r\sin\theta}{r\cos\theta+r}=\frac{\sin\theta}{1+\cos\theta}.$$
Par la question précédente :
$$\theta = 2\Arctan\left(\frac{\sin\theta}{1+\cos\theta}\right) = 2\Arctan\left(\frac{y}{x+\sqrt{x^2+y^2}}\right).$$

\end{enumerate}
\fincorrection
\correction{006975}
Par définition des fonctions $\ch$ et $\sh$, on a 
\begin{eqnarray*}
2\ch^2(x)-\sh(2x)&=&2\left(\frac{e^x+e^{-x}}{2}\right)^2-\frac{e^{2x}-e^{-2x}}{2}\\
 &=&\frac{e^{2x}+2+e^{-2x}}{2}+\frac{e^{-2x}-e^{2x}}{2}\\
 &=&1+e^{-2x}
\end{eqnarray*}

Et en utilisant les deux relations $\ln(ab)=\ln a + \ln b$ et $\ln(e^x) = x$ on calcule :
\begin{eqnarray*}
x-\ln(\ch x)-\ln 2
  &=& x-\ln\left(\frac{e^x+e^{-x}}{2}\right)-\ln 2\\
  &=& x-\ln(e^x+e^{-x}) + \ln 2 - \ln 2\\
  &=& x-\ln\big(e^x(1+e^{-2x})\big)\\
  &=& x-\ln(e^x)-\ln(1+e^{-2x})\\
  &=& x-x-\ln(1+e^{-2x})\\
  &=& -\ln(1+e^{-2x})\\
\end{eqnarray*}
d'où 
$$\frac{2\ch^2(x)-\sh(2x)}{x-\ln(\ch x)-\ln 2}=-\frac{1+e^{-2x}}{\ln(1+e^{-2x})}$$
C'est une expression de la forme $-\frac{u}{\ln u}$ avec $u=1+e^{-2x}$ :
\begin{itemize}
  \item si $x\to +\infty$, alors $u\to 1^+$, $\frac{1}{\ln u} \to +\infty$ 
  donc $-\frac{u}{\ln u}\to -\infty$ ;
  \item si $x\to -\infty$, alors $u\to +\infty$ donc d'après les 
  relations de croissances comparées, $-\frac{u}{\ln u}\to -\infty$.
\end{itemize}
\fincorrection
\correction{000764}\
\begin{enumerate}
  \item 
  \begin{enumerate}
    \item Remarquons d'abord que, par construction, $t\in]-\frac{\pi}{2},\frac{\pi}{2}[$,
    $t$ est donc dans le domaine de définition de la fonction $\tan$.
    En prenant la tangente de l'égalité $t=\Arctan(\sh x)$ on obtient directement $\tan t=
    \tan\big(\Arctan(\sh x)\big) =\sh x$. 

    \item Ensuite, $\frac{1}{\cos^2t}=1+\tan^2 t= 1 + \tan^2\big(\Arctan(\sh x)\big) 
    = 1+\sh^2x=\ch^2x$. Or la fonction $\ch$ ne prend que des valeurs positives, 
    et $t\in]-\frac{\pi}{2},\frac{\pi}{2}[$ donc $\cos t>0$. Ainsi $\frac{1}{\cos t}=\ch x$. 

    \item Enfin, $\sin t=\tan t \cdot \cos t= \sh x \cdot \frac{1}{\ch x}=\tanh x$.
  \end{enumerate}

\item Puisque $t\in]-\frac{\pi}{2},\frac{\pi}{2}[$, on a 
$0<\frac{t}{2}+\frac{\pi}{4}<\frac{\pi}{2}$, donc 
$\tan\left(\frac{t}{2}+\frac{\pi}{4}\right)$ est bien défini et 
strictement positif. Ainsi $y=\ln \left(\tan\left(\frac{t}{2}+\frac{\pi}{4}\right)\right)$ 
est bien défini.

Ensuite : 
\begin{eqnarray*}
\sh y
  &=& \frac{e^y-e^{-y}}{2}\\
  &=& \tfrac{1}{2}\tan\left(\tfrac{t}{2}+\tfrac{\pi}{4}\right)-
  \tfrac{1}{2}\frac{1}{\tan\left(\frac{t}{2}+\frac{\pi}{4}\right)}\\
  &=&\frac{\sin^2\left(\frac{t}{2}+\frac{\pi}{4}\right)-
 \cos^2\left(\frac{t}{2}+\frac{\pi}{4}\right)}{2\cos\left(\frac{t}{2}+\frac{\pi}{4}\right)\sin\left(\frac{t}{2}+\frac{\pi}{4}\right)}\\
 &=&\frac{-\cos\left(t+\frac{\pi}{2}\right)}{\sin\left(t+\frac{\pi}{2}\right)}
\end{eqnarray*}
car $\sin(2u)=2\cos u\sin u$ et $\cos(2u)=\cos^2u-\sin^2u$. 


Enfin, puisque $\cos\left(t+\frac{\pi}{2}\right)=-\sin t$ et 
$\sin\left(t+\frac{\pi}{2}\right)=\cos t$, on a $\sh y= \frac{\sin t}{\cos t} = \tan t=\sh x$. 
Puisque la fonction $\sh$ est bijective de $\R$ dans $\R$, on en déduit $y=x$.
Conclusion : $x = y = \ln \big(\tan\big(\frac{t}{2}+\frac{\pi}{4}\big)\big)$.
\end{enumerate}
\fincorrection
\correction{006976}
Puisque $\ch x+\sh x=e^x$ et $\ch x-\sh x=e^{-x}$, 
les expressions $C_n+S_n=\sum_{k=1}^ne^{kx}$ et $C_n-S_n=\sum_{k=1}^ne^{-kx}$ 
sont des sommes de termes de suites géométriques, de raison respectivement $e^x$ et $e^{-x}$.

Si $x=0$, on a directement $C_n=\sum_{k=1}^n1=n$ et $S_n=\sum_{k=1}^n0=0$.

Supposons $x\not=0$, alors $e^{x}\not=1$. 
Donc
\begin{eqnarray*}
C_n+S_n
  &=& \sum_{k=1}^ne^{kx} = \frac{e^x-e^{(n+1)x}}{1-e^x}\\
  &=& e^x\,\frac{1-e^{nx}}{1-e^x}\\
  &=& e^x\ \frac{e^{\frac{nx}{2}}(e^{-\frac{nx}{2}}-e^{\frac{nx}{2}})}{e^{\frac x2}(e^{-\frac x2}-e^{\frac x2})}\\
  &=& e^{\frac{(n+1)x}{2}}\ \frac{e^{\frac{nx}{2}}-e^{-\frac{nx}{2}}}{e^{\frac x2}-e^{-\frac x2}}\\
  &=& e^{\frac{(n+1)x}{2}}\ \frac{\sh\frac{nx}{2}}{\sh\frac{x}{2}}
\end{eqnarray*}

De même $C_n-S_n = \sum_{k=1}^ne^{-kx}$ ; c'est donc la même formule que ci-dessus en remplaçant $x$ par $-x$.
Ainsi :
$$C_n-S_n = e^{-\frac{(n+1)x}{2}}\ \frac{\sh\frac{nx}{2}}{\sh\frac{x}{2}}$$

En utilisant $C_n=\frac{(C_n+S_n)+(C_n-S_n)}{2}$ et $S_n=\frac{(C_n+S_n)-(C_n-S_n)}{2}$, on récupère donc
$$C_n=\frac{e^{\frac{(n+1)x}{2}}+e^{-\frac{(n+1)x}{2}}}{2}\,\frac{\sh\frac{nx}{2}}{\sh\frac{x}{2}}=\ch\left(\tfrac{(n+1)x}{2}\right)\,\frac{\sh\frac{nx}{2}}{\sh\frac{x}{2}}$$
$$S_n=\frac{e^{\frac{(n+1)x}{2}}-e^{-(n+1)\frac x2}}{2}\,\frac{\sh\frac{nx}{2}}{\sh\frac{x}{2}}=\sh\left(\tfrac{(n+1)x}{2}\right)\,\frac{\sh\frac{nx}{2}}{\sh\frac{x}{2}}$$
\fincorrection
\correction{006977}
\begin{eqnarray*}
(S)\ \left\{\begin{array}{l}
\ch(x)+\ch(y)=2a\\
\sh(x)+\sh(y)=2b
\end{array}\right.
&\Longleftrightarrow&
\left\{\begin{array}{l}
e^x+e^{-x}+e^y+e^{-y}=4a\\
e^x-e^{-x}+e^y-e^{-y}=4b
\end{array}\right.\\
 &\Longleftrightarrow&
\left\{\begin{array}{l}
e^x+e^y=2a+2b\\
e^x-e^{-x}+e^y-e^{-y}=4b
\end{array}\right.\\
 &\Longleftrightarrow&
\left\{\begin{array}{l}
e^x+e^y=2a+2b\\
-e^{-x}-e^{-y}=2b-2a
\end{array}\right.\\
 &\Longleftrightarrow&
\left\{\begin{array}{l}
e^x+e^y=2(a+b)\\
\frac{1}{e^{x}}+\frac{1}{e^{y}}=2(a-b)
\end{array}\right.
\end{eqnarray*}
ce qui donne, en posant  $X=e^x$ et $Y=e^y$:
\begin{eqnarray*}
(S)&\Longleftrightarrow&
\left\{\begin{array}{l}
X+Y=2(a+b)\\
\frac{1}{X}+\frac{1}{Y}=2(a-b)
\end{array}\right.\\
&\Longleftrightarrow&
\left\{\begin{array}{l}
X+Y=2(a+b)\\
\frac{X+Y}{XY}=2(a-b)
\end{array}\right.\\
&\Longleftrightarrow&
\left\{\begin{array}{l}
X+Y=2(a+b)\\
\frac{2(a+b)}{XY}=2(a-b)
\end{array}\right.
\end{eqnarray*}
Or $a\not=b$ puisque par hypothèse, $a^2-b^2=1$. Ainsi,
\begin{eqnarray*}
(S)&\Longleftrightarrow&
\left\{\begin{array}{l}
X+Y=2(a+b)\\
XY=\frac{a+b}{a-b}
\end{array}\right.\\
&\Longleftrightarrow& X\ \text{et}\ Y\ \text{sont les solutions de}\ z^2-2(a+b)z+\frac{a+b}{a-b}=0
\end{eqnarray*}

\medskip

\emph{Remarque :} On rappelle
que si $X, Y$ vérifient le système 
$\left\{\begin{array}{l}
X+Y=S\\
XY=P
\end{array}\right.$, alors $X$ et $Y$ sont les solutions de l'équation $z^2-Sz+P=0$.
\medskip

Or le discriminant du trinôme $z^2-2(a+b)z+\frac{a+b}{a-b}=0$ vaut 
$$\Delta=4(a+b)^2-4\frac{a+b}{a-b}=4(a+b)\left(a+b-\frac{1}{a-b}\right)=\frac{4(a+b)(a^2-b^2-1)}{a-b}=0$$ 
Il y a donc une racine double qui vaut $\frac{2(a+b)}{2}$, ainsi $X=Y=a+b$ et donc :
$$(S)\Longleftrightarrow e^x=e^y=a+b$$
On vérifie que $a+b\ge 0$ (car $a\ge 0$ et $b\ge 0$) et $a+b\not=0$ (car $a^2-b^2=1$).
Conclusion : le système $(S)$ admet une unique solution, donnée par $\big(x=\ln(a+b),y=\ln(a+b)\big)$. 
\fincorrection
\correction{006978}\
\begin{enumerate}
  \item 
  \begin{enumerate}
    \item On sait que $\ch^2 u=1+\sh^2u$. Comme de plus la fonction $\ch$ est à valeurs positives, 
$\ch u=\sqrt{1+\sh^2u}$ et donc $\ch(\Argsh x)=\sqrt{1+\sh^2(\Argsh x)} = \sqrt{1+x^2}$. 

    \item Alors
$$\tanh(\Argsh x)=\frac{\sh(\Argsh x)}{\ch(\Argsh x)}=\frac{x}{\sqrt{1+x^2}}.$$

    \item Et $\sh(2\Argsh x)=2\ch(\Argsh x)\sh(\Argsh x)=2x\sqrt{1+x^2}$.
  \end{enumerate}
  
\item On pourrait, comme pour la question précédente, appliquer les formules 
trigonométriques hyperboliques. Pour changer, on va plutôt utiliser les expressions explicites 
des fonctions hyperboliques réciproques. Supposons $x\ge 1$, pour que $\Argch x$ 
soit bien défini, alors on a la formule (à connaître) :
$$\Argch x=\ln\big(x+\sqrt{x^2-1}\big).$$ 

Ainsi :
\begin{eqnarray*}
\sh(\Argch x)&=&\frac{e^{\Argch x}-e^{-\Argch x}}{2}\\
 &=& \frac{x+\sqrt{x^2-1}-\frac{1}{x+\sqrt{x^2-1}}}{2}\\
 &=& \frac{x+\sqrt{x^2-1}}{2}-\frac{x-\sqrt{x^2-1}}{2\big(x+\sqrt{x^2-1}\big)\big(x-\sqrt{x^2-1}\big)}\\
 &=& \frac{x+\sqrt{x^2-1}}{2}-\frac{x-\sqrt{x^2-1}}{2\big(x^2-(x^2-1)\big)}\\
 &=& \sqrt{x^2-1}
\end{eqnarray*}

\medskip

Donc $\displaystyle\tanh(\Argch x)=\frac{\sh(\Argch x)}{\ch(\Argch x)}=\frac{\sqrt{x^2-1}}{x}$. 

\medskip

Enfin, si $u=\Argch x$:
$\ch(3u)=\ch(2u+u)=\ch(2u)\ch u+\sh(2u)\sh u$, où 
$$\left\{\begin{array}{l}
\ch(2u)=\ch^2u+\sh^2u=x^2+(x^2-1)=2x^2-1\\
\sh(2u)=2\sh u\ch u=2x\sqrt{x^2-1}
\end{array}\right.$$ 
Donc 
$\ch(3\Argch x)=(2x^2-1)x+2x\sqrt{x^2-1}\sqrt{x^2-1}=x(4x^2-3)$. 
\end{enumerate}
\fincorrection
\correction{006979}
La fonction $\Argch$ est définie sur $[1,+\infty[$. Or
\begin{eqnarray*}
\frac{1}{2}\left(x+\frac{1}{x}\right)\ge 1
 &\Longleftrightarrow&\frac{x^2+1}{x}\ge 2\\
 &\Longleftrightarrow&\frac{x^2+1-2x}{x}\ge 0\\
 &\Longleftrightarrow&\frac{(x-1)^2}{x}\ge 0\\
 &\Longleftrightarrow& x > 0 
\end{eqnarray*}
donc $f$ est définie sur $]0,+\infty[$. 

\medskip

Soit $x>0$, alors $y=\frac{1}{2}\left(x+\frac{1}{x}\right)\ge 1$
et on sait que $\Argch y=\ln(y+\sqrt{y^2-1})$. Ainsi
$\sqrt{y^2-1}=\sqrt{\frac{1}{4}\left(x+\frac{1}{x}\right)^2-1}
=\sqrt{\frac{(x^2+1)^2}{4x^2}-1}=\sqrt{\frac{(x^2-1)^2}{4x^2}}
=\left|\frac{x^2-1}{2x}\right|$,
on obtient
$$f(x)=\Argch y=\ln(y+\sqrt{y^2-1})=\ln\left(\frac{x^2+1}{2x}+\left|\frac{x^2-1}{2x}\right|\right)$$


On a supposé $x>0$, il suffit donc de distinguer les cas $x\ge 1$ et $0<x\le 1$.
\begin{itemize}
  \item Si $x\ge 1$, $\displaystyle f(x)=\ln\left(\frac{x^2+1}{2x}+\frac{x^2-1}{2x}\right)=\ln x$.
  \item Si $0<x\le 1$, $\displaystyle f(x)=\ln\left(\frac{x^2+1}{2x}+\frac{1-x^2}{2x}\right)=\ln \frac{1}{x}=-\ln x$.
\end{itemize}

Puisque $\ln x$ est positif si $x\ge 1$ et négatif si $x\le 1$, 
on obtient dans les deux cas $f(x)=|\ln x|$.

\begin{center}
\begin{tikzpicture}[scale=1.5]
      \draw[->,>=latex, gray] (-1,0)--(5,0) node[below,black] {$x$};
      \draw[->,>=latex, gray] (0,-1)--(0,2.7) node[right,black] {$y$};  
 
     \draw[ultra thick, color=red,domain=0.1:5,samples=100,smooth] plot (\x,{abs(ln(\x))}) node[above left] {$f(x)$}; 
     
     \fill (0,0) circle (1.5pt);     
     \fill (1,0) circle (1.5pt);
     \fill (0,1) circle (1.5pt);     
     \node at (0,1)[above left] {$1$};
     \node at (1,0)[below] {$1$};
     \node at (0,0)[below right] {$0$};
\end{tikzpicture}  
\end{center}

\fincorrection
\correction{006980}
Soit $f(x)=\Argsh x+\Argch x$. La fonction $f$ est bien définie, continue, et strictement croissante, 
sur $[1,+\infty[$ (comme somme de deux fonctions continues strictement croissantes). 

\begin{center}
\begin{tikzpicture}[scale=2]
      \draw[->,>=latex, gray] (-0.5,0)--(2.5,0) node[below,black] {$x$};
      \draw[->,>=latex, gray] (0,-0.5)--(0,3) node[right,black] {$y$};  
 
      \draw[ultra thick, color=red,domain=1.0:2,samples=100,smooth] plot (\x,{ln(\x+sqrt(\x^2+1)) + ln(\x+sqrt(\x^2-1))}) node[above left] {$f(x)$}; 

   \draw (0,1.1)--(1.03,1.1)--(1.03,0);

   
   \draw (0,0.9)--(0.98,0.9)--(0.98,0);
   

     \fill (0,0) circle (1pt);
     \fill (0,0.9) circle (1pt); 
     \fill (0.96,0) circle (1pt);
     \fill (1.04,0) circle (1pt);
     
     \fill (0,1.1) circle (1pt);     
     \node at (0,1)[above left] {$1$};
     \node at (0,0.9)[below left] {$f(1)$};
     \node at (1,0)[below left] {$1$};
     \node at (1.03,0)[below right] {$a$};
     \node at (0,0)[below right] {$0$};
\end{tikzpicture}  
\end{center}

De plus, $f(x)\xrightarrow[x\to +\infty]{}+\infty$, donc $f$ atteint exactement une fois 
toute valeur de l'intervalle $[f(1),+\infty[$. Comme (par la formule logarithmique)
$f(1)=\Argsh 1=\ln(1+\sqrt{2})<\ln(e)=1$, 
on a $1\in[f(1),+\infty[$. Par le théorème des valeurs intermédiaires
l'équation $f(x)=1$ admet une unique solution, que l'on notera $a$. 

\medskip

Déterminons la solution :
\begin{eqnarray*}
\sh 1&=&\sh(\Argsh a+\Argch a)\\ 
 &=&\sh(\Argsh a)\ch(\Argch a)+\sh(\Argch a)\ch(\Argsh a)\\
 &=&a^2+\sqrt{a^2-1}\sqrt{a^2+1}=a^2+\sqrt{a^4-1}
\end{eqnarray*}
donc $\sqrt{a^4-1}=\sh 1-a^2$. En élevant au carré et en simplifiant, on obtient
$a^2=\frac{1+\sh^21}{2\sh 1}=\frac{\ch^2 1}{2\sh 1}$. Comme on cherche $a$ positif 
(et que $\ch 1>0$), on en déduit $a=\frac{\ch 1}{\sqrt{2\sh 1}}$. 
Cette valeur est la seule solution possible de l'équation $f(x)=1$, il faudrait normalement 
vérifier qu'elle convient bien, puisqu'on a seulement raisonné par implication (et pas par équivalence).
Or on sait déjà que l'équation admet une unique solution: c'est donc nécessairement 
$$a=\frac{\ch 1}{\sqrt{2\sh 1}}= \tfrac{1}{2}\frac{e+\frac1e}{\sqrt{e-\frac1e}} = 1,0065\ldots.$$
\fincorrection


\end{document}

