\documentclass[11pt]{article}

 %Configuration de la feuille 
 
\usepackage{amsmath,amssymb,enumerate,graphicx,pgf,tikz,fancyhdr}
\usepackage[utf8]{inputenc}
\usetikzlibrary{arrows}
\usepackage{geometry}
\usepackage{tabvar}
\geometry{hmargin=2.2cm,vmargin=1.5cm}\pagestyle{fancy}
\lfoot{\bfseries http://www.bibmath.net}
\rfoot{\bfseries\thepage}
\cfoot{}
\renewcommand{\footrulewidth}{0.5pt} %Filet en bas de page

 %Macros utilisées dans la base de données d'exercices 

\newcommand{\mtn}{\mathbb{N}}
\newcommand{\mtns}{\mathbb{N}^*}
\newcommand{\mtz}{\mathbb{Z}}
\newcommand{\mtr}{\mathbb{R}}
\newcommand{\mtk}{\mathbb{K}}
\newcommand{\mtq}{\mathbb{Q}}
\newcommand{\mtc}{\mathbb{C}}
\newcommand{\mch}{\mathcal{H}}
\newcommand{\mcp}{\mathcal{P}}
\newcommand{\mcb}{\mathcal{B}}
\newcommand{\mcl}{\mathcal{L}}
\newcommand{\mcm}{\mathcal{M}}
\newcommand{\mcc}{\mathcal{C}}
\newcommand{\mcmn}{\mathcal{M}}
\newcommand{\mcmnr}{\mathcal{M}_n(\mtr)}
\newcommand{\mcmnk}{\mathcal{M}_n(\mtk)}
\newcommand{\mcsn}{\mathcal{S}_n}
\newcommand{\mcs}{\mathcal{S}}
\newcommand{\mcd}{\mathcal{D}}
\newcommand{\mcsns}{\mathcal{S}_n^{++}}
\newcommand{\glnk}{GL_n(\mtk)}
\newcommand{\mnr}{\mathcal{M}_n(\mtr)}
\DeclareMathOperator{\ch}{ch}
\DeclareMathOperator{\sh}{sh}
\DeclareMathOperator{\vect}{vect}
\DeclareMathOperator{\card}{card}
\DeclareMathOperator{\comat}{comat}
\DeclareMathOperator{\imv}{Im}
\DeclareMathOperator{\rang}{rg}
\DeclareMathOperator{\Fr}{Fr}
\DeclareMathOperator{\diam}{diam}
\DeclareMathOperator{\supp}{supp}
\newcommand{\veps}{\varepsilon}
\newcommand{\mcu}{\mathcal{U}}
\newcommand{\mcun}{\mcu_n}
\newcommand{\dis}{\displaystyle}
\newcommand{\croouv}{[\![}
\newcommand{\crofer}{]\!]}
\newcommand{\rab}{\mathcal{R}(a,b)}
\newcommand{\pss}[2]{\langle #1,#2\rangle}
 %Document 

\begin{document} 

\begin{center}\textsc{{\huge produit scalaire }}\end{center}

% Exercice 2589


\vskip0.3cm\noindent\textsc{Exercice 1} - Produits scalaires sur $\mathbb R^2$
\vskip0.2cm
Les applications suivantes définissent-elles un produit scalaire sur $\mathbb R^2$?
\begin{enumerate}
 \item $\varphi_1\big((x_1,x_2),(y_1,y_2)\big)=\sqrt{x_1^2+y_1^2+x_2^2+y_2^2}$;
 \item $\varphi_2\big((x_1,x_2),(y_1,y_2)\big)=4x_1y_1-x_2y_2$;
 \item $\varphi_3\big((x_1,x_2),(y_1,y_2)\big)=x_1y_1-3x_1y_2-3x_2y_1+10x_2y_2$.
\end{enumerate}


% Exercice 1032


\vskip0.3cm\noindent\textsc{Exercice 2} - Produit scalaire et matrices
\vskip0.2cm
Pour $A,B\in\mathcal M_n(\mathbb R)$, on définit
$$\langle A,B\rangle=\textrm{tr}(A^T B).$$
\begin{enumerate}
\item Démontrer que cette formule définit un produit scalaire sur $\mathcal M_n(\mathbb R)$.
\item En déduire que, pour tous $A,B\in\mathcal S_n(\mathbb R)$, on a
$$\big(\textrm{tr}(AB))^2\leq \textrm{tr}(A^2)\textrm{tr}(B^2).$$
\end{enumerate}


% Exercice 2591


\vskip0.3cm\noindent\textsc{Exercice 3} - Un produit scalaire sur les polynômes
\vskip0.2cm
Soit $n\geq 1$ et soit $a_0,\dots,a_n$ des réels distincts deux à deux. Montrer que l'application $\varphi:\mathbb R_n[X]\times\mathbb R_n[X]\to\mathbb R$
définie par $\varphi(P,Q)=\sum_{i=0}^n P(a_i)Q(a_i)$ définit un produit scalaire sur $\mathbb R_n[X]$.


% Exercice 1033


\vskip0.3cm\noindent\textsc{Exercice 4} - Des exemples de produit scalaire
\vskip0.2cm
Démontrer que les formules suivantes définissent des produits scalaires sur l'espace vectoriel associé :
\begin{enumerate}
\item $\langle f,g\rangle=f(0)g(0)+\int_0^1 f'(t)g'(t)dt$ sur $E=\mathcal C^1([0,1],\mathbb R)$;
\item $\langle f,g\rangle=\int_a^b f(t)g(t)w(t)dt$ sur $E=\mathcal C([a,b],\mathbb R)$ où $w\in E$ satisfait $w>0$ sur $]a,b[$.
\end{enumerate}


% Exercice 3259


\vskip0.3cm\noindent\textsc{Exercice 5} - Une première application de l'inégalité de Cauchy-Schwarz
\vskip0.2cm
Démontrer que pour tous $x_1,\dots,x_n\in\mathbb R,$
$$\left(\sum_{k=1}^n \frac{x_k}{2^k}\right)^2\leq\frac 13\sum_{k=1}^n x_k^2.$$


% Exercice 1475


\vskip0.3cm\noindent\textsc{Exercice 6} - Quand une inégalité en implique une autre...
\vskip0.2cm
Soit $x,y,z$ trois réels tels que $2x^2+y^2+5z^2\leq 1$. Démontrer que $(x+y+z)^2\leq\frac {17}{10}.$




\vskip0.5cm
\noindent{\small Cette feuille d'exercices a été conçue à l'aide du site \textsf{https://www.bibmath.net}}

%Vous avez accès aux corrigés de cette feuille par l'url : https://www.bibmath.net/ressources/justeunefeuille.php?id=28693
\end{document}