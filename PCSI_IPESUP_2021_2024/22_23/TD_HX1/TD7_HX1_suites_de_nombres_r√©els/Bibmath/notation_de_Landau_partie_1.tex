\documentclass[11pt]{article}

 %Configuration de la feuille 
 
\usepackage{amsmath,amssymb,enumerate,graphicx,pgf,tikz,fancyhdr}
\usepackage[utf8]{inputenc}
\usetikzlibrary{arrows}
\usepackage{geometry}
\usepackage{tabvar}
\geometry{hmargin=2.2cm,vmargin=1.5cm}\pagestyle{fancy}
\lfoot{\bfseries http://www.bibmath.net}
\rfoot{\bfseries\thepage}
\cfoot{}
\renewcommand{\footrulewidth}{0.5pt} %Filet en bas de page

 %Macros utilisées dans la base de données d'exercices 

\newcommand{\mtn}{\mathbb{N}}
\newcommand{\mtns}{\mathbb{N}^*}
\newcommand{\mtz}{\mathbb{Z}}
\newcommand{\mtr}{\mathbb{R}}
\newcommand{\mtk}{\mathbb{K}}
\newcommand{\mtq}{\mathbb{Q}}
\newcommand{\mtc}{\mathbb{C}}
\newcommand{\mch}{\mathcal{H}}
\newcommand{\mcp}{\mathcal{P}}
\newcommand{\mcb}{\mathcal{B}}
\newcommand{\mcl}{\mathcal{L}}
\newcommand{\mcm}{\mathcal{M}}
\newcommand{\mcc}{\mathcal{C}}
\newcommand{\mcmn}{\mathcal{M}}
\newcommand{\mcmnr}{\mathcal{M}_n(\mtr)}
\newcommand{\mcmnk}{\mathcal{M}_n(\mtk)}
\newcommand{\mcsn}{\mathcal{S}_n}
\newcommand{\mcs}{\mathcal{S}}
\newcommand{\mcd}{\mathcal{D}}
\newcommand{\mcsns}{\mathcal{S}_n^{++}}
\newcommand{\glnk}{GL_n(\mtk)}
\newcommand{\mnr}{\mathcal{M}_n(\mtr)}
\DeclareMathOperator{\ch}{ch}
\DeclareMathOperator{\sh}{sh}
\DeclareMathOperator{\vect}{vect}
\DeclareMathOperator{\card}{card}
\DeclareMathOperator{\comat}{comat}
\DeclareMathOperator{\imv}{Im}
\DeclareMathOperator{\rang}{rg}
\DeclareMathOperator{\Fr}{Fr}
\DeclareMathOperator{\diam}{diam}
\DeclareMathOperator{\supp}{supp}
\newcommand{\veps}{\varepsilon}
\newcommand{\mcu}{\mathcal{U}}
\newcommand{\mcun}{\mcu_n}
\newcommand{\dis}{\displaystyle}
\newcommand{\croouv}{[\![}
\newcommand{\crofer}{]\!]}
\newcommand{\rab}{\mathcal{R}(a,b)}
\newcommand{\pss}[2]{\langle #1,#2\rangle}
 %Document 

\begin{document} 

\begin{center}\textsc{{\huge notation de Landau partie 1}}\end{center}

% Exercice 587


\vskip0.3cm\noindent\textsc{Exercice 1} - Somme de factorielles
\vskip0.2cm
Montrer que 
$$\sum_{k=1}^n k!\sim_{+\infty} n!.$$


% Exercice 2598


\vskip0.3cm\noindent\textsc{Exercice 2} - \'Equivalents simples de suites
\vskip0.2cm
Trouver un équivalent le plus simple possible aux suites suivantes :
$$\begin{array}{lll}
\mathbf 1.\ u_n=\frac{1}{n-1}-\frac{1}{n+1}&\quad&\mathbf 2.\ v_n=\sqrt{n+1}-\sqrt{n-1}\\
\mathbf 3.\ w_n=\frac{n^3-\sqrt{1+n^2}}{\ln n-2n^2}&\quad&\mathbf 4.\ z_n=\sin\left(\frac1{\sqrt{n+1}}\right).
\end{array}$$


% Exercice 800


\vskip0.3cm\noindent\textsc{Exercice 3} - Comparaison entre exponentielle et factorielle
\vskip0.2cm
Soit $\gamma>0$. Le but de l'exercice est de prouver que 
$$e^{\gamma n}=o(n!).$$
Pour cela, on pose, pour $n\geq 1$, $u_n=e^{\gamma n}$ et $v_n=n!$. 
\begin{enumerate}
\item Démontrer qu'il existe un entier $n_0\in\mathbb N$ tel que, pour tout $n\geq n_0$, 
$$\frac{u_{n+1}}{u_n}\leq\frac 12\frac{v_{n+1}}{v_n}.$$
\item En déduire qu'il existe une constante $C>0$ telle que, pour tout $n\geq n_0$, on a
$$u_n\leq C\left(\frac 12\right)^{n-n_0}v_n.$$
\item Conclure.
\end{enumerate}




\vskip0.5cm
\noindent{\small Cette feuille d'exercices a été conçue à l'aide du site \textsf{http://www.bibmath.net}}

%Vous avez accès aux corrigés de cette feuille par l'url : https://www.bibmath.net/ressources/justeunefeuille.php?id=25491
\end{document}