
%%%%%%%%%%%%%%%%%% PREAMBULE %%%%%%%%%%%%%%%%%%

\documentclass[11pt,a4paper]{article}

\usepackage{amsfonts,amsmath,amssymb,amsthm}
\usepackage[utf8]{inputenc}
\usepackage[T1]{fontenc}
\usepackage[francais]{babel}
\usepackage{mathptmx}
\usepackage{fancybox}
\usepackage{graphicx}
\usepackage{ifthen}

\usepackage{tikz}   

\usepackage{hyperref}
\hypersetup{colorlinks=true, linkcolor=blue, urlcolor=blue,
pdftitle={Exo7 - Exercices de mathématiques}, pdfauthor={Exo7}}

\usepackage{geometry}
\geometry{top=2cm, bottom=2cm, left=2cm, right=2cm}

%----- Ensembles : entiers, reels, complexes -----
\newcommand{\Nn}{\mathbb{N}} \newcommand{\N}{\mathbb{N}}
\newcommand{\Zz}{\mathbb{Z}} \newcommand{\Z}{\mathbb{Z}}
\newcommand{\Qq}{\mathbb{Q}} \newcommand{\Q}{\mathbb{Q}}
\newcommand{\Rr}{\mathbb{R}} \newcommand{\R}{\mathbb{R}}
\newcommand{\Cc}{\mathbb{C}} \newcommand{\C}{\mathbb{C}}
\newcommand{\Kk}{\mathbb{K}} \newcommand{\K}{\mathbb{K}}

%----- Modifications de symboles -----
\renewcommand{\epsilon}{\varepsilon}
\renewcommand{\Re}{\mathop{\mathrm{Re}}\nolimits}
\renewcommand{\Im}{\mathop{\mathrm{Im}}\nolimits}
\newcommand{\llbracket}{\left[\kern-0.15em\left[}
\newcommand{\rrbracket}{\right]\kern-0.15em\right]}
\renewcommand{\ge}{\geqslant} \renewcommand{\geq}{\geqslant}
\renewcommand{\le}{\leqslant} \renewcommand{\leq}{\leqslant}

%----- Fonctions usuelles -----
\newcommand{\ch}{\mathop{\mathrm{ch}}\nolimits}
\newcommand{\sh}{\mathop{\mathrm{sh}}\nolimits}
\renewcommand{\tanh}{\mathop{\mathrm{th}}\nolimits}
\newcommand{\cotan}{\mathop{\mathrm{cotan}}\nolimits}
\newcommand{\Arcsin}{\mathop{\mathrm{arcsin}}\nolimits}
\newcommand{\Arccos}{\mathop{\mathrm{arccos}}\nolimits}
\newcommand{\Arctan}{\mathop{\mathrm{arctan}}\nolimits}
\newcommand{\Argsh}{\mathop{\mathrm{argsh}}\nolimits}
\newcommand{\Argch}{\mathop{\mathrm{argch}}\nolimits}
\newcommand{\Argth}{\mathop{\mathrm{argth}}\nolimits}
\newcommand{\pgcd}{\mathop{\mathrm{pgcd}}\nolimits} 

%----- Structure des exercices ------

\newcommand{\exercice}[1]{\video{0}}
\newcommand{\finexercice}{}
\newcommand{\noindication}{}
\newcommand{\nocorrection}{}

\newcounter{exo}
\newcommand{\enonce}[2]{\refstepcounter{exo}\hypertarget{exo7:#1}{}\label{exo7:#1}{\bf Exercice \arabic{exo}}\ \  #2\vspace{1mm}\hrule\vspace{1mm}}

\newcommand{\finenonce}[1]{
\ifthenelse{\equal{\ref{ind7:#1}}{\ref{bidon}}\and\equal{\ref{cor7:#1}}{\ref{bidon}}}{}{\par{\footnotesize
\ifthenelse{\equal{\ref{ind7:#1}}{\ref{bidon}}}{}{\hyperlink{ind7:#1}{\texttt{Indication} $\blacktriangledown$}\qquad}
\ifthenelse{\equal{\ref{cor7:#1}}{\ref{bidon}}}{}{\hyperlink{cor7:#1}{\texttt{Correction} $\blacktriangledown$}}}}
\ifthenelse{\equal{\myvideo}{0}}{}{{\footnotesize\qquad\texttt{\href{http://www.youtube.com/watch?v=\myvideo}{Vidéo $\blacksquare$}}}}
\hfill{\scriptsize\texttt{[#1]}}\vspace{1mm}\hrule\vspace*{7mm}}

\newcommand{\indication}[1]{\hypertarget{ind7:#1}{}\label{ind7:#1}{\bf Indication pour \hyperlink{exo7:#1}{l'exercice \ref{exo7:#1} $\blacktriangle$}}\vspace{1mm}\hrule\vspace{1mm}}
\newcommand{\finindication}{\vspace{1mm}\hrule\vspace*{7mm}}
\newcommand{\correction}[1]{\hypertarget{cor7:#1}{}\label{cor7:#1}{\bf Correction de \hyperlink{exo7:#1}{l'exercice \ref{exo7:#1} $\blacktriangle$}}\vspace{1mm}\hrule\vspace{1mm}}
\newcommand{\fincorrection}{\vspace{1mm}\hrule\vspace*{7mm}}

\newcommand{\finenonces}{\newpage}
\newcommand{\finindications}{\newpage}


\newcommand{\fiche}[1]{} \newcommand{\finfiche}{}
%\newcommand{\titre}[1]{\centerline{\large \bf #1}}
\newcommand{\addcommand}[1]{}

% variable myvideo : 0 no video, otherwise youtube reference
\newcommand{\video}[1]{\def\myvideo{#1}}

%----- Presentation ------

\setlength{\parindent}{0cm}

\definecolor{myred}{rgb}{0.93,0.26,0}
\definecolor{myorange}{rgb}{0.97,0.58,0}
\definecolor{myyellow}{rgb}{1,0.86,0}

\newcommand{\LogoExoSept}[1]{  % input : echelle       %% NEW
{\usefont{U}{cmss}{bx}{n}
\begin{tikzpicture}[scale=0.1*#1,transform shape]
  \fill[color=myorange] (0,0)--(4,0)--(4,-4)--(0,-4)--cycle;
  \fill[color=myred] (0,0)--(0,3)--(-3,3)--(-3,0)--cycle;
  \fill[color=myyellow] (4,0)--(7,4)--(3,7)--(0,3)--cycle;
  \node[scale=5] at (3.5,3.5) {Exo7};
\end{tikzpicture}}
}


% titre
\newcommand{\titre}[1]{%
\vspace*{-4ex} \hfill \hspace*{1.5cm} \hypersetup{linkcolor=black, urlcolor=black} 
\href{http://exo7.emath.fr}{\LogoExoSept{3}} 
 \vspace*{-5.7ex}\newline 
\hypersetup{linkcolor=blue, urlcolor=blue}  {\Large \bf #1} \newline 
 \rule{12cm}{1mm} \vspace*{3ex}}

%----- Commandes supplementaires ------



\begin{document}

%%%%%%%%%%%%%%%%%% EXERCICES %%%%%%%%%%%%%%%%%%

\fiche{f00157, bodin, 2012/05/19} 

\titre{Calculs sur les matrices}

Corrections d'Arnaud Bodin.

\section{Opérations sur les matrices}

\exercice{1040, liousse, 2003/10/01}
\video{XwtvirsK2HU}
\enonce{001040}{}
Effectuer le produit des  matrices  :
$$\left( \begin{array}{cc} 2 & 1  \\ 3& 2 \end{array} \right)\times
 \left( \begin{array}{cc}1 & -1  \\ 1& 2 \end{array} \right) \ \ \ \  \left( \begin{array}
 {ccc} 1 & 2 & 0  \\ 3 & 1 & 4 \end{array} \right) 
\times
\left( \begin{array}{ccc} -1 &-1& 0 \\ 1 & 4 & -1 \\ 2 & 1 & 2\end{array} \right)
 \ \ \ \ \left( \begin{array}{ccc} a &b& c \\ c & b & a \\ 1 & 1 & 1\end{array} \right)
\times
 \left( \begin{array}{ccc} 1 &a& c \\ 1 & b & b \\ 1 & c & a\end{array} \right)$$
\finenonce{001040}


\finexercice

\exercice{1061, ridde, 1999/11/01}
\video{O1CDw2RqXUw}
\enonce{001061}{}
Soit $A(\theta) = \begin{pmatrix} \cos \theta & -\sin \theta \\ \sin \theta
& \cos \theta \end{pmatrix}$ pour $\theta \in \Rr$. Calculer $A(\theta) \times A(\theta')$ et $\big(A(\theta)\big)^n$ pour
$n \ge 1$.
\finenonce{001061}


\finexercice
\exercice{1063, ridde, 1999/11/01}
\video{-ew4tp-RHJY}
\enonce{001063}{}
Soient $A$ et $B \in \mathcal{M}_n(\Rr)$ telles que $\forall X \in \mathcal{M}_n (\Rr)$,
$\text{tr} (AX) = \text{tr} (BX)$. Montrer que $A = B$.
\finenonce{001063}



\finexercice
\exercice{1064, ridde, 1999/11/01}
\video{IfhgaXB6Oh0}
\enonce{001064}{}
Que peut-on dire d'une matrice $A \in \mathcal{M}_n (\Rr)$ qui vérifie $\text{tr} (A \ {}^{t}\!{A}) = 0$ ?
\finenonce{001064}



\finexercice
\section{Inverse}


\exercice{6872, bodin, 2012/05/29}
\video{SbWnWREUACY}
\enonce{006872}{}
Calculer (s'il existe) l'inverse des matrices :
$$
\begin{pmatrix} a & b \\ c & d \\ \end{pmatrix}
\qquad 
\begin{pmatrix} 1&2&1 \\ 1&2&-1\\ -2&-2&-1 \end{pmatrix}
\qquad 
\begin{pmatrix} 
1 &\bar\alpha &\bar\alpha^2 \cr
\alpha   &1          &\bar\alpha   \cr
\alpha^2 &\alpha     &1            \cr \end{pmatrix} (\alpha \in \Cc)
\qquad
\begin{pmatrix} 
0 & 1 & 1 & 1 \\ 1 & 0 & 1 & 1 \\ 1 & 1 & 0 & 1 \\ 1 & 1 & 1 & 0 \\ 
\end{pmatrix}
$$

$$
\begin{pmatrix}
1 & 1 & \cdots &\cdots & 1 \\
0 & 1 & \ddots & & \vdots  \\
 & \ddots & \ddots & \ddots &  \vdots \\
& \cdots & 0 & 1 & 1 \\
0 & & \cdots & 0 & 1
\end{pmatrix}
\qquad 
\begin{pmatrix}
1 & 2 & 3 & \cdots & n \\
0 & 1 & 2 & \cdots & \vdots \\
  & \ddots & \ddots & \ddots &  \vdots \\
\vdots &  & 0 & 1 & 2 \\
 0 & \cdots & & 0 & 1
\end{pmatrix}
$$
\finenonce{006872}


\finexercice

\exercice{1052, legall, 1998/09/01}
\video{7AwRBOj2CYY}
\enonce{001052}{}
Soit 
$A=\begin{pmatrix} 
1 & 0 & 2 \cr
0 & -1 & 1 \cr
1 & -2 & 0 \cr
\end{pmatrix}$. 
Calculer $A^3-A$. En déduire que $A$ est inversible puis déterminer $A^{-1}$.
\finenonce{001052}




\finexercice
\exercice{3380, quercia, 2010/03/09}
\video{dad4PGxRe4U}
\enonce{003380}{$M$ antisymétrique $ \Rightarrow  I+M$ est inversible}

Soit $M \in \mathcal{M}_n(\R)$ antisymétrique.

\begin{enumerate}
  \item Montrer que $I+M$ est inversible (si $(I+M)X = 0$, calculer ${}^t\!(MX)(MX)$).
  \item Soit $A = (I-M)(I+M)^{-1}$. Montrer que ${}^t\!A = A^{-1}$.
\end{enumerate}
\finenonce{003380}



\finexercice

\exercice{1069, gourio, 2001/09/01}
\video{XdgqhVnKjTg}
\enonce{001069}{}
$A=\left( a_{i,j}\right) \in \mathcal{M}_{n}(\Rr)$
telle que :
$$\forall i =1,\ldots,n \qquad \left| a_{i,i}\right| >\sum_{j\neq i}\left|a_{i,j}\right| . $$
Montrer que $A$ est inversible.
\finenonce{001069}


\finexercice

\finfiche



 \finenonces 



 \finindications 

\noindication
\indication{001061}
Il faut connaître les formules de $\cos(\theta+\theta')$ et $\sin(\theta+\theta')$.
\finindication
\indication{001063}
Essayer avec $X$ la matrice élémentaire $E_{ij}$ (des zéros partout sauf le coefficient $1$ à la
 $i$-ème ligne et la $j$-ème colonne).
\finindication
\indication{001064}
Appliquer la formule du produit pour calculer les coefficients diagonaux de $A\ {}^{t}\!{A}$
\finindication
\noindication
\indication{001052}
Une fois que l'on a calculé $A^2$ et $A^3$ on peut en déduire $A^{-1}$ sans calculs.
\finindication
\indication{003380}
$M$ antisymétrique signifie ${}^{t}\!{M}=-M$.
\begin{enumerate}
  \item Si $Y$ est un vecteur alors ${}^t\!Y Y = \| Y\|^2$ est un réel positif ou nul.
  \item $I-M$ et $(I+M)^{-1}$ commutent.
\end{enumerate}
\finindication
\indication{001069}
Prendre un vecteur $X=\left(
\begin{array}{c}
x_1\\
\vdots\\
x_n
\end{array}
\right)$ tel que $AX=0$, considérer le rang $i_0$ tel $|x_{i_0}|=\max\big\{|x_i| \mid i=1,...,n\big\}$.
\finindication


\newpage

\correction{001040}
Si $C = A \times B$ alors on obtient le coefficient $c_{ij}$ (situé à la $i$-ème ligne et la $j$-ème colonne
de $C$) en effectuant le produit scalaire du $i$-ème vecteur-ligne de $A$ avec le $j$-éme vecteur colonne de $B$.

On trouve  
$$\left( \begin{array}{cc} 2 & 1  \\ 3& 2 \end{array} \right)\times
 \left( \begin{array}{cc}1 & -1  \\ 1& 2 \end{array} \right)
= \begin{pmatrix}
  3 & 0 \\
  5 & 1 \\  
  \end{pmatrix}$$

$$\left( \begin{array}
 {ccc} 1 & 2 & 0  \\ 3 & 1 & 4 \end{array} \right) 
\times
\left( \begin{array}{ccc} -1 &-1& 0 \\ 1 & 4 & -1 \\ 2 & 1 & 2\end{array} \right)
= \begin{pmatrix}
  1 & 7 & -2 \\
  6 & 5 & 7 \\ 
  \end{pmatrix}$$

$$\left( \begin{array}{ccc} a &b& c \\ c & b & a \\ 1 & 1 & 1\end{array} \right)
\times
 \left( \begin{array}{ccc} 1 &a& c \\ 1 & b & b \\ 1 & c & a\end{array} \right) 
 = \begin{pmatrix}
    a + b + c & a^2+b^2+c^2 & 2ac + b^2 \\
    a + b + c & 2ac + b^2   & a^2+b^2+c^2 \\
    3         &  a+b+c      & a+b+c \\
\end{pmatrix}$$
\fincorrection
\correction{001061}
\begin{align*}
A(\theta)\times A(\theta')  
  & =  \begin{pmatrix} \cos \theta & -\sin \theta \\ \sin \theta
& \cos \theta \end{pmatrix} \times \begin{pmatrix} \cos \theta' & -\sin \theta' \\ \sin \theta'
& \cos \theta' \end{pmatrix} \\
  & = \begin{pmatrix} 
\cos \theta\cos \theta' -\sin \theta \sin\theta' & - \cos \theta \sin \theta' - \sin \theta \cos \theta' \\
 \sin \theta \cos \theta'+\cos \theta \sin \theta'   & -\sin \theta \sin\theta'+\cos \theta\cos \theta'  \\
 \end{pmatrix} \\
  & =
\begin{pmatrix} \cos (\theta+\theta') & -\sin (\theta+\theta')  \\ \sin  (\theta+\theta')
& \cos  (\theta+\theta') \end{pmatrix} \\
  & = A(\theta+\theta') \\
\end{align*}

Bilan : $A(\theta)\times A(\theta') = A(\theta+\theta')$.

\bigskip

Nous allons montrer par récurrence sur $n\ge 1$ que $\big(A(\theta)\big)^n = A(n\theta)$.

\begin{itemize}
  \item C'est bien sûr vrai pour $n=1$.
  \item Fixons $n\ge 1$ et supposons que $\big(A(\theta)\big)^n = A(n\theta)$ alors
$$\big(A(\theta)\big)^{n+1} = \big(A(\theta)\big)^n \times A(\theta) = A(n\theta) \times A(\theta) = A(n\theta+\theta) = A((n+1)\theta)$$
  \item C'est donc vrai pour tout $n\ge 1$.
\end{itemize}


Remarques :
\begin{itemize}
  \item On aurait aussi la formule $A(\theta')\times A(\theta) = A(\theta+\theta') = A(\theta)\times A(\theta')$.
Les matrices $A(\theta)$ et $A(\theta')$ commutent.

  \item En fait il n'est pas plus difficile de montrer que $\big(A(\theta)\big)^{-1}=A(-\theta)$.
On sait aussi que par définition $\big(A(\theta)\big)^{0}=I$. Et on en déduit que pour $n\in \Zz$ on a 
$\big(A(\theta)\big)^n = A(n\theta)$.

  \item En terme géométrique $A(\theta)$ est la matrice de la rotation d'angle $\theta$ (centrée à l'origine).
On vient de montrer que si l'on compose un rotation d'angle $\theta$ avec un rotation d'angle $\theta'$
alors on obtient une rotation d'angle $\theta+\theta'$.
\end{itemize}

\fincorrection
\correction{001063}
Notons $E_{ij}$ la matrice élémentaire (des zéros partout sauf le coefficient $1$ à la
 $i$-ème ligne et la $j$-ème colonne).

Soit $A = (a_{ij})  \in \mathcal{M}_n(\Rr)$.
Alors 
$$A \times E_{ij} = 
\begin{pmatrix}  
0 & 0 & \cdots & 0 & a_{1i} & 0 & \cdots \\
0 & 0 & \cdots & 0 & a_{2i} & 0 & \cdots \\
\vdots&  & \cdots & & \vdots &  & \cdots \\
0 & 0 & \cdots & 0 & a_{ji} & 0 & \cdots \\
\vdots&  & \cdots & & \vdots &  & \cdots \\
0 & 0 & \cdots & 0 & a_{ni} & 0 & \cdots \\
\end{pmatrix}$$
La seule colonne non nulle est la $j$-ème colonne.

La trace est la somme des éléments sur la diagonale. Ici le seul élément non nul de la diagonale est
$a_{ji}$, on en déduit donc 
$$\text{tr} (A \times E_{ij})=a_{ji}$$
(attention à l'inversion des indices).

Maintenant prenons deux matrices $A, B$ telles que $\text{tr} (AX) = \text{tr} (BX)$
pour toute matrice $X$. Alors pour $X=E_{ij}$ on en déduit $a_{ji}=b_{ji}$.
On fait ceci pour toutes les matrices élémentaires $E_{ij}$ avec $1 \le i,j \le n$
ce qui implique $A=B$.
\fincorrection
\correction{001064}
Notons $A=(a_{ij})$, notons $B = {}^{t}\!{A}$ si les coefficients sont $B=(b_{ij})$ 
alors par définition de la transposée on a $b_{ij}= a_{ji}$.

Ensuite notons $C = A \times B$ alors par définition du produit de matrices 
le coefficients $c_{ij}$ de $C$ s'obtient par la formule :
$$c_{ij} = \sum_{k=1}^n a_{ik}b_{kj}.$$

Appliquons ceci avec $B = {}^{t}\!{A}$
$$c_{ij} = \sum_{k=1}^n a_{ik}b_{kj} = \sum_{k=1}^n a_{ik}a_{jk}.$$
Et pour un coefficient de la diagonale on a $i=j$ donc 
$$c_{ii} =  \sum_{k=1}^n a_{ik}^2.$$

La trace étant la somme des coefficients sur la diagonale on a :
$$\text{tr} (A \ {}^{t}\!{A}) = \text{tr} (C) 
= \sum_{i=1}^n c_{ii} =  \sum_{i=1}^n \sum_{k=1}^n a_{ik}^2 = \sum_{1\le i,k \le n} a_{ik}^2.$$

Si on change l'indice $k$ en $j$ on obtient 
$$\text{tr} (A \ {}^{t}\!{A})  = \sum_{1\le i,j \le n} a_{ij}^2.$$

Donc cette trace vaut la somme des carrés de tous les coefficients.

Conséquence : si $\text{tr} (A \ {}^{t}\!{A}) = 0$ alors la somme des carrés 
$\sum_{1\le i,j \le n} a_{ij}^2$ est nulle 
donc chaque carré $a_{ij}^2$ est nul. Ainsi $a_{ij}=0$ (pour tout $i,j$) autrement dit $A$ est la matrice nulle.
\fincorrection
\correction{006872}

\begin{enumerate}
  \item  si le déterminant $ad-bc$ est non nul l'inverse est 
$\frac{1}{ad-bc}\begin{pmatrix} d & -b \\ -c & a \\ \end{pmatrix}$
  \item $\frac14 \begin{pmatrix} -4 & 0 & -4 \\ 3 & 1 & 2 \\2 & -2 & 0 \\  \end{pmatrix}$

  \item si $|\alpha|\neq 1$ alors l'inverse est $\frac 1{1-\alpha\bar\alpha}
            \begin{pmatrix} 1       &-\bar\alpha        &0           \cr
                      -\alpha &1+\alpha\bar\alpha &-\bar\alpha \cr
                      0       &-\alpha            &1           \cr\end{pmatrix}$

   \item $\frac13 \begin{pmatrix}
-2 & 1 & 1 & 1 \\ 1 & -2 & 1 & 1 \\ 1 & 1 & -2 & 1 \\ 1 & 1 & 1 & -2 \\
                                      \end{pmatrix}$

  \item $\begin{pmatrix}
1 & -1 & 0 & \cdots & 0 \\
0 & 1 & -1 & 0 & \cdots\\
 & \ddots & \ddots & \ddots &  \\
\vdots & & 0 & 1 & -1  \\
0 & \cdots & \cdots & 0 & 1
\end{pmatrix}$


  \item $\begin{pmatrix}
1 & -2 & 1 & 0 &\cdots & 0 \\
 & 1 & -2 & 1 & 0 & \cdots\\
 &  & 1 & -2 & 1 & 0 \\
 & &  & \ddots & \ddots & \ddots \\
 &  & (0) &  & 1 & -2 \\
 &  &  & &  & 1
\end{pmatrix}$
\end{enumerate}
\fincorrection
\correction{001052}

On trouve 
$$A^2 =
\begin{pmatrix} 
3 & -4 & 2 \cr
1 & -1 & -1 \cr
1 & 2 & 0 \cr
\end{pmatrix}
\qquad \text{ et } \qquad 
A^3=
\begin{pmatrix} 
5 & 0 & 2 \cr
0 & 3 & 1 \cr
1 & -2 & 4 \cr
\end{pmatrix}
.$$

Un calcul donne $A^3-A = 4 I$.
En factorisant par $A$ on obtient $A\times (A^2-I) = 4I$.
Donc $A \times \frac 1 4 (A^2-I) = I$, ainsi $A$ est inversible et
$$A^{-1} = \frac 1 4 (A^2-I) = \frac 1 4
\begin{pmatrix}
2&-4&2\\
1&-2&-1\\
1&2&-1\\
\end{pmatrix}.
$$
\fincorrection
\correction{003380}

Avant de commencer la résolution nous allons faire une remarque importante :
pour $X = \begin{pmatrix}x_1 \\ x_2 \\ \vdots \\ x_n \end{pmatrix}$ un vecteur 
(considéré comme une matrice à une seule colonne)
alors nous allons calculer ${}^t\!X X$ : 
$${}^t\!X X = (x_1,x_2,\cdots,x_n) \begin{pmatrix}x_1 \\ x_2 \\ \vdots \\ x_n \end{pmatrix} = x_1^2+x_2^2+\cdots+x_n^2.$$
On note $ \| X \|^2 = {}^t\!X X$ : $\| X \|$ est la \emph{norme} ou la \emph{longueur} du vecteur $X$.
De ce calcul on déduit d'une part que ${}^t\!X X \ge 0$. Et aussi que  ${}^t\!X X \ge 0$ si et seulement si
$X$ est le vecteur nul.


\begin{enumerate}
  \item Nous allons montrer que $I+M$ est inversible en montrant que si un vecteur $X$ vérifie
$(I+M)X = 0$ alors $X=0$.

Nous allons estimer ${}^t\!(MX)(MX)$ de deux façons.
D'une part c'est un produit de la forme ${}^t\!Y Y = \| Y\|^2$ et donc ${}^t\!(MX)(MX) \ge 0$.

D'autre part :
\begin{align*}
{}^t\!(MX)(MX) 
 & = {}^t\!(MX) (-X) \quad \text{ car } (I+M)X = 0 \text{ donc } MX = -X \\
 & = {}^t\!X {}^t\!M (-X)    \quad \text{ car } {}^t\!(AB)={}^t\! B {}^t\!A \\
 & = {}^t\!X (-M) (-X)    \quad \text{ car } {}^t\!M=-M \\
 & = {}^t\!X MX \\
 & = {}^t\!X (-X) \\
 & = -{}^t\!XX \\
 & = - \|X\|^2 \\
\end{align*}
Qui est donc négatif.

Seule possibilité $\|X\|^2=0$ donc $X=0$ (= le vecteur nul) et donc $I+M$ inversible.

\item 
  \begin{enumerate}
    \item Calculons $A^{-1}$.
$$A^{-1} =\big( (I-M)\times(I+M)^{-1} \big)^{-1} = \big( (I+M)^{-1} \big)^{-1} \times(I-M)^{-1} = (I+M)\times(I-M)^{-1}$$
(n'oubliez pas que $(AB)^{-1}=B^{-1}A^{-1}$).


     \item Calculons ${}^t\!A$.
\begin{align*}
{}^t\!A 
  & = {}^t\!\big( (I-M) \times (I+M)^{-1} \big) \\
  & = {}^t\!\big( (I+M)^{-1} \big) \times {}^t\!(I-M) \qquad \text{ car } {}^t\!(AB)={}^t\!B{}^t\!A \\
  & = \big( {}^t\!(I+M) \big)^{-1}\times{}^t\!(I-M) \qquad \text{ car } {}^t\!(A^{-1})=\big({}^tA\big)^{-1} \\
  & = \big( I+{}^t\!M) \big)^{-1}\times(I-{}^t\!M) \qquad \text{ car } {}^t\!(A+B)={}^t\!A+{}^t\!B \\
  & = (I-M) ^{-1}\times (I+M) \qquad \text{ car ici } {}^{t}\!{M} = -M \\  
\end{align*}

      \item Montrons que $I+M$ et $(I-M) ^{-1}$ commutent.

Tout d'abord $I+M$ et $I-M$ commutent car $(I+M)(I-M) = I-M^2 = (I-M)(I+M)$.
Maintenant nous avons le petit résultat suivant :

\textbf{Lemme.} Si $AB=BA$ alors $AB^{-1}=B^{-1}A$.

Pour la preuve on écrit :
$$AB=BA \Rightarrow B^{-1}(AB)B^{-1}=B^{-1}(BA)B^{-1} \Rightarrow B^{-1}A=AB^{-1}.$$

En appliquant ceci à $I+M$ et $I-M$ on trouve $(I+M)\times (I-M) ^{-1}= (I-M) ^{-1}\times(I+M)$
et donc $A^{-1}={}^t\!A$.

    \end{enumerate}
\end{enumerate}

\fincorrection
\correction{001069}
Soit $X=\left(
\begin{array}{c}
x_1\\
\vdots\\
x_n
\end{array}
\right)$  un vecteur tel que $AX=0$. 
Nous allons montrer qu'alors $X$ est le vecteur nul ce qui entraîne que $A$ est inversible.

Par l'absurde supposons $X\neq0$. 
Alors, si $i_0$ est un indice tel que $|x_{i_0}|=\max\big\{|x_i| \mid i=1,\ldots,n\big\}$, on a  $|x_{i_0}|>0$.

Mais alors comme $AX=0$ on a pour tout $i=1,\ldots,n$ :
$$\sum_{j=1}^{n}a_{i,j}x_j=0$$
donc 
$$|a_{i_0,i_0}x_{i_0}|=\left|-\sum_{j\neq i_0}^{}a_{i_0,j}x_j\right|\leq\sum_{j\neq i_0}^{}|a_{i_0,j}|.|x_j|
 \leq|x_{i_0}|\sum_{j\neq i_0}^{}|a_{i_0,j}|
$$
et, puisque $|x_{i_0}|> 0$, on obtient $|a_{i_0,i_0}|\leq\sum_{j\neq i_0}^{}|a_{i_0,j}|$ contredisant les hypothèses de l'énoncé. 
Ainsi $X=0$. On a donc prouvé <<$AX=0 \Rightarrow X=0$>> ce qui équivaut à $A$ inversible.
\fincorrection


\end{document}

