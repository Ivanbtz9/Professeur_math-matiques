
%%%%%%%%%%%%%%%%%% PREAMBULE %%%%%%%%%%%%%%%%%%

\documentclass[11pt,a4paper]{article}

\usepackage{amsfonts,amsmath,amssymb,amsthm}
\usepackage[utf8]{inputenc}
\usepackage[T1]{fontenc}
\usepackage[francais]{babel}
\usepackage{mathptmx}
\usepackage{fancybox}
\usepackage{graphicx}
\usepackage{ifthen}

\usepackage{tikz}   

\usepackage{hyperref}
\hypersetup{colorlinks=true, linkcolor=blue, urlcolor=blue,
pdftitle={Exo7 - Exercices de mathématiques}, pdfauthor={Exo7}}

\usepackage{geometry}
\geometry{top=2cm, bottom=2cm, left=2cm, right=2cm}

%----- Ensembles : entiers, reels, complexes -----
\newcommand{\Nn}{\mathbb{N}} \newcommand{\N}{\mathbb{N}}
\newcommand{\Zz}{\mathbb{Z}} \newcommand{\Z}{\mathbb{Z}}
\newcommand{\Qq}{\mathbb{Q}} \newcommand{\Q}{\mathbb{Q}}
\newcommand{\Rr}{\mathbb{R}} \newcommand{\R}{\mathbb{R}}
\newcommand{\Cc}{\mathbb{C}} \newcommand{\C}{\mathbb{C}}
\newcommand{\Kk}{\mathbb{K}} \newcommand{\K}{\mathbb{K}}

%----- Modifications de symboles -----
\renewcommand{\epsilon}{\varepsilon}
\renewcommand{\Re}{\mathop{\mathrm{Re}}\nolimits}
\renewcommand{\Im}{\mathop{\mathrm{Im}}\nolimits}
\newcommand{\llbracket}{\left[\kern-0.15em\left[}
\newcommand{\rrbracket}{\right]\kern-0.15em\right]}
\renewcommand{\ge}{\geqslant} \renewcommand{\geq}{\geqslant}
\renewcommand{\le}{\leqslant} \renewcommand{\leq}{\leqslant}

%----- Fonctions usuelles -----
\newcommand{\ch}{\mathop{\mathrm{ch}}\nolimits}
\newcommand{\sh}{\mathop{\mathrm{sh}}\nolimits}
\renewcommand{\tanh}{\mathop{\mathrm{th}}\nolimits}
\newcommand{\cotan}{\mathop{\mathrm{cotan}}\nolimits}
\newcommand{\Arcsin}{\mathop{\mathrm{arcsin}}\nolimits}
\newcommand{\Arccos}{\mathop{\mathrm{arccos}}\nolimits}
\newcommand{\Arctan}{\mathop{\mathrm{arctan}}\nolimits}
\newcommand{\Argsh}{\mathop{\mathrm{argsh}}\nolimits}
\newcommand{\Argch}{\mathop{\mathrm{argch}}\nolimits}
\newcommand{\Argth}{\mathop{\mathrm{argth}}\nolimits}
\newcommand{\pgcd}{\mathop{\mathrm{pgcd}}\nolimits} 

%----- Structure des exercices ------

\newcommand{\exercice}[1]{\video{0}}
\newcommand{\finexercice}{}
\newcommand{\noindication}{}
\newcommand{\nocorrection}{}

\newcounter{exo}
\newcommand{\enonce}[2]{\refstepcounter{exo}\hypertarget{exo7:#1}{}\label{exo7:#1}{\bf Exercice \arabic{exo}}\ \  #2\vspace{1mm}\hrule\vspace{1mm}}

\newcommand{\finenonce}[1]{
\ifthenelse{\equal{\ref{ind7:#1}}{\ref{bidon}}\and\equal{\ref{cor7:#1}}{\ref{bidon}}}{}{\par{\footnotesize
\ifthenelse{\equal{\ref{ind7:#1}}{\ref{bidon}}}{}{\hyperlink{ind7:#1}{\texttt{Indication} $\blacktriangledown$}\qquad}
\ifthenelse{\equal{\ref{cor7:#1}}{\ref{bidon}}}{}{\hyperlink{cor7:#1}{\texttt{Correction} $\blacktriangledown$}}}}
\ifthenelse{\equal{\myvideo}{0}}{}{{\footnotesize\qquad\texttt{\href{http://www.youtube.com/watch?v=\myvideo}{Vidéo $\blacksquare$}}}}
\hfill{\scriptsize\texttt{[#1]}}\vspace{1mm}\hrule\vspace*{7mm}}

\newcommand{\indication}[1]{\hypertarget{ind7:#1}{}\label{ind7:#1}{\bf Indication pour \hyperlink{exo7:#1}{l'exercice \ref{exo7:#1} $\blacktriangle$}}\vspace{1mm}\hrule\vspace{1mm}}
\newcommand{\finindication}{\vspace{1mm}\hrule\vspace*{7mm}}
\newcommand{\correction}[1]{\hypertarget{cor7:#1}{}\label{cor7:#1}{\bf Correction de \hyperlink{exo7:#1}{l'exercice \ref{exo7:#1} $\blacktriangle$}}\vspace{1mm}\hrule\vspace{1mm}}
\newcommand{\fincorrection}{\vspace{1mm}\hrule\vspace*{7mm}}

\newcommand{\finenonces}{\newpage}
\newcommand{\finindications}{\newpage}


\newcommand{\fiche}[1]{} \newcommand{\finfiche}{}
%\newcommand{\titre}[1]{\centerline{\large \bf #1}}
\newcommand{\addcommand}[1]{}

% variable myvideo : 0 no video, otherwise youtube reference
\newcommand{\video}[1]{\def\myvideo{#1}}

%----- Presentation ------

\setlength{\parindent}{0cm}

\definecolor{myred}{rgb}{0.93,0.26,0}
\definecolor{myorange}{rgb}{0.97,0.58,0}
\definecolor{myyellow}{rgb}{1,0.86,0}

\newcommand{\LogoExoSept}[1]{  % input : echelle       %% NEW
{\usefont{U}{cmss}{bx}{n}
\begin{tikzpicture}[scale=0.1*#1,transform shape]
  \fill[color=myorange] (0,0)--(4,0)--(4,-4)--(0,-4)--cycle;
  \fill[color=myred] (0,0)--(0,3)--(-3,3)--(-3,0)--cycle;
  \fill[color=myyellow] (4,0)--(7,4)--(3,7)--(0,3)--cycle;
  \node[scale=5] at (3.5,3.5) {Exo7};
\end{tikzpicture}}
}


% titre
\newcommand{\titre}[1]{%
\vspace*{-4ex} \hfill \hspace*{1.5cm} \hypersetup{linkcolor=black, urlcolor=black} 
\href{http://exo7.emath.fr}{\LogoExoSept{3}} 
 \vspace*{-5.7ex}\newline 
\hypersetup{linkcolor=blue, urlcolor=blue}  {\Large \bf #1} \newline 
 \rule{12cm}{1mm} \vspace*{3ex}}

%----- Commandes supplementaires ------



\begin{document}

%%%%%%%%%%%%%%%%%% EXERCICES %%%%%%%%%%%%%%%%%%

\fiche{f00070, tumpach, 2009/10/16}

Enoncés : Barbara Tumpach

\titre{Calcul matriciel}

\textit{Le but de cette feuille d'exercices est d'apprendre les op\'erations sur les matrices~: somme, produit de matrices, transpos\'ee, puissances d'une matrice, inverse.}

\exercice{2747, tumpach, 2009/10/25}
\enonce{002747}{}
On consid\`ere les matrices suivantes~:
$$
A = \left(\begin{array}{ccc}1& 2 & -1\\ 2 & 3 & -2\\ 0 &0 & 0\end{array}\right); \quad\quad B = \left(\begin{array}{ccc}1 & 0 & -1\\ 2 & 0 & 4\\ 1 & 0 & -2\end{array}\right); \quad\quad  C = \left(\begin{array}{cc} 2 &-2 \\ 1 & 1\\ 3 & 1\end{array}\right);$$
$$D = \left(\begin{array}{c} 2 \\ 1 \\ 3\end{array}\right);\quad\quad E = \left(\begin{array}{ccc}0 & 1 & 2 \end{array}\right).
$$
Calculer \textit{lorsque cela est bien d\'efini} les produits de matrices suivants~: $AB$, $BA$, $AC$, $CA$, $AD$, $AE$, $BC$, $BD$, $BE$, $CD$, $DE$.
\finenonce{002747}



\finexercice
\exercice{2748, tumpach, 2009/10/25}
\enonce{002748}{}
Soient les matrices suivantes~:
$$
A = \left(\begin{array}{ccc}2& 5 & -1\\ 0 & 1 & 3\\ 0 &-2 & 4\end{array}\right); \quad\quad B = \left(\begin{array}{ccc}1 & 7 & -1\\ 2 & 3 & 4\\ 0 & 0 & 0\end{array}\right); \quad\quad  C = \left(\begin{array}{cc} 1 & 2 \\ 0 & 4\\ -1 & 0\end{array}\right).
$$
Calculer~:
 $(A - 2 B)C$,
$C^{T}A$,
$C^T B$,
$C^{T}(A^T - 2 B^T)$,
o\`u $C^T$ d\'esigne la matrice transpos\'ee de $C$.
\finenonce{002748}



\finexercice
\exercice{2749, tumpach, 2009/10/25}
\enonce{002749}{}
Calculer $A^n$ pour tout $n\in\mathbb{Z}$, avec successivement
$$
A = \left(\begin{array}{cc}\cos(a) & -\sin(a)\\ \sin(a) & \cos(a)\end{array}\right), \quad\quad\left(\begin{array}{cc}\cosh(a) & \sinh(a)\\ \sinh(a) & \cosh(a)\end{array}\right).
$$
\finenonce{002749}



\finexercice
\exercice{2750, tumpach, 2009/10/25}
\enonce{002750}{}
Les matrices suivantes sont-elles inversibles? Si oui, calculer leurs inverses.
$$
\left(\begin{array}{ccc}1 & 2 & 3\\ 2 & 3 & 1\\ 3 & 1 & 2\end{array}\right),
\left(\begin{array}{ccc} 1 & 0 & -1\\ 2 & 0  & 1\\1 & 1 & 3\end{array}\right),
\left(\begin{array}{ccc} 2 & 1 & -1\\0 & 3 & 0\\0 & 2 & 1\end{array}\right).
$$
\finenonce{002750}



\finexercice
\exercice{2751, tumpach, 2009/10/25}
\enonce{002751}{}
Inverser les matrices suivantes~:
$$
\left(\begin{array}{cccc} 1 & 1 & 1 & 1\\ 1 & 1 & -1 & -1\\ 1 & -1 & 1 & -1\\1 & -1 & -1 & -1\end{array}\right),
\left(\begin{array}{cccc} 1 & a & a^2 & a^3\\ 0 & 1 & a & a^2\\ 0 & 0 & 1 & a\\  0 & 0 & 0 &1\end{array}\right),
\left(\begin{array}{cccc}1 & 2 & 3 & 4\\ 0 & 1 & 2 & 3\\ 0 & 0 & 1& 2\\ 0 & 0& 0&1\end{array}\right).
$$
\finenonce{002751}



\finexercice
\exercice{2752, tumpach, 2009/10/25}
\enonce{002752}{}
L'\textit{exponentielle} d'une matrice carr\'ee $M$ est, par d\'efinition, la limite de la s\'erie 
$$
e^M = 1 + M + \frac{M^2}{2!} + \dots = \lim_{n\rightarrow+\infty} \sum_{k=0}^{n}\frac{M^k}{k!}.
$$
On admet que cette limite existe en vertu d'un th\'eor\`eme d'analyse.
\begin{enumerate}
\item Montrer que si $AB = BA$ alors $e^{A+B} = e^A e^B$. On est autoris\'e, pour traiter cette question, \`a passer \`a la limite sans pr\'ecautions.
\item Calculer $e^M$ pour les quatre matrices suivantes~:
$$
\left(\begin{array}{ccc}
a & 0 & 0\\0 & b & 0\\0 & 0& c
\end{array}\right),
\left(\begin{array}{ccc}
0 & a & b\\0 & 0 & c\\ 0 & 0 & 0
\end{array}\right),
\left(\begin{array}{cc}
0 & 1\\-1 & 0
\end{array}\right),
\left(\begin{array}{cc}
1 & 0 \\ 0  & 0
\end{array}\right).
$$
\item Chercher un exemple simple o\`u $e^{A + B}\neq e^A e^B$.
\end{enumerate}
\finenonce{002752}



\finexercice

\finfiche 



 \finenonces 



 \finindications 

\noindication
\noindication
\noindication
\noindication
\noindication
\noindication


\newpage

\nocorrection
\nocorrection
\nocorrection
\nocorrection
\nocorrection
\nocorrection


\end{document}

