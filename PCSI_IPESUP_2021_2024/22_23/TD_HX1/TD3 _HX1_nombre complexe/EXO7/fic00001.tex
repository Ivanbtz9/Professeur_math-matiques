
%%%%%%%%%%%%%%%%%% PREAMBULE %%%%%%%%%%%%%%%%%%

\documentclass[11pt,a4paper]{article}

\usepackage{amsfonts,amsmath,amssymb,amsthm}
\usepackage[utf8]{inputenc}
\usepackage[T1]{fontenc}
\usepackage[francais]{babel}
\usepackage{mathptmx}
\usepackage{fancybox}
\usepackage{graphicx}
\usepackage{ifthen}

\usepackage{tikz}   

\usepackage{hyperref}
\hypersetup{colorlinks=true, linkcolor=blue, urlcolor=blue,
pdftitle={Exo7 - Exercices de mathématiques}, pdfauthor={Exo7}}

\usepackage{geometry}
\geometry{top=2cm, bottom=2cm, left=2cm, right=2cm}

%----- Ensembles : entiers, reels, complexes -----
\newcommand{\Nn}{\mathbb{N}} \newcommand{\N}{\mathbb{N}}
\newcommand{\Zz}{\mathbb{Z}} \newcommand{\Z}{\mathbb{Z}}
\newcommand{\Qq}{\mathbb{Q}} \newcommand{\Q}{\mathbb{Q}}
\newcommand{\Rr}{\mathbb{R}} \newcommand{\R}{\mathbb{R}}
\newcommand{\Cc}{\mathbb{C}} \newcommand{\C}{\mathbb{C}}
\newcommand{\Kk}{\mathbb{K}} \newcommand{\K}{\mathbb{K}}

%----- Modifications de symboles -----
\renewcommand{\epsilon}{\varepsilon}
\renewcommand{\Re}{\mathop{\mathrm{Re}}\nolimits}
\renewcommand{\Im}{\mathop{\mathrm{Im}}\nolimits}
\newcommand{\llbracket}{\left[\kern-0.15em\left[}
\newcommand{\rrbracket}{\right]\kern-0.15em\right]}
\renewcommand{\ge}{\geqslant} \renewcommand{\geq}{\geqslant}
\renewcommand{\le}{\leqslant} \renewcommand{\leq}{\leqslant}

%----- Fonctions usuelles -----
\newcommand{\ch}{\mathop{\mathrm{ch}}\nolimits}
\newcommand{\sh}{\mathop{\mathrm{sh}}\nolimits}
\renewcommand{\tanh}{\mathop{\mathrm{th}}\nolimits}
\newcommand{\cotan}{\mathop{\mathrm{cotan}}\nolimits}
\newcommand{\Arcsin}{\mathop{\mathrm{arcsin}}\nolimits}
\newcommand{\Arccos}{\mathop{\mathrm{arccos}}\nolimits}
\newcommand{\Arctan}{\mathop{\mathrm{arctan}}\nolimits}
\newcommand{\Argsh}{\mathop{\mathrm{argsh}}\nolimits}
\newcommand{\Argch}{\mathop{\mathrm{argch}}\nolimits}
\newcommand{\Argth}{\mathop{\mathrm{argth}}\nolimits}
\newcommand{\pgcd}{\mathop{\mathrm{pgcd}}\nolimits} 

%----- Structure des exercices ------

\newcommand{\exercice}[1]{\video{0}}
\newcommand{\finexercice}{}
\newcommand{\noindication}{}
\newcommand{\nocorrection}{}

\newcounter{exo}
\newcommand{\enonce}[2]{\refstepcounter{exo}\hypertarget{exo7:#1}{}\label{exo7:#1}{\bf Exercice \arabic{exo}}\ \  #2\vspace{1mm}\hrule\vspace{1mm}}

\newcommand{\finenonce}[1]{
\ifthenelse{\equal{\ref{ind7:#1}}{\ref{bidon}}\and\equal{\ref{cor7:#1}}{\ref{bidon}}}{}{\par{\footnotesize
\ifthenelse{\equal{\ref{ind7:#1}}{\ref{bidon}}}{}{\hyperlink{ind7:#1}{\texttt{Indication} $\blacktriangledown$}\qquad}
\ifthenelse{\equal{\ref{cor7:#1}}{\ref{bidon}}}{}{\hyperlink{cor7:#1}{\texttt{Correction} $\blacktriangledown$}}}}
\ifthenelse{\equal{\myvideo}{0}}{}{{\footnotesize\qquad\texttt{\href{http://www.youtube.com/watch?v=\myvideo}{Vidéo $\blacksquare$}}}}
\hfill{\scriptsize\texttt{[#1]}}\vspace{1mm}\hrule\vspace*{7mm}}

\newcommand{\indication}[1]{\hypertarget{ind7:#1}{}\label{ind7:#1}{\bf Indication pour \hyperlink{exo7:#1}{l'exercice \ref{exo7:#1} $\blacktriangle$}}\vspace{1mm}\hrule\vspace{1mm}}
\newcommand{\finindication}{\vspace{1mm}\hrule\vspace*{7mm}}
\newcommand{\correction}[1]{\hypertarget{cor7:#1}{}\label{cor7:#1}{\bf Correction de \hyperlink{exo7:#1}{l'exercice \ref{exo7:#1} $\blacktriangle$}}\vspace{1mm}\hrule\vspace{1mm}}
\newcommand{\fincorrection}{\vspace{1mm}\hrule\vspace*{7mm}}

\newcommand{\finenonces}{\newpage}
\newcommand{\finindications}{\newpage}


\newcommand{\fiche}[1]{} \newcommand{\finfiche}{}
%\newcommand{\titre}[1]{\centerline{\large \bf #1}}
\newcommand{\addcommand}[1]{}

% variable myvideo : 0 no video, otherwise youtube reference
\newcommand{\video}[1]{\def\myvideo{#1}}

%----- Presentation ------

\setlength{\parindent}{0cm}

\definecolor{myred}{rgb}{0.93,0.26,0}
\definecolor{myorange}{rgb}{0.97,0.58,0}
\definecolor{myyellow}{rgb}{1,0.86,0}

\newcommand{\LogoExoSept}[1]{  % input : echelle       %% NEW
{\usefont{U}{cmss}{bx}{n}
\begin{tikzpicture}[scale=0.1*#1,transform shape]
  \fill[color=myorange] (0,0)--(4,0)--(4,-4)--(0,-4)--cycle;
  \fill[color=myred] (0,0)--(0,3)--(-3,3)--(-3,0)--cycle;
  \fill[color=myyellow] (4,0)--(7,4)--(3,7)--(0,3)--cycle;
  \node[scale=5] at (3.5,3.5) {Exo7};
\end{tikzpicture}}
}


% titre
\newcommand{\titre}[1]{%
\vspace*{-4ex} \hfill \hspace*{1.5cm} \hypersetup{linkcolor=black, urlcolor=black} 
\href{http://exo7.emath.fr}{\LogoExoSept{3}} 
 \vspace*{-5.7ex}\newline 
\hypersetup{linkcolor=blue, urlcolor=blue}  {\Large \bf #1} \newline 
 \rule{12cm}{1mm} \vspace*{3ex}}

%----- Commandes supplementaires ------



\begin{document}

%%%%%%%%%%%%%%%%%% EXERCICES %%%%%%%%%%%%%%%%%%
\fiche{f00001, bodin, 2007/09/01} 

\titre{Nombres complexes} 

\section{Forme cartésienne, forme polaire}
\exercice{1, bodin, 1998/09/01}
\video{g5vcj-DIa2M}
\enonce{000001}{}

 Mettre sous la forme $a+ib$ ($a,b \in \Rr$) les nombres :
$$ \frac{3+6i}{3-4i}  \quad ; \quad \left(\frac{1+i}{2-i}\right)^2 + \frac{3+6i}{3-4i} \quad;  \quad
\frac{2+5i}{1-i} + \frac{2-5i}{1+i}. $$

\finenonce{000001}



\finexercice

\exercice{3, cousquer, 2003/10/01}
\video{2kur52c78pA}
\enonce{000003}{}
 Écrire sous la forme $a+ib$ les nombres complexes suivants :
\begin{enumerate}
\item Nombre de module $2$ et d'argument $\pi /3$.
\item Nombre de module $3$ et d'argument $-\pi /8$.
\end{enumerate}

\finenonce{000003}




\finexercice

\exercice{11, bodin, 1998/09/01}
\video{XzALEyZLQYc}
\enonce{000011}{}
Calculer le module et l'argument de $u =
\frac{\sqrt{6}-i\sqrt{2}}{2}$ et $v = 1 - i$. En d\'eduire le
module et l'argument de $w = \frac{u}{v}$.

\finenonce{000011}



\finexercice
\exercice{13, bodin, 1998/09/01}
\video{-Kjd4SKjwEo}
\enonce{000013}{}
 D\'eterminer le module et l'argument des nombres
complexes :
$$e^{e^{i\alpha}} \quad \text{ et } \quad e^{i\theta}+e^{2i\theta}.$$

\finenonce{000013}




\finexercice


\section{Racines carrées, équation du second degré}
\exercice{27, bodin, 1998/09/01}
\video{BPzsFnvZypQ}
\enonce{000027}{}
 Calculer les racines carr\'ees de $1,\ i,\ 3+4i,\
8-6i,$ et $7+24i$.

\finenonce{000027}




\finexercice

\exercice{29, bodin, 1998/09/01}
\video{Q5kK2BZBffI}
\enonce{000029}{}
\begin{enumerate}
    \item Calculer les racines carr\'ees de $\frac{1+i}{\sqrt{2}}$. En d\'eduire
les valeurs de $\cos(\pi/8)$ et $\sin(\pi/8)$.
    \item Calculer les valeurs de $\cos(\pi/12)$ et $\sin(\pi/12)$.
\end{enumerate}

\finenonce{000029}



\finexercice\exercice{31, bodin, 1998/09/01}
\video{TC-3du7GQAQ}
\enonce{000031}{}
 R\'esoudre dans $\Cc$ les \'equations suivantes :
$$ z^2+z+1 = 0 \quad ; \quad z^2-(1+2i)z+i-1 = 0 \quad ; \quad z^2-\sqrt{3}z-i = 0 \quad ;$$
$$z^2-(5-14i)z-2(5i+12)=0 \  ; \  z^2-(3+4i)z-1+5i =0 \  ; \  4z^2-2z+1=0 \  ;$$
$$z^4+10z^2 +169=0 \quad ; \quad z^4+2z^2 +4=0.$$

\finenonce{000031}



\finexercice
\section{Racine $n$-ième}
\exercice{47, bodin, 1998/09/01}
\video{MJT-lI-lt-M}
\enonce{000047}{}

 Calculer la somme $S_n = 1+z+z^2+\cdots+z^n$.
\finenonce{000047} 


\finexercice
\exercice{48, bodin, 1998/09/01}
\video{b63m6T3TlsY}
\enonce{000048}{}

\begin{enumerate}
    \item R\'esoudre $z^3 = 1$ et montrer que les racines s'\'ecrivent $1$, $j$, $j^2$.
Calculer $1+j+j^2$ et en d\'eduire les racines de $1+z+z^2 =0$.
    \item R\'esoudre $z^n = 1$ et montrer que les racines s'\'ecrivent
 $1,\epsilon,\ldots,\epsilon^{n-1}$. En d\'eduire les racines de $1+z+z^2+\cdots+z^{n-1} =0$.
Calculer, pour $p \in \Nn$, $1+\epsilon^p+\epsilon^{2p}+\cdots+\epsilon^{(n-1)p}$.
\end{enumerate}
\finenonce{000048} 


\finexercice
\exercice{43, cousquer, 2003/10/01}
\video{6B7bztuiRh8}
\enonce{000043}{}
 Trouver les racines cubiques de 
$2-2i$ et de $11+2i$.

\finenonce{000043} 


\finexercice
\exercice{56, bodin, 1998/09/01}
\video{sVzO_9JjHTM}
\enonce{000056}{}
\begin{enumerate}
        \item Soient $z_1$, $z_2$, $z_3$ trois nombres complexes distincts
        ayant le m\^eme cube.

        Exprimer $z_2$ et $z_3$ en fonction de $z_1$.

        \item Donner, sous forme polaire, les solutions dans $\Cc$ de :
        $$ z^6+(7-i)z^3 -8 -8i = 0.$$
        (Indication : poser $Z=z^3$ ; calculer $(9+i)^2$)
\end{enumerate}

\finenonce{000056} 


\finexercice

\section{Géométrie}
\exercice{60, bodin, 1998/09/01}
\video{iiReM2y_jKk}
\enonce{000060}{}
D\'eterminer l'ensemble des nombres complexes $z$
tels que :
\begin{enumerate}
    \item $\displaystyle{\left|\frac{z-3}{z-5}\right|=1},$
    \item $\displaystyle{\left|\frac{z-3}{z-5}\right|= \frac{\sqrt{2}}{2}}.$
\end{enumerate}

\finenonce{000060} 


\finexercice
\exercice{69, bodin, 1998/09/01}
\video{MNGaSR9bknI}
\enonce{000069}{}

Montrer que pour $u,v \in \Cc$, on a $|u+v|^2+|u-v|^2=2(|u|^2+|v|^2).$
Donner une interprétation géométrique.
\finenonce{000069} 


\finexercice
\exercice{77, bodin, 1998/09/01}
\video{wHlb0IsMB7Q}
\enonce{000077}{}

Soit $(A_{0},A_{1},A_{2},A_{3},A_{4})$ un pentagone r\'egulier. On note $O$ son centre et on
choisit un rep\`ere orthonorm\'e $(O,\overrightarrow{u},\overrightarrow{v})$ avec
$\overrightarrow{u}=\overrightarrow{OA_{0}}$, qui nous permet d'identifier le plan avec
l'ensemble des nombres complexes $\Cc$.
%dessin
$$
%\includegraphics{../images/img000077-1}
$$

\begin{enumerate}
\item
Donner les affixes $\omega_{0},\ldots,\omega_{4}$ des points $A_{0},\ldots,A_{4}$. Montrer
que  $\omega_{k}={\omega_{1}}^k$ pour $ k\in\{0,1,2,3,4\}$. Montrer que
$1+\omega_{1}+\omega_{1}^2+\omega_{1}^3+\omega_{1}^4=0$.

\item
En d\'eduire que $\cos(\frac{2\pi}{5})$ est l'une des solutions de l'\'equation $4z^2+2z-1=0$.
En d\'eduire la valeur de $\cos(\frac{2\pi}{5})$.

\item
On consid\`ere le point $B$ d'affixe $-1$. Calculer la longueur $BA_{2}$ en fonction de
$\sin\frac{\pi}{10}$ puis de $\sqrt{5}$ (on remarquera que
$\sin\frac{\pi}{10}=\cos\frac{2\pi}{5}$).

\item
On consid\`ere le point $I$ d'affixe $\frac{i}{2}$, le cercle $\mathcal{C}$ de centre $I$ de
rayon $\frac{1}{2}$ et enfin le point $J$ d'intersection de $\mathcal{C}$ avec la demi-droite
$[BI)$. Calculer la longueur $BI$ puis  la longueur $BJ$.

\item
\textbf{Application:} Dessiner un pentagone r\'egulier \`a la r\`egle et au compas. Expliquer.
\end{enumerate}



\finenonce{000077} 


\finexercice

\section{Trigonométrie}
\exercice{20, bodin, 1998/09/01}
\video{8nXKgqMsucU}
\enonce{000020}{}
 Soit $z$ un nombre complexe de module $\rho$,
d'argument $\theta$, et soit $\overline{z}$ son conjugu\'e.
Calculer
$(z+\overline{z})(z^2+\overline{z}^2)\ldots(z^n+\overline{z}^n)$
en fonction de $\rho$ et $\theta$.

\finenonce{000020}


\finexercice\exercice{80, cousquer, 2003/10/01}
\video{yE-CGGcOrYA}
\enonce{000080}{}

 En utilisant les nombres complexes, calculer $\cos 5\theta$ et
$\sin5\theta$ en fonction de $\cos\theta$ et $\sin\theta$.
\finenonce{000080} 


\finexercice

\section{Divers}
\exercice{96, bodin, 1998/09/01}
\video{JdTvqdw6lRE}
\enonce{000096}{}
Soit $\Zz[i] = \{ a+ib \  ; \ a,b \in \Zz \}$.
\begin{enumerate}
        \item Montrer que si $\alpha$ et $\beta$ sont dans $\Zz[i]$ alors
        $\alpha + \beta$ et $\alpha\beta$ le sont aussi.

        \item Trouver les \'elements inversibles de $\Zz[i]$, c'est-\`a-dire les
        \'el\'ements $\alpha \in \Zz[i]$ tels qu'il existe $\beta \in \Zz[i]$ avec
        $\alpha\beta = 1$.

        \item V\'erifier que quel que soit $\omega \in \Cc$ il existe $\alpha \in \Zz[i]$
        tel que $|\omega - \alpha| < 1$.

        \item Montrer qu'il existe sur $\Zz[i]$ une division euclidienne,
 c'est-\`a-dire que,
        quels que soient $\alpha$ et $\beta$ dans $\Zz[i]$ il existe $q$ et $r$ dans
        $\Zz[i]$ v\'erifiant :
        $$ \alpha = \beta q + r \qquad \text{avec} \qquad |r| < |\beta|.$$
        (Indication : on pourra consid\'erer le complexe $\frac{\alpha}{\beta}$)

\end{enumerate}

\finenonce{000096} 


\finexercice

\finfiche

 \finenonces 



 \finindications 

\indication{000001}
Pour se ``débarrasser'' d'un dénominateur écrivez $\frac{z_1}{z_2} = \frac{z_1}{z_2}\cdot \frac{\bar z_2}{\bar z_2} =  \frac{z_1 \bar z_2}{ |z_2|^2}$.
\finindication
\indication{000003}
Il faut bien connaître ses formules trigonométriques.
En particulier si l'on connait $\cos(2\theta)$ ou $\sin(2\theta)$ on sait
calculer $\cos \theta$ et $\sin \theta$.
\finindication
\indication{000011}
Passez à la forme trigonométrique.
Souvenez-vous des formules sur les produits de puissances :
$$e^{ia}e^{ib} = e^{i(a+b)}\text{ et  } e^{ia} / e^{ib} = e^{i(a-b)}.$$
\finindication
\indication{000013}
Pour calculer un somme du type
$e^{iu}+e^{iv}$ il est souvent utile de factoriser par
$e^{i\frac{u+v}{2}}$.
\finindication
\indication{000027}
Pour $z= a+ib$ on cherche $\omega = \alpha + i \beta$ tel que  $(\alpha +i \beta)^2 = a+ib$.
Développez et indentifiez. Utilisez aussi que $|\omega|^2 = |z|$.
\finindication
\indication{000029}
Il s'agit de calculer les racines carrées de $\frac{1+i}{\sqrt{2}} =  e^{i\frac{\pi}{4}}$ de deux
fa\c{c}ons différentes.
\finindication
\indication{000031}
Pour les équation du type $az^4+bz^2+c=0$, poser $Z=z^2$.
\finindication
\indication{000047}
Calculer $(1-z)S_n$.
\finindication
\noindication
\noindication
\noindication
\indication{000060}
Le premier ensemble est une droite le second est un cercle.
\finindication
\indication{000069}
Pour l'interprétation géométrique cherchez le parallélogramme.
\finindication
\noindication
\indication{000020}
Utiliser la formule d'Euler pour faire appara\^{\i}tre des cosinus.
\finindication
\indication{000080}
Appliquer deux fois la formule de Moivre en remarquant
$e^{i5\theta}=(e^{i\theta})^5$.
\finindication
\noindication


\newpage

\correction{000001}
Remarquons d'abord que pour $z \in \Cc$, $z \overline{z} = |z|^2$
est un nombre r\'eel, ce qui fait qu'en multipliant le dénominateur par son conjugué nous obtenons un nombre réel.
 
$$ \frac{3+6i}{3-4i}
= \frac{(3+6i)(3+4i)}{(3-4i)(3+4i)} = \frac{9-24+12i+18i}{9+16} =
\frac{-15+30i}{25} = -\frac{3}{5}+\frac{6}{5}i.$$

\bigskip

Calculons
$$\frac{1+i}{2-i}
= \frac{(1+i)(2+i)}{5} = \frac{1+3i}{5},
$$
et
$$ \left( \frac{1+i}{2-i}  \right)^2
= \left( \frac{1+3i}{5}  \right)^2 = \frac{-8+6i}{25}=
-\frac{8}{25}+\frac{6}{25}i.
$$

Donc
$$ \left( \frac{1+i}{2-i}  \right)^2 + \frac{3+6i}{3-4i}
= -\frac{8}{25}+\frac{6}{25}i  -\frac{3}{5}+\frac{6}{5}i
=-\frac{23}{25}+\frac{36}{25}i.$$


\bigskip
Soit $z = \frac{2+5i}{1-i}$. Calculons $z + \overline{z}$, nous
savons d\'ej\`a que c'est un nombre r\'eel, plus pr\'ecis\'ement :
$z = -\frac{3}{2}+\frac{7}{2}i$ et donc $ z + \overline{z} = -3$.
\fincorrection
\correction{000003}
\begin{enumerate}
\item $z_1 = 2 e^{i\frac \pi 3} = 2(\cos \frac \pi 3 + i \sin \frac \pi 3) = 2 (\frac 12+ i\frac{\sqrt3}{2}) = 1+i\sqrt 3$.
\item $z_2 = 3e^{-i\frac \pi 8} = 3\cos {\pi \over 8}-3i\sin{\pi \over8}={3\sqrt{2+\sqrt2}\over 2}-{3i\sqrt{2-\sqrt2}\over 2}$.


Il nous reste à expliquer comment nous avons calculé $\cos \frac \pi 8$ et $\sin \frac\pi 8$: 
posons $\theta=\frac{\pi}{8}$,
alors $2\theta = \frac \pi 4$ et donc $\cos(2\theta)= \frac {\sqrt 2}{2} = \sin(2\theta)$.
Mais $\cos(2\theta) = 2\cos^2 \theta - 1$. Donc $\cos^2\theta = \frac{\cos(2\theta) + 1}{2} = \frac 14 (2 + \sqrt 2)$.
Et ensuite  $\sin^2 \theta = 1- \cos^2 \theta = \frac 14 (2 - \sqrt 2)$.
Comme $0 \le \theta = \frac \pi 8 \le \frac \pi 2$, $\cos \theta$ et $\sin \theta$ sont des nombres positifs. Donc
$$\cos \frac \pi 8 = \frac 12 \sqrt{2 + \sqrt 2} \quad, \quad \sin \frac \pi 8 = \frac 12 \sqrt{2 - \sqrt 2}.$$ 

\end{enumerate}
\fincorrection
\correction{000011}
Nous avons
$$ u = \frac{\sqrt{6}-\sqrt{2}i}{2}
= \sqrt{2}\left( \frac{\sqrt{3}}{2}-\frac{i}{2} \right) =
\sqrt{2}\left( \cos\frac{\pi}{6} -i\sin\frac{\pi}{6} \right)
=\sqrt{2} e^{-i\frac{\pi}{6}}.$$ puis
$$v = 1-i = \sqrt{2}e^{-i\frac{\pi}{4}}.$$
Il ne reste plus qu'\`a calculer le quotient :
$$ \frac{u}{v} = \frac{\sqrt{2}e^{-i\frac{\pi}{6}}}{\sqrt{2}e^{-i\frac{\pi}{4}}}
= e^{-i\frac{\pi}{6}+i\frac{\pi}{4}} = e^{i\frac{\pi}{12}}.$$
\fincorrection
\correction{000013}
D'apr\`es la formule de Moivre pour $e^{i\alpha}$ nous avons :
$$e^{e^{i\alpha}} = e^{\cos \alpha + i\sin \alpha}
= e^{\cos \alpha}e^{i\sin \alpha}.$$ Or $e^{\cos \alpha} > 0$ donc
l'\'ecriture pr\'ec\'edente est bien de la forme
``module-argument''.

\bigskip

De fa\c{c}on g\'en\'erale pour calculer un somme du type
$e^{iu}+e^{iv}$ il est souvent utile de factoriser par
$e^{i\frac{u+v}{2}}$. En effet
\begin{align*}
e^{iu}+e^{iv} &= e^{i\frac{u+v}{2}}\left( e^{i
\frac{u-v}{2}}+ e^{-i \frac{u-v}{2}}\right) \\
&= e^{i\frac{u+v}{2}} 2 \cos  \frac{u-v}{2} \\
&=  2 \cos  \frac{u-v}{2} e^{i\frac{u+v}{2}}.\\
\end{align*}
Ce qui est proche de l'\'ecriture en coordon\'ees polaires.

Pour le cas qui nous concerne :
$$z = e^{i\theta} + e^{2i\theta}
= e^{\frac{3i\theta}{2}} \left[ e^{-\frac{i\theta}{2}} +
e^{\frac{i\theta}{2}} \right] = 2\cos \frac{\theta}{2}
e^{\frac{3i\theta}{2}}.$$ Attention le module dans une
d\'ecomposion en forme polaire doit \^etre positif ! Donc si $\cos
\frac{\theta}{2}  \geq 0$  alors $2\cos\frac{\theta}{2}$ est le module de $z$
et $3\theta/2$ est son argument ; par contre si $\cos \frac{\theta}{2}  <
0$ le module est $2|\cos\frac{\theta}{2}|$ et l'argument $3\theta/2+\pi$ (le
$+\pi$ compense le changement de signe car $e^{i\pi} = -1$).
\fincorrection
\correction{000027}
\textbf{Racines carr\'ees.} Soit $z= a+ib$ un nombre complexe avec
$a,b \in \Rr$ ; nous cherchons les complexes $\omega \in \Cc$ tels
que $\omega^2 = z$. \'Ecrivons $\omega = \alpha + i \beta$. Nous
raisonnons par \'equivalence :
\begin{align*}
\omega^2 = z    & \Leftrightarrow (\alpha +i \beta)^2 = a+ib \\
        & \Leftrightarrow \alpha^2-\beta^2+2i\alpha\beta = a +i b \\
\intertext{Soit en identifiant les parties r\'eelles entre elles ainsi que les parties imaginaires :}\\
        & \Leftrightarrow
              \begin{cases}
                 \alpha^2-\beta^2 = a \\
                 2\alpha\beta = b
              \end{cases} \\
\intertext{Sans changer l'\'equivalence nous rajoutons la
condition $|\omega|^2 = |z|$.} & \Leftrightarrow
  \begin{cases}
     \alpha^2 + \beta^2 = \sqrt{a^2+b^2} \\
     \alpha^2-\beta^2 = a \\
     2\alpha\beta = b
  \end{cases} \\
\intertext{Par somme et différence des deux premières lignes :}
& \Leftrightarrow
  \begin{cases}
     \alpha^2  =\frac{a+ \sqrt{a^2+b^2}}{2} \\
     \beta^2 =  \frac{-a+ \sqrt{a^2+b^2}}{2} \\
     2\alpha\beta = b
  \end{cases} \\
& \Leftrightarrow
  \begin{cases}
     \alpha  =\pm \sqrt{\frac{a+ \sqrt{a^2+b^2}}{2}} \\
     \beta = \pm \sqrt{ \frac{-a+ \sqrt{a^2+b^2}}{2}} \\
     \alpha\beta \text{ est du m\^eme signe que } b
  \end{cases} \\
\end{align*}
Cela donne deux couples $(\alpha,\beta)$ de solutions et donc deux
racines carr\'ees (opposées) $\omega = \alpha + i\beta$ de $z$.

En pratique on r\'ep\`ete facilement ce raisonnement, par exemple
pour  $z = 8-6i$,
\begin{align*}
\omega^2 = z
&\Leftrightarrow (\alpha +i \beta)^2 = 8-6i\\
&\Leftrightarrow \alpha^2-\beta^2+2i\alpha\beta = 8-6i\\
&\Leftrightarrow
  \begin{cases}
     \alpha^2-\beta^2 = 8\\
     2\alpha\beta = -6
  \end{cases} \\
&\Leftrightarrow
  \begin{cases}
     \alpha^2 + \beta^2 = \sqrt{8^2+(-6)^2} = 10 \text{ le module de $z$} \\
     \alpha^2-\beta^2 = 8 \\
     2\alpha\beta = -6
  \end{cases}\\
&\Leftrightarrow
  \begin{cases}
     2\alpha^2  =18 \\
     \beta^2 = 1 \\
     2\alpha\beta = -6
  \end{cases}\\
&\Leftrightarrow
  \begin{cases}
     \alpha  =\pm \sqrt{9} = \pm 3 \\
     \beta = \pm 1 \\
     \alpha \text{ et } \beta \text{ de signes oppos\'es}
  \end{cases} \\
&\Leftrightarrow
  \begin{cases}
     \ \ & \alpha  = 3 \text{ et } \beta = - 1\\
     \text{ou}&\\
    \ \ &\alpha  = -3 \text{ et } \beta = +1 \\
\end{cases} \\
\end{align*}

Les racines de $ z = 8-6i$ sont donc $\omega_1 = 3-i$ et $\omega_2 = -\omega_1 =
-3+i$.

Pour les autres : 
\begin{itemize}
 \item Les racines carrées de $1$ sont : $+1$ et $-1$.
 \item  Les racines carrées de $i$ sont : $\frac{\sqrt 2}{2}(1+i)$ et $-\frac{\sqrt 2}{2}(1+i)$.
 \item  Les racines carrées de $3+4i$ sont : $2+i$ et $-2-i$. 
 \item  Les racines carrées de $7+24i$ sont : $4+3i$ et $-4-3i$. 
\end{itemize}

\fincorrection
\correction{000029}
Par la m\'ethode usuelle nous calculons les racines carr\'ees
$\omega, -\omega$ de $z = \frac{1+i}{\sqrt{2}}$, nous obtenons
$$\omega = \sqrt{\frac{\sqrt{2}+1}{2\sqrt{2}}}+i\sqrt{\frac{\sqrt{2}-1}{2\sqrt{2}}},$$
qui peut aussi s'écrire :
$$\omega = \frac 12 \sqrt{2+\sqrt 2} +i \frac 12 \sqrt{2-\sqrt 2}.$$

Mais nous remarquons que $z$ s'\'ecrit \'egalement
$$z = e^{i\frac{\pi}{4}}$$
et  $e^{i\frac{\pi}{8}}$ v\'erifie
$$\left (e^{i\frac{\pi}{8}}\right)^2 = e^{\frac{2i\pi}{8}}
= e^{i\frac{\pi}{4}}.$$
 Cela signifie que $e^{i\frac{\pi}{8}}$ est
une racine carr\'ee de $z$, donc $ e^{i\frac{\pi}{8}} = \cos
\frac{\pi}{8}+i\sin\frac{\pi}{8}$
  est \'egal \`a  $\omega$ ou $-\omega$. Comme $ \cos \frac{\pi}{8} > 0$
alors $ e^{i\frac{\pi}{8}} =  \omega$ et donc par identification
des parties r\'eelles et imaginaires :
$$\cos \frac{\pi}{8} = \frac 12 \sqrt{2+\sqrt 2}
\quad \text{ et }\quad \sin \frac{\pi}{8} =
\frac 12 \sqrt{2-\sqrt 2}.$$
\fincorrection
\correction{000031}
\textbf{\'Equations du second degr\'e.} La m\'ethode g\'enerale
pour r\'esoudre les \'equations du second degr\'e $az^2+bz+c= 0$
(avec $a,b,c \in \Cc$ et $a\not=0$) est la suivante : soit $\Delta
= b^2-4ac$ le discriminant complexe et $\delta$ une racine
carr\'ee de $\Delta$ ($\delta^2 = \Delta$) alors les solutions
sont :
$$z_1 = \frac{-b+\delta}{2a} \quad \text{ et } \quad z_2 = \frac{-b-\delta}{2a}.$$
Dans le cas o\`u les coefficients sont r\'eels, on retrouve la
m\'ethode bien connue. Le seul travail dans le cas complexe est de
calculer une racine $\delta$ de $\Delta$.

Exemple : pour $z^2-\sqrt{3}z-i =0$, $\Delta = 3+4i$, dont une
racine carr\'ee est $\delta = 2+i$,  les solutions sont donc :
$$z_1 = \frac{\sqrt{3}+2+i}{2}\quad  \text{ et }\quad  z_2 = \frac{\sqrt{3}-2-i}{2}.$$

Les solutions des autres équations sont :
\begin{itemize}
 \item L'équation $z^2+z+1=0$ a pour solutions : $\frac12 (-1+i\sqrt{3})$, $\frac12 (-1-i\sqrt{3})$.
 \item L'équation $z^2-(1+2i)z+i-1=0$ a pour solutions : $1+i$, $i$.
 \item L'équation $z^2-\sqrt{3}z-i=0$ a pour solutions : $\frac12(2-\sqrt{3}+i)$,  $\frac12(-2-\sqrt{3}-i)$
 \item L'équation $z^2-(5-14i)z-2(5i+12)=0$ a pour solutions : $5-12i$, $-2i$.
 \item L'équation $z^2-(3+4i)z-1+5i =0$ a pour solutions : $2+3i$, $1+i$.
 \item L'équation $4z^2-2z+1=0$ a pour solutions : $\frac 14(1+i\sqrt{3})$, $\frac 14(1-i\sqrt{3})$.
 \item L'équation $z^4+10z^2 +169=0$ a pour solutions : $2+3i$, $-2-3i$, $2-3i$, $-2+3i$.
 \item L'équation $z^4+2z^2 +4=0$ a pour solutions : $\frac{\sqrt{2}}{2}(1+i\sqrt{3})$, $\frac{\sqrt{2}}{2}(1-i\sqrt{3})$,  $\frac{\sqrt{2}}{2}(-1+i\sqrt{3})$, $\frac{\sqrt{2}}{2}(-1-i\sqrt{3})$.
\end{itemize}

\fincorrection
\correction{000047}
$$S_n = 1+z+z^2+\cdots+z^n = \sum_{k=0}^{n}z^k.$$
Nous devons retrouver le r\'esultat sur la somme $S_n =
\frac{1-z^{n+1}}{1-z}$d'une suite g\'eom\'etrique dans le cas o\`u
$z\not=1$ est un r\'eel. Soit maintenant $z \not= 1$ un nombre
complexe. Calculons $S_n(1-z)$.
\begin{align*}
S_n(1-z) & =(1+z+z^2+\cdots+z^n)(1-z) \text{ d\'eveloppons }\\
         &= 1+z+z^2+\cdots+z^n - z-z^2-\cdots-z^{n+1} \text{ les termes interm\'ediaires s'annulent }\\
         &= 1-z^{n+1}.
\end{align*}
Donc $$S_n = \frac{1-z^{n+1}}{1-z}, \text{ pour } z\not= 1.$$
\fincorrection
\correction{000048}
\textbf{Calcul de racine $n$-i\`eme.} Soit $z\in\Cc$ tel que
$z^n=1$, d\'ej\`a $|z|^n=1$ et donc $|z|=1$. \'Ecrivons $z =
e^{i\theta}$. L'\'equation devient
$$e^{in\theta} = e^{0} =1
\Leftrightarrow n\theta = 0 + 2k\pi, \ k\in \Zz \Leftrightarrow \theta =
\frac{2k\pi}{n}, \ k\in \Zz.$$ Les solution sont donc
$$\mathcal{S} = \left\lbrace e^{\frac{2ik\pi}{n}}, \ k\in \Zz\right\rbrace.$$
Comme le polyn\^ome $z^n-1$ est de degr\'e $n$ il a au plus $n$
racines. Nous choisissons pour repr\'esentants :
$$\mathcal{S} = \left\lbrace e^{\frac{2ik\pi}{n}}, \ k= 0,\ldots,n-1\right\rbrace.$$
De plus si $\epsilon = e^{\frac{2i\pi}{n}}$ alors $\mathcal{S} =
\left\lbrace \epsilon^k, \ k= 0,\ldots,n-1\right\rbrace.$ Ces
racines sont les sommets d'un polygone r\'egulier \`a $n$
c\^ot\'es inscrit dans le cercle unit\'e.



Soit $P(z) = \sum_{k=0}^{n-1}z^k=\frac{1-z^{n}}{1-z}$ pour
$z\not=1$. Donc quelque soit $z\in\mathcal{S}\setminus\{ 1 \}$
$P(z) = 0$, nous avons ainsi trouver $n-1$ racines pour $P$ de
degr\'e $n-1$, donc l'ensemble des racines de $P$ est exactement
$\mathcal{S}\setminus\{ 1 \}$.




Pour conclure soit  $Q_p(z) = \sum_{k=0}^{n-1}\epsilon^{kp}$.\\
Si $p = 0 +\ell n$, $\ell \in \Zz$ alors $\epsilon^{kp}=\epsilon^{k\ell n}=(\epsilon^n)^{k \ell}
=1^{k \ell}=1$. Donc $Q_p(z)  = \sum_{k=0}^{n-1} 1 = n$. \\
Sinon $Q_p(z)$ est la somme d'une suite g\'eom\'etrique de raison
$\epsilon^p$ :
$$ Q_p(z) = \frac{1-\left( \epsilon^p \right)^n}{1-\epsilon^p}
= \frac{1-\left( \epsilon^n \right)^p}{1-\epsilon^p} =
\frac{1-1}{1-\epsilon^p}=0.$$
\fincorrection
\correction{000043}
\begin{enumerate}
\item Les trois racines cubiques ont m\^eme module $\sqrt2$, et leurs arguments
sont $-\pi /12$, $7\pi /12$ et $5\pi/4$. Des valeurs approch\'ees sont
$1{,}36603-0{,}36603 i$, $-0{,}36603+1{,}36603 i$ et $-1-i$.
\item $-1-2i$, $(-1-2i)j$ et $(-1-2i)j^2$ o\`u $j={-1+i\sqrt3\over2}$ (racine
cubique de 1).
\end{enumerate}
\fincorrection
\correction{000056}
Soient $z_{1},z_{2},z_{3}$ trois nombres complexes
\emph{distincts} ayant le m\^eme cube.
\begin{enumerate}
\item
$z_{1}\neq0$ car sinon on aurait $z_{1}=z_{2}=z_{3}=0$. Ainsi
$(\frac{z_{2}}{z_{1}})^3=(\frac{z_{3}}{z_{1}})^3=1$. Comme les
trois nombres $1,(\frac{z_{2}}{z_{1}})$ et $(\frac{z_{3}}{z_{1}})$
sont distincts on en d\'eduit que ce sont les trois racines
cubiques de 1. Ces racines sont $1, j=e^{\frac{2i\pi}{3}}$ et $
j^2=e^{-\frac{2i\pi}{3}}$. A une permutation pr\`es des indices 2
et 3 on a donc :
$$
  z_{2}=jz_{1} \qquad\text{ et }\qquad z_{3}=j^2z_{1}.
$$
\item Soit $z\in\C$. On a les \'equivalences suivantes :
\begin{align*}
z^6+(7-i)z^3-8-8i=0 &\Leftrightarrow z^3\text{ est solution de }Z^2+(7-i)Z-8-8i=0\\
\intertext{ Etudions l'\'equation $Z^2+(7-i)Z-8-8i=0$.
$\Delta=(7-i)^2+4(8+8i)=80+18i=(9+i)^2$. Les solutions sont donc
$-8$ et $1+i$. Nous pouvons reprendre notre suite d'\'equivalences
: }
z^6+(7-i)z^3-8-8i=0 &\Leftrightarrow z^3\in\{-8,1+i\} \\
  &\Leftrightarrow z^3=(-2)^3 \quad\text{ ou }\quad z^3=(\sqrt[6]{2}e^{i\frac{\pi}{12}})^3\\
  &\Leftrightarrow z\in\{-2,
              -2e^{\frac{2i\pi}{3}},
              -2e^{-\frac{2i\pi}{3}}\}
         \text{ ou }
        z\in\{\sqrt[6]{2}e^{i\frac{\pi}{12}},
              \sqrt[6]{2}e^{i\frac{9\pi}{12}},
              \sqrt[6]{2}e^{i\frac{17\pi}{12}}\}\\
  &\Leftrightarrow z\in\{-2,
               2e^{\frac{i\pi}{3}},
               2e^{-\frac{i\pi}{3}},
              \sqrt[6]{2}e^{i\frac{\pi}{12}},
              \sqrt[6]{2}e^{i\frac{3\pi}{4}},
              \sqrt[6]{2}e^{i\frac{17\pi}{12}}\}.
\end{align*}
L'ensemble des solutions est donc :
           $$\{-2,
               2e^{\frac{i\pi}{3}},
               2e^{-\frac{i\pi}{3}},
              \sqrt[6]{2}e^{i\frac{\pi}{12}},
              \sqrt[6]{2}e^{i\frac{3\pi}{4}},
              \sqrt[6]{2}e^{i\frac{17\pi}{12}}\}.$$
\end{enumerate}
\fincorrection
\correction{000060}
Nous identifions $\Cc$ au plan affine et $z=x+iy$ \`a  $(x,y) \in
\Rr\times \Rr$.

Remarquons que pour les deux ensembles $z=5$ n'est pas solution,
donc
$$\left| \frac{z-3}{z-5} \right| = 1
\Leftrightarrow |z-3| = |z-5|.$$ Ce qui signifie pr\'eci\'sement que les
points d'affixe $z$ sont situ\'es \`a \'egale distance des points
$A,B$ d'affixes respectives $3 = (3,0)$ et $5=(5,0)$. L'ensemble
solution est la m\'ediatrice du segment $[A,B]$.

\bigskip

Ensuite pour
\begin{align*}
\left| \frac{z-3}{z-5} \right| = \frac{\sqrt{2}}{2}
&\Leftrightarrow |z-3|^2 = \frac{1}{2}|z-5|^2 \\
&\Leftrightarrow (z-3)\overline{(z-3)} = \frac{1}{2}(z-5)\overline{(z-5)}\\
&\Leftrightarrow z\overline{z}-(z+\overline{z})=7\\
&\Leftrightarrow |z-1|^2=8\\
&\Leftrightarrow |z-1|=2\sqrt{2}\\
\end{align*}
L'ensemble solution est donc le cercle de centre le point d'affixe
$1 = (1,0)$ et de rayon $2\sqrt{2}$.
\fincorrection
\correction{000069}
$$|u+v|^2+|u-v|^2=(u+v)(\bar u +\bar v) + (u-v)(\bar u -\bar v)
= 2u\bar u+2v\bar v = 2|u|^2+2|v|^2.$$
G\'eom\'etriquement il s'agit de l'identit\'e du parall\'elogramme.
Les points d'affixes $0,u,v,u+v$ forment un parall\'elogramme.
$|u|$ et $|v|$ sont les longueurs des cot\'es, et
$|u+v|, |u-v|$ sont les longueurs des diagonales.
Il n'est pas \'evident de montrer ceci sans les nombres complexes !!
\fincorrection
\correction{000077}
\begin{enumerate}
\item
Comme $(A_{0},\ldots,A_{4})$ est un pentagone r\'egulier, on a
$OA_{0}=OA_{1}=OA_{2}=OA_{3}=OA_{4}=1$ et $
  (\overrightarrow{OA_{0}},\overrightarrow{OA_{1}})=\frac{2\pi}{5}[2\pi],
  (\overrightarrow{OA_{0}},\overrightarrow{OA_{2}})=\frac{4\pi}{5}[2\pi],
  (\overrightarrow{OA_{0}},\overrightarrow{OA_{3}})=-\frac{4\pi}{5}[2\pi],
  (\overrightarrow{OA_{0}},\overrightarrow{OA_{4}})=-\frac{2\pi}{5}[2\pi],
 $.
On en d\'eduit:
 $
  \omega_{0}=1,
  \omega_{1}=e^{\frac{2i\pi}{5}},
  \omega_{2}=e^{\frac{4i\pi}{5}},
  \omega_{3}=e^{-\frac{4i\pi}{5}}=e^{\frac{6i\pi}{5}},
  \omega_{4}=e^{-\frac{2i\pi}{5}}=e^{\frac{8i\pi}{5}},
 $.
On a bien $\omega_{i}=\omega_{1}^i$. Enfin, comme
$\omega_{1}\neq0$, $1+\omega_{1}+\ldots+\omega_{1}^4=
\frac{1-\omega_{1}^5}{1-\omega_{1}}=\frac{1-1}{1-\omega_{1}}=0$.

\item $\mathop{\mathrm{Re}}\nolimits(1+\omega_{1}+\ldots+\omega_{1}^4)=
1+2\cos(\frac{2\pi}{5})+2\cos(\frac{4\pi}{5})$. Comme
$\cos(\frac{4\pi}{5})=2\cos^2(\frac{2\pi}{5})-1$ on en d\'eduit:
$4\cos^2(\frac{2\pi}{5})+2\cos(\frac{2\pi}{5})-1=0$.
$\cos(\frac{2\pi}{5})$ est donc bien une solution de l'\'equation
$4z^2+2z-1=0$. Etudions cette \'equation: $\Delta=20=2^2.5$. Les
solutions sont donc $\frac{-1-\sqrt{5}}{4}$ et
$\frac{-1+\sqrt{5}}{4}$. Comme $\cos(\frac{2\pi}{5})>0$, on en
d\'eduit que $\cos(\frac{2\pi}{5})=\frac{\sqrt{5}-1}{4}$.

\item
 $
  BA_{2}^2=|\omega_{2}+1|^2
          =|\cos(\frac{4\pi}{5})+i\sin(\frac{4\pi}{5})+1|^2
          =1+2\cos(\frac{4\pi}{5})+\cos^2(\frac{4\pi}{5})+\sin^2(\frac{4\pi}{5})
          =4\cos^2(\frac{2\pi}{5})
  $. Donc $BA_{2}=\frac{\sqrt{5}-1}{2}$.

\item
$BI=|i/2+1|=\frac{\sqrt{5}}{2}$. $BJ=BI-1/2=\frac{\sqrt{5}-1}{2}$.

\item
Pour tracer un pentagone r\'egulier, on commence par tracer un
cercle $C_{1}$ et deux diam\`etres orthogonaux, qui jouent le r\^ole
du cercle passant par les sommets et des axes de coordonn\'ees. On
trace ensuite le milieu d'un des rayons: on obtient le point I de
la question 4. On trace le  cercle de centre $I$ passant par le
centre de $C_{1}$: c'est le cercle $\mathcal{C}$. On trace le
segment $BI$ pour obtenir son point $J$ d'intersection avec
$\mathcal{C}$. On trace enfin le cercle de centre $B$ passant par
$J$: il coupe $C_{1}$ en $A_{2}$ et $A_{3}$, deux sommets du
pentagone. Il suffit pour obtenir tous les sommets de reporter la
distance $A_{2}A_{3}$ sur $C_{1}$, une fois depuis $A_{2}$, une
fois depuis $A_{3}$. (en fait le cercle de centre $B$ et passant
par $J'$, le point de $\mathcal{C}$ diam\'etralement oppos\'e \`a
$J$, coupe $C_{1}$ en $A_{1}$ et $A_{4}$, mais nous ne l'avons pas
justifi\'e par le calcul : c'est un exercice !)
\end{enumerate}
% $$
% \includegraphics[65mm,50mm]{penta.eps}
% $$
\fincorrection
\correction{000020}
\'Ecrivons $z = \rho e^{i\theta}$, alors $\overline{z} = \rho
e^{-i\theta}$. Donc
\begin{align*}
P &= \prod_{k=1}^n \left(z^k+{\overline{z}}^k \right)\\
&= \prod_{k=1}^n \rho^k \left(  (e^{i\theta})^k + (e^{-i\theta})^k \right)\\
&= \prod_{k=1}^n \rho^k \left(  e^{ik\theta} + e^{-ik\theta}) \right)\\
&= \prod_{k=1}^n 2 \rho^k \cos {k\theta}\\
&= 2^n.\rho.\rho^2.\ldots.\rho^n \prod_{k=1}^n \cos {k\theta}\\
&= 2^n\rho^{\frac{n(n+1)}{2}} \prod_{k=1}^n \cos {k\theta}.\\
\end{align*}
\fincorrection
\correction{000080}
Nous avons par la formule de Moivre
$$\cos5\theta+i\sin5\theta=e^{i5\theta}=(e^{i\theta})^5=
(\cos\theta+i\sin\theta)^5.$$
On d\'eveloppe ce dernier produit, puis on identifie parties r\'eelles
et parties imaginaires. On obtient :
\begin{eqnarray*}
\cos 5\theta & = & \cos^5\theta -10\cos^3\theta \sin^2\theta
                    +5\cos\theta \sin^4\theta\\
\sin 5\theta & = & 5\cos^4\theta \sin \theta -10\cos^2\theta \sin^3\theta
                   +\sin^5\theta 
\end{eqnarray*}

Remarque : Gr\^ace \`a la formule $\cos^2\theta +\sin^2\theta =1$, on pourrait
 continuer les
calculs et exprimer $\cos 5\theta $ en fonction de $\cos\theta $, et
$\sin 5\theta $ en fonction de $\sin\theta $.
\fincorrection
\correction{000096}
\begin{enumerate}
\item Soit $\alpha,\beta\in\Z[i]$. Notons $\alpha=a+ib$ et $\beta=c+id$ avec $a,b,c,d\in\Z$. Alors
$\alpha+\beta=(a+c)+i(b+d)$ et $a+c\in\Z$, $b+d\in\Z$ donc
$\alpha+\beta\in\Z[i]$. De m\^eme, $\alpha\beta=(ac-bd)+i(ad+bc)$ et
$ac-bd\in\Z$, $ad+bc\in\Z$ donc $\alpha\beta\in\Z[i]$.

\item
Soit $\alpha\in\Z[i]$ inversible. Il existe donc $\beta\in\Z[i]$
tel que $\alpha\beta=1$. Ainsi, $\alpha\neq0$ et
$\frac{1}{\alpha}\in\Z[i]$. Remarquons que tout \'el\'ement non
nul de $\Z[i]$ est de module sup\'erieur ou \'egal \`a 1: en effet
$\forall z\in\C, |z|\geq \sup(|\mathop{\mathrm{Re}}\nolimits(z)|,|\mathop{\mathrm{Im}}\nolimits(z)|)$ et si
$z\in\Z[i]\setminus\{0\}$, $\sup(|\mathop{\mathrm{Re}}\nolimits(z)|,|\mathop{\mathrm{Im}}\nolimits(z)|)\geq 1$. Si
$|\alpha|\neq 1$ alors $|\alpha|>1$ et $|1/\alpha|<1$. On en
d\'eduit $1/\alpha=0$ ce qui est impossible. Ainsi $|\alpha|=1$,
ce qui implique $\alpha\in\{1,-1,i,-i\}$.

R\'eciproquement,
$1^{-1}=1\in\Z[i],(-1)^{-1}=-1\in\Z[i],i^{-1}=-i\in\Z[i],(-i)^{-1}=i\in\Z[i]$.
Les \'el\'ements inversibles de $\Z[i]$ sont donc $1,-1,i$ et
$-i$.

\item
Soit $\omega\in\C$. Notons $\omega=x+iy$ avec $x,y\in\R$. soit
$E(x)$ la partie enti\`ere de $x$, i.e. le plus grand entier
inf\'erieur ou \'egal \`a $x$: $E(x)\leq x<E(x)+1$. Si $x\leq E(x)+1/2$, 
notons $n_{x}=E(x)$, et si $x> E(x)+1/2$, notons
$n_{x}=E(x)+1$. $n_{x}$ est le, ou l'un des s'il y en a deux,
nombre entier le plus proche de $x$: $|x-n_{x}|\leq1/2$. Notons
$n_{y}$ l'entier associ\'e de la m\^eme mani\`ere \`a $y$. Soit
alors $\alpha=n_{x}+i \cdot n_{y}$. $z\in\Z[i]$ et
$|\omega-\alpha|^2=(x-n_{x})^2+(y-n_{y})^2\leq 1/4+1/4=1/2$. Donc
$|\omega-\alpha|<1$.

% $$
% \includegraphics[64mm,44mm]{gauss.wmf}
% $$
\item
Soit $\alpha,\beta\in\Z[i]$, avec $\beta\neq0$. Soit alors
$q\in\Z[i]$ tel que $|\frac{\alpha}{\beta}-q|<1$. Soit
$r=\alpha-\beta q$. Comme $\alpha\in\Z[i]$ et $\beta q\in\Z[i]$,
$r\in\Z[i]$. De plus
$|\frac{r}{\beta}|=|\frac{\alpha}{\beta}-q|<1$ donc $|r|<|\beta|$.
\end{enumerate}
\fincorrection


\end{document}

