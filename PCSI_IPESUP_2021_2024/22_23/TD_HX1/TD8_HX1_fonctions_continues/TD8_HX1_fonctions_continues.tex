\documentclass[a4paper,11pt]{article}

\usepackage{inputenc}
\usepackage[T1]{fontenc}
\usepackage[frenchb]{babel}
\usepackage{fancyhdr,fancybox} % pour personnaliser les en-têtes
\usepackage{lastpage,setspace}
\usepackage{amsfonts,amssymb,amsmath,amsthm,mathrsfs}
\usepackage{relsize,exscale,bbold}
\usepackage{paralist}
\usepackage{xspace,multicol,diagbox,array}
\usepackage{xcolor}
\usepackage{variations}
\usepackage{xypic}
\usepackage{eurosym,stmaryrd}
\usepackage{graphicx}
\usepackage[np]{numprint}
\usepackage{hyperref} 
\usepackage{tikz}
\usepackage{colortbl}
\usepackage{multirow}
\usepackage{MnSymbol,wasysym}
\usepackage[top=1.5cm,bottom=1.5cm,right=1.2cm,left=1.5cm]{geometry}
\usetikzlibrary{calc, arrows, plotmarks, babel,decorations.pathreplacing}
\setstretch{1.25}
%\usepackage{lipsum} %\usepackage{enumitem} %\setlist[enumerate]{itemsep=1mm} bug avec enumerate



\newtheorem{thm}{Théorème}
\newtheorem{rmq}{Remarque}
\newtheorem{prop}{Propriété}
\newtheorem{cor}{Corollaire}
\newtheorem{lem}{Lemme}
\newtheorem{prop-def}{Propriété-définition}

\theoremstyle{definition}

\newtheorem{defi}{Définition}
\newtheorem{ex}{Exemple}
\newtheorem*{rap}{Rappel}
\newtheorem{cex}{Contre-exemple}
\newtheorem{exo}{Exercice} % \large {\fontfamily{ptm}\selectfont EXERCICE}
\newtheorem{nota}{Notation}
\newtheorem{ax}{Axiome}
\newtheorem{appl}{Application}
\newtheorem{csq}{Conséquence}
\def\di{\displaystyle}



\renewcommand{\thesection}{\Roman{section}}\renewcommand{\thesubsection}{\arabic{subsection} }\renewcommand{\thesubsubsection}{\alph{subsubsection} }


\newcommand{\bas}{~\backslash}\newcommand{\ba}{\backslash}
\newcommand{\C}{\mathbb{C}}\newcommand{\R}{\mathbb{R}}\newcommand{\Q}{\mathbb{Q}}\newcommand{\Z}{\mathbb{Z}}\newcommand{\N}{\mathbb{N}}\newcommand{\V}{\overrightarrow}\newcommand{\Cs}{\mathscr{C}}\newcommand{\Ps}{\mathscr{P}}\newcommand{\Rs}{\mathscr{R}}\newcommand{\Gs}{\mathscr{G}}\newcommand{\Ds}{\mathscr{D}}\newcommand{\happy}{\huge\smiley}\newcommand{\sad}{\huge\frownie}\newcommand{\danger}{\begin{tikzpicture}[x=1.5pt,y=1.5pt,rotate=-14.2]
	\definecolor{myred}{rgb}{1,0.215686,0}
	\draw[line width=0.1pt,fill=myred] (13.074200,4.937500)--(5.085940,14.085900)..controls (5.085940,14.085900) and (4.070310,15.429700)..(3.636720,13.773400)
	..controls (3.203130,12.113300) and (0.917969,2.382810)..(0.917969,2.382810)
	..controls (0.917969,2.382810) and (0.621094,0.992188)..(2.097660,1.359380)
	..controls (3.574220,1.726560) and (12.468800,3.984380)..(12.468800,3.984380)
	..controls (12.468800,3.984380) and (13.437500,4.132810)..(13.074200,4.937500)
	--cycle;
	\draw[line width=0.1pt,fill=white] (11.078100,5.511720)--(5.406250,11.875000)..controls (5.406250,11.875000) and (4.683590,12.812500)..(4.367190,11.648400)
	..controls (4.050780,10.488300) and (2.375000,3.675780)..(2.375000,3.675780)
	..controls (2.375000,3.675780) and (2.156250,2.703130)..(3.214840,2.964840)
	..controls (4.273440,3.230470) and (10.640600,4.847660)..(10.640600,4.847660)
	..controls (10.640600,4.847660) and (11.332000,4.953130)..(11.078100,5.511720)
	--cycle;
	\fill (6.144520,8.839900)..controls (6.460940,7.558590) and (6.464840,6.457090)..(6.152340,6.378910)
	..controls (5.835930,6.300840) and (5.320300,7.277400)..(5.003900,8.554750)
	..controls (4.683590,9.835940) and (4.679690,10.941400)..(4.996090,11.019600)
	..controls (5.312490,11.097700) and (5.824210,10.121100)..(6.144520,8.839900)
	--cycle;
	\fill (7.292960,5.261780)..controls (7.382800,4.898500) and (7.128900,4.523500)..(6.730460,4.421880)
	..controls (6.328120,4.324220) and (5.929680,4.535220)..(5.835930,4.898500)
	..controls (5.746080,5.261780) and (5.999990,5.640630)..(6.402340,5.738340)
	..controls (6.804690,5.839840) and (7.203110,5.625060)..(7.292960,5.261780)
	--cycle;
	\end{tikzpicture}}\newcommand{\alors}{\Large\Rightarrow}\newcommand{\equi}{\Leftrightarrow}
\newcommand{\fonction}[5]{\begin{array}{l|rcl}
		#1: & #2 & \longrightarrow & #3 \\
		& #4 & \longmapsto & #5 \end{array}}


\definecolor{vert}{RGB}{11,160,78}
\definecolor{rouge}{RGB}{255,120,120}
\definecolor{bleu}{RGB}{15,5,107}



\pagestyle{fancy}
\lhead{Groupe IPESUP}\chead{}\rhead{Année~2022-2023}\lfoot{M. Botcazou \& M.Dupré}\cfoot{\thepage/3}\rfoot{PCSI }\renewcommand{\headrulewidth}{0.4pt}\renewcommand{\footrulewidth}{0.4pt}


\begin{document}
 	
	

\noindent\shadowbox{
	\begin{minipage}{1\linewidth}
		\centering
		\huge{\textbf{ TD 8 : Fonctions continues }}
	\end{minipage}
}
\medskip
%%%%%%%%%% Bibmath %%%%%%%%%%%%%%

%https://www.bibmath.net/ressources/index.php?action=affiche&quoi=bde/analyse/unevariable/continuite&type=fexo

%https://www.bibmath.net/ressources/index.php?action=affiche&quoi=bde/analyse/unevariable/continuiteuniforme&type=fexo





\begin{minipage}[t]{1\linewidth}
\section*{Études locales et manipulations:}
	
	\begin{minipage}[t]{0.48\linewidth}
		\raggedright

		\begin{exo}\textbf{(*)}\quad\\[0.2cm]
		Montrer en revenant à la définition que $f(x)=\frac{3x-1}{x-5}$ est continue en tout point de $\R\setminus\{5\}$.
		
		\centering
		\rule{1\linewidth}{0.6pt}
	\end{exo}

	\begin{exo}\textbf{(*)}\quad\\[0.2cm]
	Soit $A$ une partie non vide de $\R$. Pour $x\in\R$, on pose $f(x)=\mbox{Inf}\{|y-x|,\;y\in A\}$. Montrer que $f$ est continue en tout point de $\R$.
	
			\centering
	\rule{1\linewidth}{0.6pt}
	\end{exo}	
			
	\begin{exo}\textbf{(**)}\quad\\[0.2cm]
		\begin{enumerate}
			\item 	Soit $f:\mathbb R\to\mathbb R$ définie par $f(x)=\lfloor x\rfloor +\sqrt{x-\lfloor x\rfloor }$.
			\'Etudier la continuité de $f$ sur $\mathbb R$.
			\item Soit $g\in\Cs^0(\R)$, à quelle condition la fonction f définie par $f(x)=\lfloor x\rfloor +g(x-\lfloor x\rfloor )$ est-elle continue?
		\end{enumerate}
	
		
	\centering
		\rule{1\linewidth}{0.6pt}
	\end{exo}
	
	
	
	\begin{exo}\textbf{(**)}\quad\\[0.2cm]
	 	Soit $f:\mathbb R\to\mathbb R$ la fonction définie par 
	 	$$f(x)=\left\{
	 	\begin{array}{cl}
	 	x&\textrm{si }x\in\mathbb Q\\
	 	x+1&\textrm{si }x\notin \mathbb Q.
	 	\end{array}\right.$$
	 	Montrer que $f$ est discontinue en tout point.
	 	
	 	
		\centering
\rule{1\linewidth}{0.6pt}
\end{exo}

	\begin{exo}\textbf{(*)}\quad\\[0.2cm]
	Dire si les fonctions suivantes sont prolongeables par continuité à $\mathbb R$ tout entier :
	\begin{enumerate}
		\item $f(x)=\sin(1/x)$ si $x\neq 0$;
		\item $g(x)=\sin(x)\sin(1/x)$ si $x\neq 0$;
		\item $h(x)=\cos(x)\cos(1/x)$ si $x\neq 0$.
	\end{enumerate}
	
	\centering
	\rule{1\linewidth}{0.6pt}
\end{exo}

		\begin{exo}\textbf{(**)}\quad\\[0.2cm]
	Trouver pour $(a,b)\in (\R^{+*})^{2}$ :\ $\lim\limits_{x\rightarrow 0^{+}}\left(\frac{a^{x}+b^{x}}{2}\right)^{\frac{1}{x}}.$
	
	\centering
	\rule{1\linewidth}{0.6pt}
\end{exo}




\end{minipage}	
\hfill\vrule\hfill
\begin{minipage}[t]{0.48\linewidth}
\raggedright


	\begin{exo}\textbf{(*)}\quad\\[0.2cm]
Les fonctions suivantes sont-elles prolongeables par continuit\'e sur
$\R$ ?
$$a)\ g(x)=\frac{1}{x}\ln\frac{e^x+e^{-x}}{2}\ ;$$
$$b)\ h(x)=\frac{1}{1-x}-\frac{2}{1-x^2}\ .$$
		
		\centering
		\rule{1\linewidth}{0.6pt}
	\end{exo}
	
	
	
	\begin{exo}\textbf{(**)}\quad\\[0.2cm]
		\begin{enumerate}
			\item  D\'emontrer que $\displaystyle{ \lim_{x\rightarrow 0}\frac{\sqrt{1+x}-\sqrt{1-x}}{ x}=1}$.
			\item  Soient $m,n$ des entiers positifs.
			
			 \'Etudier $\displaystyle{\lim_{x\rightarrow 0}\frac{\sqrt{1+x^m}-
					\sqrt{1-x^m}}{ x^n}}$.
			\item D\'emontrer que $\displaystyle{ \lim_{x\rightarrow 0}\frac{1}{ x}(\sqrt{1+x+x^2}-1)=
				\frac{1}{ 2}}$.
		\end{enumerate}
	
		\centering
		\rule{1\linewidth}{0.6pt}
	\end{exo}

	\begin{exo}\textbf{(*)}\quad\\[0.2cm]
 Calculer lorsqu'elles existent les limites suivantes
$$\begin{array}{lll}
a)\ \lim\limits_{x\rightarrow 0}\frac{x^2+2\,|x|}{x} &\quad  b)\
\lim\limits_{x\rightarrow -\infty}\frac{x^2+2\,|x|}{x}\\
\\
c) \ \lim\limits_{x\rightarrow\pi}\frac{\sin^2x}{1+\cos x}&
\quad  d)\ \lim\limits_{x\rightarrow 0}\frac{\sqrt{1+x}-\sqrt{1+x^2}}{x}\\
\\
e)\ \lim\limits_{x\rightarrow 0}\frac{\sqrt[3]{1+x^2}-1}{x^2}&
\quad  f)\ \lim\limits_{x\rightarrow 1}\frac{x-1}{x^n-1}\\
\\
g)\ \lim\limits_{x\rightarrow 2}\frac{x^2-4}{x^2-3\,x+2}&
\quad  h)\ \lim\limits_{x\rightarrow +\infty}\sqrt{x+5}-\sqrt{x-3}\\
\\
\end{array}$$


		\centering
\rule{1\linewidth}{0.6pt}
\end{exo}

	\begin{exo}\textbf{(***)}\quad\\[0.2cm]
	Étudier l'existence d'une limite et la continuité éventuelle en chacun de ses points de la fonction définie sur $]0,+\infty[$ par $f(x)=0$ si $x$ est irrationnel et $f(x)=\frac{1}{p+q}$ si $x$ est rationnel égal à $\frac{p}{q}$, la fraction $\frac{p}{q}$ étant irréductible.
	
	\centering
	\rule{1\linewidth}{0.6pt}
\end{exo}


\end{minipage}
\end{minipage}

\medskip
\begin{minipage}[t]{1\linewidth}
	\begin{minipage}[t]{0.48\linewidth}
		\raggedright
		
		
		

		
		
		
		\begin{exo}\textbf{(**)}\quad\\[0.2cm]
			Calculer, lorsqu'elles existent, les limites suivantes :
			$$
			\lim_{x\rightarrow \alpha} \frac{x^{n+1}-\alpha^{n+1}}{x^n-\alpha^n},
			$$
			
			$$
			\lim_{x\rightarrow 0} \frac{\tan x - \sin x}{\sin x(\cos 2x - \cos x)},
			$$
			
			$$
			\lim_{x\rightarrow +\infty} \sqrt{x+\sqrt{x+\sqrt{x}}}-\sqrt{x},
			$$
			
			$$
			\lim_{x\rightarrow \alpha^+} \frac{\sqrt{x}-\sqrt{\alpha}-\sqrt{x-\alpha}}{\sqrt{x^2-\alpha^2}}, \quad (\alpha >0)
			$$
			
			$$
			\lim_{x\rightarrow 0} xE\left(\frac{1}{x}\right),
			$$
			
			$$
			\lim_{x\rightarrow 2} \frac{e^x-e^2}{x^2+x-6},
			$$
			
			$$
			\lim_{x\rightarrow +\infty} \frac{x^4}{1+x^\alpha\sin^2x}, \text{ en fonction de $\alpha \in \R$.}
			$$
			
			\centering
			\rule{1\linewidth}{0.6pt}
		\end{exo}
		
		
		\medskip		
		
		
	\end{minipage}	
	\hfill\vrule\hfill
	\begin{minipage}[t]{0.48\linewidth}
		\raggedright
		
		
		\begin{exo}\textbf{(**)}\quad\\[0.2cm]
	 D\'eterminer les limites suivantes, en justifiant vos calculs.
	\begin{enumerate}
		\item $\displaystyle\lim_{x \rightarrow0^+}{x+2 \over x^2 \ln x}$
		
		%\item $\displaystyle\lim_{x \rightarrow0^+}2x \ln(x+\sqrt x)$
		
		\item $\displaystyle\lim_{x \rightarrow+ \infty}{x^3-2x^2+3 \over x \ln x}$
		
		%\item $\displaystyle\lim_{x \rightarrow+ \infty}{e^{\sqrt x+1} \over x+2}$
		
		\item $\displaystyle\lim_{x \rightarrow0^+}{\ln(3x+1) \over2x}$
		
		\item $\displaystyle\lim_{x \rightarrow0^+}{x^x-1 \over\ln(x+1)}$
		
		\item $\displaystyle\lim_{x \rightarrow- \infty}{2 \over x+1}\ln
		\Bigl({x^3+4 \over1-x^2}\Bigr)$
		
		%\item $\displaystyle\lim_{x \rightarrow{(-1)}^+}(x^2-1) \ln(7x^3+4x^2+3)$
		
		%\item $\displaystyle\lim_{x \rightarrow2^+}{(x-2)}^2 \ln(x^3-8)$
		
		%\item $\displaystyle\lim_{x \rightarrow0^+}{x(x^x-1) \over\ln(x+1)}$
		
		%\item $\displaystyle\lim_{x \rightarrow+ \infty}(x \ln x -x \ln(x+2))$
		
		\item $\displaystyle\lim_{x \rightarrow+ \infty}{e^x-e^{x^2} \over x^2-x}$ 
		
		%\item $\displaystyle\lim_{x \rightarrow0^+}{{(1+x)}^{\ln x}}$
		
		\item $\displaystyle\lim_{x \rightarrow+ \infty}{\Bigl({x+1 \over x-3}\Bigr)^x}$
		
		%\item $\displaystyle\lim_{x \rightarrow+ \infty}{\Bigl({x^3+5 \over
				%x^2+2}\Bigr)^{x+1 \over x^2+1}}$
		
		\item $\displaystyle\lim_{x \rightarrow+ \infty}{\Bigl({e^x+1 \over
				x+2}\Bigr)^{1 \over x+1}}$
		
		%\item $\displaystyle\lim_{x \rightarrow0^+}\bigl(\ln(1+x)\bigr)^{1\over\ln x}$
		
		\item $\displaystyle\lim_{x \rightarrow+ \infty}{x^{(x^{x-1})} \over
			x^{(x^{x})}}$
		
		%\item $\displaystyle\lim_{x \rightarrow+ \infty}{(x+1)^x \over x^{x+1}}$
		
		\item $\displaystyle\lim_{x \rightarrow+ \infty}{x \sqrt{\ln(x^2+1)} \over
			1+e^{x-3}}$
	\end{enumerate}

			\centering
			\rule{1\linewidth}{0.6pt}
		\end{exo}
		
		
	\end{minipage}
\end{minipage}

\begin{minipage}[t]{1\linewidth}
	\section*{Équations fonctionnelles}
	\begin{minipage}[t]{0.48\linewidth}
		\raggedright
		
		
		
		\begin{exo}\textbf{(*)}\quad\\[0.2cm]
			On cherche à déterminer toutes les fonctions continues $f:\mathbb R\to\mathbb R$  vérifiant, pour tout $x\in\mathbb R$, $f(2x)-f(x)=x.$
			\begin{enumerate}
				\item Soit $f$ une telle fonction. Démontrer que, pour tout $x\in\mathbb R$ et pour tout $n\geq 1$, on a 
				$$f(x)-f(x/2^n)=\sum_{k=1}^n\frac{x}{2^k}.$$
				\item Répondre au problème posé.
			\end{enumerate}
			
			\centering
			\rule{1\linewidth}{0.6pt}
		\end{exo}
		
		
		

			\begin{exo}\textbf{(***)}\quad\\[0.2cm]
		Soit $f$ une fonction définie sur un voisinage de $0$ telle que $\lim\limits_{x\rightarrow 0}f(x)=0$ et $\lim\limits_{x\rightarrow 0}\frac{f(2x)-f(x)}{x}=0$.
		
		 Montrer que $\lim_{x\rightarrow 0}\frac{f(x)}{x}=0$.
		 
		  (Indication. Considérer $g(x)=\frac{f(2x)-f(x)}{x}$.)
		
		\centering
		\rule{1\linewidth}{0.6pt}
	\end{exo}
		
		
\medskip		
		
		
	\end{minipage}	
	\hfill\vrule\hfill
	\begin{minipage}[t]{0.48\linewidth}
		\raggedright
		
		
		\begin{exo}\textbf{(*)}\quad\\[0.2cm]
			Soit $f : \R \rightarrow \R$ continue en $0$ telle que pour chaque $x \in \R$,
			$f(x) = f(2x)$. Montrer que $f$ est constante.
			
			\centering
			\rule{1\linewidth}{0.6pt}
		\end{exo}
		
		
		
		\begin{exo}\textbf{(**)}\quad\\[0.2cm]
			Soient $I$ un intervalle de $\R$ et $f : I \rightarrow \R$ continue, telle que pour chaque $x \in I$, $f (x)^2 = 1$.
			Montrer que $f = 1$  ou $f = -1$.
			
			\centering
			\rule{1\linewidth}{0.6pt}
		\end{exo}
	
			\begin{exo}\textbf{(***)}\quad\\[0.2cm]
		Soit $f$ une fonction réelle d'une variable réelle définie et continue sur un voisinage de $+\infty$. 
		
		On suppose que la fonction $f(x+1)-f(x)$ admet dans $\R$ une limite $\ell$ quand $x$ tend vers $+\infty$. 
		
		Etudier l'existence et la valeur eventuelle de $\lim_{x\rightarrow +\infty}\frac{f(x)}{x}$.
		
		\centering
		\rule{1\linewidth}{0.6pt}
	\end{exo}
		

		
	\end{minipage}
\end{minipage}



\medskip		

\begin{minipage}[t]{1\linewidth}
	\section*{Fonctions continues sur un intervalle}

	
	\begin{minipage}[t]{0.48\linewidth}
		\raggedright
		
			%\subsection*{Théorèmes et propriétés importantes}
	
				\begin{exo}\textbf{(*)}\quad\\[0.2cm]
			Soit $I$ un intervalle, $k>0$ et $f:I\to\mathbb R$ vérifiant :
			$$\forall (x,y)\in I^2,\ |f(x)-f(y)|\leq k|x-y|.$$
			Démontrer que $f$ est continue sur $I$. La fonction $f$ est-elle uniformément continue ?
			
			\centering
			\rule{1\linewidth}{0.6pt}
		\end{exo}
	
		\begin{exo}\textbf{(*)}\quad\\[0.2cm]
			Démontrer que l'équation $\cos x=\frac 1x$ admet une infinité de solutions dans $\mathbb R_+^*$.
			
			\centering
			\rule{1\linewidth}{0.6pt}
		\end{exo}
	
			\begin{exo}\textbf{(**)}\quad\\[0.2cm]
		Soit $f:\mathbb R_+\to\mathbb R$ une fonction continue surjective. 
		\begin{enumerate}
			\item Démontrer que $0$ admet un nombre infini d'antécédents.
			\item Plus généralement, démontrer que tout réel admet un nombre infini d'antécédents.
		\end{enumerate}
		
		\centering
		\rule{1\linewidth}{0.6pt}
	\end{exo}
		
		
		
	
	

\begin{exo}\textbf{(**)}\quad\\[0.2cm]
	Soit $f:\mathbb R_+\to \mathbb R_+$ continue. On suppose que $x\mapsto \frac{f(x)}x$ admet une limite
	finie $l<1$ en $+\infty$. Démontrer que $f$ admet un point fixe.
	
	\centering
	\rule{1\linewidth}{0.6pt}
\end{exo}

	\begin{exo}\textbf{(**)}\quad\\[0.2cm]
		Une personne parcourt 4 km en 1 heure.
		Montrer qu'il existe un intervalle de 30 mn pendant lequel elle parcourt exactement 2 km.
		
			\centering
		\rule{1\linewidth}{0.6pt}
		\end{exo}
	
	
		\begin{exo}\textbf{(**)}\quad\\[0.2cm]
		Trouver $f$  bijective de $[0,1]$ sur lui-même et discontinue en chacun de ses points.
		
		\centering
		\rule{1\linewidth}{0.6pt}
	\end{exo}

	\begin{exo}\textbf{(***)}\quad\\[0.2cm]%https://www.bibmath.net/ressources/justeunexo.php?id=1943
	Soit $f:I\to\mathbb R$ une fonction monotone. Montrer que l'ensemble de ses points de discontinuité est fini ou dénombrable.
	
	\centering
	\rule{1\linewidth}{0.6pt}
\end{exo}

	
		
	\end{minipage}	
	\hfill\vrule\hfill
	\begin{minipage}[t]{0.48\linewidth}
		\raggedright
			%\subsection*{Continuité uniforme}
		

			\begin{exo}\textbf{(**)}\quad\\[0.2cm]
			Soit $f:[0,1]\to[0,1]$ une fonction continue. Démontrer que $f$ admet toujours au moins un point fixe.
			
			\centering
			\rule{1\linewidth}{0.6pt}
		\end{exo}
		
		
		\begin{exo}\textbf{(**)}\quad\\[0.2cm]
			Soit $f$ une fonction continue sur $\R$ admettant une période $T$. Prouver que $f$ est uniformément continue.
			
			\centering
			\rule{1\linewidth}{0.6pt}
		\end{exo}
		

	
	\begin{exo}\textbf{(***)}\quad\\[0.2cm]
	Soient $f,g:[a,b]\to\mathbb R$ deux fonctions continues. Pour $t\in\mathbb R$, on pose
	$$h(t)=\sup\{f(x)+tg(x);\ x\in[a,b]\}.$$
	Montrer que $h$ est lipschitzienne.
	
				\centering
	\rule{1\linewidth}{0.6pt}
\end{exo}

	\begin{exo}\textbf{(***)}\quad\\[0.2cm]
	Une fonction $f$ définie sur un intervalle $I \subset \mathbb{R}$ est SCI (pour semi-continue inférieurement) si\\[-0.4cm]
	
	 $$\forall x_{0} \in I, \forall \varepsilon>0, \exists \eta>0, \forall x \in I,$$\\[-0.8cm]
	
	$$ \left|x-x_{0}\right| \leq \eta \quad \Rightarrow \quad f(x) \geq f\left(x_{0}\right)-\varepsilon .
	$$
	\begin{enumerate}
		\item
		\begin{enumerate}
			\item Montrer qu'une fonction continue sur un intervalle est SCI.
			\item Déterminer une fonction SCI qui n'est pas continue.
		\end{enumerate}
	\item
	\begin{enumerate}
		\item Montrer que la somme de deux fonctions SCI sur un intervalle $I$ est encore une fonction SCI.
		\item Montrer que le produit de deux fonctions SCI n'est pas forcément une fonction SCI.
	\end{enumerate}

		\item Soit $f$ une fonction SCI sur un segment $I$.
	\begin{enumerate}
		\item Montrer que $f$ est minorée.
		\item Montrer qu'il existe $x_{0} \in I$ tel que $f\left(x_{0}\right)=\inf _{x \in I} f(x)$.
	\end{enumerate}
	\end{enumerate}

				\centering
\rule{1\linewidth}{0.6pt}

\end{exo}	
		
		
	\end{minipage}
\end{minipage}





\end{document}