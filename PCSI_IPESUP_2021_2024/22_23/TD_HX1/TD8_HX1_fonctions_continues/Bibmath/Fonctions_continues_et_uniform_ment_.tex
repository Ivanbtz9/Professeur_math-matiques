\documentclass[11pt]{article}

 %Configuration de la feuille 
 
\usepackage{amsmath,amssymb,enumerate,graphicx,pgf,tikz,fancyhdr}
\usepackage[utf8]{inputenc}
\usetikzlibrary{arrows}
\usepackage{geometry}
\usepackage{tabvar}
\geometry{hmargin=2.2cm,vmargin=1.5cm}\pagestyle{fancy}
\lfoot{\bfseries http://www.bibmath.net}
\rfoot{\bfseries\thepage}
\cfoot{}
\renewcommand{\footrulewidth}{0.5pt} %Filet en bas de page

 %Macros utilisées dans la base de données d'exercices 

\newcommand{\mtn}{\mathbb{N}}
\newcommand{\mtns}{\mathbb{N}^*}
\newcommand{\mtz}{\mathbb{Z}}
\newcommand{\mtr}{\mathbb{R}}
\newcommand{\mtk}{\mathbb{K}}
\newcommand{\mtq}{\mathbb{Q}}
\newcommand{\mtc}{\mathbb{C}}
\newcommand{\mch}{\mathcal{H}}
\newcommand{\mcp}{\mathcal{P}}
\newcommand{\mcb}{\mathcal{B}}
\newcommand{\mcl}{\mathcal{L}}
\newcommand{\mcm}{\mathcal{M}}
\newcommand{\mcc}{\mathcal{C}}
\newcommand{\mcmn}{\mathcal{M}}
\newcommand{\mcmnr}{\mathcal{M}_n(\mtr)}
\newcommand{\mcmnk}{\mathcal{M}_n(\mtk)}
\newcommand{\mcsn}{\mathcal{S}_n}
\newcommand{\mcs}{\mathcal{S}}
\newcommand{\mcd}{\mathcal{D}}
\newcommand{\mcsns}{\mathcal{S}_n^{++}}
\newcommand{\glnk}{GL_n(\mtk)}
\newcommand{\mnr}{\mathcal{M}_n(\mtr)}
\DeclareMathOperator{\ch}{ch}
\DeclareMathOperator{\sh}{sh}
\DeclareMathOperator{\vect}{vect}
\DeclareMathOperator{\card}{card}
\DeclareMathOperator{\comat}{comat}
\DeclareMathOperator{\imv}{Im}
\DeclareMathOperator{\rang}{rg}
\DeclareMathOperator{\Fr}{Fr}
\DeclareMathOperator{\diam}{diam}
\DeclareMathOperator{\supp}{supp}
\newcommand{\veps}{\varepsilon}
\newcommand{\mcu}{\mathcal{U}}
\newcommand{\mcun}{\mcu_n}
\newcommand{\dis}{\displaystyle}
\newcommand{\croouv}{[\![}
\newcommand{\crofer}{]\!]}
\newcommand{\rab}{\mathcal{R}(a,b)}
\newcommand{\pss}[2]{\langle #1,#2\rangle}
 %Document 

\begin{document} 

\begin{center}\textsc{{\huge TD8 fcts continues}}\end{center}

% Exercice 215


\vskip0.3cm\noindent\textsc{Exercice 1} - Avec la partie entière
\vskip0.2cm
Soit $f:\mathbb R\to\mathbb R$ définie par $f(x)=\lfloor x\rfloor +\sqrt{x-\lfloor x\rfloor }$.
\'Etudier la continuité de $f$ sur $\mathbb R$.


% Exercice 218


\vskip0.3cm\noindent\textsc{Exercice 2} - Indicatrice de $\mathbb Q$
\vskip0.2cm
Soit $f:\mathbb R\to\mathbb R$ la fonction définie par 
$$f(x)=\left\{
\begin{array}{ll}
1&\textrm{si }x\in\mathbb Q\\
0&\textrm{si }x\notin \mathbb Q.
\end{array}\right.$$
Montrer que $f$ est discontinue en tout point.


% Exercice 219


\vskip0.3cm\noindent\textsc{Exercice 3} - Une fonction bizarre
\vskip0.2cm
Soit $f:\mathbb R\to \mathbb R$ la fonction définie par
$$f(x)=\left\{
\begin{array}{ll}
0&\textrm{si $x$ est irrationnel ou $x=0$.}\\
\frac{1}{q}&\textrm{si $x=p/q$, avec $p\in \mathbb Z$, $q\geq 1$ et $p\wedge q=1$ }\\
\end{array}\right.$$
En quels points $f$ est-elle continue?


% Exercice 217


\vskip0.3cm\noindent\textsc{Exercice 4} - Prolongement par continuité et fonctions trigonométriques
\vskip0.2cm
Dire si les fonctions suivantes sont prolongeables par continuité à $\mathbb R$ tout entier :
\begin{enumerate}
\item $f(x)=\sin(1/x)$ si $x\neq 0$;
\item $g(x)=\sin(x)\sin(1/x)$ si $x\neq 0$;
\item $h(x)=\cos(x)\cos(1/x)$ si $x\neq 0$.
\end{enumerate}


% Exercice 3206


\vskip0.3cm\noindent\textsc{Exercice 5} - Une fonction lipschitzienne est continue
\vskip0.2cm
Soit $I$ un intervalle, $k>0$ et $f:I\to\mathbb R$ vérifiant :
$$\forall (x,y)\in I^2,\ |f(x)-f(y)|\leq k|x-y|.$$
Démontrer que $f$ est continue sur $I$.


% Exercice 213


\vskip0.3cm\noindent\textsc{Exercice 6} - Fonctions monotones
\vskip0.2cm
Soit $f:I\to\mathbb R$ une fonction monotone. Montrer que l'ensemble de ses points de discontinuité est fini ou dénombrable.


% Exercice 3207


\vskip0.3cm\noindent\textsc{Exercice 7} - Une équation fonctionnelle
\vskip0.2cm
On cherche à déterminer toutes les fonctions continues $f:\mathbb R\to\mathbb R$  vérifiant, pour tout $x\in\mathbb R$, $f(2x)-f(x)=x.$
\begin{enumerate}
\item Soit $f$ une telle fonction. Démontrer que, pour tout $x\in\mathbb R$ et pour tout $n\geq 1$, on a 
$$f(x)-f(x/2^n)=\sum_{k=1}^n\frac{x}{2^k}.$$
\item Répondre au problème posé.
\end{enumerate}


% Exercice 2864


\vskip0.3cm\noindent\textsc{Exercice 8} - Une infinité de solutions
\vskip0.2cm
Démontrer que l'équation $\cos x=\frac 1x$ admet une infinité de solutions dans $\mathbb R_+^*$.


% Exercice 225


\vskip0.3cm\noindent\textsc{Exercice 9} - Point fixe
\vskip0.2cm
Soit $f:[0,1]\to[0,1]$ une fonction continue. Démontrer que $f$ admet toujours au moins un point fixe.


% Exercice 226


\vskip0.3cm\noindent\textsc{Exercice 10} - Point fixe
\vskip0.2cm
Soit $f:\mathbb R_+\to \mathbb R_+$ continue. On suppose que $x\mapsto \frac{f(x)}x$ admet une limite
finie $l<1$ en $+\infty$. Démontrer que $f$ admet un point fixe.


% Exercice 2604


\vskip0.3cm\noindent\textsc{Exercice 11} - Infinité d'antécédents
\vskip0.2cm
Soit $f:\mathbb R_+\to\mathbb R$ une fonction continue surjective. 
\begin{enumerate}
\item Démontrer que $0$ admet un nombre infini d'antécédents.
\item Plus généralement, démontrer que tout réel admet un nombre infini d'antécédents.
\end{enumerate}


% Exercice 195


\vskip0.3cm\noindent\textsc{Exercice 12} - Fonctions continues périodiques
\vskip0.2cm
Soit $f$ une fonction continue sur $\mtr$ admettant une période $T$. Prouver que $f$ est uniformément continue.


% Exercice 196


\vskip0.3cm\noindent\textsc{Exercice 13} - Avec une limite à l'infini
\vskip0.2cm
Soit $f:\mathbb R_+\to\mathbb R$ une fonction continue admettant une limite (finie) en $+\infty$. Montrer que $f$ est uniformément continue.


% Exercice 237


\vskip0.3cm\noindent\textsc{Exercice 14} - Une fonction étonnament lipschitzienne
\vskip0.2cm
Soient $f,g:[a,b]\to\mathbb R$ deux fonctions continues. Pour $t\in\mathbb R$, on pose
$$h(t)=\sup\{f(x)+tg(x);\ x\in[a,b]\}.$$
Montrer que $h$ est lipschitzienne.




\vskip0.5cm
\noindent{\small Cette feuille d'exercices a été conçue à l'aide du site \textsf{http://www.bibmath.net}}

%Vous avez accès aux corrigés de cette feuille par l'url : https://www.bibmath.net/ressources/justeunefeuille.php?id=25491
\end{document}