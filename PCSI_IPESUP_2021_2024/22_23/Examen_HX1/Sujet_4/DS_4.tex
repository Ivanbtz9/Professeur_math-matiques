\documentclass[a4paper,11pt]{article}


\usepackage[frenchb]{babel}

\usepackage{amsfonts,amssymb,amsmath,amsthm,mathrsfs}
\usepackage[T1]{fontenc}
\usepackage{fancyhdr,fancybox} % pour personnaliser les en-tétes
\usepackage{lastpage}
\usepackage{paralist}
\usepackage{xspace,multicol}
\usepackage{xcolor}
\usepackage{variations}
\usepackage{eurosym}
\usepackage{graphicx}
\usepackage[np]{numprint}
\usepackage{hyperref} 
\usepackage{lipsum}

\usepackage{enumitem}

\usepackage{colortbl}
\usepackage{multirow}
\usepackage[top=1.5cm,bottom=1.5cm,right=1.2cm,left=1.5cm]{geometry}

\newtheorem{defi}{Définition}
\newtheorem{ex}{Exemple}
\newtheorem{cex}{Contre-exemple}
\newtheorem{exer}{Exercice} % \large {\fontfamily{ptm}\selectfont EXERCICE}
\newtheorem{nota}{Notation}
\newtheorem{ax}{Axiome}
\newtheorem{appl}{Application}
\newtheorem{csq}{Conséquence}




\setlist[enumerate]{itemsep=1mm}


\pagestyle{fancy}
\lhead{Groupe IPESUP}\chead{}\rhead{Année~2022-2023}\lfoot{M.Botcazou \& M.Dupré}\cfoot{\thepage/3}\rfoot{\textbf{Tournez la page S.V.P.}}\renewcommand{\headrulewidth}{0.4pt}\renewcommand{\footrulewidth}{0.4pt}


\begin{document}

	\noindent\shadowbox{
		\begin{minipage}{1\linewidth}
		$$\huge{\textbf{ Examen n°4}}$$
		$$\left(\text{Temps : }4\text{ heures}\right) $$
	
		\end{minipage}
	}
\medskip

\bigskip

\begin{enumerate}

\item La présentation, la lisibilité, l'orthographe, \textbf{la qualité de la rédaction et la précision des raisonnements} entreront pour
une part importante dans l'appréciation des copies. Autrement dit, toute rédaction "fumeuse" ou toute justification bancale n'apportera qu'une faible quantité de points.
\item Les étudiants sont invités é encadrer dans la mesure du possible les résultats de leurs calculs. \textbf{Les réponses doivent toutes étre soigneusement justifiées}
\item Les calculatrices sont interdites.
\end{enumerate}







\newpage

	\lhead{Groupe IPESUP}\chead{}\rhead{Année~2022-2023}\lfoot{M.Botcazou \& M.Dupré}\cfoot{\thepage/3}\rfoot{\textbf{FIN.}}\renewcommand{\headrulewidth}{0.4pt}\renewcommand{\footrulewidth}{0.4pt}
	
	
\begin{exer}
\begin{enumerate}
\item Soit $f: \mathbb{R}_{+} \rightarrow \mathbb{R}_{+}$continue. On suppose que $x \mapsto \frac{f(x)}{x}$ admet une limite finie $l<1$ en $+\infty$. Démontrer que $f$ admet un point fixe
\item 
On dit qu'une matrice $A=(a_{i,j})$ est \textbf{stochastique} si :
\begin{itemize}
\item $a_{i,j} \geq 0 \ \forall \ (i,j) \in [|1,n|]^2$
\item $\displaystyle\sum_{j=1}^{n}{a_{i,j}} = 1 \ \forall \ i \in [|1,n|]$
\end{itemize}
Montrer que le produit de deux matrices stochastiques est stochastique.
\item Pour tout vecteur du plan fixé $\left(\begin{array}{c}x_0 \\ y_0\end{array}\right) \in \mathbb{R}^2$, on considère la suite définie par récurrence :
$$
\left(\begin{array}{l}
x_{n+1} \\
y_{n+1}
\end{array}\right)=\left(\begin{array}{cc}
0 & -1 \\
1 & 0
\end{array}\right)\left(\begin{array}{l}
x_n \\
y_n
\end{array}\right)
$$
(a) En partant de $\left(\begin{array}{l}x_0 \\ y_0\end{array}\right)=\left(\begin{array}{l}1 \\ 0\end{array}\right)$, représenter les 8 premiers termes de la suite. \\
(b) Cette suite est elle convergente?
\end{enumerate}
\end{exer}
	

\begin{exer}{\textbf{(Fonctions continues au sens de Césaro)}} \newline
\textit{On a vu en cours que dire que $f$ est continue sur $\mathbb{R}$ revient à dire que pour toute suite $(x_n)$, pour tout $a \in \mathbb{R}$ :
$$ \lim\limits_{n \to \infty} x_n = a \implies \lim\limits_{n \to \infty} f(x_n) = f(a) $$ (caractérisation séquentielle de la continuité) \newline \newline
On va étudier une réciproque avec une hypothèse légèrement différente dans cet exercice et montrer que les seules fonctions qui la vérifient sont beaucoup plus restreintes que les fonctions continues sur $\mathbb{R}$ !
}


Soit $f : \mathbb{R} \longrightarrow \mathbb{R}$. On dit que $f$ est \textbf{continue au sens de Cesaro} en $a$ si pour toute suite $(x_n)$ :
$$ \lim\limits_{n \to \infty} \dfrac{x_1+...+x_n}{n} = a \implies \lim\limits_{n \to \infty} \dfrac{f(x_1)+...+f(x_n)}{n} = f(a) $$
On veut déterminer toutes les fonctions continue au sens de Cesaro sur $\mathbb{R}$ : on raisonne par analyse-synthèse.
\begin{enumerate}
\item Soit $a \in \mathbb{R}$. 

\begin{enumerate}
\item Montrer que $f$ est continue au sens de Cesaro revient à dire $f-a$ l'est aussi. 
\item Montrer qu'on peut alors supposer que $f(0) = 0$
\end{enumerate}
\item On va maintenant utiliser une suite particulière pour se ramener à une équation fonctionnelle sur $f$ qu'on sait résoudre.
\begin{enumerate}
\item Montrer que si $f$ est continue au sens de Cesaro sur $\mathbb{R}$, alors :
$$\forall (x,y) \in \mathbb{R}^2, f \left( \dfrac{x+y}{2} \right) = \dfrac{f(x) + f(y)}{2}$$ \newline
\textit{Indication :} penser à une suite sympa
\item En supposant $f(0) = 0$, montrer que $f \left( \dfrac{x}{2} \right) = \dfrac{f(x)}{2}$, puis que $f(x+y) = f(x) + f(y)$
\end{enumerate}
\item 
\begin{enumerate}
\item Montrer que $\forall r \in \mathbb{Q}$, $f(rx) = rf(x)$. 
\item Redémontrer que pour tout réel $x$, il existe une suite $(x_n) \in \mathbb{Q}^{\mathbb{N}}$ tel que $\lim\limits_{n \to \infty} x_n = x$
\item En déduire que $\forall r \in \mathbb{R}$, $f(rx) = rf(x)$
\end{enumerate}
\item Conclure
\end{enumerate}
\end{exer}







\begin{exer}\textbf{(Les puissances d'une matrice triangulaire supérieure dont les coefficients diagonaux sont de module complexe inférieur à 1 tendent vers 0)}\\
\textit{L'objectif de l'exercice est de montrer que les puissances d'une matrice triangulaire supérieure à coefficients complexes de module strictement inférieurs à 1 tendent vers 0 en l'infini. \\
Une première question intéressante à se poser : que cela signifie t-il qu'une suite de matrices tend vers 0? On a défini la convergence pour une suite de réels ou à la rigueur de complexes au chapitre 7, mais jamais pour une suite de matrices. \\
Ici, on dira que la suite de matrices tend vers $0$ si et seulement si la suite des coefficients tendent tous vers 0 en l'infini.
}
 \newline
Autrement dit, si $T = (t_{i,j})_{1 \leq i,j \leq n}$ avec $t_{i,j} = 0$ pour $i>j$.
et $|t_{i,i}| < 1$ pour tout $i \in [|1,n|]$, $\lim\limits_{p \to \infty} T^p = 0$. \newline
\textbf{Notation :} on notera $(t_{i,j}^{(p)})_{1 \leq i,j \leq n}$ les coefficients de la matrice $T^p$. 
On va montrer que tous les coefficients tendent vers 0, en procédant par récurrence selon la "distance" à la diagonale. Attention à ne pas confondre avec $t_{i,i}^n$ \\ \\
Les questions 2 et 3 traite le cas des coefficients diagonaux et ceux au-dessus de la diagonale (initialisation). La question 4 correspond à l'hérédité.
\begin{enumerate}
\item \textbf{Préliminaires}
\begin{enumerate}
\item Soit $T=(t_{i,j})$ et $S=(s_{i,j})$ deux matrices triangulaires supérieures. Montrer que $TS$ est triangulaire supérieure. En déduire que $T^k$ est triangulaire supérieure.
\end{enumerate}
\item \textbf{Coefficients diagonaux : les $t_{i,i}^{(p)}$} \\
Montrer que $\lim\limits_{n \to \infty} t_{i,i}^{(n)}  = 0$ \\ \\
\item \textbf{Coefficients juste au-dessus de la diagonale :}
\begin{enumerate}
\item Calculer $t_{i,i+1}^{(2)}$, puis montrer que :
$$ t_{i,i+1}^{(p+1)} = t_{i,i+1}^{(p)} t_{i,i} + t_{i+1,i+1}^{p} t_{i,i+1}$$
\item Soit $(u_n)_{n \geq 1}$ une suite récurrente définie par : $u_{n+1} = a u_n + b_n$ avec $a \in \mathbb{R}$ et $(b_n)$ une suite réelle. Montrer que $$u_n = u_1 a^{n-1} + \displaystyle\sum_{k=1}^{n-1}{b_k a^{n-1-k}}$$ pour $n \geq 1$
\item Trouver une expression simplifiée de $t_{i,i+1}^{(p)}$ en utilisant la formule précédente, puis montrer que cette suite tend vers $0$ \\
\end{enumerate}
\item \textbf{Cas général :}
Soit $k \in [|1,n-1|]$.
On suppose que  $\lim\limits_{n \to \infty} t_{i,i+l}^{(p)} = 0$ pour tout $0 \leq l \leq k-1$ $i \in [|1,n-l|]$, . On veut montrer que $\lim\limits_{n \to \infty} t_{i,i+k}^{(p)} = 0$ pour $i \in [|1,n-k|]$,
\begin{enumerate}
\item Montrer 
%$$ t_{i,i+l}^{(p+1)} = t_{i,i+l}^{(p)} t_{i,i} + \displaystyle\sum_{k=1}^{l-1}{t_{i,i+k}^{(p)} t_{i+k,i+l}$$
\item Soit $x \in ]-1,1[$
\begin{enumerate}
\item Montrer que pour tout $l \in \mathbb{N}, \displaystyle\sum_{k=0}^{l}{|x|^k} \leq \dfrac{1}{1-|x|}$
\item Soit $(u_n)$ une suite qui tend vers $0$. Montrer que la suite $(w_n)$ définie par :
$$ \forall n \in \mathbb{N}, w_n = \displaystyle\sum_{k=0}^{n}{u_k x^{n-k}}$$ tend vers $0$
\end{enumerate}
\item En utilisant la question 3(b), les questions 4(a) et 4(b), conclure que $\lim\limits_{p \to \infty} t_{i,i+l}^{(p)} = 0$.
\end{enumerate}
\item Conclure
\end{enumerate}
%\end{enumerate}
\end{exer}



	
%\end{enumerate}

\begin{exer}{\textbf{Opérateurs à noyau}}
\textit{On étudie dans cet exercice une propriété des opérateurs à noyaux. Ca sert notamment en machine learning pour approximer des fonctions de décision. Vous pouvez taper "Méthodes à noyau" sur Google (après le DS évidemment !)}
Soient $K \in \mathcal{C}^0\left([0,1]^2, \mathbf{R}^{+^*}\right), f$ et $g$ dans $\mathcal{C}^0\left([0,1], \mathbf{R}^{+^*}\right)$ vérifiant, pour tout $x$ de $[0,1]: \int_0^1 K(x, y) f(y) d y=g(x), \int_0^1 K(x, y) g(y) d y=f(x)$. 
L'objectif du problème est de montrer que $f = g$ \\
(\textit{Remarque} : il y a ci-dessus une fonction continue à 2 variables, ce qu'on a pas encore vu en cours. Pour ce problème, vous pouvez simplement considérer que la fonction est continue comme fonction de $x$ à $y$ fixé et continue comme fonction de $y$ à $x$ fixé).

On note, si $h \in \mathcal{C}^0([0,1], \mathbf{R})$ :
$$
T : h \mapsto \left(T(h) :x \in[0,1] \quad \mapsto \quad T(h)(x) =\int_0^1 K(x, y) h(y) d y \right)
$$
$T$ associe donc à une fonction continue $h$ une fonction $T(h)$ dont on peut montrer qu'elle est continue.
On a donc $T(f)=g$ et $T(g)=f$.
\begin{enumerate}
\item \textbf{Préliminaires :}
\begin{enumerate}
\item Montrer que si $h_1$ et $h_2 \in \mathcal{C}^0\left([0,1], \mathbf{R}^{+^*}\right)$, on a $T(h_1 + \lambda h_2) = T (h_1) + \lambda T(h_2)$ (on dit que $T$ est linéaire, on reverra ça pour l'algèbre linéaire) \newline
\item L'objectif de cette question est de montrer que $T$ est un opérateur strictement positif, autrement dit que si $h \geq 0$, $T(h) > 0$ 
\begin{enumerate}
\item Montrer que si $h \in \mathcal{C}^0\left([0,1], \mathbf{R}^{+}\right), h \neq 0$, il existe un intervalle $A$ sur lequel $h > 0$
\item Conclure \\
\textit{Indication :} pour une fonction $f$ positive, $A$ un intervalle de $[0,1]$, on a $\displaystyle\int_0^1 f(y) d y \geq \displaystyle\int_{A} f(y) d y$
\end{enumerate}

\end{enumerate}
\textbf{L'objectif des questions 2 et 3 est de se ramener à une fonction positive et de montrer une contradiction avec le caractère strictement positif de T}.
\item 
Montrer qu'il existe $r>0$ et $x_0 \in [0,1]$ tel que $f - rg \geq 0$ et $f(x_0) - r g(x_0) = 0$. \\
\textit{Indication :} on pourra appliquer le théorème des bornes atteintes (le TBA...) à $\dfrac{f}{g}$. \\
\item Dans cette question, on suppose par l'absurde que $f-r g$ n'est pas identiquement nulle
\begin{enumerate}

\item Montrer que si $x \in[0,1]$ , $T(f-r g)(x) > 0$ puis $T^2(f-r g)(x)>0$ ($T^2 = T \circ T$)
\item En déduire enfin que $f - rg > 0$ et conclure à une absurdité
\end{enumerate}
\item On a donc montré que $f = rg$. Montrer que $r^2 = 1$, puis que $r=1$
Conclure que $f=g$
\end{enumerate}
\end{exer} 

%\end{exer}


\end{document}
