\documentclass[11pt]{article}

 %Configuration de la feuille 
 
\usepackage{amsmath,amssymb,enumerate,graphicx,pgf,tikz,fancyhdr}
\usepackage[utf8]{inputenc}
\usetikzlibrary{arrows}
\usepackage{geometry}
\usepackage{tabvar}
\geometry{hmargin=2.2cm,vmargin=1.5cm}\pagestyle{fancy}
\lfoot{\bfseries http://www.bibmath.net}
\rfoot{\bfseries\thepage}
\cfoot{}
\renewcommand{\footrulewidth}{0.5pt} %Filet en bas de page

 %Macros utilisées dans la base de données d'exercices 

\newcommand{\mtn}{\mathbb{N}}
\newcommand{\mtns}{\mathbb{N}^*}
\newcommand{\mtz}{\mathbb{Z}}
\newcommand{\mtr}{\mathbb{R}}
\newcommand{\mtk}{\mathbb{K}}
\newcommand{\mtq}{\mathbb{Q}}
\newcommand{\mtc}{\mathbb{C}}
\newcommand{\mch}{\mathcal{H}}
\newcommand{\mcp}{\mathcal{P}}
\newcommand{\mcb}{\mathcal{B}}
\newcommand{\mcl}{\mathcal{L}}
\newcommand{\mcm}{\mathcal{M}}
\newcommand{\mcc}{\mathcal{C}}
\newcommand{\mcmn}{\mathcal{M}}
\newcommand{\mcmnr}{\mathcal{M}_n(\mtr)}
\newcommand{\mcmnk}{\mathcal{M}_n(\mtk)}
\newcommand{\mcsn}{\mathcal{S}_n}
\newcommand{\mcs}{\mathcal{S}}
\newcommand{\mcd}{\mathcal{D}}
\newcommand{\mcsns}{\mathcal{S}_n^{++}}
\newcommand{\glnk}{GL_n(\mtk)}
\newcommand{\mnr}{\mathcal{M}_n(\mtr)}
\DeclareMathOperator{\ch}{ch}
\DeclareMathOperator{\sh}{sh}
\DeclareMathOperator{\vect}{vect}
\DeclareMathOperator{\card}{card}
\DeclareMathOperator{\comat}{comat}
\DeclareMathOperator{\imv}{Im}
\DeclareMathOperator{\rang}{rg}
\DeclareMathOperator{\Fr}{Fr}
\DeclareMathOperator{\diam}{diam}
\DeclareMathOperator{\supp}{supp}
\newcommand{\veps}{\varepsilon}
\newcommand{\mcu}{\mathcal{U}}
\newcommand{\mcun}{\mcu_n}
\newcommand{\dis}{\displaystyle}
\newcommand{\croouv}{[\![}
\newcommand{\crofer}{]\!]}
\newcommand{\rab}{\mathcal{R}(a,b)}
\newcommand{\pss}[2]{\langle #1,#2\rangle}
 %Document 

\begin{document} 

\begin{center}\textsc{{\huge }}\end{center}

% Exercice 1047


\vskip0.3cm\noindent\textsc{Exercice 1} - Trouver une base orthonormale
\vskip0.2cm
Déterminer une base orthonormale de $\mathbb R_2[X]$ muni du produit scalaire 
$$\langle P,Q\rangle=\int_{-1}^1 P(t)Q(t)dt.$$


% Exercice 2615


\vskip0.3cm\noindent\textsc{Exercice 2} - Projection dans un espace de matrices
\vskip0.2cm
Soit $E=\mathcal M_2(\mathbb R)$ que l'on munit du produit scalaire
$$\langle M,N\rangle=\textrm{Tr}(M^TN).$$
On pose $F=\left\{\begin{pmatrix}
                         a&b\\
                         -b&a
                        \end{pmatrix};\ (a,b)\in\mathbb R^2\right\}.$
\begin{enumerate}
 \item Déterminer une base orthonormée de $F^\perp$.
 \item Calculer la projection de $J=\begin{pmatrix}1&1\\
                                     1&1
                                    \end{pmatrix}$
sur $F^\perp$.
\item Calculer la distance de $J$ à $F.$
\end{enumerate}


% Exercice 1714


\vskip0.3cm\noindent\textsc{Exercice 3} - Méthode des moindres carrés
\vskip0.2cm
Soit $n$ et $p$ deux entiers naturels avec $p\leq n$. On munit $\mathbb R^n$ du produit scalaire canonique et on identifie $\mathbb R^n$ avec $\mathcal M_{n,1}(\mathbb R)$. On considère une matrice $A\in\mathcal M_{n,p}(\mathbb R)$ de rang $p$ et $B\in\mathcal M_{n,1}(\mathbb R)$. 
\begin{enumerate}
\item Démontrer qu'il existe une unique matrice $X_0$ de $\mathcal M_{p,1}(\mathbb R)$ telle que 
$$\|AX_0-B\|=\inf\{\|AX-B\|;\ X\in\mathcal M_{p,1}(\mathbb R)\}.$$
\item Montrer que $X_0$ est l'unique solution de 
$$A^T AX=A^T B.$$
\item Application : déterminer 
$$\inf\{(x+y-1)^2+(x-y)^2+(2x+y+2)^2;\ (x,y)\in\mathbb R^2\}.$$
\end{enumerate}




\vskip0.5cm
\noindent{\small Cette feuille d'exercices a été conçue à l'aide du site \textsf{https://www.bibmath.net}}

%Vous avez accès aux corrigés de cette feuille par l'url : https://www.bibmath.net/ressources/justeunefeuille.php?id=29269
\end{document}