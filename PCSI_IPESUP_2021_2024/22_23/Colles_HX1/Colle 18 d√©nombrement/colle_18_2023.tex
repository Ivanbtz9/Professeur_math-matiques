\documentclass[a4paper,11pt]{article}

\usepackage{inputenc}
\usepackage[T1]{fontenc}
\usepackage[frenchb]{babel}
\usepackage{fancyhdr,fancybox} % pour personnaliser les en-têtes
\usepackage{lastpage,setspace}
\usepackage{amsfonts,amssymb,amsmath,amsthm,mathrsfs}
\usepackage{relsize,exscale,bbold}
\usepackage{paralist}
\usepackage{xspace,multicol,diagbox,array}
\usepackage{xcolor}
\usepackage{variations}
\usepackage{xypic}
\usepackage{eurosym,stmaryrd}
\usepackage{graphicx}
\usepackage[np]{numprint}
\usepackage{hyperref} 
\usepackage{tikz}
\usepackage{colortbl}
\usepackage{multirow}
\usepackage{MnSymbol,wasysym}
\usepackage[top=1.5cm,bottom=1.5cm,right=1.2cm,left=1.5cm]{geometry}
\usetikzlibrary{calc, arrows, plotmarks, babel,decorations.pathreplacing}
\setstretch{1.25}
%\usepackage{lipsum} %\usepackage{enumitem} %\setlist[enumerate]{itemsep=1mm} bug avec enumerate



\newtheorem{thm}{Théorème}
\newtheorem{rmq}{Remarque}
\newtheorem{prop}{Propriété}
\newtheorem{cor}{Corollaire}
\newtheorem{lem}{Lemme}
\newtheorem{prop-def}{Propriété-définition}

\theoremstyle{definition}

\newtheorem{defi}{Définition}
\newtheorem{ex}{Exemple}
\newtheorem*{rap}{Rappel}
\newtheorem{cex}{Contre-exemple}
\newtheorem{exo}{Exercice} % \large {\fontfamily{ptm}\selectfont EXERCICE}
\newtheorem{nota}{Notation}
\newtheorem{ax}{Axiome}
\newtheorem{appl}{Application}
\newtheorem{csq}{Conséquence}
\def\di{\displaystyle}



\renewcommand{\thesection}{\Roman{section}}\renewcommand{\thesubsection}{\arabic{subsection} }\renewcommand{\thesubsubsection}{\alph{subsubsection} }


\newcommand{\bas}{~\backslash}\newcommand{\ba}{\backslash}
\newcommand{\C}{\mathbb{C}}\newcommand{\R}{\mathbb{R}}\newcommand{\K}{\mathbb{K}}\newcommand{\Q}{\mathbb{Q}}\newcommand{\Z}{\mathbb{Z}}\newcommand{\N}{\mathbb{N}}\newcommand{\V}{\overrightarrow}\newcommand{\Cs}{\mathscr{C}}\newcommand{\Ps}{\mathscr{P}}\newcommand{\Rs}{\mathscr{R}}\newcommand{\Gs}{\mathscr{G}}\newcommand{\Ds}{\mathscr{D}}\newcommand{\happy}{\huge\smiley}\newcommand{\sad}{\huge\frownie}\newcommand{\danger}{\begin{tikzpicture}[x=1.5pt,y=1.5pt,rotate=-14.2]
	\definecolor{myred}{rgb}{1,0.215686,0}
	\draw[line width=0.1pt,fill=myred] (13.074200,4.937500)--(5.085940,14.085900)..controls (5.085940,14.085900) and (4.070310,15.429700)..(3.636720,13.773400)
	..controls (3.203130,12.113300) and (0.917969,2.382810)..(0.917969,2.382810)
	..controls (0.917969,2.382810) and (0.621094,0.992188)..(2.097660,1.359380)
	..controls (3.574220,1.726560) and (12.468800,3.984380)..(12.468800,3.984380)
	..controls (12.468800,3.984380) and (13.437500,4.132810)..(13.074200,4.937500)
	--cycle;
	\draw[line width=0.1pt,fill=white] (11.078100,5.511720)--(5.406250,11.875000)..controls (5.406250,11.875000) and (4.683590,12.812500)..(4.367190,11.648400)
	..controls (4.050780,10.488300) and (2.375000,3.675780)..(2.375000,3.675780)
	..controls (2.375000,3.675780) and (2.156250,2.703130)..(3.214840,2.964840)
	..controls (4.273440,3.230470) and (10.640600,4.847660)..(10.640600,4.847660)
	..controls (10.640600,4.847660) and (11.332000,4.953130)..(11.078100,5.511720)
	--cycle;
	\fill (6.144520,8.839900)..controls (6.460940,7.558590) and (6.464840,6.457090)..(6.152340,6.378910)
	..controls (5.835930,6.300840) and (5.320300,7.277400)..(5.003900,8.554750)
	..controls (4.683590,9.835940) and (4.679690,10.941400)..(4.996090,11.019600)
	..controls (5.312490,11.097700) and (5.824210,10.121100)..(6.144520,8.839900)
	--cycle;
	\fill (7.292960,5.261780)..controls (7.382800,4.898500) and (7.128900,4.523500)..(6.730460,4.421880)
	..controls (6.328120,4.324220) and (5.929680,4.535220)..(5.835930,4.898500)
	..controls (5.746080,5.261780) and (5.999990,5.640630)..(6.402340,5.738340)
	..controls (6.804690,5.839840) and (7.203110,5.625060)..(7.292960,5.261780)
	--cycle;
	\end{tikzpicture}}\newcommand{\alors}{\Large\Rightarrow}\newcommand{\equi}{\Leftrightarrow}
\newcommand{\fonction}[5]{\begin{array}{l|rcl}
		#1: & #2 & \longrightarrow & #3 \\
		& #4 & \longmapsto & #5 \end{array}}


\definecolor{vert}{RGB}{11,160,78}
\definecolor{rouge}{RGB}{255,120,120}
\definecolor{bleu}{RGB}{15,5,107}


\pagestyle{fancy}
\lhead{Optimal Sup Spé, groupe IPESUP}\chead{Année~2022-2023}\rhead{Niveau: Première année de PCSI }\lfoot{M. Botcazou}\cfoot{\thepage}\rfoot{mail: i.botcazou@gmx.fr }\renewcommand{\headrulewidth}{0.4pt}\renewcommand{\footrulewidth}{0.4pt}

\begin{document}
	
	
	\begin{center}
		\Large \sc colle 18 = Dénombrement
	\end{center}
\raggedright


\section*{Connaître son cours:}
\begin{enumerate}
	\item Montrer que toute partie de l'ensemble $\llbracket1,n\rrbracket$ pour $n\in\N^*$ est finie.
	\item Soit $p , q \in \N^*,$ montrer qu'il existe une injection de $\llbracket1,p\rrbracket$ dans $\llbracket1,q\rrbracket$ si, et seulement si, $p \leq q $.
	\item Soit $E$ et $F$ deux ensembles finis. Alors, $E \cup F$ est fini et $\displaystyle |E \cup F | = |E | + |F | - |E \cap F |$.
\end{enumerate}

\section*{Exercices:} 	


\begin{exo}\textbf{(*)}\quad\\[0.2cm]
	Combien les mots suivants possèdent-ils d’anagrammes ?
	$$1.\ \text{"ABRACADABRA".} \quad \quad \quad \quad 2.\ \text{"LIPSCHITZIENNE".}$$
	
	
	\centering\rule{1\linewidth}{0.6pt}\end{exo}
	
\begin{exo}\textbf{(**)}\quad\textit{(Nombres de Bell)}\\[0.25cm]%https://progresser-en-maths.com/exercice-corrige-nombres-de-bell/
	Soit $E_{n}$ un ensemble fini de cardinal $n \geq 1$. On appelle partition de $E_{n}$, tout ensemble de parties de $E_{n}$ non vides, deux à deux disjointes, et dont la réunion est égale à $E_{n}$. On note $B_{n}$ le nombre de partitions de $E_{n}$ et on convient que $B_{0}=1$. Les $B_{n}, n \in \mathbb{N}$ sont appelés \emph{nombres de Bell}.
	
	1. Calculer $B_{1}, B_{2}$ et $B_{3}$
	
	2. Etablir la relation, dite d'Aitken, $\displaystyle\forall n \in \mathbb{N}, B_{n+1}=\sum_{i=0}^{n}\left(\begin{array}{c}n \\ i\end{array}\right) B_{i}$
	
	3. Démontrer par récurrence forte que $\displaystyle\forall n \in \mathbb{N}, B_{n}=\sum_{i=0}^{+\infty} e^{-1} \frac{i^{n}}{i !}$
	
	\centering
	\rule{1\linewidth}{0.6pt}
\end{exo}

	


\begin{exo}\textbf{(*)}\quad\\[0.25cm]
	On appelle diagonale d’un
	polygone convexe tout segment joignant deux de ses sommets non
	consécutifs. Si un polygone possède autant de diagonales que de côtés, combien possède-t-il de côtés ?
	
	\centering
\rule{1\linewidth}{0.6pt}
\end{exo}

\begin{exo}\textbf{(***)} \quad\textit{(Nombres de Stirling de première espèce)}\\[0.25cm%https://www.bibmath.net/ressources/index.php?action=affiche&quoi=bde/proba/denombrement-theo&type=fexo]
Soit $n$ et $k$ deux entiers strictement positifs. 
\begin{enumerate}
	\item Montrer qu’il n’existe qu’un nombre fini de partitions de l’ensemble $\{1,\dots,n\}$ en $k$ parties. Dans la suite, on notera $S(n,k)$ le nombre de ces partitions. On pose de plus $S(0,0)=1$ et $S(n,0)=S(0,k)=0$.
	\item Que vaut $S(n,k)$ pour $k>n$?
	\item Que vaut $S(n,1)$?
	\item Démontrer que $S(n,k)=S(n-1,k-1)+kS(n-1,k)$.
	\item Rédiger une fonction récursive Python permettant de calculer $S(n,k)$.
\end{enumerate}

	\centering
	\rule{1\linewidth}{0.6pt}
\end{exo}











\end{document}
