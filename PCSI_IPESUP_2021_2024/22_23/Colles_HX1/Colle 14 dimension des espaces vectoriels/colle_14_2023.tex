\documentclass[a4paper,11pt]{article}

\usepackage{inputenc}
\usepackage[T1]{fontenc}
\usepackage[frenchb]{babel}
\usepackage{fancyhdr,fancybox} % pour personnaliser les en-têtes
\usepackage{lastpage,setspace}
\usepackage{amsfonts,amssymb,amsmath,amsthm,mathrsfs}
\usepackage{relsize,exscale,bbold}
\usepackage{paralist}
\usepackage{xspace,multicol,diagbox,array}
\usepackage{xcolor}
\usepackage{variations}
\usepackage{xypic}
\usepackage{eurosym,stmaryrd}
\usepackage{graphicx}
\usepackage[np]{numprint}
\usepackage{hyperref} 
\usepackage{tikz}
\usepackage{colortbl}
\usepackage{multirow}
\usepackage{MnSymbol,wasysym}
\usepackage[top=1.5cm,bottom=1.5cm,right=1.2cm,left=1.5cm]{geometry}
\usetikzlibrary{calc, arrows, plotmarks, babel,decorations.pathreplacing}
\setstretch{1.25}
%\usepackage{lipsum} %\usepackage{enumitem} %\setlist[enumerate]{itemsep=1mm} bug avec enumerate



\newtheorem{thm}{Théorème}
\newtheorem{rmq}{Remarque}
\newtheorem{prop}{Propriété}
\newtheorem{cor}{Corollaire}
\newtheorem{lem}{Lemme}
\newtheorem{prop-def}{Propriété-définition}

\theoremstyle{definition}

\newtheorem{defi}{Définition}
\newtheorem{ex}{Exemple}
\newtheorem*{rap}{Rappel}
\newtheorem{cex}{Contre-exemple}
\newtheorem{exo}{Exercice} % \large {\fontfamily{ptm}\selectfont EXERCICE}
\newtheorem{nota}{Notation}
\newtheorem{ax}{Axiome}
\newtheorem{appl}{Application}
\newtheorem{csq}{Conséquence}
\def\di{\displaystyle}



\renewcommand{\thesection}{\Roman{section}}\renewcommand{\thesubsection}{\arabic{subsection} }\renewcommand{\thesubsubsection}{\alph{subsubsection} }


\newcommand{\bas}{~\backslash}\newcommand{\ba}{\backslash}
\newcommand{\C}{\mathbb{C}}\newcommand{\R}{\mathbb{R}}\newcommand{\K}{\mathbb{K}}\newcommand{\Q}{\mathbb{Q}}\newcommand{\Z}{\mathbb{Z}}\newcommand{\N}{\mathbb{N}}\newcommand{\V}{\overrightarrow}\newcommand{\Cs}{\mathscr{C}}\newcommand{\Ps}{\mathscr{P}}\newcommand{\Rs}{\mathscr{R}}\newcommand{\Gs}{\mathscr{G}}\newcommand{\Ds}{\mathscr{D}}\newcommand{\happy}{\huge\smiley}\newcommand{\sad}{\huge\frownie}\newcommand{\danger}{\begin{tikzpicture}[x=1.5pt,y=1.5pt,rotate=-14.2]
	\definecolor{myred}{rgb}{1,0.215686,0}
	\draw[line width=0.1pt,fill=myred] (13.074200,4.937500)--(5.085940,14.085900)..controls (5.085940,14.085900) and (4.070310,15.429700)..(3.636720,13.773400)
	..controls (3.203130,12.113300) and (0.917969,2.382810)..(0.917969,2.382810)
	..controls (0.917969,2.382810) and (0.621094,0.992188)..(2.097660,1.359380)
	..controls (3.574220,1.726560) and (12.468800,3.984380)..(12.468800,3.984380)
	..controls (12.468800,3.984380) and (13.437500,4.132810)..(13.074200,4.937500)
	--cycle;
	\draw[line width=0.1pt,fill=white] (11.078100,5.511720)--(5.406250,11.875000)..controls (5.406250,11.875000) and (4.683590,12.812500)..(4.367190,11.648400)
	..controls (4.050780,10.488300) and (2.375000,3.675780)..(2.375000,3.675780)
	..controls (2.375000,3.675780) and (2.156250,2.703130)..(3.214840,2.964840)
	..controls (4.273440,3.230470) and (10.640600,4.847660)..(10.640600,4.847660)
	..controls (10.640600,4.847660) and (11.332000,4.953130)..(11.078100,5.511720)
	--cycle;
	\fill (6.144520,8.839900)..controls (6.460940,7.558590) and (6.464840,6.457090)..(6.152340,6.378910)
	..controls (5.835930,6.300840) and (5.320300,7.277400)..(5.003900,8.554750)
	..controls (4.683590,9.835940) and (4.679690,10.941400)..(4.996090,11.019600)
	..controls (5.312490,11.097700) and (5.824210,10.121100)..(6.144520,8.839900)
	--cycle;
	\fill (7.292960,5.261780)..controls (7.382800,4.898500) and (7.128900,4.523500)..(6.730460,4.421880)
	..controls (6.328120,4.324220) and (5.929680,4.535220)..(5.835930,4.898500)
	..controls (5.746080,5.261780) and (5.999990,5.640630)..(6.402340,5.738340)
	..controls (6.804690,5.839840) and (7.203110,5.625060)..(7.292960,5.261780)
	--cycle;
	\end{tikzpicture}}\newcommand{\alors}{\Large\Rightarrow}\newcommand{\equi}{\Leftrightarrow}
\newcommand{\fonction}[5]{\begin{array}{l|rcl}
		#1: & #2 & \longrightarrow & #3 \\
		& #4 & \longmapsto & #5 \end{array}}


\definecolor{vert}{RGB}{11,160,78}
\definecolor{rouge}{RGB}{255,120,120}
\definecolor{bleu}{RGB}{15,5,107}


\pagestyle{fancy}
\lhead{Optimal Sup Spé, groupe IPESUP}\chead{Année~2022-2023}\rhead{Niveau: Première année de PCSI }\lfoot{M. Botcazou}\cfoot{\thepage}\rfoot{mail: i.botcazou@gmx.fr }\renewcommand{\headrulewidth}{0.4pt}\renewcommand{\footrulewidth}{0.4pt}

\begin{document}
	
	
	\begin{center}
		\Large \sc colle 14 = dimension des Espaces vectoriels
	\end{center}
\raggedright
%https://www.bibmath.net/ressources/justeunefeuille.php?id=27491

\section*{Connaître son cours:}
\begin{enumerate}
	\item Soit $(P_k)_{k \in\llbracket 1,n\rrbracket} $ une famille de polynômes non nuls échelonnée en degré, montrer que cette famille est libre. Donner un exemple de famille libre de polynômes qui n'est pas échelonnée en degré.
	\item Soit $F$ et $G$ deux sous-espaces supplémentaires d’un $\K$-espace vectoriel $E$. Montrer que la famille obtenue par
	concaténation d'une base de $F$ et d’une base de $G$ est une base de $E $.
	\item Montrer que deux $\K$-espaces vectoriels de dimension finie sont isomorphes si, et seulement s’ils ont la même dimension.
	
\end{enumerate}

\section*{Exercices:} 	

\begin{exo}\textbf{(**)}\quad\\[0.25cm]
Soit $E$ l’espace vectoriel des applications de $\mathbb R$ dans $\mathbb R$. On note $L:E\to E$ l’application qui à $f\in E$ associe $L(f)$ définie par $L(f):x\mapsto \dfrac{f(x)+ f(-x)}{2}$.
\begin{enumerate}
	\item Montrer que $L$ est un endomorphisme de E.
	\item Préciser le noyau et l’image de L.
	\item L’application $L$ est-elle injective ? surjective ?
	\item Montrer que l’application $L$ est une projection. 
\end{enumerate}	
	
	\centering
	\rule{1\linewidth}{0.6pt}
\end{exo}
	
\begin{exo}\textbf{(**)}\quad\\[0.25cm]
Soit $f$ l'endomorphisme de $\mathbb R^3$ tel que $f(x,y,z)=(-3x+2y-4z,2x+2z,4x-2y+5z)$. Montrer que $f$ est la projection sur un plan $P$ parallèlement à une droite $D$. Donner une équation cartésienne du plan $P$ et un vecteur directeur de $D$.	
	
	\centering
	\rule{1\linewidth}{0.6pt}
\end{exo}

	


\begin{exo}\textbf{(**)}\quad\\[0.25cm]%An114
Soit $E$ un espace vectoriel et $u,v\in\mathcal L(E)$. On suppose que $u\circ v=v\circ u$. 

Démontrer que $\textrm{ker}(u)$ et $\textrm{Im}(u)$ sont stables par $v$.%, c'est-à-dire que\quad\\[-0.4cm] $$v(\ker (u))\subset \ker (u)\textrm{ et }v(\textrm{Im}(u))\subset \textrm{Im}(u).$$\quad\\[-0.4cm]

	\centering
\rule{1\linewidth}{0.6pt}
\end{exo}

\begin{exo}\textbf{(**)}\quad\\[0.25cm]%An115
Soit $f\in\mathcal L(E)$ et soient $\alpha,\beta$ deux réels distincts.
\begin{enumerate}
	\item Démontrer que $E=\textrm{Im}(f-\alpha Id_E)+\textrm{Im}(f-\beta Id_E)$.\newline
	On suppose de plus que $\alpha$ et $\beta$ sont non nuls et que $$(f-\alpha Id_E)\circ (f-\beta Id_E)=0.$$
	\item Démontrer que $f$ est inversible, et calculer $f^{-1}$.
	\item Démontrer que $E=\ker(f-\alpha Id_E)\oplus \ker(f-\beta Id_E)$.
	\item Exprimer en fonction de $f$ le projecteur $p$ sur $\ker(f-\alpha Id_E)$ parallèlement
	à $\ker(f-\beta Id_E)$.
\end{enumerate}	
	
	\centering
	\rule{1\linewidth}{0.6pt}
\end{exo}


\newpage

\begin{exo}\textbf{(***)}\quad\\[0.25cm]%An115
	Soit $E$ un espace vectoriel de dimension finie $n$ et soit $f\in\mathcal L(E)$.
	\begin{enumerate}
		\item Soit $k\geq 1$. Démontrer que $\ker(f^{k})\subset \ker(f^{k+1})$ 
		et $\textrm{Im}(f^{k+1})\subset \textrm{Im}(f^k).$
		\item \begin{enumerate}
			\item Démontrer que si $\ker(f^k)=\ker(f^{k+1})$, alors $\ker(f^{k+1})= \ker(f^{k+2})$.
			\item Démontrer qu'il existe $p\in\mathbb N$ tel que
			\begin{itemize}
				\item si $k<p$, alors $\ker(f^k)\neq \ker(f^{k+1})$;
				\item si $k\geq p$, alors $\ker(f^k)= \ker(f^{k+1})$.
			\end{itemize}
			\item Démontrer que $p\leq n$;
		\end{enumerate}
		\item Démontrer que si $k<p$, alors $\textrm{Im}(f^k)\neq \textrm{Im}(f^{k+1})$ et
		si $k\geq p$, alors $\textrm{Im}(f^k)=\textrm{Im}(f^{k+1})$.
		\item Démontrer que $\ker(f^p)$ et $\textrm{Im}(f^p)$ sont supplémentaires.
		\item Démontrer qu'il existe deux sous-espaces $F$ et $G$ de $E$ tels que $F$ et $G$ sont supplémentaires, $f_{|F}$ est nilpotent et $f_{|G}$ induit un automorphisme de $G$.
		\item Soit $d_k=\dim\big(\textrm{Im}(f^k)\big)$. Montrer que la suite $(d_k-d_{k+1})$ est décroissante.
	\end{enumerate}
	
	\centering
	\rule{1\linewidth}{0.6pt}
\end{exo}

\begin{exo}\textbf{(***)}\quad\\[0.25cm]
\noindent\textbf{ Partie A - Exemple d'un projecteur}
 
 Notons $E=\mathbb{R}[X]$ l'ensemble des polynômes réels, $\mathscr{P}$ et $\mathscr{I}$ les sous-espaces vectoriels des polynômes pairs et impairs respectivement.
 \begin{enumerate}
 	\item Montrer que $\mathscr{I}$ est un supplémentaire de $\mathscr{P}$ dans $E$.
 	\item   Soit l'application linéaire
 	$$\varphi : \left\{\begin{array}{ccl}
 	E &\longrightarrow & E\\
 	P &\longmapsto & \dfrac{P(X)+P(-X)}{2} + X\dfrac{P(X)-P(-X)}{2}
 	\end{array}\right.$$
 	\begin{enumerate}
 		\item Déterminer $\operatorname{Im} \varphi$ puis établir que
 		
 		$$
 		\operatorname{Ker} \varphi=\{(1-X) P(X), P \in \mathscr{I}\} .
 		$$
 		\item  Montrer que $\varphi$ est un projecteur de $E$.
 		\item En déduire que $\operatorname{Ker} \varphi$ est un supplémentaire de $\mathscr{P}$.
 	\end{enumerate}
 
 \end{enumerate}
 
\noindent\textbf{Partie B - sous-espaces qui admettent un supplémentaire commun}
 
 Soit $E$ un espace vectoriel, $F_{1}$ et $F_{2}$ deux sous-espaces vectoriels de $E$
 
 \begin{enumerate}
 	\item Supposons, dans cette question, que $F_{1}$ et $F_{2}$ sont supplémentaires dans $E$ et qu'il existe un isomorphisme $u: F_{1} \rightarrow F_{2}$.
 	
 	Montrer que $G=\left\{x-u(x), x \in F_{1}\right\}$ est un espace vectoriel puis qu'il est un supplémentaire commun à $F_{1}$ et $F_{2}$.
 	\item Réciproquement supposons dans cette question que $F_{1}$ et $F_{2}$ admettent un supplémentaire commun $G$. Montrer que $F_{1}$ et $F_{2}$ sont isomorphes.
 	
 	
 \end{enumerate}
 	\centering
	\rule{1\linewidth}{0.6pt}
\end{exo}






\end{document}
