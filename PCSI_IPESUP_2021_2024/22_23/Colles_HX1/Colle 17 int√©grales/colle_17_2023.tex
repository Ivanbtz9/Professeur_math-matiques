\documentclass[a4paper,11pt]{article}

\usepackage{inputenc}
\usepackage[T1]{fontenc}
\usepackage[frenchb]{babel}
\usepackage{fancyhdr,fancybox} % pour personnaliser les en-têtes
\usepackage{lastpage,setspace}
\usepackage{amsfonts,amssymb,amsmath,amsthm,mathrsfs}
\usepackage{relsize,exscale,bbold}
\usepackage{paralist}
\usepackage{xspace,multicol,diagbox,array}
\usepackage{xcolor}
\usepackage{variations}
\usepackage{xypic}
\usepackage{eurosym,stmaryrd}
\usepackage{graphicx}
\usepackage[np]{numprint}
\usepackage{hyperref} 
\usepackage{tikz}
\usepackage{colortbl}
\usepackage{multirow}
\usepackage{MnSymbol,wasysym}
\usepackage[top=1.5cm,bottom=1.5cm,right=1.2cm,left=1.5cm]{geometry}
\usetikzlibrary{calc, arrows, plotmarks, babel,decorations.pathreplacing}
\setstretch{1.25}
%\usepackage{lipsum} %\usepackage{enumitem} %\setlist[enumerate]{itemsep=1mm} bug avec enumerate



\newtheorem{thm}{Théorème}
\newtheorem{rmq}{Remarque}
\newtheorem{prop}{Propriété}
\newtheorem{cor}{Corollaire}
\newtheorem{lem}{Lemme}
\newtheorem{prop-def}{Propriété-définition}

\theoremstyle{definition}

\newtheorem{defi}{Définition}
\newtheorem{ex}{Exemple}
\newtheorem*{rap}{Rappel}
\newtheorem{cex}{Contre-exemple}
\newtheorem{exo}{Exercice} % \large {\fontfamily{ptm}\selectfont EXERCICE}
\newtheorem{nota}{Notation}
\newtheorem{ax}{Axiome}
\newtheorem{appl}{Application}
\newtheorem{csq}{Conséquence}
\def\di{\displaystyle}



\renewcommand{\thesection}{\Roman{section}}\renewcommand{\thesubsection}{\arabic{subsection} }\renewcommand{\thesubsubsection}{\alph{subsubsection} }


\newcommand{\bas}{~\backslash}\newcommand{\ba}{\backslash}
\newcommand{\C}{\mathbb{C}}\newcommand{\R}{\mathbb{R}}\newcommand{\K}{\mathbb{K}}\newcommand{\Q}{\mathbb{Q}}\newcommand{\Z}{\mathbb{Z}}\newcommand{\N}{\mathbb{N}}\newcommand{\V}{\overrightarrow}\newcommand{\Cs}{\mathscr{C}}\newcommand{\Ps}{\mathscr{P}}\newcommand{\Rs}{\mathscr{R}}\newcommand{\Gs}{\mathscr{G}}\newcommand{\Ds}{\mathscr{D}}\newcommand{\happy}{\huge\smiley}\newcommand{\sad}{\huge\frownie}\newcommand{\danger}{\begin{tikzpicture}[x=1.5pt,y=1.5pt,rotate=-14.2]
	\definecolor{myred}{rgb}{1,0.215686,0}
	\draw[line width=0.1pt,fill=myred] (13.074200,4.937500)--(5.085940,14.085900)..controls (5.085940,14.085900) and (4.070310,15.429700)..(3.636720,13.773400)
	..controls (3.203130,12.113300) and (0.917969,2.382810)..(0.917969,2.382810)
	..controls (0.917969,2.382810) and (0.621094,0.992188)..(2.097660,1.359380)
	..controls (3.574220,1.726560) and (12.468800,3.984380)..(12.468800,3.984380)
	..controls (12.468800,3.984380) and (13.437500,4.132810)..(13.074200,4.937500)
	--cycle;
	\draw[line width=0.1pt,fill=white] (11.078100,5.511720)--(5.406250,11.875000)..controls (5.406250,11.875000) and (4.683590,12.812500)..(4.367190,11.648400)
	..controls (4.050780,10.488300) and (2.375000,3.675780)..(2.375000,3.675780)
	..controls (2.375000,3.675780) and (2.156250,2.703130)..(3.214840,2.964840)
	..controls (4.273440,3.230470) and (10.640600,4.847660)..(10.640600,4.847660)
	..controls (10.640600,4.847660) and (11.332000,4.953130)..(11.078100,5.511720)
	--cycle;
	\fill (6.144520,8.839900)..controls (6.460940,7.558590) and (6.464840,6.457090)..(6.152340,6.378910)
	..controls (5.835930,6.300840) and (5.320300,7.277400)..(5.003900,8.554750)
	..controls (4.683590,9.835940) and (4.679690,10.941400)..(4.996090,11.019600)
	..controls (5.312490,11.097700) and (5.824210,10.121100)..(6.144520,8.839900)
	--cycle;
	\fill (7.292960,5.261780)..controls (7.382800,4.898500) and (7.128900,4.523500)..(6.730460,4.421880)
	..controls (6.328120,4.324220) and (5.929680,4.535220)..(5.835930,4.898500)
	..controls (5.746080,5.261780) and (5.999990,5.640630)..(6.402340,5.738340)
	..controls (6.804690,5.839840) and (7.203110,5.625060)..(7.292960,5.261780)
	--cycle;
	\end{tikzpicture}}\newcommand{\alors}{\Large\Rightarrow}\newcommand{\equi}{\Leftrightarrow}
\newcommand{\fonction}[5]{\begin{array}{l|rcl}
		#1: & #2 & \longrightarrow & #3 \\
		& #4 & \longmapsto & #5 \end{array}}


\definecolor{vert}{RGB}{11,160,78}
\definecolor{rouge}{RGB}{255,120,120}
\definecolor{bleu}{RGB}{15,5,107}


\pagestyle{fancy}
\lhead{Optimal Sup Spé, groupe IPESUP}\chead{Année~2022-2023}\rhead{Niveau: Première année de PCSI }\lfoot{M. Botcazou}\cfoot{\thepage}\rfoot{mail: i.botcazou@gmx.fr }\renewcommand{\headrulewidth}{0.4pt}\renewcommand{\footrulewidth}{0.4pt}

\begin{document}
	
	
	\begin{center}
		\Large \sc colle 17 = Calcul intégral
	\end{center}
\raggedright


\section*{Connaître son cours:}
\begin{enumerate}
	\item Soit $f : [a , b ] \to \R$ continue, monter qu’il existe $c \in [a , b ]$ tel que
	$\displaystyle f(c) = \dfrac{1}{b-a} \int_{a}^{b} f(t)dt$. 
	\item Soit $f$ une fonction continue sur $[a , b ] $, positive et non nulle en au moins un point de $[a , b ] $.
	\item Soit $f$ une fonction de classe $\mathcal C^{n+1}$ sur $I$ un intervalle réel et $a \in I $. Donner la formule de Taylor avec reste intégral en $a$.
\end{enumerate}

\section*{Exercices:} 	


\begin{exo}\textbf{(**)}\quad\\[0.2cm]
	Soit $$\displaystyle I_{n} = \int_0^1 \frac{x^n}{1 + x}d x$$
	\begin{enumerate}
		\item Montrer que $\displaystyle \lim_{n\to +\infty} I_{n}=0$.
		\item  Calculer $\displaystyle I_n + I_{n + 1}$.
		\item D\'eterminer $\displaystyle \lim_{n \rightarrow  + \infty} \left(\sum_{k = 1}^n \frac{(-1)^{k + 1}}k\right)$
	\end{enumerate}	
	
	\centering\rule{1\linewidth}{0.6pt}\end{exo}
	
\begin{exo}\textbf{(***)}\quad\\[0.25cm]
	
	Soit $n\in\N$, déterminer la borne inférieure de la partie de $\R$ définie par
	
	$$\displaystyle E= \left\{c \in \R, \  \forall f \in  \mathscr{C}^0 ([0, 1], \R^+ ), \ 
	\int_{0}^{1}f (\sqrt[n]{t}) dt \ \leq \ c \int_{0}^{1}f (t) dt\right\}.$$
	
	\centering
	\rule{1\linewidth}{0.6pt}
\end{exo}

	


\begin{exo}\textbf{(**)}\quad\\[0.25cm]%An114
Considérons les intégrales
$$\displaystyle I = \int_0^{\dfrac{\pi}{2}} \frac{\sin(x)}{\sqrt{1 + 2\cos(x)\sin(x)}}d x \quad \quad J = \int_0^{\dfrac{\pi}{2}} \frac{\cos(x)}{\sqrt{1 + 2\cos(x)\sin(x)}}d x$$

\begin{enumerate}
	\item Calculer $I + J$
	\item Montrer que $I = J$ par un judicieux changement de variables.
	\item En déduire la valeur de $I $.
\end{enumerate}
	\centering
\rule{1\linewidth}{0.6pt}
\end{exo}

\begin{exo}\textbf{(***)}\quad\\[0.25cm]%An115
	Soit $x_n$ l’argument du premier maximum local de $\displaystyle f_n : x \mapsto \sum_{k=1}^{n} \frac{\sin(kx)}{k}$ sur $\R^+$. Montrer que $$\displaystyle f_n(x_n) \to \int_{0}^{\pi}\dfrac{\sin(t)}{t}dt$$
	
	\centering
	\rule{1\linewidth}{0.6pt}
\end{exo}











\end{document}
