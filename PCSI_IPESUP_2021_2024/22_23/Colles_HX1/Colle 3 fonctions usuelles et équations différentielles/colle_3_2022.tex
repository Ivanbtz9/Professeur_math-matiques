\documentclass[a4paper,11pt]{article}

\usepackage{inputenc}
\usepackage[T1]{fontenc}
\usepackage[frenchb]{babel}
\usepackage{fancyhdr,fancybox} % pour personnaliser les en-têtes
\usepackage{lastpage,setspace}
\usepackage{amsfonts,amssymb,amsmath,amsthm,mathrsfs}
\usepackage{relsize,exscale,bbold}
\usepackage{paralist}
\usepackage{xspace,multicol,diagbox,array}
\usepackage{xcolor}
\usepackage{variations}
\usepackage{xypic}
\usepackage{eurosym,stmaryrd}
\usepackage{graphicx}
\usepackage[np]{numprint}
\usepackage{hyperref} 
\usepackage{tikz}
\usepackage{colortbl}
\usepackage{multirow}
\usepackage{MnSymbol,wasysym}
\usepackage[top=1.5cm,bottom=1.5cm,right=1.2cm,left=1.5cm]{geometry}
\usetikzlibrary{calc, arrows, plotmarks, babel,decorations.pathreplacing}
\setstretch{1.25}
%\usepackage{lipsum} %\usepackage{enumitem} %\setlist[enumerate]{itemsep=1mm} bug avec enumerate



\newtheorem{thm}{Théorème}
\newtheorem{rmq}{Remarque}
\newtheorem{prop}{Propriété}
\newtheorem{cor}{Corollaire}
\newtheorem{lem}{Lemme}
\newtheorem{prop-def}{Propriété-définition}

\theoremstyle{definition}

\newtheorem{defi}{Définition}
\newtheorem{ex}{Exemple}
\newtheorem*{rap}{Rappel}
\newtheorem{cex}{Contre-exemple}
\newtheorem{exo}{Exercice} % \large {\fontfamily{ptm}\selectfont EXERCICE}
\newtheorem{nota}{Notation}
\newtheorem{ax}{Axiome}
\newtheorem{appl}{Application}
\newtheorem{csq}{Conséquence}
\def\di{\displaystyle}



\renewcommand{\thesection}{\Roman{section}}\renewcommand{\thesubsection}{\arabic{subsection} }\renewcommand{\thesubsubsection}{\alph{subsubsection} }


\newcommand{\bas}{~\backslash}\newcommand{\ba}{\backslash}
\newcommand{\C}{\mathbb{C}}\newcommand{\R}{\mathbb{R}}\newcommand{\Q}{\mathbb{Q}}\newcommand{\Z}{\mathbb{Z}}\newcommand{\N}{\mathbb{N}}\newcommand{\V}{\overrightarrow}\newcommand{\Cs}{\mathscr{C}}\newcommand{\Ps}{\mathscr{P}}\newcommand{\Rs}{\mathscr{R}}\newcommand{\Gs}{\mathscr{G}}\newcommand{\Ds}{\mathscr{D}}\newcommand{\happy}{\huge\smiley}\newcommand{\sad}{\huge\frownie}\newcommand{\danger}{\begin{tikzpicture}[x=1.5pt,y=1.5pt,rotate=-14.2]
	\definecolor{myred}{rgb}{1,0.215686,0}
	\draw[line width=0.1pt,fill=myred] (13.074200,4.937500)--(5.085940,14.085900)..controls (5.085940,14.085900) and (4.070310,15.429700)..(3.636720,13.773400)
	..controls (3.203130,12.113300) and (0.917969,2.382810)..(0.917969,2.382810)
	..controls (0.917969,2.382810) and (0.621094,0.992188)..(2.097660,1.359380)
	..controls (3.574220,1.726560) and (12.468800,3.984380)..(12.468800,3.984380)
	..controls (12.468800,3.984380) and (13.437500,4.132810)..(13.074200,4.937500)
	--cycle;
	\draw[line width=0.1pt,fill=white] (11.078100,5.511720)--(5.406250,11.875000)..controls (5.406250,11.875000) and (4.683590,12.812500)..(4.367190,11.648400)
	..controls (4.050780,10.488300) and (2.375000,3.675780)..(2.375000,3.675780)
	..controls (2.375000,3.675780) and (2.156250,2.703130)..(3.214840,2.964840)
	..controls (4.273440,3.230470) and (10.640600,4.847660)..(10.640600,4.847660)
	..controls (10.640600,4.847660) and (11.332000,4.953130)..(11.078100,5.511720)
	--cycle;
	\fill (6.144520,8.839900)..controls (6.460940,7.558590) and (6.464840,6.457090)..(6.152340,6.378910)
	..controls (5.835930,6.300840) and (5.320300,7.277400)..(5.003900,8.554750)
	..controls (4.683590,9.835940) and (4.679690,10.941400)..(4.996090,11.019600)
	..controls (5.312490,11.097700) and (5.824210,10.121100)..(6.144520,8.839900)
	--cycle;
	\fill (7.292960,5.261780)..controls (7.382800,4.898500) and (7.128900,4.523500)..(6.730460,4.421880)
	..controls (6.328120,4.324220) and (5.929680,4.535220)..(5.835930,4.898500)
	..controls (5.746080,5.261780) and (5.999990,5.640630)..(6.402340,5.738340)
	..controls (6.804690,5.839840) and (7.203110,5.625060)..(7.292960,5.261780)
	--cycle;
	\end{tikzpicture}}\newcommand{\alors}{\Large\Rightarrow}\newcommand{\equi}{\Leftrightarrow}
\newcommand{\fonction}[5]{\begin{array}{l|rcl}
		#1: & #2 & \longrightarrow & #3 \\
		& #4 & \longmapsto & #5 \end{array}}


\definecolor{vert}{RGB}{11,160,78}
\definecolor{rouge}{RGB}{255,120,120}
\definecolor{bleu}{RGB}{15,5,107}


\pagestyle{fancy}
\lhead{Optimal Sup Spé, groupe IPESUP}\chead{Année~2022-2023}\rhead{Niveau: Première année de PCSI }\lfoot{M. Botcazou}\cfoot{\thepage}\rfoot{mail: i.botcazou@gmx.fr }\renewcommand{\headrulewidth}{0.4pt}\renewcommand{\footrulewidth}{0.4pt}

\begin{document}
	
	
	\begin{center}
		\Large \sc colle 3 = fonctions usuelles, primitives et équations différentielles 
	\end{center}

\section*{Connaître son cours:}
\begin{enumerate}
	\item Vérifier pour $x\neq0$ que: $\arctan x + \arctan\frac{1}{x} = \text{sgn}(x)\frac{\pi}{2}.$
	\item Pour tout $a,b\in \R$ donner l'expression de $min(a,b)$ et $max(a,b)$ à l'aide de la fonction valeur absolue. 
	\item  Soient $n\in\N$ et la fonction $f:x\mapsto-\ln(x)$. Donner les dérivées  n-ième  $f^{(n)}$ de la fonction $f$.
\end{enumerate}



\hfill\\\hfill\\
\begin{minipage}{1\linewidth}
	\begin{minipage}[t]{0.48\linewidth}
		\raggedright
		\section*{Fonctions usuelles:}%https://www.bibmath.net/ressources/index.php?action=affiche&quoi=bde/analyse/unevariable/foncusutrigo&type=fexo
		\begin{exo}\textit{ }\quad\\[0.25cm]
			Soit $\alpha\in\mathbb R$ et $f$ la fonction définie sur $\mathbb R$ par $f(x)=\cos(x)+\cos(\alpha x)$. On veut démontrer que $f$ est périodique si et seulement si $\alpha\in\mathbb Q$.
			\begin{enumerate}
				\item On suppose que $\alpha=p/q\in\mathbb Q$. Démontrer que $f$ est périodique.
				\item On suppose que $\alpha\notin\mathbb Q$. Résoudre l'équation $f(x)=2$. En déduire que $f$ n'est pas périodique.
			\end{enumerate}
			
			\centering
			\rule{1\linewidth}{0.6pt}
		\end{exo}
		
		
		
		\begin{exo}\textit{}\quad\\
			Soit $f$ la fonction définie par 
			$$f(x)=\arcsin\left(2x\sqrt{1-x^2}\right).$$
			\begin{enumerate}
				\item Quel est l'ensemble de définition de $f$?
				\item En posant $x=\sin t$, simplifier l'écriture de $f$.
			\end{enumerate}
			
			\centering
			\rule{1\linewidth}{0.6pt}
		\end{exo}
		
		\begin{exo}\quad\\
			Pour $n\in\mathbb N$, on pose $f_n(x)=\cos(n\arccos x)$ et $g_n(x)=\frac{\sin(n \arccos x)}{\sqrt{1-x^2}}$.
			Prouver que $f_n$ et $g_n$ sont des fonctions polynomiales.
			
			\centering
			\rule{1\linewidth}{0.6pt}
		\end{exo}
		
	
		
	\end{minipage}	
	\hfill\vrule\hfill
	\begin{minipage}[t]{0.48\linewidth}
		\raggedright
	\section*{Primitives et équations différentielles:}%https://www.bibmath.net/ressources/index.php?action=affiche&quoi=bde/analyse/equadiff/eqlineairessecordre&type=fexo
	\begin{exo}\textit{}\quad\\[0.25cm]
		\begin{enumerate}
			\item Résoudre l'équation différentielle $(x^2+1)y'+2xy=3x^2+1$ sur $\R$.
			Tracer des courbes intégrales. Trouver la solution vérifiant $y(0) = 3$.
			
			
			\item Résoudre l'équation différentielle $y'\sin x-y\cos x+1=0$ sur $]0;\pi[$.
			Tracer des courbes intégrales. Trouver la solution vérifiant $y(\frac\pi4) = 1$.
			
		\end{enumerate}
		
		
		\centering
		\rule{1\linewidth}{0.6pt}
	\end{exo}
	
	
	
	\begin{exo}\textit{}\quad\\
		On considère l'équation différentielle
		$$y'-e^xe^y=a$$
		Déterminer ses solutions, en précisant soigneusement leurs intervalles de définition, pour
		\begin{enumerate}
			\item $a=0$
			\item $a=-1$ (faire le changement de fonction inconnue $z(x)=x+y(x)$)
		\end{enumerate}
		
		\centering
		\rule{1\linewidth}{0.6pt}
	\end{exo}
	
	\begin{exo}\quad\\
		On considère $y''-4y'+4y=d(x)$. 
		
		Résoudre l'équation homogène, puis trouver une solution particulière 
		lorsque $d(x)=e^{-2x}$, puis $d(x)=e^{2x}$. 
		Donner la forme générale des solutions quand $d(x)=\frac{1}{2} \text{ch}(2x)$.
		
		\centering
		\rule{1\linewidth}{0.6pt}
	\end{exo}
		
		
		
		
	\end{minipage}
\end{minipage}



		
	
		
		

\end{document}