\documentclass[11pt]{article}

 %Configuration de la feuille 
 
\usepackage{amsmath,amssymb,enumerate,graphicx,pgf,tikz,fancyhdr}
\usepackage[utf8]{inputenc}
\usetikzlibrary{arrows}
\usepackage{geometry}
\usepackage{tabvar}
\geometry{hmargin=2.2cm,vmargin=1.5cm}\pagestyle{fancy}
\lfoot{\bfseries http://www.bibmath.net}
\rfoot{\bfseries\thepage}
\cfoot{}
\renewcommand{\footrulewidth}{0.5pt} %Filet en bas de page

 %Macros utilisées dans la base de données d'exercices 

\newcommand{\mtn}{\mathbb{N}}
\newcommand{\mtns}{\mathbb{N}^*}
\newcommand{\mtz}{\mathbb{Z}}
\newcommand{\mtr}{\mathbb{R}}
\newcommand{\mtk}{\mathbb{K}}
\newcommand{\mtq}{\mathbb{Q}}
\newcommand{\mtc}{\mathbb{C}}
\newcommand{\mch}{\mathcal{H}}
\newcommand{\mcp}{\mathcal{P}}
\newcommand{\mcb}{\mathcal{B}}
\newcommand{\mcl}{\mathcal{L}}
\newcommand{\mcm}{\mathcal{M}}
\newcommand{\mcc}{\mathcal{C}}
\newcommand{\mcmn}{\mathcal{M}}
\newcommand{\mcmnr}{\mathcal{M}_n(\mtr)}
\newcommand{\mcmnk}{\mathcal{M}_n(\mtk)}
\newcommand{\mcsn}{\mathcal{S}_n}
\newcommand{\mcs}{\mathcal{S}}
\newcommand{\mcd}{\mathcal{D}}
\newcommand{\mcsns}{\mathcal{S}_n^{++}}
\newcommand{\glnk}{GL_n(\mtk)}
\newcommand{\mnr}{\mathcal{M}_n(\mtr)}
\DeclareMathOperator{\ch}{ch}
\DeclareMathOperator{\sh}{sh}
\DeclareMathOperator{\vect}{vect}
\DeclareMathOperator{\card}{card}
\DeclareMathOperator{\comat}{comat}
\DeclareMathOperator{\imv}{Im}
\DeclareMathOperator{\rang}{rg}
\DeclareMathOperator{\Fr}{Fr}
\DeclareMathOperator{\diam}{diam}
\DeclareMathOperator{\supp}{supp}
\newcommand{\veps}{\varepsilon}
\newcommand{\mcu}{\mathcal{U}}
\newcommand{\mcun}{\mcu_n}
\newcommand{\dis}{\displaystyle}
\newcommand{\croouv}{[\![}
\newcommand{\crofer}{]\!]}
\newcommand{\rab}{\mathcal{R}(a,b)}
\newcommand{\pss}[2]{\langle #1,#2\rangle}
 %Document 

\begin{document} 

\begin{center}\textsc{{\huge }}\end{center}

% Exercice 998


\vskip0.3cm\noindent\textsc{Exercice 1} - Déterminant tridiagonal
\vskip0.2cm
Soit $x\in\mathbb R$. Calculer
$$\left|
\begin{array}{ccccc}
1+x^2&-x&0&\dots&0\\
-x&1+x^2&-x&\ddots&\vdots\\
0&\ddots&\ddots&\ddots&0\\
\vdots&\ddots&-x&1+x^2&-x\\
0&\dots&0&-x&1+x^2
\end{array}
\right|.
$$


% Exercice 999


\vskip0.3cm\noindent\textsc{Exercice 2} - Déterminant circulant
\vskip0.2cm
Soient $a_1,\dots,a_n$ des nombres complexes, $\omega=e^{2i\pi/n}$, et $A$ et $M$ les matrices suivantes :
$$A=\left(
\begin{array}{ccccc}
a_1&a_2&a_3&\dots&a_n\\
a_n&a_1&a_2&\dots&a_{n-1}\\
\vdots&\vdots&\vdots&\vdots&\vdots\\
a_2&a_3&\dots&\dots&a_{1}
\end{array}\right),$$
$$M=\left(
\begin{array}{ccccc}
1&1&\dots&\dots&1\\
1&\omega&\omega^2&\dots&\omega^{n-1}\\
\vdots&\vdots&\vdots&\vdots\\
1&\omega^{n-1}&\omega^{2(n-1)}&\dots&\omega^{(n-1)(n-1)}
\end{array}
\right).$$
Calculer $\det(AM)$ et en déduire $\det(A)$.




\vskip0.5cm
\noindent{\small Cette feuille d'exercices a été conçue à l'aide du site \textsf{https://www.bibmath.net}}

%Vous avez accès aux corrigés de cette feuille par l'url : https://www.bibmath.net/ressources/justeunefeuille.php?id=28469
\end{document}