\documentclass[11pt]{article}

 %Configuration de la feuille 
 
\usepackage{amsmath,amssymb,enumerate,graphicx,pgf,tikz,fancyhdr}
\usepackage[utf8]{inputenc}
\usetikzlibrary{arrows}
\usepackage{geometry}
\usepackage{tabvar}
\geometry{hmargin=2.2cm,vmargin=1.5cm}\pagestyle{fancy}
\lfoot{\bfseries http://www.bibmath.net}
\rfoot{\bfseries\thepage}
\cfoot{}
\renewcommand{\footrulewidth}{0.5pt} %Filet en bas de page

 %Macros utilisées dans la base de données d'exercices 

\newcommand{\mtn}{\mathbb{N}}
\newcommand{\mtns}{\mathbb{N}^*}
\newcommand{\mtz}{\mathbb{Z}}
\newcommand{\mtr}{\mathbb{R}}
\newcommand{\mtk}{\mathbb{K}}
\newcommand{\mtq}{\mathbb{Q}}
\newcommand{\mtc}{\mathbb{C}}
\newcommand{\mch}{\mathcal{H}}
\newcommand{\mcp}{\mathcal{P}}
\newcommand{\mcb}{\mathcal{B}}
\newcommand{\mcl}{\mathcal{L}}
\newcommand{\mcm}{\mathcal{M}}
\newcommand{\mcc}{\mathcal{C}}
\newcommand{\mcmn}{\mathcal{M}}
\newcommand{\mcmnr}{\mathcal{M}_n(\mtr)}
\newcommand{\mcmnk}{\mathcal{M}_n(\mtk)}
\newcommand{\mcsn}{\mathcal{S}_n}
\newcommand{\mcs}{\mathcal{S}}
\newcommand{\mcd}{\mathcal{D}}
\newcommand{\mcsns}{\mathcal{S}_n^{++}}
\newcommand{\glnk}{GL_n(\mtk)}
\newcommand{\mnr}{\mathcal{M}_n(\mtr)}
\DeclareMathOperator{\ch}{ch}
\DeclareMathOperator{\sh}{sh}
\DeclareMathOperator{\vect}{vect}
\DeclareMathOperator{\card}{card}
\DeclareMathOperator{\comat}{comat}
\DeclareMathOperator{\imv}{Im}
\DeclareMathOperator{\rang}{rg}
\DeclareMathOperator{\Fr}{Fr}
\DeclareMathOperator{\diam}{diam}
\DeclareMathOperator{\supp}{supp}
\newcommand{\veps}{\varepsilon}
\newcommand{\mcu}{\mathcal{U}}
\newcommand{\mcun}{\mcu_n}
\newcommand{\dis}{\displaystyle}
\newcommand{\croouv}{[\![}
\newcommand{\crofer}{]\!]}
\newcommand{\rab}{\mathcal{R}(a,b)}
\newcommand{\pss}[2]{\langle #1,#2\rangle}
 %Document 

\begin{document} 

\begin{center}\textsc{{\huge }}\end{center}

% Exercice 330


\vskip0.3cm\noindent\textsc{Exercice 1} - Hypothèse affaiblie
\vskip0.2cm
Soit $f$ définie sur un intervalle ouvert $I$ contenant $0$, continue sur $I$.
On suppose en outre que $\lim_{x\to 0}\frac{f(2x)-f(x)}{x}=0$. Montrer que $f$ est dérivable en 0.



% Exercice 350


\vskip0.3cm\noindent\textsc{Exercice 2} - Calcul de dérivée $n$-ième
\vskip0.2cm
Soit $n\in\mathbb N$. Montrer que la dérivée d'ordre $n+1$ de $x^ne^{1/x}$ est
$$\frac{(-1)^{n+1}}{x^{n+2}}e^{1/x}.$$


% Exercice 346


\vskip0.3cm\noindent\textsc{Exercice 3} - Polynômes de Laguerre
\vskip0.2cm
On pose, pour tout entier naturel $n$ et pour tout réel $x$, 
$$h_n(x)=x^ne^{-x}\textrm{ et }L_n(x)=\frac{e^x}{n!}h_n^{(n)}(x).$$
\begin{enumerate}
\item Montrer que, pour tout entier $n$, $L_n$ est une fonction polynômiale. Préciser son degré et son coefficient dominant.
\item Plus précisément, montrer que, pour tout $k\in\{0,\dots,n\}$, il existe $Q_k\in\mathbb R[X]$ tel que, pour tout $x\in\mathbb R$, on a
$$h_n^{(k)}(x)=x^{n-k}e^{-x}Q_k(x).$$
\end{enumerate}


% Exercice 3067


\vskip0.3cm\noindent\textsc{Exercice 4} - Limite de la dérivée et limite de $f(x)/x$.
\vskip0.2cm
Soit $f:]0,+\infty[\to\mathbb R$ une fonction dérivable et $\ell\in\mathbb R$ tel que $\lim_{x\to+\infty}f'(x)=\ell$. L'objectif de cet exercice est de démontrer que $\lim_{x\to+\infty}\frac{f(x)}x=\ell$. 
\begin{enumerate}
\item On suppose dans cette question que $\ell=0$. Soit $\veps>0$.
\begin{enumerate}
\item Montrer qu'il existe $A>0$ tel que, pour tout $x\geq A$, on a 
$$\left|\frac{f(x)}{x}\right|\leq \left|\frac{f(A)}{x}\right|+\veps.$$
\item En déduire le résultat dans ce cas.
\end{enumerate}
\item Démontrer le résultat dans le cas général.
\item Réciproquement, est-il vrai que pour toute fonction dérivable $f:]0,+\infty[\to \mathbb R$ telle que $\lim_{x\to+\infty}\frac{f(x)}x=\ell$, alors on a $\lim_{x\to+\infty}f'(x)=\ell$?
\end{enumerate}


% Exercice 349


\vskip0.3cm\noindent\textsc{Exercice 5} - Polynômes
\vskip0.2cm
Déterminer la dérivée d'ordre $n$ de la fonction $f$ définie par $f(x)=(x-a)^n (x-b)^n$
($a,b$ sont des réels). En étudiant le cas $a=b$, trouver la valeur de $\sum_{k=0}^n \binom{n}{k}^2$.




\vskip0.5cm
\noindent{\small Cette feuille d'exercices a été conçue à l'aide du site \textsf{https://www.bibmath.net}}

%Vous avez accès aux corrigés de cette feuille par l'url : https://www.bibmath.net/ressources/justeunefeuille.php?id=26503
\end{document}