
 %Macros utilisées dans la base de données d'exercices 

\newcommand{\mtn}{\mathbb{N}}
\newcommand{\mtns}{\mathbb{N}^*}
\newcommand{\mtz}{\mathbb{Z}}
\newcommand{\mtr}{\mathbb{R}}
\newcommand{\mtk}{\mathbb{K}}
\newcommand{\mtq}{\mathbb{Q}}
\newcommand{\mtc}{\mathbb{C}}
\newcommand{\mch}{\mathcal{H}}
\newcommand{\mcp}{\mathcal{P}}
\newcommand{\mcb}{\mathcal{B}}
\newcommand{\mcl}{\mathcal{L}}
\newcommand{\mcm}{\mathcal{M}}
\newcommand{\mcc}{\mathcal{C}}
\newcommand{\mcmn}{\mathcal{M}}
\newcommand{\mcmnr}{\mathcal{M}_n(\mtr)}
\newcommand{\mcmnk}{\mathcal{M}_n(\mtk)}
\newcommand{\mcsn}{\mathcal{S}_n}
\newcommand{\mcs}{\mathcal{S}}
\newcommand{\mcd}{\mathcal{D}}
\newcommand{\mcsns}{\mathcal{S}_n^{++}}
\newcommand{\glnk}{GL_n(\mtk)}
\newcommand{\mnr}{\mathcal{M}_n(\mtr)}
\DeclareMathOperator{\ch}{ch}
\DeclareMathOperator{\sh}{sh}
\DeclareMathOperator{\vect}{vect}
\DeclareMathOperator{\card}{card}
\DeclareMathOperator{\comat}{comat}
\DeclareMathOperator{\imv}{Im}
\DeclareMathOperator{\rang}{rg}
\DeclareMathOperator{\Fr}{Fr}
\DeclareMathOperator{\diam}{diam}
\DeclareMathOperator{\supp}{supp}
\newcommand{\veps}{\varepsilon}
\newcommand{\mcu}{\mathcal{U}}
\newcommand{\mcun}{\mcu_n}
\newcommand{\dis}{\displaystyle}
\newcommand{\croouv}{[\![}
\newcommand{\crofer}{]\!]}
\newcommand{\rab}{\mathcal{R}(a,b)}
\newcommand{\pss}[2]{\langle #1,#2\rangle}
 %Document 

\begin{document} 

\begin{center}\textsc{{\huge COLLE 2 sommes produits et fonctions usuelles}}\end{center}

% Exercice 451


\vskip0.3cm\noindent\textsc{Exercice 1} - Somme et somme des carrés
\vskip0.2cm
Soit $n\geq 1$ et $x_1,\dots,x_n$ des réels vérifiant 
$$\sum_{k=1}^n x_k=n\textrm{ et }\sum_{k=1}^n x_k^2=n.$$
Démontrer que, pour tout $k$ dans $\{1,\dots,n\}$, $x_k=1$.


% Exercice 462


\vskip0.3cm\noindent\textsc{Exercice 2} - Une somme
\vskip0.2cm
Soient $n,p$ des entiers naturels avec $n\geq p$. Démontrer que
$$\sum_{k=p}^n \dbinom{k}{p}=\dbinom{n+1}{p+1}.$$


% Exercice 463


\vskip0.3cm\noindent\textsc{Exercice 3} - Somme avec des nombres complexes
\vskip0.2cm
Calculer $(1+i)^{4n}$. En déduire les valeurs de 
$$\sum_{p=0}^{2n}(-1)^p \dbinom{4n}{2p}\textrm{ et }\sum_{p=0}^{2n-1}(-1)^p \dbinom{4n}{2p+1}.$$


% Exercice 3048


\vskip0.3cm\noindent\textsc{Exercice 4} - Périodicité
\vskip0.2cm
Soit $\alpha\in\mathbb R$ et $f$ la fonction définie sur $\mathbb R$ par $f(x)=\cos(x)+\cos(\alpha x)$. On veut démontrer que $f$ est périodique si et seulement si $\alpha\in\mathbb Q$.
\begin{enumerate}
\item On suppose que $\alpha=p/q\in\mathbb Q$. Démontrer que $f$ est périodique.
\item On suppose que $\alpha\notin\mathbb Q$. Résoudre l'équation $f(x)=2$. En déduire que $f$ n'est pas périodique.
\end{enumerate}


% Exercice 372


\vskip0.3cm\noindent\textsc{Exercice 5} - Simplifier!
\vskip0.2cm
Soit $f$ la fonction définie par 
$$f(x)=\arcsin\left(2x\sqrt{1-x^2}\right).$$
\begin{enumerate}
\item Quel est l'ensemble de définition de $f$?
\item En posant $x=\sin t$, simplifier l'écriture de $f$.
\end{enumerate}


% Exercice 381


\vskip0.3cm\noindent\textsc{Exercice 6} - Polynômes de Chebychev
\vskip0.2cm
Pour $n\in\mathbb N$, on pose $f_n(x)=\cos(n\arccos x)$ et $g_n(x)=\frac{\sin(n \arccos x)}{\sqrt{1-x^2}}$.
Prouver que $f_n$ et $g_n$ sont des fonctions polynomiales.




\vskip0.5cm
\noindent{\small Cette feuille d'exercices a été conçue à l'aide du site \textsf{http://www.bibmath.net}}

%Vous avez accès aux corrigés de cette feuille par l'url : https://www.bibmath.net/ressources/justeunefeuille.php?id=25215
\end{document}