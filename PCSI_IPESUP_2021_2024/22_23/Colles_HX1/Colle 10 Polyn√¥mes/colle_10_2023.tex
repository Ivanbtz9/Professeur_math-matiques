\documentclass[a4paper,11pt]{article}

\usepackage{inputenc}
\usepackage[T1]{fontenc}
\usepackage[frenchb]{babel}
\usepackage{fancyhdr,fancybox} % pour personnaliser les en-têtes
\usepackage{lastpage,setspace}
\usepackage{amsfonts,amssymb,amsmath,amsthm,mathrsfs}
\usepackage{relsize,exscale,bbold}
\usepackage{paralist}
\usepackage{xspace,multicol,diagbox,array}
\usepackage{xcolor}
\usepackage{variations}
\usepackage{xypic}
\usepackage{eurosym,stmaryrd}
\usepackage{graphicx}
\usepackage[np]{numprint}
\usepackage{hyperref} 
\usepackage{tikz}
\usepackage{colortbl}
\usepackage{multirow}
\usepackage{MnSymbol,wasysym}
\usepackage[top=1.5cm,bottom=1.5cm,right=1.2cm,left=1.5cm]{geometry}
\usetikzlibrary{calc, arrows, plotmarks, babel,decorations.pathreplacing}
\setstretch{1.25}
%\usepackage{lipsum} %\usepackage{enumitem} %\setlist[enumerate]{itemsep=1mm} bug avec enumerate



\newtheorem{thm}{Théorème}
\newtheorem{rmq}{Remarque}
\newtheorem{prop}{Propriété}
\newtheorem{cor}{Corollaire}
\newtheorem{lem}{Lemme}
\newtheorem{prop-def}{Propriété-définition}

\theoremstyle{definition}

\newtheorem{defi}{Définition}
\newtheorem{ex}{Exemple}
\newtheorem*{rap}{Rappel}
\newtheorem{cex}{Contre-exemple}
\newtheorem{exo}{Exercice} % \large {\fontfamily{ptm}\selectfont EXERCICE}
\newtheorem{nota}{Notation}
\newtheorem{ax}{Axiome}
\newtheorem{appl}{Application}
\newtheorem{csq}{Conséquence}
\def\di{\displaystyle}



\renewcommand{\thesection}{\Roman{section}}\renewcommand{\thesubsection}{\arabic{subsection} }\renewcommand{\thesubsubsection}{\alph{subsubsection} }


\newcommand{\bas}{~\backslash}\newcommand{\ba}{\backslash}
\newcommand{\C}{\mathbb{C}}\newcommand{\R}{\mathbb{R}}\newcommand{\K}{\mathbb{K}}\newcommand{\Q}{\mathbb{Q}}\newcommand{\Z}{\mathbb{Z}}\newcommand{\N}{\mathbb{N}}\newcommand{\V}{\overrightarrow}\newcommand{\Cs}{\mathscr{C}}\newcommand{\Ps}{\mathscr{P}}\newcommand{\Rs}{\mathscr{R}}\newcommand{\Gs}{\mathscr{G}}\newcommand{\Ds}{\mathscr{D}}\newcommand{\happy}{\huge\smiley}\newcommand{\sad}{\huge\frownie}\newcommand{\danger}{\begin{tikzpicture}[x=1.5pt,y=1.5pt,rotate=-14.2]
	\definecolor{myred}{rgb}{1,0.215686,0}
	\draw[line width=0.1pt,fill=myred] (13.074200,4.937500)--(5.085940,14.085900)..controls (5.085940,14.085900) and (4.070310,15.429700)..(3.636720,13.773400)
	..controls (3.203130,12.113300) and (0.917969,2.382810)..(0.917969,2.382810)
	..controls (0.917969,2.382810) and (0.621094,0.992188)..(2.097660,1.359380)
	..controls (3.574220,1.726560) and (12.468800,3.984380)..(12.468800,3.984380)
	..controls (12.468800,3.984380) and (13.437500,4.132810)..(13.074200,4.937500)
	--cycle;
	\draw[line width=0.1pt,fill=white] (11.078100,5.511720)--(5.406250,11.875000)..controls (5.406250,11.875000) and (4.683590,12.812500)..(4.367190,11.648400)
	..controls (4.050780,10.488300) and (2.375000,3.675780)..(2.375000,3.675780)
	..controls (2.375000,3.675780) and (2.156250,2.703130)..(3.214840,2.964840)
	..controls (4.273440,3.230470) and (10.640600,4.847660)..(10.640600,4.847660)
	..controls (10.640600,4.847660) and (11.332000,4.953130)..(11.078100,5.511720)
	--cycle;
	\fill (6.144520,8.839900)..controls (6.460940,7.558590) and (6.464840,6.457090)..(6.152340,6.378910)
	..controls (5.835930,6.300840) and (5.320300,7.277400)..(5.003900,8.554750)
	..controls (4.683590,9.835940) and (4.679690,10.941400)..(4.996090,11.019600)
	..controls (5.312490,11.097700) and (5.824210,10.121100)..(6.144520,8.839900)
	--cycle;
	\fill (7.292960,5.261780)..controls (7.382800,4.898500) and (7.128900,4.523500)..(6.730460,4.421880)
	..controls (6.328120,4.324220) and (5.929680,4.535220)..(5.835930,4.898500)
	..controls (5.746080,5.261780) and (5.999990,5.640630)..(6.402340,5.738340)
	..controls (6.804690,5.839840) and (7.203110,5.625060)..(7.292960,5.261780)
	--cycle;
	\end{tikzpicture}}\newcommand{\alors}{\Large\Rightarrow}\newcommand{\equi}{\Leftrightarrow}
\newcommand{\fonction}[5]{\begin{array}{l|rcl}
		#1: & #2 & \longrightarrow & #3 \\
		& #4 & \longmapsto & #5 \end{array}}


\definecolor{vert}{RGB}{11,160,78}
\definecolor{rouge}{RGB}{255,120,120}
\definecolor{bleu}{RGB}{15,5,107}


\pagestyle{fancy}
\lhead{Optimal Sup Spé, groupe IPESUP}\chead{Année~2022-2023}\rhead{Niveau: Première année de PCSI }\lfoot{M. Botcazou}\cfoot{\thepage}\rfoot{mail: i.botcazou@gmx.fr }\renewcommand{\headrulewidth}{0.4pt}\renewcommand{\footrulewidth}{0.4pt}

\begin{document}
	
	
	\begin{center}
		\Large \sc colle 10 = Polynômes
	\end{center}
\raggedright

\section*{Connaître son cours:}
\begin{enumerate}
	\item Montrer qu'un complexe $a$ est une racine de $P \in \K[X]$ si, et seulement si, $X - a$ divise $P $.
	\item  Rappeler le Théorème de d'\emph{Alembert-Gauss} et montrer qu'un polynôme $P \in \C[X]$ non constant est surjectif de $\C$ dans $\C$. Est-ce vrai de $\R$ dans $\R$ ?
	\item Soit $P, Q \in \K[X]$, rappeler la définition du produit de $P$ et $Q$ le polynôme noté $P.Q$.
	
	Montrer que \  deg$(P.Q)=$ deg$(P) $ $+$ deg$(Q)$.
	
\end{enumerate}

\section*{Exercices:}	


\begin{exo}\textbf{(*)}\quad\\[0.25cm]

Soient $a_1,\ldots,a_n$ des réels deux à deux distincts.
Pour tout $i=1,\ldots,n$, on pose
 
 $$\displaystyle L_i(X)=\prod\limits_{\substack{1\le j\le n \\ j\not= i}}\frac{X-a_j}{a_i-a_j}$$
\begin{enumerate}
	\item Calculer $L_i(a_j)$ pour $j=1,\ldots,n$.
	\item Soient $b_1,\ldots,b_n$ des réels fixés. 
	Montrer que $\displaystyle P(X)=\sum_{i=0}^nb_iL_i(X)$ est l'unique polynôme de degré inférieur ou égal à $n-1$ qui vérifie:
	
	$P(a_j)=b_j  \ \text{ pour tout }j=1,\ldots,n.$
	\item Trouver le polynôme $P$ de degré inférieur ou égal à $3$ tel que 
	$P(0)=1\ , \ P(1)=0\ , \ P(-1)=-2\ \text{et}\  P(2)=4.$
\end{enumerate}		

	\centering
\rule{1\linewidth}{0.6pt}
\end{exo}
	
\begin{exo}\textbf{(**)}\quad\\[0.25cm]
	Soit $a,b\in\R$, déterminer la dérivée d'ordre $n$ de la fonction polynomiale $f$ définie par $f(x)=(x-a)^n (x-b)^n$.
	En étudiant le cas $a=b$, trouver la valeur de $\sum_{k=0}^n \binom{n}{k}^2$.
	
	\centering
	\rule{1\linewidth}{0.6pt}
\end{exo}	



\begin{exo}\textbf{(*)}\quad\\[0.25cm]
 Soit $P$ un polynôme différent de $X$.

Montrer que $P(X)-X$ divise $P(P(X))-X$.
	
	\centering
	\rule{1\linewidth}{0.6pt}
\end{exo}

\begin{exo}\textbf{(**)}\quad\\[0.25cm]
	Trouver un polynôme de degré $5$ tel que $P(X)+10$ soit divisible par $(X+2)^3$ 
	
	et $P(X)-10$ soit divisible par $(X-2)^3$.
	
	\centering
	\rule{1\linewidth}{0.6pt}
\end{exo}


\newpage

\begin{exo}\textbf{(*)}\quad\\[0.25cm]
				\begin{enumerate}
		\item Décomposer en éléments simples la fraction rationnelle $\displaystyle\frac{1}{X(X+1)(X+2)}$.
		\item En déduire la limite de la suite $(S_n)$ suivante : $\displaystyle S_n=\sum_{k=1}^n \frac{1}{k(k+1)(k+2)}$.
	\end{enumerate}

	\centering
	\rule{1\linewidth}{0.6pt}
\end{exo}

\begin{exo}\textbf{(**)}\quad\\[0.25cm]
	Montrer que l'ensemble des polynômes unitaires, de degré $n > 0$, à coefficients entiers et à racines complexes dans $\mathbb U$ est fini.
	
	\centering
	\rule{1\linewidth}{0.6pt}
\end{exo}

\end{document}
