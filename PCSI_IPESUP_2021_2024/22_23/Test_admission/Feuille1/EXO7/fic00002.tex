
%%%%%%%%%%%%%%%%%% PREAMBULE %%%%%%%%%%%%%%%%%%

\documentclass[11pt,a4paper]{article}

\usepackage{amsfonts,amsmath,amssymb,amsthm}
\usepackage[utf8]{inputenc}
\usepackage[T1]{fontenc}
\usepackage[francais]{babel}
\usepackage{mathptmx}
\usepackage{fancybox}
\usepackage{graphicx}
\usepackage{ifthen}

\usepackage{tikz}   

\usepackage{hyperref}
\hypersetup{colorlinks=true, linkcolor=blue, urlcolor=blue,
pdftitle={Exo7 - Exercices de mathématiques}, pdfauthor={Exo7}}

\usepackage{geometry}
\geometry{top=2cm, bottom=2cm, left=2cm, right=2cm}

%----- Ensembles : entiers, reels, complexes -----
\newcommand{\Nn}{\mathbb{N}} \newcommand{\N}{\mathbb{N}}
\newcommand{\Zz}{\mathbb{Z}} \newcommand{\Z}{\mathbb{Z}}
\newcommand{\Qq}{\mathbb{Q}} \newcommand{\Q}{\mathbb{Q}}
\newcommand{\Rr}{\mathbb{R}} \newcommand{\R}{\mathbb{R}}
\newcommand{\Cc}{\mathbb{C}} \newcommand{\C}{\mathbb{C}}
\newcommand{\Kk}{\mathbb{K}} \newcommand{\K}{\mathbb{K}}

%----- Modifications de symboles -----
\renewcommand{\epsilon}{\varepsilon}
\renewcommand{\Re}{\mathop{\mathrm{Re}}\nolimits}
\renewcommand{\Im}{\mathop{\mathrm{Im}}\nolimits}
\newcommand{\llbracket}{\left[\kern-0.15em\left[}
\newcommand{\rrbracket}{\right]\kern-0.15em\right]}
\renewcommand{\ge}{\geqslant} \renewcommand{\geq}{\geqslant}
\renewcommand{\le}{\leqslant} \renewcommand{\leq}{\leqslant}

%----- Fonctions usuelles -----
\newcommand{\ch}{\mathop{\mathrm{ch}}\nolimits}
\newcommand{\sh}{\mathop{\mathrm{sh}}\nolimits}
\renewcommand{\tanh}{\mathop{\mathrm{th}}\nolimits}
\newcommand{\cotan}{\mathop{\mathrm{cotan}}\nolimits}
\newcommand{\Arcsin}{\mathop{\mathrm{arcsin}}\nolimits}
\newcommand{\Arccos}{\mathop{\mathrm{arccos}}\nolimits}
\newcommand{\Arctan}{\mathop{\mathrm{arctan}}\nolimits}
\newcommand{\Argsh}{\mathop{\mathrm{argsh}}\nolimits}
\newcommand{\Argch}{\mathop{\mathrm{argch}}\nolimits}
\newcommand{\Argth}{\mathop{\mathrm{argth}}\nolimits}
\newcommand{\pgcd}{\mathop{\mathrm{pgcd}}\nolimits} 

%----- Structure des exercices ------

\newcommand{\exercice}[1]{\video{0}}
\newcommand{\finexercice}{}
\newcommand{\noindication}{}
\newcommand{\nocorrection}{}

\newcounter{exo}
\newcommand{\enonce}[2]{\refstepcounter{exo}\hypertarget{exo7:#1}{}\label{exo7:#1}{\bf Exercice \arabic{exo}}\ \  #2\vspace{1mm}\hrule\vspace{1mm}}

\newcommand{\finenonce}[1]{
\ifthenelse{\equal{\ref{ind7:#1}}{\ref{bidon}}\and\equal{\ref{cor7:#1}}{\ref{bidon}}}{}{\par{\footnotesize
\ifthenelse{\equal{\ref{ind7:#1}}{\ref{bidon}}}{}{\hyperlink{ind7:#1}{\texttt{Indication} $\blacktriangledown$}\qquad}
\ifthenelse{\equal{\ref{cor7:#1}}{\ref{bidon}}}{}{\hyperlink{cor7:#1}{\texttt{Correction} $\blacktriangledown$}}}}
\ifthenelse{\equal{\myvideo}{0}}{}{{\footnotesize\qquad\texttt{\href{http://www.youtube.com/watch?v=\myvideo}{Vidéo $\blacksquare$}}}}
\hfill{\scriptsize\texttt{[#1]}}\vspace{1mm}\hrule\vspace*{7mm}}

\newcommand{\indication}[1]{\hypertarget{ind7:#1}{}\label{ind7:#1}{\bf Indication pour \hyperlink{exo7:#1}{l'exercice \ref{exo7:#1} $\blacktriangle$}}\vspace{1mm}\hrule\vspace{1mm}}
\newcommand{\finindication}{\vspace{1mm}\hrule\vspace*{7mm}}
\newcommand{\correction}[1]{\hypertarget{cor7:#1}{}\label{cor7:#1}{\bf Correction de \hyperlink{exo7:#1}{l'exercice \ref{exo7:#1} $\blacktriangle$}}\vspace{1mm}\hrule\vspace{1mm}}
\newcommand{\fincorrection}{\vspace{1mm}\hrule\vspace*{7mm}}

\newcommand{\finenonces}{\newpage}
\newcommand{\finindications}{\newpage}


\newcommand{\fiche}[1]{} \newcommand{\finfiche}{}
%\newcommand{\titre}[1]{\centerline{\large \bf #1}}
\newcommand{\addcommand}[1]{}

% variable myvideo : 0 no video, otherwise youtube reference
\newcommand{\video}[1]{\def\myvideo{#1}}

%----- Presentation ------

\setlength{\parindent}{0cm}

\definecolor{myred}{rgb}{0.93,0.26,0}
\definecolor{myorange}{rgb}{0.97,0.58,0}
\definecolor{myyellow}{rgb}{1,0.86,0}

\newcommand{\LogoExoSept}[1]{  % input : echelle       %% NEW
{\usefont{U}{cmss}{bx}{n}
\begin{tikzpicture}[scale=0.1*#1,transform shape]
  \fill[color=myorange] (0,0)--(4,0)--(4,-4)--(0,-4)--cycle;
  \fill[color=myred] (0,0)--(0,3)--(-3,3)--(-3,0)--cycle;
  \fill[color=myyellow] (4,0)--(7,4)--(3,7)--(0,3)--cycle;
  \node[scale=5] at (3.5,3.5) {Exo7};
\end{tikzpicture}}
}


% titre
\newcommand{\titre}[1]{%
\vspace*{-4ex} \hfill \hspace*{1.5cm} \hypersetup{linkcolor=black, urlcolor=black} 
\href{http://exo7.emath.fr}{\LogoExoSept{3}} 
 \vspace*{-5.7ex}\newline 
\hypersetup{linkcolor=blue, urlcolor=blue}  {\Large \bf #1} \newline 
 \rule{12cm}{1mm} \vspace*{3ex}}

%----- Commandes supplementaires ------



\begin{document}

%%%%%%%%%%%%%%%%%% EXERCICES %%%%%%%%%%%%%%%%%%
\fiche{f00002, bodin, 2007/09/01} 

\titre{Logique, ensembles, raisonnements} 

\section{Logique}
\exercice{108, bodin, 1998/09/01}
\video{2tmyTz5_2Rw}
\enonce{000108}{}

Compl\'eter les pointill\'es par le connecteur logique qui s'impose : $\Leftrightarrow ,\ \Leftarrow,\ \Rightarrow.$
\begin{enumerate}
    \item $  x\in \Rr\ \ x^2=4 \ \ldots\ldots \ x=2$ ;
    \item $  z\in \Cc\ \ z=\overline{z} \ \ldots\ldots \ z\in\Rr$ ;
    \item $ x\in \Rr\ \ x=\pi \ \ldots\ldots \ e^{2ix}=1$.
\end{enumerate}
\finenonce{000108} 


\finexercice
\exercice{106, bodin, 1998/09/01}
\video{nfJizOU7DbA}
\enonce{000106}{}

Soient les quatre assertions suivantes :
$$ (a) \ \exists x\in \Rr \quad \forall y\in \Rr \quad x+y > 0 \quad ; \quad
 (b) \ \forall x\in \Rr \quad \exists y\in \Rr \quad x+y > 0  \ ;$$
$$ (c) \ \forall x\in \Rr \quad \forall y\in \Rr \quad x+y > 0  \quad ; \quad
(d) \ \exists x\in \Rr \quad \forall y\in \Rr \quad y^2 > x .$$
\begin{enumerate}
    \item Les assertions $a$, $b$, $c$, $d$ sont-elles vraies ou fausses ?
    \item Donner leur n\'egation.
\end{enumerate}
\finenonce{000106} 


\finexercice
\exercice{109, bodin, 1998/09/01}
\video{q07OjCbDg3o}
\enonce{000109}{}

Dans $\R^2$, on d\'efinit les ensembles $F_{1}=\{(x,y)\in\R^2,\ y\leq0\}$ et
$F_{2}=\{(x,y)\in\R^2,\ xy\geq1,\ x\geq0\}$. On note $M_1M_2$ la distance usuelle entre deux points $M_1$ et $M_2$ de $\R^2$.
\'Evaluer les propositions suivantes :
\begin{enumerate}
\item $
 \forall\epsilon\in]0,+\infty[\quad
 \exists M_{1}\in F_{1}\quad
 \exists M_{2}\in F_{2}
 \qquad M_{1}M_{2}<\epsilon$
\item $
 \exists M_{1}\in F_{1}\quad
 \exists M_{2}\in F_{2}\quad
 \forall\epsilon\in]0,+\infty[
 \qquad M_{1}M_{2}<\epsilon$
\item $
 \exists\epsilon\in]0,+\infty[\quad 
 \forall M_{1}\in F_{1}\quad
 \forall M_{2}\in F_{2}
 \qquad M_{1}M_{2}<\epsilon$
\item $
 \forall M_{1}\in F_{1}\quad
 \forall M_{2}\in F_{2}\quad
 \exists\epsilon\in]0,+\infty[
 \qquad M_{1}M_{2}<\epsilon$
\end{enumerate}
Quand elles sont fausses, donner leur n\'egation.
\finenonce{000109} 


\finexercice
\exercice{110, gourio, 2001/09/01}
\video{QSbHxGnf-9M}
\enonce{000110}{}
Nier la proposition: ``tous les habitants de la rue du Havre qui ont les
yeux bleus gagneront au loto et prendront leur retraite avant 50 ans''.

\finenonce{000110} 


\finexercice
\exercice{112, bodin, 1998/09/01}
\video{FH3vTlYT23M}
\enonce{000112}{}
 Nier les assertions suivantes :
\begin{enumerate}
    \item tout triangle rectangle poss\`ede un angle droit ;
    \item dans toutes les \'ecuries, tous les chevaux sont noirs ;
    \item pour tout entier $x$, il existe un entier $y$ tel que, pour tout entier $z$, 
la relation $z<x$ implique le relation $z<x+1$ ;
    \item $\forall \epsilon >0 \quad \exists \alpha >0 \qquad (|x-7/5|<\alpha \Rightarrow |5x-7|<\epsilon)$.
\end{enumerate}


\finenonce{000112} 


\finexercice
\exercice{120, bodin, 1998/09/01}
\video{-Kid1jaNv5A}
\enonce{000120}{}

Soient $f,g$ deux fonctions de $\Rr$ dans $\Rr$. Traduire
en termes de quantificateurs les expressions suivantes :
\begin{enumerate}
  \item $f$ est major\'ee;
  \item $f$ est born\'ee;
  \item $f$ est paire;
  \item $f$ est impaire;
  \item $f$ ne s'annule jamais;
  \item $f$ est p\'eriodique;
  \item $f$ est croissante;
  \item $f$ est strictement d\'ecroissante;
  \item $f$ n'est pas la fonction nulle;
  \item $f$ n'a jamais les m\^{e}mes valeurs en deux points distincts;
  \item $f$ atteint toutes les valeurs de $\Nn$;
  \item $f$ est inf\'erieure \`a $g$;
  \item $f$ n'est pas inf\'erieure \`a $g$.
\end{enumerate}
\finenonce{000120} 


\finexercice
\exercice{107, bodin, 1998/09/01}
\video{ZCLCoGR3hcc}
\enonce{000107}{}
 Soit  $f$  une application de  ${\Rr}$  dans
${\Rr}$. Nier, de la mani\`ere la plus pr\' ecise possible, les \'
enonc\' es qui suivent :
\begin{enumerate}
    \item Pour tout  $x\in {\Rr}\ f(x)\leq 1$.
    \item L'application  $f$  est croissante.
    \item L'application  $f$  est croissante et positive.
    \item Il existe  $x\in {\Rr}^+$  tel que  $f(x)\leq 0$.
    \item  Il existe $x\in \Rr$ tel que quel que soit $y\in \Rr$, si $x<y$ alors $f(x)>f(y)$.
\end{enumerate}
\noindent On ne demande pas de d\' emontrer quoi que ce soit, juste d'\' ecrire le contraire
d'un \' enonc\' e.

\finenonce{000107} 


\finexercice
\exercice{119, ridde, 1999/11/01}
\video{UK22qba7LV0}
\enonce{000119}{}

Montrer que 
$$\forall \epsilon >0 \quad \exists N \in \Nn \text{ tel que }
 (n \geq N \Rightarrow 2-\epsilon < \frac{2n + 1}{n + 2} < 2 + \epsilon).$$
\finenonce{000119} 


\finexercice

\section{Ensembles}
\exercice{123, bodin, 1998/09/01}
\video{I6E-O_PNk1Y}
\enonce{000123}{}
 Soit $A,B$ deux ensembles, montrer
$\complement(A\cup B) =\complement A\cap \complement B$ et
$\complement (A\cap B) =\complement A\cup \complement B$.

\finenonce{000123} 


\finexercice
\exercice{122, bodin, 1998/09/01}
\video{MxucdpW0WDI}
\enonce{000122}{}

 Montrer par contraposition les assertions suivantes, $E$ \'etant
un ensemble :
\begin{enumerate}
\item $\forall A,B \in \mathcal{P}(E) \quad (A\cap B=A\cup B)\Rightarrow A=B$,
\item $\forall A,B,C \in \mathcal{P}(E) \quad
(A\cap B=A\cap C \text{ et } A\cup B=A\cup C)\Rightarrow B=C$.
\end{enumerate}
\finenonce{000122} 


\finexercice
\exercice{124, bodin, 1998/09/01}
\video{64sBWnSD4nM}
\enonce{000124}{}

Soient $E$ et $F$ deux ensembles, $f:E\rightarrow F$. D\'emontrer que :\\
$\forall A,B \in \mathcal{P}(E) \quad (A\subset B)\Rightarrow (f(A)\subset f(B))$,\\
$\forall A,B \in \mathcal{P}(E) \quad f(A\cap B)\subset f(A)\cap f(B)$,\\
$\forall A,B \in \mathcal{P}(E) \quad f(A\cup B) = f(A)\cup f(B)$,\\
$\forall A,B \in \mathcal{P}(F) \quad f^{-1}(A\cup B) = f^{-1}(A)\cup f^{-1}(B)$,\\
$\forall A \in \mathcal{P}(F) \quad f^{-1}(F\setminus A)=E\setminus f^{-1}(A)$.

\finenonce{000124} 


\finexercice
\exercice{137, cousquer, 2003/10/01}
\video{a7iULB800tg}
\enonce{000137}{}

 Montrez que chacun des ensembles suivants est un intervalle que vous calculerez.
$$I=\bigcap_{n=1}^{+\infty}\left[-\frac{1}{n},2+\frac{1}{n}\right[\ \quad \mbox{
et }\quad \ J=\bigcup_{n=2}^{+\infty}\left[1+\frac{1}{n},n\right]$$

\finenonce{000137} 


\finexercice

\section{Absurde et contraposée}
\exercice{150, bodin, 1998/09/01}
\video{RDbMQTxMIuA}
\enonce{000150}{}
 Soit $(f_n)_{n\in\Nn}$ une suite d'applications de
l'ensemble $\Nn$ dans lui-m\^eme. On d\'efinit une application $f$
de $\Nn$ dans $\Nn$ en posant $f(n)=f_n(n)+1$. D\'emontrer qu'il
n'existe aucun $p\in \Nn$ tel que $f=f_p$.

\finenonce{000150} 


\finexercice
\exercice{151, bodin, 1998/09/01}
\video{RHiN60xhmoo}
\enonce{000151}{}
\begin{enumerate}
\item
Soit $p_{1},p_{2},\ldots ,p_{r}$, $r$ nombres premiers. Montrer que l'entier
$N=p_{1}p_{2}\ldots p_{r}+1$ n'est divisible par aucun des entiers $p_{i}$.
\item
Utiliser la question pr\'ec\'edente pour montrer par l'absurde qu'il existe une infinit\'e de
nombres premiers.
\end{enumerate}

\finenonce{000151} 


\finexercice

\section{Récurrence}
\exercice{153, bodin, 1998/09/01}
\video{SUMAdDoMPto}
\enonce{000153}{}
\label{ex104}
Montrer :
\begin{enumerate}
    \item $\displaystyle{\sum_{k=1}^{n}k=\frac{n(n+1)}{2} \quad \forall n \in \Nn^{*} \,.}$
    \item $\displaystyle{\sum_{k=1}^{n}k^2=\frac{n(n+1)(2n+1)}{6} \quad \forall n \in \Nn^{*} \,.}$
\end{enumerate}

\finenonce{000153} 


\finexercice
\exercice{157, ridde, 1999/11/01}
\video{okAsZpWWB0M}
\enonce{000157}{}
Soit $X$ un ensemble. Pour $f \in \mathcal{F} (X, X)$, on d\'efinit $f^0 = id$ et
par r\'ecurrence pour $n \in \Nn$ $f^{n + 1} = f^n \circ f$.
\begin{enumerate}
\item Montrer que $\forall n \in \Nn$ $f^{n + 1} = f \circ f^n$.
\item Montrer que si $f$ est bijective alors $\forall n \in \Nn$ $ (f^{-1})^n
 = (f^n)^{-1}$.
 \end{enumerate}

\finenonce{000157} 


\finexercice
\exercice{155, bodin, 1998/09/01}
\video{6iVRvfkMLTw}
\enonce{000155}{}
Soit la suite $(x_n)_{n\in \Nn}$ d\'efinie par
$x_0=4$ et $\displaystyle{x_{n+1}=\frac{2x_n^2-3}{x_n+2}}$.
\begin{enumerate}
    \item Montrer que : $\forall n\in\Nn\quad x_n>3$.
    \item Montrer que : $\forall n\in\Nn \quad x_{n+1}-3>\frac{3}{2}(x_n-3)$.
    \item Montrer que : $\forall n\in\Nn \quad x_n \geqslant \left(\frac{3}{2}\right)^n+3$.
    \item La suite $(x_n)_{n\in\Nn}$ est-elle convergente ?

\end{enumerate}

\finenonce{000155} 


\finexercice

\finfiche

 \finenonces 



 \finindications 

\noindication
\indication{000106}
Attention : la n\'egation d'une in\'egalit\'e stricte est une in\'egalit\'e large
(et r\'eci\-pro\-quement).
\finindication
\indication{000109}
Faire un dessin de $F_1$ et de $F_2$.
Essayer de voir si la difficult\'e pour r\'ealiser les assertions vient de
$\epsilon$ ``petit'' (c'est-\`a-dire proche de $0$) ou de $\epsilon$ ``grand''
(quand il tend vers $+\infty$).
\finindication
\noindication
\noindication
\noindication
\noindication
\indication{000119}
En fait, on a toujours : $\frac{2n+1}{n+2} \leq 2$.
Puis chercher une condition sur $n$ pour que
l'in\'egalit\'e 
$$2-\epsilon < \frac{2n + 1}{n + 2}$$
soit vraie.
\finindication
\indication{000123}
Il est plus facile de raisonner en prenant un \'el\'ement $x\in E$.
Par exemple, soit $F, G$ des sous-ensembles de $E$. Montrer que $F\subset G$
revient à montrer que pour tout $x\in F$ alors $x\in G$.
Et montrer $F=G$ est \'equivalent \`a $x\in F$ si et seulement si $x\in G$,
et ce pour tout $x$ de $E$.
Remarque : pour montrer $F=G$ on peut aussi montrer $F\subset G$ puis $G\subset F$.

Enfin, se rappeler que $x\in \complement F$ si et seulement si $x\notin F$.
\finindication
\noindication
\noindication
\noindication
\indication{000150}
Par l'absurde, supposer qu'il existe $p\in \Nn$ tel que $f=f_p$.
Puis pour un tel $p$, \'evaluer $f$ et $f_p$ en une valeur bien choisie.
\finindication
\indication{000151}
Pour la premi\`ere question vous pouvez raisonner par contraposition ou par l'absurde.
\finindication
\noindication
\indication{000157}
Pour les deux questions, travailler par r\'ecurrence.
\finindication
\indication{000155}
\begin{enumerate}
    \item R\'ecurrence : calculer $x_{n+1}-3$.
    \item Calculer  $x_{n+1}-3 - \frac{3}{2}(x_n-3)$.
    \item R\'ecurrence.
\end{enumerate}
\finindication


\newpage

\correction{000108}
\begin{enumerate}
  \item $\Leftarrow$
  \item $\Leftrightarrow$
  \item $\Rightarrow$
\end{enumerate}
\fincorrection
\correction{000106}
\begin{enumerate}
\item (a) est fausse. Car sa n\'egation qui est 
 $\forall x\in \Rr \ \exists y\in \Rr \quad x+y \leq 0$
est vraie. \'Etant donn\'e $x\in \Rr$ il existe toujours un $y\in\Rr$ tel que
$x+y \leq 0$, par exemple on peut prendre $y=-(x+1)$ et alors $x+y=x-x-1=-1 \leq 0$.

\item (b) est vraie, pour un $x$ donn\'e, on peut prendre (par exemple) $y=-x+1$
et alors $x+y=1>0$.
La n\'egation de (b) est 
 $\exists x\in \Rr \ \forall y\in \Rr \quad x+y \leq 0$.

\item (c) : $\forall x\in \Rr \ \forall y\in \Rr \quad x+y > 0$
est fausse, par exemple $x=-1$, $y=0$. La n\'egation est 
$\exists x\in \Rr\ \exists y\in \Rr\ x+y \leq 0$.

\item (d) est vraie, on peut prendre $x=-1$. La n\'egation est:
$\forall x\in \Rr \ \exists y\in \Rr \quad y^2 \leq x$. 
\end{enumerate}
\fincorrection
\correction{000109}
\begin{enumerate}
\item
 Cette proposition est vraie. En effet soit $\epsilon > 0$,  d\'efinissons $M_1 = (\frac{2}{\epsilon},0)  \in F_1$ et
 $M_2 = (\frac{2}{\epsilon},\frac{\epsilon}{2}) \in F_2$, alors $M_1M_2=\frac{\epsilon}{2} < \epsilon$. Ceci \'etant vrai quelque soit
$\epsilon >0$ la proposition est donc d\'emontr\'ee.


\item
Soit deux points fix\'es $M_1$, $M_2$ v\'erifiant cette
proposition, la distance $d= M_1M_2$ est
 aussi petite que l'on veut donc elle est nulle, donc $M_1 = M_2$ ; or les ensembles $F_1$ et $F_2$
sont disjoints. Donc la proposition est fausse. La n\'egation de
cette proposition est :
$$ \forall M_1 \in F_1 \ \  \forall M_2 \in F_2 \quad \exists \epsilon \in ]0,+\infty[ \quad  \quad M_1M_2 \geqslant \epsilon $$ et cela exprime le fait que les ensembles $F_1$ et $F_2$ sont disjoints.

\item
Celle ci est \'egalement fausse, en effet supposons qu'elle soit
vraie, soit alors $\epsilon$ correspondant \`a cette proposition.
Soit $M_1=(\epsilon +2,0)$ et $M_2 = (1,1)$, on a $M_1M_2 >
\epsilon+1$ ce qui est absurde. La n\'egation est :
$$\forall \epsilon \in ]0,+\infty[ \quad  \exists M_1 \in F_1 \ \  \exists M_2 \in F_2 \quad   \quad M_1M_2 \geqslant \epsilon $$
C'est-\`a-dire que l'on peut trouver deux points aussi
\'eloign\'es l'un de l'autre  que l'on veut.

\item
Cette proposition est vraie, il suffit de choisir
$\epsilon=M_1M_2+1$. Elle signifie que la distance entre deux
points donn\'es est un nombre fini !

\end{enumerate}
\fincorrection
\correction{000110}
``Il existe un habitant de la rue du Havre qui a les yeux bleus, qui
ne gagnera pas au loto ou qui prendra sa retraite apr\`es 50 ans.''
\fincorrection
\correction{000112}
\begin{enumerate}
    \item ``Il existe un triangle rectangle qui n'a pas d'angle droit." Bien sûr cette dernière phrase est fausse !
    \item ``Il existe une \'ecurie dans laquelle il y a (au moins) un cheval
dont la couleur n'est pas noire."
    \item Sachant que la proposition en langage math\'ematique s'\'ecrit
$$\forall x\in\Zz \ \  \exists y\in\Zz\ \  \forall z\in\Zz \quad (z<x \Rightarrow z<x+1),$$
la n\'egation est
$$\exists x\in\Zz\ \  \forall y\in\Zz\ \  \exists z\in\Zz \quad (z<x \text{ et } z\geq x+1).$$
    \item $\exists \epsilon>0\ \  \forall \alpha>0 \quad (|x-7/5|<\alpha \text{ et } |5x-7|\geq\epsilon).$
\end{enumerate}
\fincorrection
\correction{000120}
\begin{enumerate}
  \item $\exists M \in \Rr \quad \forall x \in \Rr \qquad f(x) \leq M$;
  \item $\exists M \in \Rr\quad \exists m \in \Rr \quad \forall x \in \Rr \qquad m \leq f(x) \leq M$;
  \item $\forall x \in \Rr \qquad f(x) = f(-x)$;
  \item $\forall x \in \Rr \qquad f(x) = -f(-x)$;
  \item $\forall x \in \Rr \qquad f(x) \not= 0$;
  \item $\exists a \in \Rr^* \quad \forall x \in \Rr \qquad f(x+a) = f(x)$;
  \item $\forall (x,y) \in \Rr^2 \qquad (x\leq y \Rightarrow f(x) \leq f(y))$;
  \item $\forall (x,y) \in \Rr^2 \qquad (x < y \Rightarrow f(x) > f(y))$;
  \item $\exists x \in \Rr \qquad  f(x) \not= 0$;
  \item $\forall (x,y) \in \Rr^2 \qquad (x\not= y \Rightarrow f(x) \not= f(y))$;
  \item $\forall n\in \Nn \quad \exists x \in \Rr \qquad f(x)=n$;
  \item $\forall x \in \Rr \qquad f(x) \leq g(x)$;
  \item $\exists x \in \Rr \qquad f(x) > g(x)$.
\end{enumerate}
\fincorrection
\correction{000107}
Dans ce corrig\'e, nous donnons une justification, ce qui n'\'etait
pas demand\'e.
\begin{enumerate}
\item Cette assertion se d\'ecompose de la mani\`ere suivante : ( Pour tout $x\in
  {\Rr}$) ($f(x)\leq 1$). La n\'egation de ``( Pour tout $x\in
  {\Rr}$)" est ``Il existe $x\in {\Rr}$" et la n\'egation de ``($f(x)\leq 1$)" est
$ f(x)>1$. Donc la n\'egation de l'assertion compl\`ete est : ``Il
existe $x\in {\Rr},  f(x)>1$".

\item Rappelons comment se traduit l'assertion ``L'application $f$ est
  croissante" : ``pour tout couple de r\'eels $(x_1,x_2)$, si $x_1\leq x_2$
  alors $f(x_1) \leq f(x_2)$". Cela se d\'ecompose en : ``(pour tout couple de
  r\'eels $x_1$ et $x_2$) ($x_1\leq x_2$ implique $f(x_1) \leq f(x_2)$)".
La n\'egation de la premi\`ere partie est : ``(il existe un couple de
r\'eels $(x_1,x_2)$)" et la n\'egation de la deuxi\`eme partie est :
``($x_1\leq x_2$ et $f(x_1) > f(x_2)$)". Donc la n\'egation de
l'assertion compl\`ete est : ``Il existe $x_1\in \Rr$ et $x_2\in
\Rr$ tels que $x_1 \leq x_2$ et $f(x_1) > f(x_2)$".

\item La n\'egation est : ``l'application $f$ n'est pas croissante ou n'est pas
  positive". On a d\'ej\`a traduit ``l'application $f$ n'est pas croissante",
  traduisons ``l'application $f$ n'est pas positive" : ``il existe $x \in \Rr,
  f(x)<0$". Donc la n\'egation de l'assertion compl\`ete est : ``
Il existe $x_1\in \Rr$ et $x_2\in \Rr$ tels que $x_1<x_2$ et
$f(x_1) \geq f(x_2)$, ou il existe $x \in \Rr,
  f(x)<0$".

\item Cette assertion se d\'ecompose de la mani\`ere suivante : ``(Il existe $x\in
  {\Rr}^+$) ($f(x)\leq 0$)". La n\'egation de la premi\`ere partie est : ``(pour
  tout $x\in  {\Rr}^+$)", et celle de la seconde est :``($f(x)> 0$)".
  Donc la n\'egation de l'assertion compl\`ete est : ``Pour tout $x\in {\Rr}^+$,  $f(x)> 0$".

\item Cette assertion se d\'ecompose de la mani\`ere suivante : 
``($\exists x \in {\Rr}$)($\forall y \in {\Rr}$)($ x<y \Rightarrow f(x)>f(y)$)".  
La n\'egation de la premi\`ere partie est ``($\forall x \in  {\Rr}$)", celle de la
seconde est ``($\exists y \in {\Rr}$)", et celle de la troisi\`eme est
``($ x<y \mathrm{\ et\ } f(x)\leq f(y)$)". Donc la n\'egation de
l'assertion compl\`ete est : `` $\forall x \in {\Rr}, \exists y \in
{\Rr} \mathrm{ tels que } x<y \mathrm{\ et\ } f(x)\leq f(y)$".
\end{enumerate}
\fincorrection
\correction{000119}
Remarquons d'abord que pour $n \in \Nn$, $\frac{2n+1}{n+2} \leq 2$
car $2n+1 \leq 2(n+2)$.
\'Etant donn\'e $\epsilon > 0$, nous avons donc 
$$\forall n \in \Nn \quad \frac{2n+1}{n+2} < 2 + \epsilon$$
Maintenant nous cherchons une condition sur $n$ pour que
l'in\'egalit\'e 
$$2-\epsilon < \frac{2n + 1}{n + 2}$$
soit vraie.
\begin{align*}
2-\epsilon < \frac{2n + 1}{n + 2} 
    &\Leftrightarrow (2-\epsilon)(n+2) < 2n+1 \\
    &\Leftrightarrow 3  < \epsilon (n+2) \\
   &\Leftrightarrow n >  \frac{3}{\epsilon}-2 \\
\end{align*}

Ici $\epsilon$ nous est donn\'e, nous prenons un $N\in \Nn$ tel
que $N > \frac{3}{\epsilon}-2$, alors pour tout $n \geq N$ nous avons
$n \geq N >  \frac{3}{\epsilon}-2$ et par
cons\'equent: $2-\epsilon < \frac{2n + 1}{n + 2}$.
Conclusion: \'etant donn\'e $\epsilon > 0$, nous avons trouv\'e un 
$N\in \Nn$ tel que pour tout $n \geq N$ on ait
$2-\epsilon < \frac{2n + 1}{n + 2}$ et $\frac{2n+1}{n+2} < 2 + \epsilon$.

En fait nous venons de prouver que la suite 
de terme $(2n+1)/(n+2)$ tend vers $2$ quand $n$ tend vers $+\infty$.
\fincorrection
\correction{000123}
\begin{align*}
x\in\complement (A\cup B) & \Leftrightarrow x\notin A\cup B \\
&\Leftrightarrow x\notin A \text{ et } x\notin B\\
&\Leftrightarrow x\in\complement A \text{ et } x\in\complement B\\
&\Leftrightarrow x\in \complement A \cap \complement B.\\
\end{align*}
\begin{align*}
x\in\complement (A\cap B) & \Leftrightarrow x\notin A\cap B\\
&\Leftrightarrow x\notin A \text{ ou } x\notin B\\
&\Leftrightarrow x\in\complement A \text{ ou } x\in\complement\\
&\Leftrightarrow x\in \complement A \cup \complement B.\\
\end{align*}
\fincorrection
\correction{000122}
Nous allons d\'emontrer l'assertion $1.$ de deux mani\`eres diff\'erentes.
\begin{enumerate}
  \item Tout d'abord de fa\c{c}on ``directe". Nous supposons
que $A$ et $B$ sont tels que $A\cap B = A \cup B$. Nous devons montrer 
que $A=B$. 

Pour cela \'etant donn\'e $x \in A$ montrons qu'il est aussi dans $B$.
Comme $x\in A$ alors $x \in A\cup B$ donc $x \in A \cap  B$ (car $A\cup B= A \cap B$). Ainsi $x \in B$. 

Maintenant nous prenons $x\in B$ et le m\^eme raisonnement implique $x \in A$.
Donc tout \'el\'ement de $A$ est dans $B$ et tout \'el\'ement de $B$ est dans $A$.
Cela veut dire $A=B$.

  \item Ensuite, comme demand\'e, nous le montrons par contraposition.
Nous supposons que $A\not= B$ et non devons montrer que 
$A\cap B \not= A\cup B$.

Si $A\not= B$ cela veut dire qu'il existe un \'el\'ement $x \in A\setminus B$
ou alors un \'el\'ement $x\in B \setminus A$. Quitte \`a \'echanger $A$ et $B$, nous supposons qu'il existe $x\in A \setminus B$. Alors $x \in A \cup B$ mais
$x \notin A\cap B$. Donc $A\cap B \not= A\cup B$.
\end{enumerate}
\fincorrection
\correction{000124}
Montrons quelques assertions.

$f(A\cap B) \subset f(A)\cap f(B)$.\\
Si $y\in f(A\cap B)$, il existe $x\in A\cap B$ tel que $y=f(x)$,
or $x\in A$ donc $y=f(x) \in f(A)$ et de m\^eme $x\in B$ donc
$y\in f(B)$. D'o\`u $y\in f(A)\cap f(B)$. Tout \'el\'ement de
$f(A\cap B)$ est un \'el\'ement de $f(A)\cap f(B)$ donc $f(A\cap
B) \subset f(A)\cap f(B)$.

Remarque : l'inclusion r\'eciproque est fausse. Exercice : trouver
un contre-exemple.

\bigskip

$f^{-1}(F\setminus A) = E\setminus f^{-1}(A)$.\\
\begin{align*}
x\in f^{-1}(F\setminus A) &\Leftrightarrow f(x) \in F\setminus A\\
&\Leftrightarrow f(x) \notin A\\
&\Leftrightarrow x \notin f^{-1}(A) \quad \text{ car } f^{-1}(A) = \{ x\in E \ / \ f(x) \in A \}\\
&\Leftrightarrow x\in E\setminus f^{-1}(A)\\
\end{align*}
\fincorrection
\correction{000137}
$I=\left[0,2\right]\;\mbox{
et }\;J=\left]1,+\infty\right[.$
\fincorrection
\correction{000150}
Par l'absurde, supposons qu'il existe $p\in \Nn$ tel que $f=f_p$.
Deux applications sont \'egales si et seulement si elles prennent
les m\^emes valeurs.
$$\forall n\in \Nn\ \ f(n) = f_p(n).$$
En particulier pour $n=p$, $f(p)=f_p(p)$. D'autre part la
d\'efinition de $f$ nous donne $f(p) = f_p(p)+1$. Nous obtenons
une contradiction car $f(p)$ ne peut prendre deux valeurs
distinctes. En conclusion, quelque soit $p\in \Nn$, $f\not= f_p$.
\fincorrection
\correction{000151}
\begin{enumerate}
\item
Montrons en fait la contrapos\'ee.

S'il existe $i$ tel que $p_i$ divise $N=p_1p_2 \ldots p_r +1$ ($i$
est fix\'e) alors il existe $k \in \Zz$ tel que $N = kp_i$ donc
$$p_i(k-p_1p_2\ldots p_{i-1}p_{i+1}\ldots p_r) = 1$$
 soit $p_iq = 1$ (avec
$q = k-p_1p_2\ldots p_{i-1}p_{i+1}\ldots p_r $ un nombre entier).
Donc $p_i \in \Zz$ et $1/p_i = q \in \Zz$,  alors $p_i$ vaut $1$
ou $-1$. Et donc $p_i$ n'est pas un nombre premier.

Conclusion : par contraposition il est vrai que $N$ n'est
divisible par aucun des $p_i$

\item Raisonnons par l'absurde : s'il n'existe qu'un nombre fini $r$ de nombres premiers $p_1,\ldots,p_r$
alors $N=p_1p_2 \ldots p_r +1$ est un nombre premier car divisible
par aucun nombre premier autre que lui m\^eme (c'est le 1.).

 Mais $N$ est strictement sup\'erieur \`a tous les $p_i$. Conclusion on a construit un
nombre premier $N$ diff\'erent des $p_i$, il y a donc au moins
$r+1$ nombres premiers, ce qui est absurde.
\end{enumerate}
\fincorrection
\correction{000153}
R\'edigeons la deuxi\`eme \'egalit\'e.
Soit $\mathcal{A}_n$, $n\in \Nn^*$ l'assertion suivante:
$$(\mathcal{A}_n) \ \ \ \sum_{k=1}^n k^2= \frac{n(n+1)(2n+1)}{6}.$$
\begin{itemize}
  \item $\mathcal{A}_0$ est vraie ($1=1$).
  \item \'Etant donn\'e $n\in\Nn^*$ supposons que $\mathcal{A}_n$
soit vraie. Alors
\begin{align*}
\sum_{k=1}^{n+1} k^2
  &= \sum_{k=1}^n k^2 \ \ + (n+1)^2 \\
  &= \frac{n(n+1)(2n+1)}{6} + (n+1)^2\\
  &= \frac{n(n+1)(2n+1)+6(n+1)^2}{6} \\
  &= \frac{(n+1)(n(2n+1)+6(n+1))}{6} \\
  &= \frac{(n+1)(n+2)(2(n+1)+1)}{6} \\
\end{align*}
Ce qui prouve $\mathcal{A}_{n+1}$.
  \item Par le principe de r\'ecurrence nous venons de montrer 
que $\mathcal{A}_n$ est vraie pour tout $n\in\Nn^*$.
\end{itemize}
\fincorrection
\correction{000157}
\begin{enumerate}
\item Montrons la proposition demand\'ee par r\'ecurrence:
soit $\mathcal{A}_{n}$ l'assertion $f^{n + 1} = f \circ f^n$.
Cette assertion est vraie pour $n=0$. Pour $n\in \Nn$ supposons
$\mathcal{A}_{n}$ vraie. Alors
$$f^{n + 2} = f^{n + 1} \circ f = (f \circ f^n) \circ f = f \circ (f^n \circ f) = f \circ f^{n + 1}.$$
Nous avons utiliser la definition de $f^{n + 2}$, puis la proposition $\mathcal{A}_{n}$,
puis l'associativit\'e de la composition, puis la d\'efinition de $f^{n + 1}$.
Donc $\mathcal{A}_{n+1}$ est vraie. Par le principe de r\'ecurrence
$$\forall \in \Nn \ \ f^n\circ f = f\circ f^n.$$

\item On proc\`ede de m\^eme par r\'ecurrence:
soit $\mathcal{A}_{n}$ l'assertion $ (f^{-1})^n  = (f^n)^{-1}$.
Cette assertion est vraie pour $n=0$. Pour $n\in \Nn$ supposons
$\mathcal{A}_{n}$ vraie. Alors
$$(f^{-1})^{n+1} = (f^{-1})^{n} \circ f^{-1} = (f^n)^{-1} \circ f^{-1} = (f\circ f^n)^{-1} =
 ( f^n \circ f)^{-1} = ( f^{n+1})^{-1} .$$
Donc $\mathcal{A}_{n+1}$ est vraie. Par le principe de r\'ecurrence
$$\forall \in \Nn \ \ (f^{-1})^n
 = (f^n)^{-1}.$$
\end{enumerate}
\fincorrection
\correction{000155}
\begin{enumerate}
    \item Montrons par r\'ecurrence $\forall n \in \Nn\  x_n > 3$.
Soit l'hypoth\`ese de r\'ecurrence :
$$(\mathcal{H}_n) : \quad x_n >3.$$

\begin{itemize}
    \item[$\bullet$] La proposition $\mathcal{H}_0$ est vraie car $x_0 = 4 > 3$.

    \item[$\bullet$] Soit $n\geq 0$, supposons $\mathcal{H}_n$ vraie et montrons
que $\mathcal{H}_{n+1}$ est alors vraie.

$$x_{n+1}-3 = \frac{2{x_n}^2-3}{x_n+2}-3 = \frac{2{x_n}^2-3x_n-9}{x_n+2}.$$
Par hypoth\`ese de r\'ecurrence $x_n > 3$, donc $x_n+2 > 0$ et
$2{x_n}^2-3x_n-9>0$ (ceci par \'etude de la fonction $x \mapsto
2{x}^2-3x-9$ pour $x>3$). Donc $x_{n+1}-3 $ et $\mathcal{H}_{n+1}$
est vraie.

    \item[$\bullet$] Nous avons montré
$$\forall n \in \Nn \quad \mathcal{H}_{n} \Rightarrow \mathcal{H}_{n+1}$$
et comme $\mathcal{H}_{0}$ est vraie alors $\mathcal{H}_{n}$ est
vraie quelque soit $n$. Ce qui termine la d\'emonstration.
\end{itemize}

    \item Montrons que  $x_{n+1}-3 - \frac{3}{2}(x_n-3)$ est positif.
$$x_{n+1}-3 - \frac{3}{2}(x_n-3) =
\frac{2{x_n}^2-3}{x_n+2}-\frac{3}{2}(x_n-3) =
\frac{1}{2}\frac{{x_n}^2-3x_n}{x_n+2}$$ Ce dernier terme est
positif car $x_n >3$.

    \item
 Montrons par r\'ecurrence $\forall n \in \Nn\  x_n > \left(\frac{3}{2}\right)^n+3$.
Soit notre nouvelle l'hypoth\`ese de r\'ecurrence :
$$(\mathcal{H}_n) \quad x_n >\left( \frac{3}{2}\right)^n+3.$$

\begin{itemize}
    \item[$\bullet$] La proposition $\mathcal{H}_0$ est vraie.

    \item[$\bullet$] Soit $n\geq 0$, supposons que $\mathcal{H}_n$ vraie et montrons
que $\mathcal{H}_{n+1}$ est v\'erifi\'ee.

D'apr\`es la question pr\'ec\'edente $ x_{n+1}-3 >
\frac{3}{2}(x_n-3)$ et par hypoth\`ese de r\'ecurrence $x_n
>\left( \frac{3}{2}\right)^n+3$ ; en r\'eunissant ces deux
in\'egalit\'es nous avons $ x_{n+1}-3 >
\frac{3}{2}(\left(\frac{3}{2}\right)^n) =
\left(\frac{3}{2}\right)^{n+1}$.

    \item[$\bullet$] Nous concluons en r\'esumant la situation :\\
$\mathcal{H}_{0}$ est vraie, et $\mathcal{H}_{n} \Rightarrow
\mathcal{H}_{n+1}$ quelque soit $n$. Donc $\mathcal{H}_{n}$ est
toujours vraie.
\end{itemize}
    \item La suite $(x_n)$ tend vers $+\infty$ et n'est donc pas convergente.
\end{enumerate}
\fincorrection


\end{document}

