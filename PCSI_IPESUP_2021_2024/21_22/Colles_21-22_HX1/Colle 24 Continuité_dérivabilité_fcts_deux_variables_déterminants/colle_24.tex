\documentclass[a4paper,10pt]{article}



\usepackage{fancyhdr} % pour personnaliser les en-têtes
\usepackage[utf8]{inputenc}
\usepackage[T1]{fontenc}
\usepackage{lastpage}
\usepackage[frenchb]{babel}
\usepackage{amsfonts,amssymb}
\usepackage{amsmath,amsthm,mathtools}
\usepackage{paralist}
\usepackage{xspace,xypic}
\usepackage{xcolor,multicol,tabularx}
\usepackage{variations}
\usepackage{xypic}
\usepackage{eurosym,multicol}
\usepackage{graphicx}
\usepackage{mathdots}%faire des points suspendus en diagonale
\usepackage[np]{numprint}
\usepackage{hyperref} 
\usepackage{relsize,exscale}
\usepackage{listings} % pour écrire des codes avec coloration syntaxique  

\usepackage{tikz}
\usetikzlibrary{calc, arrows, plotmarks,decorations.pathreplacing}
\usepackage{colortbl}
\usepackage{multirow}
\usepackage[top=2cm,bottom=1.5cm,right=2cm,left=1.5cm]{geometry}

\newtheorem{thm}{Théorème}
\newtheorem*{pro}{Propriété}
\newtheorem*{exemple}{Exemple}

\theoremstyle{definition}
\newtheorem*{remarque}{Remarque}
\theoremstyle{definition}
\newtheorem{exo}{Exercice}
\newtheorem{definition}{Définition}


\newcommand{\vtab}{\rule[-0.4em]{0pt}{1.2em}}
\newcommand{\V}{\overrightarrow}
\renewcommand{\thesection}{\Roman{section} }
\renewcommand{\thesubsection}{\arabic{subsection} }
\renewcommand{\thesubsubsection}{\alph{subsubsection} }
\newcommand*{\transp}[2][-3mu]{\ensuremath{\mskip1mu\prescript{\smash{\mathrm t\mkern#1}}{}{\mathstrut#2}}}%

\newcommand{\K}{\mathbb{K}}
\newcommand{\C}{\mathbb{C}}
\newcommand{\R}{\mathbb{R}}
\newcommand{\Q}{\mathbb{Q}}
\newcommand{\Z}{\mathbb{Z}}
\newcommand{\N}{\mathbb{N}}
\newcommand{\p}{\mathbb{P}}

\renewcommand{\Im}{\mathop{\mathrm{Im}}\nolimits}



\definecolor{vert}{RGB}{11,160,78}
\definecolor{rouge}{RGB}{255,120,120}
% Set the beginning of a LaTeX document
\pagestyle{fancy}
\lhead{Optimal Sup Spé, groupe IPESUP}\chead{Année~2021-2022}\rhead{Niveau: Première année de PCSI }\lfoot{M. Botcazou}\cfoot{\thepage}\rfoot{mail: ibotca52@gmail.com }\renewcommand{\headrulewidth}{0.4pt}\renewcommand{\footrulewidth}{0.4pt}

\begin{document}
 	
	
	\begin{center}
		\Large \sc colle 24 =  Dérivabilité, continuité et déterminants: 
	\end{center}%, fonctions à deux variables

\section*{Exercices mixtes:}%[-0.25cm]

		\raggedright
		
		
		\begin{exo}\textit{(Extrait d'un Concours Centrale-Supélec, Filière MP 1998)}\quad\\[0.25cm]
			%Sujet: https://www.concours-centrale-supelec.fr/CentraleSupelec/1998/MP/sujets/math1.pdf
			%Correction : http://www.booleanopera.fr/DM/MPSI/DM_Absolument_Monotones_Corrige.pdf
				
			Soient $-\infty\leq a<b\leq+\infty$  et $f$ une fonction de classe $\mathcal{C}^{\infty}$ sur $]a;b[$ à valeur dans $\R$.\\[0.25cm]
			\begin{enumerate}[$\square$]
				\item $f$ est dite absolument monotone (en abrégé AM) si
				$$\forall n\in\N, \ \forall x\in ]a;b[, \ f^{(n)}(x)\geq0.  $$
				\item $f$ est dite complètement monotone (en abrégé CM) si
				$$\forall n\in\N, \ \forall x\in ]a;b[, \ (-1)^nf^{(n)}(x)\geq0.  $$
			\end{enumerate}
		\noindent\textbf{\Large{Questions:}}
			\begin{enumerate}
				\item Soient $f$ et $g$ deux fonctions \emph{AM} définies sur $]a;b[$. Montrer que $f+g$ et $fg$
				sont \emph{AM}. Qu’en est-il pour les fonctions \emph{CM} ?
				\item Si $f$ est une fonction \emph{AM} sur $]a;b[$, montrer par récurrence que $e^f$ l’est
				aussi.
				\item Soient $f:]a;b[ \mapsto \R$ et $g:]-b;-a[ \mapsto \R$ définie par : $g(x)=f(-x)$ . Montrer que $f$ est AM sur $]a;b[$ si, et seulement si, $g$ est CM sur $]-b;-a[$ .
				\item\begin{enumerate}
					\item Vérifier que la fonction $-\ln$ est CM sur $]0;1[$ .
					\item Montrer que $f:]0;1[ \mapsto \R$ définie par:
					$$f(x) = \dfrac{1}{\sqrt{1-x^2}}$$
					est \emph{AM} sur $]0;1[$. 
					\item Montrer que la fonction $\arcsin$ est AM sur $]0;1[$.
					\item Montrer que la fonction $\tan$ est AM sur $]0;\dfrac{\pi}{2}[$.
				\end{enumerate}
				\item\begin{enumerate}
					\item On suppose dans cette question que $a\in\R$ et $f$ est AM sur $] a ; b $[.\\ Montrer qu’il existe $\lambda\in\R$ tel que:
					$$\lim\limits_{x\rightarrow a^+}f(x) = \lambda$$
					\item On prolonge $f$ en posant $f ( a ) = \lambda$. Montrer que $f$ est dérivable à droite en $a$ et que $f'$ est continue à droite en $a$.
					\item Plus généralement, montrer que $f$ est indéfiniment dérivable à droite en $a$ avec des dérivées positives ou nulles. Le même phénomène se produit-il en
					b?
				\end{enumerate}
			\end{enumerate}
			
			\centering
			\rule{1\linewidth}{0.6pt}
		\end{exo}
		

		\begin{exo}\quad\\[0.25cm]
			Soient $(z_0, \dots, z_n)$ des nombres complexes deux à deux distincts. Montrer que la famille 
			$$\left((X-z_0)^n, \dots , (X-z_n)^n\right)$$
			est une base de $\C_n[X]$.
			
			\centering
			\rule{1\linewidth}{0.6pt}
		\end{exo}
	
			\begin{exo}\quad\\[0.25cm]
		Soit $A,B\in \mathcal{M}_n(\R)$. On suppose que $A$ et $B$ sont semblables sur $\C$, c'est à dire qu'il existe $P\in \mathcal{G}l_n(\C)$ tel que $A=PBP^{-1}$. Montrer que $A$ et $B$ sont semblables sur $\R$.
		
		\centering
		\rule{1\linewidth}{0.6pt}
	\end{exo}
	
	\newpage
	 	
		\begin{exo}\quad\\[0.25cm]
			Soient $x\in \R$ et $n\in\N^*$, on définit:
			
			
			$$f_n(x) \ = \ \begin{vmatrix}
			1+x^2 & -x &0& \cdots & 0 \\
			-x & 1+x^2 &-x & & \vdots \\
			0 & \ddots &\ddots &\ddots & 0 \\
			\vdots &  & -x & 1+x^2 & -x \\
			0 & \cdots &0 & -x & 1+x^2 \\
			\end{vmatrix}$$
			\begin{enumerate}
				\item Trouver une relation de récurrence double sur $f_n(x)$.
				\item Expliciter pour tout $x\in \R$ et tout $n\in\N^*$ la valeur de $f_n(x)$
				\item Montrer que pour $x\in\R\backslash\{0;1\}$:
				$$f_n(x) =  \dfrac{\dfrac{1}{x^{n+1}}-x^{n+1}}{\dfrac{1}{x}-x}x^n$$ 
			\end{enumerate}
			\centering
			\rule{1\linewidth}{0.6pt}
		\end{exo}			
		
		\begin{exo}\quad\\[0.25cm]
			Soient n$\geq$1, p$\geq$0. Calculer le déterminant suivant :\\[0.4cm] 
			$$ A = \begin{vmatrix}
			\begin{pmatrix}
			n\\0
			\end{pmatrix} & \begin{pmatrix}
			n\\1
			\end{pmatrix} & \cdots & \begin{pmatrix}
			n\\p
			\end{pmatrix} \\\\
			\begin{pmatrix}
			n+1\\0
			\end{pmatrix} & \begin{pmatrix}
			n+1\\1
			\end{pmatrix} &\cdots & \begin{pmatrix}
			n+1\\p
			\end{pmatrix} \\[0.2cm]
			\vdots  &\vdots &  & \vdots \\[0.2cm]
			
			\begin{pmatrix}
				n+p\\0
			\end{pmatrix} & \begin{pmatrix}
				n+p\\1
			\end{pmatrix} &\cdots & \begin{pmatrix}
				n+p\\p
			\end{pmatrix}
		\end{vmatrix}$$
		\\[0.2cm]
		\centering
		\rule{1\linewidth}{0.6pt}
		\end{exo}
		



\end{document}