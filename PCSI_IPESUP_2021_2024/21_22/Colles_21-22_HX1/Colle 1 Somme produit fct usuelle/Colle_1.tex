\documentclass[10pt,a4paper]{article}


\usepackage[T1]{fontenc}
\usepackage[francais]{babel}
\usepackage{times}
\usepackage[utf8]{inputenc}
\usepackage{enumitem}
\usepackage{multicol}
\usepackage{fancyhdr}
\usepackage{tcolorbox}
\usepackage{tablists}
\usepackage[a4paper,bottom = 50pt]{geometry}
\usepackage{color}
\usepackage{amsmath,amssymb,amsthm, mathrsfs,pifont}
\usepackage{pgf,tikz,pgfplots,tkz-tab}
\usepackage{hyperref}
\usepackage{cancel}
\usepackage{array}
\usepackage{xcolor} 
\usepackage{amssymb}
\usepackage{amsthm}
\usepackage{graphicx}
\usepackage{pstricks}

\usepackage{geometry}


\setcellgapes{1pt}
\makegapedcells
\newcolumntype{R}[1]{>{\raggedleft\arraybackslash }b{#1}}
\newcolumntype{L}[1]{>{\raggedright\arraybackslash }b{#1}}
\newcolumntype{C}[1]{>{\centering\arraybackslash }b{#1}}

\geometry{top=2cm, bottom=2cm, left=2cm, right=2cm}

\newcommand{\R}{\mathbb{R}}
\newcommand{\Z}{\mathbb{Z}}
\newcommand{\N}{\mathbb{N}}
\newcommand{\notni}{\not\owns}

\newtheorem{thm}{Théorème}
\newtheorem*{pro}{Propriété}
\newtheorem*{exemple}{Exemple}

\theoremstyle{definition}
\newtheorem*{remarque}{Remarque}
\theoremstyle{definition}
\newtheorem{exo}{Exercice}
\newtheorem{definition}{Définition}



\begin{document}
	
	\leftline{\bfseries Optimal Sup Spé, groupe IPESUP \hfill Année~2021-2022}
	\leftline{\bfseries Niveau: Première année de PCSI  }
	\leftline{\bfseries M.Botcazou \hfill mail: ibotca52@gmail.com  }
	\rule[0.5ex]{\textwidth}{0.1mm}	
	
	\begin{center}
		\Large \sc colle 1 = Sommes, produits et fonctions usuelles	
	\end{center}
\section*{Sommes :}

\begin{center}
\begin{minipage}[t]{0.45\linewidth}
\raggedright
\begin{exo}\quad\\
On pose pour tout $n\in\N^*$ 
$$ H_n  \ = \ \sum_{k=1}^{n} \dfrac{1}{k}$$
Montrer que:  
$$\forall n\in\N^* , \  \ \sum_{i=1}^{n} H_i  \ = \ (n+1)H_n - n$$
\end{exo}
\begin{center}
\rule{1\linewidth}{0.6pt}
\end{center}

\begin{exo}\quad\\ 
Calculer $\sum\limits_{k=1}^{n} k \ln\left(1+\dfrac{1}{k}\right)$ pour tout     $n\in\N^*$ en faisant apparaître un téléscopage.
\end{exo}

\begin{center}
\rule{1\linewidth}{0.6pt}
\end{center}
\end{minipage}	
\hfill\vrule\hfill
\begin{minipage}[t]{0.45\linewidth}
\raggedright

\begin{exo}\quad\\
Montrer en raisonnant par récurrence que pour tout $ n\in\N^*$
$$\sum\limits_{k=1}^{2n} \dfrac{(-1)^{k-1}}{k} \ = \ \sum\limits_{k=1}^{n} \dfrac{1}{n+k} $$
\end{exo}
\begin{center}
\rule{1\linewidth}{0.6pt}
\end{center}
\begin{exo}\quad\\ 

Calculer $\sum\limits_{k=2}^{n-1} \dfrac{3^k}{2^{2k-1}}$ pour tout     $n\in\N,~n\geq 3$\\ \quad\\

En déduire:  
$$\lim\limits_{n \rightarrow +\infty} ~\sum\limits_{k=2}^{n-1} \dfrac{3^k}{2^{2k-1}} $$
\end{exo}
\begin{center}
\rule{1\linewidth}{0.6pt}
\end{center}


	\end{minipage}
\end{center}

\quad\\
\section*{Produits :}

\begin{center}
\begin{minipage}[t]{0.45\linewidth}
\raggedright
\begin{exo}\quad\\
\begin{enumerate}
\item Montrer que pour tout $n\in\N$
$$\dfrac{(2n+1)!}{(n+1)!} \ \geq  \  (n+1)^n$$
\item En déduire par récurrence que pour tout $n\in\N^*$ : 
$$\prod_{k=0}^{n-1}(2k+1)! \ \geq \ (n!)^n$$
\end{enumerate}

\end{exo}
\begin{center}
\rule{1\linewidth}{0.6pt}
\end{center}
\begin{exo}\quad\\ 
Simplifier les produits suivants:
\begin{multicols}{2}
\begin{enumerate}
\item $\prod\limits_{k=1}^{n}\sqrt{k(k+1)}$
\item$\prod\limits_{k=1}^{n}(-5)^{k^2-k}$
\end{enumerate}
\end{multicols}
\end{exo}
\begin{center}
\rule{1\linewidth}{0.6pt}
\end{center}
\begin{exo}\quad\\ 
\begin{enumerate}
\item Montrer que:
$$\forall x > 0 , ~~ x-\dfrac{x^2}{2} \ \leq \ \ln(1+x)  \ \leq \ x$$
\end{enumerate}
\end{exo}
\end{minipage}	
\hfill\vrule\hfill
\begin{minipage}[t]{0.45\linewidth}
\raggedright
\begin{enumerate}
\item[2.] Déterminer la limite de:
$$u_n \ = \ \prod\limits_{k=1}^{n}\left(1 + \dfrac{k}{n^2}\right)$$
\end{enumerate}
\begin{center}
\rule{1\linewidth}{0.6pt}
\end{center}
\begin{exo}\quad\\
\begin{enumerate}
\item Factoriser $(k^3-1)$ par $(k-1)$ et $(k^3+1)$ par $(k+1)$ pour tout $k\geq 2$
\item En déduire une simplification du produit 
$$\prod\limits_{k=2}^{n}~\dfrac{k^3-1}{k^3+1}$$
\item~En déduire l'existence et la valeur de $$\lim\limits_{n \rightarrow +\infty} \prod\limits_{k=2}^{n}~\dfrac{k^3-1}{k^3+1} $$ que l'on notera aussi~~~ $\prod\limits_{k=2}^{+\infty}~\dfrac{k^3-1}{k^3+1}$
\end{enumerate}
\end{exo}
\begin{center}
\rule{1\linewidth}{0.6pt}
\end{center}




	\end{minipage}
\end{center}

\quad\\
\section*{Fonctions usuelles :}

\begin{center}
\begin{minipage}[t]{0.45\linewidth}
\raggedright
\begin{exo}\quad\\
Suivant la valeur de $x$ déterminer le signe de
\begin{enumerate}
\item $ f(x) \ = \ \sqrt{x-1} - \sqrt{2x-3}$
\item $ g(x) \ = \ \sqrt{|x-1|} - \sqrt{|2x-3|}$
\item $ h(x) \ = \ \ln(x+3) + \ln(x+2) - \ln(x+11)$
\end{enumerate}
\end{exo}
\begin{center}
\rule{1\linewidth}{0.6pt}
\end{center}
%\begin{exo}\quad\\ 
%Soit f une fonction continue sur un intervalle $I$ de $\R$ telle que:
%$$\forall x \in I ,~f(x)^2 \ = \ 1$$
%Montrer que la fonction $f$ est constante sur $I$
%\end{exo}
\begin{exo}
Montrer que pour tout $x\neq 0$,
$$\sum\limits_{k=0}^{n} \cosh(kx) \ = \ \dfrac{\cosh\left(\dfrac{nx}{2}\right)\sinh\left(\dfrac{(n+1)x}{2}\right)}{\sinh\left(\dfrac{x}{2}\right)}$$
\end{exo}
\begin{center}
\rule{1\linewidth}{0.6pt}
\end{center}

\begin{exo}\quad\\ 
\begin{enumerate}
\item Déterminer les réels $a$ et $b$ tels que:
$$\forall x \in\R \backslash \{-1;0\}, ~\dfrac{1}{x(x+1)} \ = \ \dfrac{a}{x} + \dfrac{b}{x+1}$$
\end{enumerate}
\end{exo}

\end{minipage}	
\hfill\vrule\hfill
\begin{minipage}[t]{0.45\linewidth}
\raggedright
\begin{enumerate}
\item[2.] Pour $n$ dans $\N^*$, calculer la dérivée n-ième de  
$$f : x \in\R \backslash \{-1;0\}\longmapsto\dfrac{1}{x(x+1)} $$
\item[3.] Trouver les nombres réels $x$ tels que:
$$f^{(n)}(x) \ = \ 0$$
\end{enumerate}
\begin{center}
\rule{1\linewidth}{0.6pt}
\end{center}
%\begin{exo}\quad\\
%Pour $x\in\R^{*}_{+}$ soit $f(x) \ = \ \sin(\dfrac{1}{x})$
%\begin{enumerate}
%\item Quelle est la limite de $f$ lorsque $x$ tend \\ vers $+\infty$
%\item La fonction $f$ a-t-elle une limite en $0$ ?
%\item Quelle est la limite de $x\longmapsto xf(x)$ lorsque $x$ tend vers $0$?
%\end{enumerate}
%\end{exo}
\begin{exo}
Démontrer que, pour tout $x\in\R$ et tout $n\geq 1$, on a
$$\left(\dfrac{1+ \tanh(x)}{1 - \tanh(x)}\right)^n \ = \ \dfrac{1+ \tanh(nx)}{1- \tanh(nx)}$$ 
\end{exo}
\begin{center}
\rule{1\linewidth}{0.6pt}
\end{center}
\begin{exo}
Résoudre l'équation $\cosh(x)=2$.
\end{exo}
\begin{center}
\rule{1\linewidth}{0.6pt}
\end{center}


	\end{minipage}
\end{center}

\quad\\

\section*{Exercice supplémentaire :}

\noindent Soient $n\in\N^*$ et $a_1,...a_n,b_1,...,b_n$ des nombres réels. \\
On définit la fonction $f$ par:

$$\forall x \in \R ~, ~f(x) \ = \ \sum\limits_{i=1}^{n}(a_ix+b_i)^2$$

Montrer l'inégalité de Cauchy-Schwarz :

 $$\left|\sum\limits_{i=1}^{n}a_ib_i\right| \ \leq \ \sqrt{\sum\limits_{i=1}^{n}a_i^2} \times \sqrt{\sum\limits_{i=1}^{n}b_i^2}$$
\quad \\

$\left(\textit{\underline{Indication}: remarquer que la fonction f est à valeur dans $\R^+$}\right)$
\end{document}