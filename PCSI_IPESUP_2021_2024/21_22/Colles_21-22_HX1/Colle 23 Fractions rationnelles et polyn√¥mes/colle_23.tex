\documentclass[a4paper,10pt]{article}



\usepackage{fancyhdr} % pour personnaliser les en-têtes
\usepackage[utf8]{inputenc}
\usepackage[T1]{fontenc}
\usepackage{lastpage}
\usepackage[frenchb]{babel}
\usepackage{amsfonts,amssymb}
\usepackage{amsmath,amsthm,mathtools}
\usepackage{paralist}
\usepackage{xspace,xypic}
\usepackage{xcolor,multicol,tabularx}
\usepackage{variations}
\usepackage{xypic}
\usepackage{eurosym,multicol}
\usepackage{graphicx}
\usepackage{mathdots}%faire des points suspendus en diagonale
\usepackage[np]{numprint}
\usepackage{hyperref} 
\usepackage{relsize,exscale}
\usepackage{listings} % pour écrire des codes avec coloration syntaxique  

\usepackage{tikz}
\usetikzlibrary{calc, arrows, plotmarks,decorations.pathreplacing}
\usepackage{colortbl}
\usepackage{multirow}
\usepackage[top=2cm,bottom=1.5cm,right=2cm,left=1.5cm]{geometry}

\newtheorem{thm}{Théorème}
\newtheorem*{pro}{Propriété}
\newtheorem*{exemple}{Exemple}

\theoremstyle{definition}
\newtheorem*{remarque}{Remarque}
\theoremstyle{definition}
\newtheorem{exo}{Exercice}
\newtheorem{definition}{Définition}


\newcommand{\vtab}{\rule[-0.4em]{0pt}{1.2em}}
\newcommand{\V}{\overrightarrow}
\renewcommand{\thesection}{\Roman{section} }
\renewcommand{\thesubsection}{\arabic{subsection} }
\renewcommand{\thesubsubsection}{\alph{subsubsection} }
\newcommand*{\transp}[2][-3mu]{\ensuremath{\mskip1mu\prescript{\smash{\mathrm t\mkern#1}}{}{\mathstrut#2}}}%

\newcommand{\K}{\mathbb{K}}
\newcommand{\C}{\mathbb{C}}
\newcommand{\R}{\mathbb{R}}
\newcommand{\Q}{\mathbb{Q}}
\newcommand{\Z}{\mathbb{Z}}
\newcommand{\N}{\mathbb{N}}
\newcommand{\p}{\mathbb{P}}

\renewcommand{\Im}{\mathop{\mathrm{Im}}\nolimits}



\definecolor{vert}{RGB}{11,160,78}
\definecolor{rouge}{RGB}{255,120,120}
% Set the beginning of a LaTeX document
\pagestyle{fancy}
\lhead{Optimal Sup Spé, groupe IPESUP}\chead{Année~2021-2022}\rhead{Niveau: Première année de PCSI }\lfoot{M. Botcazou}\cfoot{\thepage}\rfoot{mail: ibotca52@gmail.com }\renewcommand{\headrulewidth}{0.4pt}\renewcommand{\footrulewidth}{0.4pt}

\begin{document}
 	
	
	\begin{center}
		\Large \sc colle 23 = Polynômes, Fractions rationnelles,Fonctions dérivables et Fonctions à plusieurs variables
	\end{center}




\section*{Polynômes et Fractions rationnelles:}\hfill\\%[-0.25cm]
\begin{minipage}{1\linewidth}
	\begin{minipage}[t]{0.48\linewidth}
		\raggedright
		
			\begin{exo}\quad\\
			Décomposer les fractions suivantes en éléments simples sur $\R$, 
			par identification des coefficients.
			\begin{multicols}{2}
				\begin{enumerate}
					\item $F=\frac{X}{X^2-4}$
					\item $G=\frac{X^3-3X^2+X-4}{X-1}$
					\item $H=\frac{2X^3+X^2-X+1}{X^2-2X+1}$
					\item $K=\frac{X+1}{X^4+1}$
				\end{enumerate}
			\end{multicols}
			
			
			\centering
			\rule{1\linewidth}{0.6pt}
		\end{exo}
	
	\begin{exo}\quad\\
		
		On pose $Q_0=(X-1)(X-2)^2$, $Q_1=X(X-2)^2$ et $Q_2=X(X-1)$. 
		\`A l'aide de la décomposition en éléments simples de $\frac{1}{X(X-1)(X-2)^2}$, 
		trouver des polynômes $A_0,\ A_1,\ A_2$ tels que $A_0Q_0+A_1Q_1+A_2Q_2=1$.
		
		\centering
		\rule{1\linewidth}{0.6pt}
	\end{exo}


			

\begin{exo}\quad\\
	Calculer les intégrales suivantes: %https://www.bibmath.net/ressources/index.php?action=affiche&quoi=bde/analyse/integration/integration-calcul&type=fexo
	\begin{multicols}{2}
		\begin{enumerate}
			\item $\mathlarger\int_{0}^{\frac{\pi}{4}}\dfrac{\sin^3(t)}{1+ \cos^2(t)}dt$
			\item $\mathlarger\int_{\frac{\pi}{3}}^{\frac{\pi}{2}}\dfrac{dt}{\sin(t)}$
		\end{enumerate}
	\end{multicols}
	
	\centering
	\rule{1\linewidth}{0.6pt}
\end{exo}

	



		
		
		
	\end{minipage}	
	\hfill\vrule\hfill
	\begin{minipage}[t]{0.48\linewidth}
		\raggedright
		
	\begin{exo}\quad\\
Soit $T_n(x)=\cos\big(n \arccos(x)\big)$ pour $x\in [-1,1]$.
\begin{enumerate}
	\item 
	\begin{enumerate}
		\item Montrer que pour tout $\theta\in[0,\pi]$, $T_n(\cos\theta)=\cos(n\theta)$.
		\item Calculer $T_0$ et $T_1$.
		\item Montrer la relation de récurrence $T_{n+2}(x) = 2xT_{n+1}(x)-T_n(x)$, pour tout $n \ge0$.
		\item En déduire que $T_n$ une fonction polynomiale de degré $n$.
	\end{enumerate}
	
	\item Soit $P(X)=\lambda(X-a_1)\cdots(X-a_n)$ un polynôme, où les $a_k$ sont deux à deux distincts 
	et $\lambda\not=0$. Montrer que 
	$$\frac{1}{P(X)}=\sum_{k=1}^n\frac{\frac{1}{P'(a_k)}}{X-a_k}$$
	
	\item Décomposer $\frac{1}{T_n}$ en éléments simples.
\end{enumerate}
\centering
\rule{1\linewidth}{0.6pt}
\end{exo}			

\begin{exo}\quad\\
	Intégrer les fractions rationnelles suivantes:
	\begin{multicols}{2}
		\begin{enumerate}
			\item $\mathlarger\int_{}^{}\dfrac{x^3}{x^2-x-6}dx$
			\item $\mathlarger\int_{}^{}\dfrac{x^3}{x^2+4x+4}dx$
		\end{enumerate}
	\end{multicols}
	
	\centering
	\rule{1\linewidth}{0.6pt}
\end{exo}
		
	\end{minipage}
\end{minipage}	


\section*{Fonctions à plusieurs variables:}\hfill\\%[-0.25cm]
\begin{minipage}{1\linewidth}
	\begin{minipage}[t]{0.48\linewidth}
		\raggedright
		
		\begin{exo}\quad\\[0.2cm]
			Pour  $(x,y)\in\R^2$, on pose \\[0.2cm]$f(x,y) =\left\{
			\begin{array}{l}
			\frac{xy(x^2-y^2)}{x^2+y^2}\;\text{si}\;(x,y)\neq(0,0)\\
			\rule{0mm}{5mm}0\;\text{si}\;(x,y)=(0,0)
			\end{array}
			\right.$.\hfil\\[0.2cm] Montrer que $f$ est de classe $C^1$ (au moins) sur $\R^2$.
			
			
			\centering
			\rule{1\linewidth}{0.6pt}
		\end{exo}
		
		\begin{exo}\quad\\[0.2cm]
			Soit $\begin{array}[t]{cccc}f~:&\R^2&\longrightarrow&\R\\
			&(x,y)&\mapsto&\left\{
			\begin{array}{l}
			0\;\text{si}\;y=0\\
			y^2\sin\left(\frac{x}{y}\right)\;\text{si}\;y\neq0
			\end{array}
			\right.
			\end{array}
			$.
			\begin{enumerate}
				\item  Etudier la continuité de $f$.
				
				\item  Etudier l'existence et la valeur éventuelle de dérivées partielles d'ordre 1 sur $\R^2$.
				\item Étudier $\frac{\partial^2f}{\partial x\partial y}$ et $\frac{\partial^2f}{\partial y\partial x}$ en $(0,0)$.
			\end{enumerate}
			
			\centering
			\rule{1\linewidth}{0.6pt}
		\end{exo}
		
		
	\end{minipage}	
	\hfill\vrule\hfill
	\begin{minipage}[t]{0.48\linewidth}
		\raggedright
		
		
		\begin{exo}\quad\\[0.2cm]
			Trouver les extrema locaux de 
			
			\begin{enumerate}
				\item  $\begin{array}[t]{cccc}
				f~:&\R^2&\rightarrow&\R\\
				&(x,y)&\mapsto&x^2+xy+y^2-3x-6y
				\end{array}$ 
				\item  $\begin{array}[t]{cccc}
				f~:&\R^2&\rightarrow&\R\\
				&(x,y)&\mapsto&x^2+2y^2-2xy-2y+1
				\end{array}$
				\item  $\begin{array}[t]{cccc}
				f~:&\R^2&\rightarrow&\R\\
				&(x,y)&\mapsto&x^4+y^4-4xy
				\end{array}$ 
			\end{enumerate}
			
			\centering
			\rule{1\linewidth}{0.6pt}
		\end{exo}
		
		
		\begin{exo}\quad\\[0.2cm] %https://www.bibmath.net/ressources/index.php?action=affiche&quoi=bde/analyse/calculdiff/extrema&type=fexo
			On désire fabriquer une boite ayant la forme d'un parallélépipède rectangle, sans couvercle sur le dessus. Le volume de cette boite doit être égal à $0,5m^3$ et pour optimiser la quantité de mâtière utilisée, on désire que la somme des aires des faces soit aussi petite que possible. Quelles dimensions doit-on choisir pour fabriquer la boite?
			
			\centering
			\rule{1\linewidth}{0.6pt}
		\end{exo}
		
			
		
		
		
		
		
		
		
	\end{minipage}
\end{minipage}


\end{document}