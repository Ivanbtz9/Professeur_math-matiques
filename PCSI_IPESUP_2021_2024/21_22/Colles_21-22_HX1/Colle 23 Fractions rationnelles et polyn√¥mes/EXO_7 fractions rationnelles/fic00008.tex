
%%%%%%%%%%%%%%%%%% PREAMBULE %%%%%%%%%%%%%%%%%%

\documentclass[11pt,a4paper]{article}

\usepackage{amsfonts,amsmath,amssymb,amsthm}
\usepackage[utf8]{inputenc}
\usepackage[T1]{fontenc}
\usepackage[francais]{babel}
\usepackage{mathptmx}
\usepackage{fancybox}
\usepackage{graphicx}
\usepackage{ifthen}

\usepackage{tikz}   

\usepackage{hyperref}
\hypersetup{colorlinks=true, linkcolor=blue, urlcolor=blue,
pdftitle={Exo7 - Exercices de mathématiques}, pdfauthor={Exo7}}

\usepackage{geometry}
\geometry{top=2cm, bottom=2cm, left=2cm, right=2cm}

%----- Ensembles : entiers, reels, complexes -----
\newcommand{\Nn}{\mathbb{N}} \newcommand{\N}{\mathbb{N}}
\newcommand{\Zz}{\mathbb{Z}} \newcommand{\Z}{\mathbb{Z}}
\newcommand{\Qq}{\mathbb{Q}} \newcommand{\Q}{\mathbb{Q}}
\newcommand{\Rr}{\mathbb{R}} \newcommand{\R}{\mathbb{R}}
\newcommand{\Cc}{\mathbb{C}} \newcommand{\C}{\mathbb{C}}
\newcommand{\Kk}{\mathbb{K}} \newcommand{\K}{\mathbb{K}}

%----- Modifications de symboles -----
\renewcommand{\epsilon}{\varepsilon}
\renewcommand{\Re}{\mathop{\mathrm{Re}}\nolimits}
\renewcommand{\Im}{\mathop{\mathrm{Im}}\nolimits}
\newcommand{\llbracket}{\left[\kern-0.15em\left[}
\newcommand{\rrbracket}{\right]\kern-0.15em\right]}
\renewcommand{\ge}{\geqslant} \renewcommand{\geq}{\geqslant}
\renewcommand{\le}{\leqslant} \renewcommand{\leq}{\leqslant}

%----- Fonctions usuelles -----
\newcommand{\ch}{\mathop{\mathrm{ch}}\nolimits}
\newcommand{\sh}{\mathop{\mathrm{sh}}\nolimits}
\renewcommand{\tanh}{\mathop{\mathrm{th}}\nolimits}
\newcommand{\cotan}{\mathop{\mathrm{cotan}}\nolimits}
\newcommand{\Arcsin}{\mathop{\mathrm{arcsin}}\nolimits}
\newcommand{\Arccos}{\mathop{\mathrm{arccos}}\nolimits}
\newcommand{\Arctan}{\mathop{\mathrm{arctan}}\nolimits}
\newcommand{\Argsh}{\mathop{\mathrm{argsh}}\nolimits}
\newcommand{\Argch}{\mathop{\mathrm{argch}}\nolimits}
\newcommand{\Argth}{\mathop{\mathrm{argth}}\nolimits}
\newcommand{\pgcd}{\mathop{\mathrm{pgcd}}\nolimits} 

%----- Structure des exercices ------

\newcommand{\exercice}[1]{\video{0}}
\newcommand{\finexercice}{}
\newcommand{\noindication}{}
\newcommand{\nocorrection}{}

\newcounter{exo}
\newcommand{\enonce}[2]{\refstepcounter{exo}\hypertarget{exo7:#1}{}\label{exo7:#1}{\bf Exercice \arabic{exo}}\ \  #2\vspace{1mm}\hrule\vspace{1mm}}

\newcommand{\finenonce}[1]{
\ifthenelse{\equal{\ref{ind7:#1}}{\ref{bidon}}\and\equal{\ref{cor7:#1}}{\ref{bidon}}}{}{\par{\footnotesize
\ifthenelse{\equal{\ref{ind7:#1}}{\ref{bidon}}}{}{\hyperlink{ind7:#1}{\texttt{Indication} $\blacktriangledown$}\qquad}
\ifthenelse{\equal{\ref{cor7:#1}}{\ref{bidon}}}{}{\hyperlink{cor7:#1}{\texttt{Correction} $\blacktriangledown$}}}}
\ifthenelse{\equal{\myvideo}{0}}{}{{\footnotesize\qquad\texttt{\href{http://www.youtube.com/watch?v=\myvideo}{Vidéo $\blacksquare$}}}}
\hfill{\scriptsize\texttt{[#1]}}\vspace{1mm}\hrule\vspace*{7mm}}

\newcommand{\indication}[1]{\hypertarget{ind7:#1}{}\label{ind7:#1}{\bf Indication pour \hyperlink{exo7:#1}{l'exercice \ref{exo7:#1} $\blacktriangle$}}\vspace{1mm}\hrule\vspace{1mm}}
\newcommand{\finindication}{\vspace{1mm}\hrule\vspace*{7mm}}
\newcommand{\correction}[1]{\hypertarget{cor7:#1}{}\label{cor7:#1}{\bf Correction de \hyperlink{exo7:#1}{l'exercice \ref{exo7:#1} $\blacktriangle$}}\vspace{1mm}\hrule\vspace{1mm}}
\newcommand{\fincorrection}{\vspace{1mm}\hrule\vspace*{7mm}}

\newcommand{\finenonces}{\newpage}
\newcommand{\finindications}{\newpage}


\newcommand{\fiche}[1]{} \newcommand{\finfiche}{}
%\newcommand{\titre}[1]{\centerline{\large \bf #1}}
\newcommand{\addcommand}[1]{}

% variable myvideo : 0 no video, otherwise youtube reference
\newcommand{\video}[1]{\def\myvideo{#1}}

%----- Presentation ------

\setlength{\parindent}{0cm}

\definecolor{myred}{rgb}{0.93,0.26,0}
\definecolor{myorange}{rgb}{0.97,0.58,0}
\definecolor{myyellow}{rgb}{1,0.86,0}

\newcommand{\LogoExoSept}[1]{  % input : echelle       %% NEW
{\usefont{U}{cmss}{bx}{n}
\begin{tikzpicture}[scale=0.1*#1,transform shape]
  \fill[color=myorange] (0,0)--(4,0)--(4,-4)--(0,-4)--cycle;
  \fill[color=myred] (0,0)--(0,3)--(-3,3)--(-3,0)--cycle;
  \fill[color=myyellow] (4,0)--(7,4)--(3,7)--(0,3)--cycle;
  \node[scale=5] at (3.5,3.5) {Exo7};
\end{tikzpicture}}
}


% titre
\newcommand{\titre}[1]{%
\vspace*{-4ex} \hfill \hspace*{1.5cm} \hypersetup{linkcolor=black, urlcolor=black} 
\href{http://exo7.emath.fr}{\LogoExoSept{3}} 
 \vspace*{-5.7ex}\newline 
\hypersetup{linkcolor=blue, urlcolor=blue}  {\Large \bf #1} \newline 
 \rule{12cm}{1mm} \vspace*{3ex}}

%----- Commandes supplementaires ------



\begin{document}

%%%%%%%%%%%%%%%%%% EXERCICES %%%%%%%%%%%%%%%%%%
\fiche{f00008, bodin, 2007/09/01} 

\titre{Fractions rationnelles}

Corrections de Léa Blanc-Centi. 

\section{Fractions rationnelles}

\exercice{6964, blanc-centi, 2014/04/08}
\video{f93zdMxazWw}
\enonce{006964}{}

Existe-t-il une fraction rationnelle $F$ telle que 
$$\big(F(X)\big)^2=(X^2+1)^3 \ \ ?$$

\finenonce{006964} 



\finexercice\exercice{6965, blanc-centi, 2014/04/08}
\video{7jn7oh2DdnQ}
\enonce{006965}{}
Soit $F=\frac{P}{Q}$ une fraction rationnelle écrite sous forme irréductible. 
On suppose qu'il existe une fraction rationnelle $G$ telle que 
$$G\left(\frac{P(X)}{Q(X)}\right)=X$$
\begin{enumerate}
\item Si $G=\frac{a_nX^n+\cdots+a_1X+a_0}{b_nX^n+\cdots+b_1X+b_0}$, montrer que $P$ divise $(a_0-b_0X)$ et que $Q$ divise $(a_n-b_nX)$.
\item En déduire que $F=\frac{P}{Q}$ est de la forme $F(X)=\frac{aX+b}{cX+d}$.
\item Pour $Y=\frac{aX+b}{cX+d}$, exprimer $X$ en fonction de $Y$. En déduire l'expression de $G$.
\end{enumerate}
\finenonce{006965} 


\finexercice

\exercice{6966, blanc-centi, 2014/04/08}
\video{BDJWi5Od7uA}
\enonce{006966}{}
Soit $n\in\Nn^*$ et $P(X)=c(X-a_1)\cdots(X-a_n)$ 
(où les $a_i$ sont des nombres complexes et où $c\not=0$).
\begin{enumerate}
\item Exprimer à l'aide de $P$ et de ses dérivées les sommes suivantes:
$$\sum_{k=1}^n\frac{1}{X-a_k}\quad\quad
\sum_{k=1}^n\frac{1}{(X-a_k)^2}\quad\quad
\sum_{\substack{1\le k,\ell\le n \\ k\not= \ell}}\frac{1}{(X-a_k)(X-a_\ell)}$$
\item Montrer que si $z$ est racine de $P'$ mais pas de $P$, alors il existe 
$\lambda_1,\hdots,\lambda_n$ des réels positifs ou nuls tels que 
$\sum_{k=1}^n\lambda_k=1$ et $z=\sum_{k=1}^n\lambda_ka_k$. 
Si toutes les racines de $P$ sont réelles, que peut-on en déduire sur les racines de $P'$ ?
\end{enumerate}
\finenonce{006966} 


\finexercice
\section{Décompositions en éléments simples}

\exercice{6967, exo7, 2014/04/08}
\video{KHEyahxXAKk}
\enonce{006967}{}
% De 444, cousquer
Décomposer les fractions suivantes en éléments simples sur $\Rr$, 
par identification des coefficients.
\begin{enumerate}
\item $F=\frac{X}{X^2-4}$
\item $G=\frac{X^3-3X^2+X-4}{X-1}$
\item $H=\frac{2X^3+X^2-X+1}{X^2-2X+1}$
\item $K=\frac{X+1}{X^4+1}$
\end{enumerate}
\finenonce{006967} 


\finexercice

\exercice{6968, exo7, 2014/04/08}
\video{cMc2VRPKaKI}
\enonce{006968}{}
Décomposer les fractions suivantes en éléments simples sur $\Rr$, 
en raisonnant par substitution pour obtenir les coefficients.
\begin{enumerate}
\item % De 444, cousquer
$F=\frac{X^5+X^4+1}{X^3-X}$

\item % De 444, cousquer
$G=\frac{X^3+X+1}{(X-1)^3(X+1)}$

\item % De 444, cousquer
$H=\frac{X}{(X^2+1)(X^2+4)}$

\item % De 445, cousquer
$K=\frac{2X^4+X^3+3X^2-6X+1}{2X^3-X^2}$
\end{enumerate}

\finenonce{006968} 


\finexercice

\exercice{6969, exo7, 2014/04/08}
\video{S53f12bRBhE}
\enonce{006969}{}
Décomposer les fractions suivantes en éléments simples sur $\Rr$.
\begin{enumerate}
\item \`A l'aide de divisions euclidiennes successives :
% De 447, cousquer
$$F=\frac{4X^6-2X^5+11X^4-X^3+11X^2+2X+3}{X(X^2+1)^3}$$

\item \`A l'aide d'une division selon les puissances croissantes :
% De 446, cousquer
$$G=\frac{4X^4-10X^3+8X^2-4X+1}{X^3(X-1)^2}$$

\item Idem pour :
% De 444, cousquer
$$H=\frac{X^4+2X^2+1}{X^5-X^3}$$

\item A l'aide du changement d'indéterminée $X=Y+1$ :
% De 444, cousquer
$$K=\frac{X^5+X^4+1}{X(X-1)^4}$$

\end{enumerate}
\finenonce{006969} 


\finexercice
\exercice{6970, exo7, 2014/04/08}
\video{ocbw3l8Wwl0}
\enonce{006970}{}
% De 444, cousquer
\begin{enumerate}
\item Décomposer les fractions suivantes en éléments simples sur $\Cc$.
$$\frac{(3-2 i )X-5+3 i }{X^2+ i  X+2} \qquad\qquad \frac{X+ i }{X^2+ i } \qquad\qquad \frac{2X}{(X+ i )^2}$$

\item Décomposer les fractions suivantes en éléments simples sur $\Rr$, puis sur $\Cc$.
$$\frac{X^5+X+1}{X^4-1} \qquad\qquad \frac{X^2-3}{(X^2+1)(X^2+4)} \qquad\qquad \frac{X^2+1}{X^4+1}$$
\end{enumerate}
\finenonce{006970} 


\finexercice
\section{Applications}

\exercice{6971, blanc-centi, 2014/04/08}
\video{FPyfOQL32Hg}
\enonce{006971}{}
On pose $Q_0=(X-1)(X-2)^2$, $Q_1=X(X-2)^2$ et $Q_2=X(X-1)$. 
\`A l'aide de la décomposition en éléments simples de $\frac{1}{X(X-1)(X-2)^2}$, 
trouver des polynômes $A_0,\ A_1,\ A_2$ tels que $A_0Q_0+A_1Q_1+A_2Q_2=1$. 
Que peut-on en déduire sur $Q_1$, $Q_2$ et $Q_3$?
\finenonce{006971} 


\finexercice
\exercice{6972, blanc-centi, 2014/04/08}
\enonce{006972}{}
Soit $T_n(x)=\cos\big(n \arccos(x)\big)$ pour $x\in [-1,1]$.
\begin{enumerate}
  \item 
  \begin{enumerate}
    \item Montrer que pour tout $\theta\in[0,\pi]$, $T_n(\cos\theta)=\cos(n\theta)$.
    \item Calculer $T_0$ et $T_1$.
    \item Montrer la relation de récurrence $T_{n+2}(x) = 2xT_{n+1}(x)-T_n(x)$, pour tout $n \ge0$.
    \item En déduire que $T_n$ une fonction polynomiale de degré $n$.
  \end{enumerate}

  \item Soit $P(X)=\lambda(X-a_1)\cdots(X-a_n)$ un polynôme, où les $a_k$ sont deux à deux distincts 
et $\lambda\not=0$. Montrer que 
$$\frac{1}{P(X)}=\sum_{k=1}^n\frac{\frac{1}{P'(a_k)}}{X-a_k}$$

\item Décomposer $\frac{1}{T_n}$ en éléments simples.
\end{enumerate}
\finenonce{006972} 


\finexercice

\finfiche

 \finenonces 



 \finindications 

\indication{006964}
\'Ecrire $F=\frac{P}{Q}$ sous forme irréductible.
\finindication
\indication{006965}
\'Ecrire $G=\frac{A}{B}$ sous forme irréductible (on pourra choisir par exemple 
$n=\mathrm{max}(\mathrm{deg}A,\mathrm{deg}B)$).
\finindication
\indication{006966}
Considérer $P'/P$ et sa dérivée, et enfin $P''/P$.
\finindication
\indication{006967}
Pour $G$ et $H$, commencer par faire une division euclidienne 
pour trouver la partie polynomiale.
\finindication
\indication{006968}
Les fractions $F, K$ ont une partie polynomiale, elles s'écrivent

$F=X^2+X+1+\frac{X^2+X+1}{X^3-X}$

$K=X+1+\frac{4X^2-6X+1}{2X^3-X^2}$
\finindication
\indication{006969}
Pour $F$, commencer par écrire $F=\frac{a}{X}+F_1$ où $F_1=\frac{N}{(X^2+1)^3}$ puis diviser $N$ par $X^2+1$.
Pour $K$, commencer par obtenir $K=1+\frac{1}{X}+K_1$, puis faire le changement d'indéterminée dans $K_1$.
\finindication
\noindication
\noindication
\indication{006972}
Pour 1. exprimer $\cos\big((n+2)\theta\big)$ et $\cos(n\theta)$ 
en fonction de $\cos\big((n+1)\theta\big)$.
Pour 3. chercher les racines de $T_n$ : 
$\omega_k=\cos\left(\frac{(2k+1)\pi}{2n}\right)$ pour $k=0,\ldots,n-1$.
\finindication


\newpage

\correction{006964}
\'Ecrivons $F(X)=\frac{P(X)}{Q(X)}$ avec $P$ et $Q$ deux polynômes premiers entre eux, avec $Q$ unitaire.
La condition $\big(F(X)\big)^2=(X^2+1)^3$ devient $P^2=(X^2+1)^3Q^2$. 
Ainsi $Q^2$ divise $P^2$. D'où $Q^2=1$, puisque $P^2$ et $Q^2$ sont premiers entre eux. 
Donc $Q=1$ (ou $-1$). Ainsi $F=P$ est un polynôme et $P^2=(X^2+1)^3$. 

En particulier $P^2$ est de degré $6$, donc $P$ doit être de degré 3. 
\'Ecrivons $P=aX^3+bX^2+cX+d$,
on développe l'identité $P^2=(X^2+1)^3$ :

{\small
$$\begin{array}{c}
X^6+3X^4+3X^2+1 = \\
a^2X^6 + 2abX^5 + (2ac+b^2)X^4 + (2ad+2bc)X^3 + (2bd+c^2)X^2 + 2cdX+d^2
 \end{array}
$$
}

On identifie les coefficients :
pour le coefficient de $X^6$, on a $a=\pm1$,
puis pour le coefficient de $X^5$, on a $b =0$ ;
pour le coefficient de $1$, on a $d=\pm 1$, 
puis pour le coefficient de $X$, on a $c=0$.
Mais alors le coefficient de $X^3$ doit vérifier $2ad+2bc=0$, ce qui est faux.

Ainsi aucun polynôme ne vérifie l'équation $P^2=(X^2+1)^3$, et 
par le raisonnement du début, aucune fraction non plus.
\fincorrection
\correction{006965}\ 
\begin{enumerate}
\item Posons $G=\frac{A}{B}$ et $F = \frac{P}{Q}$ (avec $A,B,P,Q$ des polynômes).
On réécrit l'identité $G(F(X))=X$ sous la forme $A(F(X))=XB(F(X))$. 
Posons $n=\mathrm{max}(\mathrm{deg}A,\mathrm{deg}B)$.
Alors $n\ge 1$ car sinon, 
$A$ et $B$ seraient constants et $G(\frac{P}{Q})=X$ aussi.

On a donc $A=\sum_{k=0}^na_kX^k$ et $B=\sum_{k=0}^nb_kX^k$, où $(a_n,b_n)\neq(0,0)$, et l'identité devient
$$\sum_{k=0}^na_k\left(\frac{P}{Q}\right)^k=X\sum_{k=0}^nb_k\left(\frac{P}{Q}\right)^k$$
En multipliant par $Q^n$, cela donne
$$\sum_{k=0}^n a_kP^kQ^{n-k}= \sum_{k=0}^n b_k X P^kQ^{n-k}.$$
Donc 
$$(a_0-b_0X)Q^n \quad + \quad (\cdots + (a_k-b_kX) P^kQ^{n-k} + \cdots)\quad + \quad (a_n-b_nX)P^n = 0$$
où les termes dans la parenthèse centrale sont tous divisibles par $P$ et par $Q$. 
Comme $Q$ divise aussi le premier terme, alors $Q$ divise $(a_n-b_nX)P^n$.
D'après le lemme de Gauss, puisque $P$ et $Q$ sont premiers entre eux, alors $Q$ divise $(a_n-b_nX)$. 
De même, $P$ divise tous les termes de la parenthèse centrale et le dernier, donc $P$ divise aussi $(a_0-b_0X)Q^n$, 
donc $P$ divise $(a_0-b_0X)$.

\item Supposons de plus qu'on a écrit $G=\frac{A}{B}$ sous forme irréductible, 
c'est-à-dire avec $\pgcd(A,B)=1$. 
Vu que $a_n$ et $b_n$ ne sont pas tous les deux nuls, alors $a_n-b_nX$ n'est pas le polynôme nul.
Comme $Q$ divise $a_n-b_nX$ alors nécessairement $Q$ est de degré au plus $1$ ; on écrit $Q(X)=cX+d$. 
Par ailleurs, $a_0-b_0X$ n'est pas non plus le polynôme nul, car sinon on aurait $a_0=b_0=0$ et 
donc $A$ et $B$ seraient tous les deux sans terme constant, donc divisibles par $X$ 
(ce qui est impossible puisqu'ils sont premiers entre eux).  
Donc $P$ est aussi de degré au plus $1$ et on écrit $P(X)=aX+b$. 
Conclusion : $F(X)=\frac{aX+b}{cX+d}$.
Notez que $a$ et $b$ ne sont pas tous les deux nuls en même temps (de même pour $b$ et $d$).

\item Si $Y = \frac{aX+b}{cX+d}$ avec $(a,b) \neq (0,0)$,
alors $X = -\frac{dY-b}{cY-a}$. 
Autrement dit si on note
$\phi(X)= \frac{aX+b}{cX+d}$, alors sa bijection réciproque est 
$\phi^{-1}(Y) = -\frac{dY-b}{cY-a}$.

Nous avons prouvé que $G\left( \frac{aX+b}{cX+d}\right) =X$.
Cette identité s'écrit $G\big( \phi(X) \big)=X$.
Appliquée en $X = \phi^{-1}(Y)$ elle devient
$G\big( \phi( \phi^{-1}(Y) ) \big)=\phi^{-1}(Y)$, c'est-à-dire
$G(Y) = \phi^{-1}(Y)$.
Ainsi $G(Y) = -\frac{dY-b}{cY-a}$.
\end{enumerate}
\fincorrection
\correction{006966}
\begin{enumerate}
\item 
  \begin{enumerate}
    \item Puisque $P(X)=c(X-a_1)\cdots(X-a_n)$ : 
\begin{eqnarray*}
P'(X)=&c(X-a_2)\cdots(X-a_n)+c(X-a_1)(X-a_3)\cdots(X-a_n)\\
 &+\cdots+c(X-a_1)\cdots(X-a_{k-1})(X-a_{k+1})\cdots(X-a_n)\\
 &\ \ \ \ \ \ +\cdots+c(X-a_1)\cdots(X-a_{n-1})
\end{eqnarray*}
La dérivée est donc la somme des termes de la forme : $\frac{c(X-a_1)\cdots(X-a_n)}{X-a_k} = \frac{P(X)}{X-a_k}$.

Ainsi 
$$P'(X) = \frac{P(X)}{X-a_1}+ \cdots + \frac{P(X)}{X-a_k}+ \cdots + \frac{P(X)}{X-a_n}.$$
Donc :
$$\frac{P'}{P}=\sum_{k=1}^n\frac{1}{X-a_k}$$

  \item Puisque $\sum_{k=1}^n\frac{1}{(X-a_k)^2}$ est la dérivée de $-\sum_{k=1}^n\frac{1}{X-a_k}$, on obtient
  par dérivation de $-\frac{P'}{P}$ :
$$\frac{P'^2-PP''}{P^2}=\sum_{k=1}^n\frac{1}{(X-a_k)^2}$$

  \item On a remarqué que la dérivée de $P'$ est la somme de facteurs $c(X-a_1)\cdots(X-a_n)$
avec un des facteurs en moins, donc de la forme $\frac{c(X-a_1)\cdots(X-a_n)}{X-a_k} = \frac{P}{X-a_k}$. 
De même $P''$ est la somme de facteurs $c(X-a_1)\cdots(X-a_n)$
avec deux facteurs en moins, c'est-à-dire de la forme $\frac{c(X-a_1)\cdots(X-a_n)}{(X-a_k)(X-a_\ell)} = \frac{P}{(X-a_k)(X-a_\ell)}$ :
$$P'' = \sum_{\substack{1\le k,\ell\le n \\ k\not= \ell}}\frac{P}{(X-a_k)(X-a_\ell)} \quad \text{ donc } \quad
\frac{P''}{P} = \sum_{\substack{1\le k,\ell\le n \\ k\not= \ell}}\frac{1}{(X-a_k)(X-a_\ell)}$$

% \begin{eqnarray*}
% \sum_{k\not=l}\frac{1}{(X-a_k)(X-a_l)}&=&\sum_{k=1}^n\frac{1}{X-a_k}\left(\sum_{l\not= k}\frac{1}{X-a_l}\right)\\
%  &=&\sum_{k=1}^n\frac{1}{X-a_k}\left(\sum_{l=1}^n\frac{1}{X-a_l}-\frac{1}{X-a_k}\right)\\
%  &=&\sum_{k=1}^n\frac{1}{X-a_k}\frac{P'}{P}-\sum_{k=1}^n\frac{1}{(X-a_k)^2}\\
%  &=&\frac{P'^2}{P^2}-\frac{P'^2-PP''}{P^2}=\frac{P''}{P}
% \end{eqnarray*}
  \end{enumerate}

\item On applique l'identité  $\frac{P'(X)}{P(X)}=\sum_{k=1}^n\frac{1}{X-a_k}$ en $z$ avec les hypothèses 
$P(z)\not=0$ et $P'(z)=0$.
On en déduit $\displaystyle \sum_{k=1}^n\frac{1}{z-a_k} = 0$.
L'expression conjuguée est aussi nulle :
$$\sum_{k=1}^n\frac{1}{\overline{z-a_k}}=\sum_{k=1}^n\frac{z-a_k}{|z-a_k|^2} = 0$$
Posons $\mu_k = \frac{1}{|z-a_k|^2}$.
Alors 
$$\sum_{k=1}^n \mu_k(z-a_k) = 0 \quad \text{ donc } \left(\sum_{k=1}^n \mu_k \right) z = \sum_{k=1}^n \mu_k a_k$$
Posons $\lambda_k = \mu_k / \left(\sum_{k=1}^n \mu_k \right)$,
alors : 
  \begin{itemize}
    \item Les $\lambda_k$ sont des réels positifs.
    \item $\sum_{k=1}^n\lambda_k=1$
    \item Et $z=\sum_{k=1}^n\lambda_ka_k$.
  \end{itemize}

En particulier si les $a_k$ sont tous des nombres réels, alors $z$ est aussi un nombre réel.
On vient de prouver que si un polynôme $P$ a toutes ses racines réelles, 
alors $P'$ a aussi toutes ses racines réelles. On a même plus : si on ordonne les racines réelles 
de $P$ en $a_1 \le a_2 \le \cdots \le a_n$ alors une racine $z$ de $P'$ est réelle et vérifie $a_1 \le z \le a_n$.

Plus généralement, l'interprétation géométrique de ce que l'on vient de prouver s'appelle le théorème de Gauss-Lucas :
<<Les racines de $P'$ sont dans l'enveloppe convexe des racines (réelles ou complexes) de $P$.>>
\end{enumerate}
\fincorrection
\correction{006967}\
\begin{enumerate}
\item $F=\frac{X}{X^2-4}$. 

Commençons par factoriser le dénominateur : $X^2-4=(X-2)(X+2)$, 
d'où une décomposition en éléments simples du type
$F=\frac{a}{X-2}+\frac{b}{X+2}$. 
En réduisant au même dénominateur, il vient
$\frac{X}{X^2-4}=\frac{(a+b)X+2(a-b)}{X^2-4}$ et en identifiant les coefficients, on obtient le système
$\left\{\begin{array}{l}a+b=1\\2(a-b)=0\end{array}\right.$. Ainsi $a=b=\frac{1}{2}$ et 
$$\frac{X}{X^2-4} = \frac{\frac12}{X-2} + \frac{\frac12}{X+2}$$

\item $G=\frac{X^3-3X^2+X-4}{X-1}$. 

Lorsque le degré du numérateur (ici $3$) est supérieur 
ou égal au degré du dénominateur (ici $1$), il
faut effectuer la division euclidienne du numérateur par le dénominateur pour 
faire apparaître la partie polynomiale (ou partie entière).
Ici la division euclidienne s'écrit $X^3-3X^2+X-4=(X-1)(X^2-2X-1)-5$. 
Ainsi en divisant les deux membres par $X-1$ on obtient 
$$\frac{X^3-3X^2+X-4}{X-1} = X^2-2X-1-\frac{5}{X-1}$$
La fraction est alors déjà décomposée en éléments simples.

\item $H=\frac{2X^3+X^2-X+1}{X^2-2X+1}$. 

Commençons par faire la division euclidienne du numérateur par le dénominateur :
$2X^3+X^2-X+1=(X^2-2X+1)(2X+5)+7X-4$, ce qui donne
$H=2X+5+\frac{7X-4}{X^2-2X+1}$. 
Il reste à décomposer en éléments simples la fraction rationnelle $H_1=\frac{7X-4}{X^2-2X+1}$. 
Puisque le dénominateur se factorise en $(X-1)^2$, elle sera de la forme 
$H_1=\frac{a}{(X-1)^2}+\frac{b}{X-1}$.
En réduisant au même dénominateur, il vient
$\frac{7X-4}{X^2-2X+1}=\frac{bX+a-b}{X^2-2X+1}$ et en identifiant les coefficients, on obtient 
$b=7$ et $a=3$. Finalement,
$$\frac{2X^3+X^2-X+1}{X^2-2X+1} =
2X+5+\frac{3}{(X-1)^2}+\frac{7}{X-1}$$

\item $K=\frac{X+1}{X^4+1}$. 

Ici, il n'y a pas de partie polynomiale puisque le degré du numérateur 
est strictement inférieur au degré du dénominateur. 
Le dénominateur admet quatre racines complexes 
$e^{\frac{i\pi}{4}}$, $e^{\frac{3i\pi}{4}}$, $e^{\frac{5i\pi}{4}}=e^{-\frac{3i\pi}{4}}$ et 
$e^{\frac{7i\pi}{4}}=e^{-\frac{i\pi}{4}}$. 
En regroupant les racines complexes conjuguées, on obtient sa factorisation sur $\Rr$:
\begin{eqnarray*}
X^4+1 &=& \big( (X-e^{\frac{i\pi}{4}})(X-e^{-\frac{i\pi}{4}}) \big)\big( (X-e^{\frac{3i\pi}{4}})(X-e^{-\frac{3i\pi}{4}}) \big) \\
      &=&\big(X^2-2\cos\tfrac{\pi}{4}+1\big)\big(X^2-2\cos\tfrac{3\pi}{4}+1\big)\\
      &=&(X^2-\sqrt{2}X+1)(X^2+\sqrt{2}X+1)
\end{eqnarray*}
Puisque les deux facteurs $(X^2-\sqrt{2}X+1)$ et $(X^2+\sqrt{2}X+1)$ sont irréductibles sur $\R$, la décomposition en éléments simples de $K$ est de la forme
$K=\frac{aX+b}{X^2-\sqrt{2}X+1}+\frac{cX+d}{X^2+\sqrt{2}X+1}$.

En réduisant au même dénominateur et en identifiant les coefficients avec ceux de $K=\frac{X+1}{X^4+1}$, 
on obtient le système
$$\left\{\begin{array}{l}
a+c=0\\
\sqrt{2}a+b-\sqrt{2}c+d=0\\
a+\sqrt{2}b+c-\sqrt{2}d=1\\
b+d=1
\end{array}\right.$$
Système que l'on résout en $a=\frac{-\sqrt{2}}{4}$, $c=\frac{\sqrt{2}}{4}$, $b=\frac{2+\sqrt{2}}{4}$ et $d=\frac{2-\sqrt{2}}{4}$. Ainsi
$$\frac{X+1}{X^4+1} = \frac{-\frac{\sqrt{2}}{4}X+\frac{2+\sqrt{2}}{4}}{X^2-\sqrt{2}X+1} + \frac{\frac{\sqrt{2}}{4}X+\frac{2-\sqrt{2}}{4}}{X^2+\sqrt{2}X+1}$$ 
\end{enumerate}
\fincorrection
\correction{006968}\ 
\begin{enumerate}

\item $F=\frac{X^5+X^4+1}{X^3-X}$.

Pour obtenir la partie polynomiale, on fait une division euclidienne : 
$X^5+X^4+1=(X^3-X)(X^2+X+1)+ X^2+X+1$. Ce qui donne $F=X^2+X+1+F_1$, 
où $F_1=\frac{X^2+X+1}{X^3-X}$. Puisque $X^3-X=X(X-1)(X+1)$, 
la décomposition en éléments simples est de la forme 
$$F_1 = \frac{X^2+X+1}{X(X-1)(X+1)}=\frac{a}{X}+\frac{b}{X-1}+\frac{c}{X+1}$$

Pour obtenir $a$ :
\begin{itemize}
  \item on multiplie l'égalité par $X$ : $\frac{X(X^2+X+1)}{X(X-1)(X+1)}=X \left(\frac{a}{X}+\frac{b}{X-1}+\frac{c}{X+1}\right)$,
  \item on simplifie $\frac{X^2+X+1}{(X-1)(X+1)}= a+\frac{bX}{X-1}+\frac{cX}{X+1}$,
  \item on remplace $X$ par $0$ et on obtient $-1= a+0+0$, donc $a=-1$.
\end{itemize}

 De même, en multipliant par $X-1$ et en remplaçant $X$ par $1$, il vient $b=\frac{3}{2}$.
 Puis en multipliant par $X+1$ et en remplaçant $X$ par $-1$, on trouve $c=\frac{1}{2}$.

D'où
$$\frac{X^5+X^4+1}{X^3-X} = X^2+X+1-\frac{1}{X}+\frac{\tfrac12}{X+1}+\frac{\tfrac32}{X-1}$$

\item $G=\frac{X^3+X+1}{(X-1)^3(X+1)}$. 

La partie polynomiale est nulle. La décomposition en éléments simples est de la forme
$G=\frac{a}{(X-1)^3}+\frac{b}{(X-1)^2}+\frac{c}{X-1}+\frac{d}{X+1}$.

\begin{itemize}
  \item En multipliant les deux membres de l'égalité par $(X-1)^3$, 
en simplifiant puis en remplaçant $X$ par $1$, 
on obtient $a=\frac32$. 

  \item De même, en multipliant par $X+1$, 
en simplifiant puis en remplaçant $X$ par $-1$, 
on obtient $d=\frac18$.

  \item En multipliant par $X$ et en regardant la limite 
  quand $X\to +\infty$, on obtient $1=c+d$. Donc $c=\frac78$.

  \item En remplaçant $X$ par $0$, il vient $-1=-a+b-c+d$.
  Donc $b = \frac54$.
\end{itemize}

Ainsi :
$$G = \frac{X^3+X+1}{(X-1)^3(X+1)} = \frac{\tfrac32}{(X-1)^3} + \frac{\tfrac54}{(X-1)^2} 
+ \frac{\tfrac78}{X-1}  + \frac{\tfrac18}{X+1}$$

\item $H=\frac{X}{(X^2+1)(X^2+4)}$.

Puisque $X^2+1$ et $X^2+4$ sont irréductibles sur $\Rr$, la décomposition en éléments simples sera de la forme 
$$\frac{X}{(X^2+1)(X^2+4)} = \frac{aX+b}{X^2+1} + \frac{cX+d}{X^2+4}$$

  \begin{itemize}
    \item En remplaçant $X$ par $0$, on obtient $0=b+\frac{1}{4}d$.
    \item En multipliant les deux membres par $X$, on obtient 
    $\frac{X^2}{(X^2+1)(X^2+4)} = \frac{aX^2+bX}{X^2+1} + \frac{cX^2+dX}{X^2+4}$.
    En calculant la limite quand $X\to +\infty$, on a $0=a+c$.
    \item Enfin, en évaluant les fractions en $X=1$ et $X=-1$, 
    on obtient $\frac{1}{10}=\frac{a+b}{2}+\frac{c+d}{5}$ et 
    $\frac{-1}{10}=\frac{-a+b}{2}+\frac{-c+d}{5}$.
  \end{itemize}

La résolution du système donne $b=d=0$, $a=\frac{1}{3}$, $c=-\frac{1}{3}$ et donc 
$$\frac{X}{(X^2+1)(X^2+4)} = \frac{\frac{1}{3}X}{X^2+1} - \frac{\frac{1}{3}X}{X^2+4}$$

\item $K=\frac{2X^4+X^3+3X^2-6X+1}{2X^3-X^2}$.

Pour la partie polynomiale, on fait la division euclidienne:
$$2X^4+X^3+3X^2-6X+1=(2X^3-X^2)(X+1)+(4X^2-6X+1)$$ 
ce qui donne 
$K=X+1+K_1$ où $K_1=\frac{4X^2-6X+1}{2X^3-X^2}$.
Pour trouver la décomposition en éléments simples de $K_1$, on factorise son numérateur: 
$2X^3-X^2=2X^2(X-\frac{1}{2})$, ce qui donne une décomposition de la forme
$K_1=\frac{a}{X^2}+\frac{b}{X}+\frac{c}{X-\frac{1}{2}}$.

On obtient alors $a$ en multipliant les deux membres de l'égalité par
$X^2$ puis en remplaçant $X$ par 0: $a=-1$. On obtient de
m\^eme $c$ en multipliant par $X-\frac{1}{2}$ et en remplaçant $X$
par $\frac{1}{2}$: $c=-2$. Enfin on trouve $b$ en identifiant pour une
valeur particuli\`ere non encore utilisée, par exemple $X=1$, ou mieux en
multipliant les deux membres par $X$ et en passant \`a la limite pour
$X\to+\infty$: $b=4$. Finalement:
$$\frac{2X^4+X^3+3X^2-6X+1}{2X^3-X^2}=X+1-\frac{1}{X^2}+\frac{4}{X}-\frac{2}{X-\frac{1}{2}}$$
\end{enumerate}

\fincorrection
\correction{006969}\ 
\begin{enumerate}
\item $F=\frac{4X^6-2X^5+11X^4-X^3+11X^2+2X+3}{X(X^2+1)^3}$.

  \begin{enumerate}
  \item 
La décomposition en éléments simples de $F$ est de la forme
$F=\frac{a}{X}+\frac{bX+c}{(X^2+1)^3}+\frac{dX+e}{(X^2+1)^2}+\frac{fX+g}{X^2+1}$.
Il est difficile d'obtenir les coefficients par substitution.
% $a,b,c$, (pour ces derniers : multiplication des deux membres par $X^2+1$, simplification, puis remplacement de $X$ par $i$ ou $-i$, avec séparation des parties
% réelle et imaginaire), mais c'est insuffisant pour conclure: il faut
% encore soustraire $\frac{bX+c}{(X^2+1)^3}$, simplifier par $X^2+1$, calculer $d$
% et $e$,\ldots

  \item
On va ici se contenter de trouver $a$ :
on multiplie $F$ par $X$, puis on remplace $X$ par $0$, on obtient $a=3$.

  \item On fait la soustraction $F_1=F-\frac{a}{X}$. 
  On sait que la fraction $F_1$ \emph{doit} se simplifier par $X$. 
  On trouve $F_1=\frac{X^5-2X^4+2X^3-X^2+2X+2}{(X^2+1)^3}$.
  
  \item La fin de la décomposition se fait par divisions euclidiennes successives.
  Tout d'abord la division du numérateur $X^5-2X^4+2X^3-X^2+2X+2$
par $X^2+1$:
$$X^5-2X^4+2X^3-X^2+2X+2=(X^2+1)(X^3-2X^2+X+1)+X+1$$
puis on recommence en divisant le quotient obtenu par $X^2+1$, pour obtenir 
$$X^5-2X^4+2X^3-X^2+2X+2=(X^2+1)\big((X^2+1)(X-2)+3\big)+X+1$$
On divise cette identité par $(X^2+1)^3$ :

$F_1 = \frac{(X^2+1)\big((X^2+1)(X-2)+3\big)+X+1}{(X^2+1)^3} 
= \frac{X+1}{(X^2+1)^3}+\frac{3}{(X^2+1)^2}+\frac{X-2}{X^2+1}$

Ainsi 
$$F=
\frac{3}{X}+\frac{X+1}{(X^2+1)^3}+\frac{3}{(X^2+1)^2}+\frac{X-2}{X^2+1}$$
  \end{enumerate}
  
\smallskip
Remarque~: cette méthode des divisions successives est tr\`es pratique quand
la fraction \`a décomposer a un dénominateur \emph{simple}, c'est \`a dire
comportant un dénominateur du type $Q^n$ o\`u $Q$ est du premier degré, ou
du second degré sans racine réelle. 

\item $G=\frac{4X^4-10X^3+8X^2-4X+1}{X^3(X-1)^2}$.

La décomposition en éléments simples de $G$ est de la forme
$\frac{a}{X^3}+\frac{b}{X^2}+\frac{c}{X}+\frac{d}{(X-1)^2}+\frac{e}{X-1}$.
% On pourrait obtenir facilement $a$ et $d$ par multiplication par $X^3$ et par $(X-1)^2$,
% mais qu'il resterait encore trois coefficients à déterminer. Il y a ici une
La méthode la plus efficace pour déterminer les coefficients est d'effectuer une division suivant les puissances
croissantes, ici à l'ordre $2$ (de sorte que le reste soit divisible par $X^3$ comme le dénominateur).
On calcule la division suivant les puissances
croissantes, à l'ordre $2$ du numérateur $1-4X+8X ^2-10X^3+4X^4$ par $(X-1)^2$, ou plutôt par $1-2X+X^2$ :
$$1-4X+8X^2-10X^3+4X^4=(1-2X+X^2)(1-2X+3X^2)+(-2X^3+X^4)$$
Remarquer que le reste $-2X^3+X^4$ est divisible par $X^3$.

En divisant les deux membres de cette identité par $X^3(X-1)^2$, 
on obtient $a$, $b$ et $c$ d'un seul coup:
$$
\begin{array}{rcl}
G & = & \frac{4X^4-10X^3+8X^2-4X+1}{X^3(X-1)^2}   \\
  & = & \frac{(X-1)^2(1-2X+3X^2)+(-2X^3+X^4)}{X^3(X-1)^2} \\
  & = & \frac{1}{X^3}-\frac{2}{X^2}+\frac{3}{X}+\frac{X-2}{(X-1)^2}
\end{array}
$$

Il reste à trouver $d$ et $e$: par exemple en faisant la division euclidienne de $X-2$ par $X-1$: $X-2=(X-1)-1$.
$$G=\frac{1}{X^3}-\frac{2}{X^2}+\frac{3}{X}-\frac{1}{(X-1)^2}+\frac{1}{X-1}$$

\item $H=\frac{X^4+2X^2+1}{X^5-X^3} = \frac{X^4+2X^2+1}{X^3(X-1)(X+1)}$.

La décomposition sera de la forme 
$H=\frac{a}{X^3}+\frac{b}{X^2}+\frac{c}{X}+\frac{d}{X-1}+\frac{e}{X+1}$. 
Pour obtenir $a, b, c$, on fait la division du numérateur par $(X-1)(X+1) = X^2-1$ 
selon les puissances croissantes, à l'ordre $2$ (de sorte que le reste soit divisible par $X^3$ qui est la puissance de $X$ au dénominateur de $H$,
en fait on l'obtient à l'ordre $3$) :
$$1+2X^2+X^4=(-1+X^2)(-1-3X^2)+4X^4$$
ce qui donne directement
$$
\begin{array}{rcl}
H 
&=& \frac{X^4+2X^2+1}{X^3(X-1)(X+1)} \\
&=& \frac{(X^2-1)(-1-3X^2)+4X^4}{X^3(X^2-1)} \\
&=& -\frac{1}{X^3}-\frac{3}{X}+\frac{4X}{X^2-1}  
\end{array}
$$
Il reste à décomposer $\frac{4X}{X^2-1}=\frac{d}{X-1}+\frac{e}{X+1}$. On trouve $d=e=2$, d'où
$$H = \frac{X^4+2X^2+1}{X^5-X^3} =-\frac{1}{X^3}-\frac{3}{X}+\frac{2}{X-1}+\frac{2}{X+1}$$ 

\medskip
Remarque : la méthode de division selon les puissances croissantes est 
efficace pour un exposant assez grand (en gros
à partir de $3$) dans une fraction du type $\frac{P(X)}{X^n Q(X)}$. 
Elle peut être utilisée pour une fraction du type 
$\frac{P(X)}{(X-a)^nQ(X)}$, mais il faut commencer par le changement de variable
$Y=X-a$ avant de faire la division, puis bien entendu revenir à la
variable $X$.


\item $K=\frac{X^5+X^4+1}{X(X-1)^4}$.

Puisque le degré du numérateur $N$ est supérieur ou égal à celui du dénominateur $D$, 
il y a une partie polynomiale.
$N$ et $D$ étant de même degré, avec le même coefficient dominant, 
la partie polynomiale vaut $1$ et $K$ se décompose sous la forme 
$K = 1+\frac{a}{X} + \frac{b}{(X-1)^4} + \frac{c}{(X-1)^3} + \frac{d}{(X-1)^2} + \frac{e}{X-1}$. 
Le coefficient $a$ s'obtient facilement en multipliant $K$ par $X$ puis en remplaçant $X$ par $0$ : $a=1$. 

Soit $K_1=K-1-\frac{1}{X}=\frac{4X^3-2X^2-2X+3}{(X-1)^4}$. Le changement d'indéterminée $X=Y+1$ donne 
$K_1=\frac{4(Y+1)^3-2(Y+1)^2-2(Y+1)+3}{Y^4} $. En développant, on obtient directement
$$K_1=\frac{4Y^3+10Y^2+6Y+3}{Y^4}= \frac{3}{Y^4} + \frac{6}{Y^3} + 
\frac{10}{Y^2} + \frac{4}{Y}$$
et donc (avec $Y=X-1$) :
$K_1=\frac{4X^3-2X^2-2X+3}{(X-1)^4}=\frac{3}{(X-1)^4} + \frac{6}{(X-1)^3} + 
\frac{10}{(X-1)^2} + \frac{4}{X-1}$ .
Finalement,
$$K = \frac{X^5+X^4+1}{X(X-1)^4} = 1 + \frac{1}{X} + \frac{3}{(X-1)^4} + \frac{6}{(X-1)^3} + 
\frac{10}{(X-1)^2} + \frac{4}{X-1}$$
\end{enumerate}
\fincorrection
\correction{006970}\
\begin{enumerate}
\item 
$$\begin{array}{rcl}
\frac{(3-2 i )X-5+3 i}{X^2+ i  X+2} &=&\frac{2+ i}{X- i}+\frac{1-3 i}{X+2 i}\\
\frac{X+ i }{X^2+ i} &=&\frac{\frac{2-\sqrt2}{4}+\frac{\sqrt2}{4} i}{X-\frac{\sqrt2-\sqrt2 i }{2}} +
\frac{\frac{2+\sqrt2}{4}-\frac{\sqrt2}{4} i }{X+\frac{\sqrt2-\sqrt2 i}{2}}\\
\frac{X}{(X+ i)^2} &=&\frac{1}{X+ i}+\frac{-i}{(X+ i)^2}
\end{array}$$

\item 
$$\begin{array}{rcl}
\frac{X^5+X+1}{X^4-1} 
& =& X + \frac{\frac34}{X-1} + \frac{\frac14}{X+1} - \frac{X+\frac{1}{2}}{X^2+1} \\
 &=& X + \frac{\frac34}{X-1} + \frac{\frac14}{X+1} + 
\frac{-\frac{1}{2}+\frac{1}{4} i }{X- i } + 
\frac{-\frac{1}{2}-\frac{1}{4} i }{X+ i }\\
\frac{X^2-3}{(X^2+1)(X^2+4)} 
&=& -\frac{\frac43}{X^2+1} + \frac{\frac73}{X^2+4} \\
 & =& \frac{\frac{2}{3} i }{X- i } + \frac{-\frac{2}{3} i }{X+ i } + 
\frac{-\frac{7}{12} i }{X-2 i } + \frac{\frac{7}{12} i }{X+2 i }\\
\frac{X^2+1}{X^4+1} 
&=&\frac{\frac12}{X^2-\sqrt{2}X+1}+\frac{\frac12}{X^2+\sqrt{2}X+1}\\
& =&\frac{-\frac{\sqrt{2}}{4} i }{X-\frac{\sqrt{2}}{2}-\frac{\sqrt{2}}{2} i } + 
\frac{ \frac{\sqrt{2}}{4} i }{X-\frac{\sqrt{2}}{2}+\frac{\sqrt{2}}{2} i } \\
 & &\ \ \ \ \ \  \ \ \ + 
\frac{ \frac{\sqrt{2}}{4} i }{X+\frac{\sqrt{2}}{2}+\frac{\sqrt{2}}{2} i } + 
\frac{-\frac{\sqrt{2}}{4} i }{X+\frac{\sqrt{2}}{2}-\frac{\sqrt{2}}{2} i }
\end{array}$$

\end{enumerate}
\fincorrection
\correction{006971}
La décomposition en élément simple s'écrit :
$$\frac{1}{X(X-1)(X-2)^2}=\frac{-\frac14}{X}+\frac{1}{X-1}+\frac{\frac12}{(X-2)^2}+\frac{-\frac34}{X-2}.$$ 
En multipliant cette identité par le dénominateur $X(X-1)(X-2)^2$, il vient :
$$1 =-\tfrac{1}{4}Q_0+Q_1+\left(\tfrac{1}{2}-\tfrac{3}{4}(X-2)\right)Q_2.$$
Ainsi $A_0=-\frac{1}{4}$, $A_1=1$ et $A_1=(2-\frac{3}{4}X)$ conviennent. 
On a obtenu une relation de Bézout entre $Q_1$, $Q_2$ et $Q_3$ qui prouve que 
ces trois polynômes sont premiers dans leur ensemble : $\pgcd(Q_1,Q_2,Q_3)=1$.
\fincorrection
\correction{006972}

\begin{enumerate}
  \item 
  \begin{enumerate}
    \item Si on pose $x=\cos \theta$ alors l'égalité $T_n(x)=\cos\big(n \arccos(x)\big)$
    devient $T_n(\cos\theta)=\cos(n\theta)$, car $\arccos (\cos \theta) = \theta$ pour $\theta \in [0,\pi]$.
    
    \item $T_0(x) = 1$, $T_1(x)=x$.
    
    \item En écrivant $(n+2)\theta = (n+1)\theta + \theta$ et $n\theta = (n+1)\theta - \theta$ on obtient :
    $$\begin{array}{c}
    \cos\big( (n+2)\theta \big) = \cos\big((n+1)\theta\big)\cos \theta - \sin\big((n+1)\theta\big)\sin \theta     \\
    \cos\big( n\theta \big) = \cos\big((n+1)\theta\big)\cos \theta + \sin\big((n+1)\theta\big)\sin \theta     \\    
      \end{array}$$
    Lorsque l'on fait la somme de ces deux égalités on obtient :
    $$\cos\big( (n+2)\theta \big) + \cos\big( n\theta \big) = \cos\big((n+1)\theta\big)\cos \theta$$
    
    Avec $x = \cos \theta$ cela donne :
    $$T_{n+2}(x) + T_n(x) = 2x T_{n+1}(x)$$
    
    \item $T_0$ et $T_1$ étant des polynômes alors, par récurrence,  $T_n(x)$ est un polynôme.
    De plus, toujours par la formule de récurrence, il est facile de voir que le degré de $T_n$ est $n$.
  \end{enumerate}

% Puisque $\arccos (\cos\theta)=\theta$, on a bien $T_n(\cos\theta)=\cos(n\theta)$. D'après la formule de Moivre,
% $$\cos(n\theta)=\mathrm{Re}(e^{in\theta)})=\mathrm{Re}((\cos\theta+i\sin\theta)^n)$$
% $$=\mathrm{Re}\left(\sum_{k=0}^nC_n^ki^k(\sin\theta)^k(\cos\theta)^{n-k}\right)$$
% Or $i^k$ est réel si et seulement si $k$ est pair (et dans ce cas, $i^k=(-1)^{k/2}$), donc en faisant le changement d'indice $k=2p$:
% \begin{eqnarray*}
%  \cos(n\theta)&=&\sum_{k\ pair,\ 1\le k\le n}C_n^ki^k(\sin\theta)^k(\cos\theta)^{n-k}\\
%  &=&\sum_{p=0}^{E(n/2)}C_n^{2p}(-1)^p(\sin\theta)^{2p}(\cos\theta)^{n-2p}\\
%  &=&\sum_{p=0}^{E(n/2)}C_n^{2p}(-1)^p(1-(\cos\theta)^2)^{p}(\cos\theta)^{n-2p}
% \end{eqnarray*}
% qui est polynomiale en $\cos\theta$.
% Comme tout $x\in[-1;1]$ peut s'écrire sous la forme $x=\cos\theta$, on a 
% $$T_n(x)=\sum_{p=0}^{E(n/2)}C_n^{2p}(-1)^p(1-x^2)^{p}x^{n-2p}$$
% Cette fonction polynomiale (sur $[-1;1]$) est de degré au plus $n$, et le coefficient du terme de degré $n$ vaut
% $a_n=\sum_{p=0}^{E(n/2)}C_n^{2p}$ qui est non nul (somme de termes strictement positifs).

\item Puisque les racines de $P=\lambda(X-a_1)\cdots(X-a_n)$ sont deux à deux distinctes, 
la décomposition en éléments simples de $\frac{1}{P}$ est de la forme 
$\frac{c_1}{X-a_1}+\cdots+\frac{c_n}{X-a_n}$. 

Expliquons comment calculer le coefficient $c_1$. On multiplie la fraction 
$\frac{1}{P}$ par $(X-a_1)$ ce qui donne  

$
\frac{X-a_1}{P} = c_1 + c_2 \frac{X-a_1}{X-a_2}+\cdots+ c_n \frac{X-a_1}{X-a_n}
\text{ et }
\frac{X-a_1}{P} = \frac{1}{\lambda(X-a_2)\cdots (X-a_n)}
$

On évalue ces égalités en $X=a_1$ ce qui donne
$$c_1 = \frac{1}{\lambda(a_1-a_2)\cdots (a_1-a_n)} = \frac{1}{\lambda\prod_{j\not= 1}(a_1-a_j)}$$

On obtiendrait de même le coefficient $c_k$ en multipliant $\frac{1}{P}$ par $(X-a_k)$,
puis en remplaçant $X$ par $a_k$, ce qui donne: $c_k=\frac{1}{\lambda\prod_{j\not= k}(a_k-a_j)}$.

Or la dérivée de $P$ est 
\begin{eqnarray*}
P'(X)=&\lambda(X-a_2)\cdots(X-a_n)+\lambda(X-a_1)(X-a_3)\cdots(X-a_n)\\
 &+\cdots+\lambda(X-a_1)\cdots(X-a_{k-1})(X-a_{k+1})\cdots(X-a_n)\\
 &\ \ \ \ \ +\cdots+\lambda(X-a_1)\cdots(X-a_{n-1})
\end{eqnarray*}
et donc $P'(a_1)=\lambda\prod_{j\not= 1}(a_1-a_j)$
et plus généralement $P'(a_k)=\lambda\prod_{j\not= k}(a_k-a_j)$. 
On a bien prouvé $c_k = \frac{1}{P'(a_k)}$ et ainsi la décomposition en éléments simple 
de $1/P$ est :
$$\frac{1}{P(X)}=\sum_{k=1}^n\frac{\frac{1}{P'(a_k)}}{X-a_k}$$

\item 
  \begin{enumerate}
    \item Cherchons d'abord les racines de $T_n(x)$.
Soit $x\in[-1,1]$ :
\begin{eqnarray*}
T_n(x)=0
  &\Longleftrightarrow& \cos\big(n \arccos(x)\big) = 0 \\
  &\Longleftrightarrow& n\arccos(x) \equiv \frac{\pi}{2} \pmod{\pi}\\
  &\Longleftrightarrow& \exists k\in\Zz\ \ \arccos(x)= \frac{\pi}{2n}+\frac{k\pi}{n}
\end{eqnarray*}
Comme par définition $\arccos(x) \in[0,\pi]$, les entiers $k$ possibles sont $k=0,\ldots,n-1$. Ainsi
$$\arccos x=\frac{\pi}{2n}+\frac{k\pi}{n} \quad \Longleftrightarrow \quad x=\cos\left(\frac{\pi}{2n}+\frac{k\pi}{n}\right)$$ 
Posons donc $\omega_k=\cos\left(\frac{(2k+1)\pi}{2n}\right)$ pour $k=0,\ldots,n-1$. 
Les $\omega_k$ sont les racines de $T_n$. Finalement $T_n(x)= \lambda \prod_{k=0}^{n-1}(x-\omega_k)$.

  \item Ainsi $T_n(x)= \lambda \prod_{k=0}^{n-1}(x-\omega_k)$ et les $\omega_k$ sont deux à deux distincts.
  On sait par la question précédente que la décomposition en éléments simple de $1/T_n$ s'écrit 
$$\frac{1}{T_n(X)}=\sum_{k=0}^{n-1} \frac{1/T_n'(\omega_k)}{X-\omega_k}$$
Repartant de $T_n(x)=\cos\big(n \arccos(x)\big)$, on calcule 
$T_n'(x)=\frac{n}{\sqrt{1-x^2}}\sin\big(n\arccos(x)\big)$.
En utilisant que 
$$\sin\big(n \arccos(\omega_k)\big) = \sin \left( n\left(\frac{(2k+1)\pi}{2n}\right)\right) 
= \sin\left(\frac{\pi}{2}+k\pi \right) = (-1)^k$$
et 
$$\sqrt{1-\cos^2 \theta} = \sqrt{\sin^2 \theta} = \sin \theta \qquad \text{ pour } \theta \in [0,\pi]$$
on trouve que 
$T_n'(\omega_k)=\frac{n(-1)^k}{\sin\left(\frac{(2k+1)\pi}{2n}\right)}$.

Ainsi 
$$\frac{1}{T_n(X)}=\sum_{k=0}^{n-1}\frac{\frac{(-1)^k}{n}\sin\left(\frac{(2k+1)\pi}{2n}\right)}{X-\cos\left(\frac{(2k+1)\pi}{2n}\right)}$$
  \end{enumerate}
\end{enumerate}
\fincorrection


\end{document}

