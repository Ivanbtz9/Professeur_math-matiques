
%%%%%%%%%%%%%%%%%% PREAMBULE %%%%%%%%%%%%%%%%%%

\documentclass[11pt,a4paper]{article}

\usepackage{amsfonts,amsmath,amssymb,amsthm}
\usepackage[utf8]{inputenc}
\usepackage[T1]{fontenc}
\usepackage[francais]{babel}
\usepackage{mathptmx}
\usepackage{fancybox}
\usepackage{graphicx}
\usepackage{ifthen}

\usepackage{tikz}   

\usepackage{hyperref}
\hypersetup{colorlinks=true, linkcolor=blue, urlcolor=blue,
pdftitle={Exo7 - Exercices de mathématiques}, pdfauthor={Exo7}}

\usepackage{geometry}
\geometry{top=2cm, bottom=2cm, left=2cm, right=2cm}

%----- Ensembles : entiers, reels, complexes -----
\newcommand{\Nn}{\mathbb{N}} \newcommand{\N}{\mathbb{N}}
\newcommand{\Zz}{\mathbb{Z}} \newcommand{\Z}{\mathbb{Z}}
\newcommand{\Qq}{\mathbb{Q}} \newcommand{\Q}{\mathbb{Q}}
\newcommand{\Rr}{\mathbb{R}} \newcommand{\R}{\mathbb{R}}
\newcommand{\Cc}{\mathbb{C}} \newcommand{\C}{\mathbb{C}}
\newcommand{\Kk}{\mathbb{K}} \newcommand{\K}{\mathbb{K}}

%----- Modifications de symboles -----
\renewcommand{\epsilon}{\varepsilon}
\renewcommand{\Re}{\mathop{\mathrm{Re}}\nolimits}
\renewcommand{\Im}{\mathop{\mathrm{Im}}\nolimits}
\newcommand{\llbracket}{\left[\kern-0.15em\left[}
\newcommand{\rrbracket}{\right]\kern-0.15em\right]}
\renewcommand{\ge}{\geqslant} \renewcommand{\geq}{\geqslant}
\renewcommand{\le}{\leqslant} \renewcommand{\leq}{\leqslant}

%----- Fonctions usuelles -----
\newcommand{\ch}{\mathop{\mathrm{ch}}\nolimits}
\newcommand{\sh}{\mathop{\mathrm{sh}}\nolimits}
\renewcommand{\tanh}{\mathop{\mathrm{th}}\nolimits}
\newcommand{\cotan}{\mathop{\mathrm{cotan}}\nolimits}
\newcommand{\Arcsin}{\mathop{\mathrm{arcsin}}\nolimits}
\newcommand{\Arccos}{\mathop{\mathrm{arccos}}\nolimits}
\newcommand{\Arctan}{\mathop{\mathrm{arctan}}\nolimits}
\newcommand{\Argsh}{\mathop{\mathrm{argsh}}\nolimits}
\newcommand{\Argch}{\mathop{\mathrm{argch}}\nolimits}
\newcommand{\Argth}{\mathop{\mathrm{argth}}\nolimits}
\newcommand{\pgcd}{\mathop{\mathrm{pgcd}}\nolimits} 

%----- Structure des exercices ------

\newcommand{\exercice}[1]{\video{0}}
\newcommand{\finexercice}{}
\newcommand{\noindication}{}
\newcommand{\nocorrection}{}

\newcounter{exo}
\newcommand{\enonce}[2]{\refstepcounter{exo}\hypertarget{exo7:#1}{}\label{exo7:#1}{\bf Exercice \arabic{exo}}\ \  #2\vspace{1mm}\hrule\vspace{1mm}}

\newcommand{\finenonce}[1]{
\ifthenelse{\equal{\ref{ind7:#1}}{\ref{bidon}}\and\equal{\ref{cor7:#1}}{\ref{bidon}}}{}{\par{\footnotesize
\ifthenelse{\equal{\ref{ind7:#1}}{\ref{bidon}}}{}{\hyperlink{ind7:#1}{\texttt{Indication} $\blacktriangledown$}\qquad}
\ifthenelse{\equal{\ref{cor7:#1}}{\ref{bidon}}}{}{\hyperlink{cor7:#1}{\texttt{Correction} $\blacktriangledown$}}}}
\ifthenelse{\equal{\myvideo}{0}}{}{{\footnotesize\qquad\texttt{\href{http://www.youtube.com/watch?v=\myvideo}{Vidéo $\blacksquare$}}}}
\hfill{\scriptsize\texttt{[#1]}}\vspace{1mm}\hrule\vspace*{7mm}}

\newcommand{\indication}[1]{\hypertarget{ind7:#1}{}\label{ind7:#1}{\bf Indication pour \hyperlink{exo7:#1}{l'exercice \ref{exo7:#1} $\blacktriangle$}}\vspace{1mm}\hrule\vspace{1mm}}
\newcommand{\finindication}{\vspace{1mm}\hrule\vspace*{7mm}}
\newcommand{\correction}[1]{\hypertarget{cor7:#1}{}\label{cor7:#1}{\bf Correction de \hyperlink{exo7:#1}{l'exercice \ref{exo7:#1} $\blacktriangle$}}\vspace{1mm}\hrule\vspace{1mm}}
\newcommand{\fincorrection}{\vspace{1mm}\hrule\vspace*{7mm}}

\newcommand{\finenonces}{\newpage}
\newcommand{\finindications}{\newpage}


\newcommand{\fiche}[1]{} \newcommand{\finfiche}{}
%\newcommand{\titre}[1]{\centerline{\large \bf #1}}
\newcommand{\addcommand}[1]{}

% variable myvideo : 0 no video, otherwise youtube reference
\newcommand{\video}[1]{\def\myvideo{#1}}

%----- Presentation ------

\setlength{\parindent}{0cm}

\definecolor{myred}{rgb}{0.93,0.26,0}
\definecolor{myorange}{rgb}{0.97,0.58,0}
\definecolor{myyellow}{rgb}{1,0.86,0}

\newcommand{\LogoExoSept}[1]{  % input : echelle       %% NEW
{\usefont{U}{cmss}{bx}{n}
\begin{tikzpicture}[scale=0.1*#1,transform shape]
  \fill[color=myorange] (0,0)--(4,0)--(4,-4)--(0,-4)--cycle;
  \fill[color=myred] (0,0)--(0,3)--(-3,3)--(-3,0)--cycle;
  \fill[color=myyellow] (4,0)--(7,4)--(3,7)--(0,3)--cycle;
  \node[scale=5] at (3.5,3.5) {Exo7};
\end{tikzpicture}}
}


% titre
\newcommand{\titre}[1]{%
\vspace*{-4ex} \hfill \hspace*{1.5cm} \hypersetup{linkcolor=black, urlcolor=black} 
\href{http://exo7.emath.fr}{\LogoExoSept{3}} 
 \vspace*{-5.7ex}\newline 
\hypersetup{linkcolor=blue, urlcolor=blue}  {\Large \bf #1} \newline 
 \rule{12cm}{1mm} \vspace*{3ex}}

%----- Commandes supplementaires ------



\begin{document}

%%%%%%%%%%%%%%%%%% EXERCICES %%%%%%%%%%%%%%%%%%

\fiche{f00122, rouget, 2010/10/16}

\titre{Séries}

Exercices de Jean-Louis Rouget.
Retrouver aussi cette fiche sur \texttt{\href{http://www.maths-france.fr}{www.maths-france.fr}}

\begin{center}
* très facile\quad** facile\quad*** difficulté moyenne\quad**** difficile\quad***** très difficile\\
I~:~Incontournable
\end{center}


\exercice{5688, rouget, 2010/10/16}
\enonce{005688}{}
\label{ex:rou1ter}
Nature de la série de terme général 

\begin{center}
\begin{tabular}{llll}
\textbf{1) (*)} $\ln\left(\frac{n^2+n+1}{n^2+n-1}\right)$&\textbf{2) (*)}  $\frac{1}{n+(-1)^n\sqrt{n}}$&\textbf{3) (**)} $\left(\frac{n+3}{2n+1}\right)^{\ln n}$  &\textbf{4) (**)} $\frac{1}{\ln(n)\ln(\ch n)}$\\
\textbf{5) (**)} $\Arccos\sqrt[3]{1-\frac{1}{n^2}}$&\textbf{6) (*)} $\frac{n^2}{(n-1)!}$&\textbf{7)} $\left(\cos\frac{1}{\sqrt{n}}\right)^n-\frac{1}{\sqrt{e}}$&\textbf{8) (**)} $\ln\left(\frac{2}{\pi}\Arctan\frac{n^2+1}{n}\right)$\\
\textbf{9) (*)} $\int_{0}^{\pi/2}\frac{\cos^2x}{n^2+\cos^2x}\;dx$&  
\textbf{10) (**)} $n^{-\sqrt{2}\sin(\frac{\pi}{4}+\frac{1}{n})}$&
\textbf{11) (**)} $e-\left(1+\frac{1}{n}\right)^n$
\end{tabular}
\end{center}
\finenonce{005688}


\finexercice
\exercice{5689, rouget, 2010/10/16}
\enonce{005689}{}
Nature de la série de terme général 

\textbf{1) (***)} $\sqrt[4]{n^4+2n^2}-\sqrt[3]{P(n)}$  où $P$ est un polynôme. \qquad\textbf{2) (**)} $\frac{1}{n^\alpha}S(n)$ où $S(n) =\sum_{p=2}^{+\infty}\frac{1}{p^n}$. 

\textbf{3) (**)} $u_n$ où $\forall n\in\Nn^*$, $u_n=\frac{1}{n}e^{-u_{n-1}}$.

\textbf{4) (****)} $u_n=\frac{1}{p_n}$ où $p_n$ est le $n$-ème nombre premier

(indication : considérer 
$\sum_{n=1}^{N}\ln\left(\frac{1}{1-\frac{1}{p_n}}\right)=\sum_{n=1}^{N}\ln(1+p_n+p_n^2+\ldots)$).

\textbf{5) (***)} $u_n=\frac{1}{n(c(n))^\alpha}$  où $c(n)$ est le nombre de chiffres de $n$ en 
base $10$.

\textbf{6) (*)} $\frac{\left(\prod_{k=2}^{n}\ln k\right)^a}{(n!)^b}$ $a > 0$ et $b> 0$.\qquad \textbf{7) (**)} $\Arctan\left(\left(1+\frac{1}{n}\right)^a\right) -\Arctan\left(\left(1-\frac{1}{n}\right)^a\right)$.

\textbf{8) (**)} $\frac{1}{n^\alpha}\sum_{k=1}^{n}k^{3/2}$.\qquad \textbf{9) (***) } $\left(\prod_{k=1}^{n}\left(1+\frac{k}{n^\alpha}\right)\right)-1$.

\finenonce{005689}


\finexercice
\exercice{5690, rouget, 2010/10/16}
\enonce{005690}{}
Nature de la série de terme général 

\textbf{1) (***)} $\sin\left(\frac{\pi n^2}{n+1}\right)$\qquad\textbf{2) (**)} $\frac{(-1)^n}{n+(-1)^{n-1}}$\qquad\textbf{3) (**)} $\ln\left(1+\frac{(-1)^n}{\sqrt{n}}\right)$\qquad\textbf{4) (***)} $\frac{e^{in\alpha}}{n}$, $\frac{\cos(n\alpha)}{n}$ et $\frac{\sin(n\alpha)}{n}$

%\textbf{5) (**)} $(-1)^n\frac{\ln n}{n}$\qquad\item  $(-1)^n\frac{P(n)}{Q(n)}$ où $P$ et $Q$ sont deux polynômes non nuls\qquad

\textbf{7) (****)} $(\sin(n!\pi e))^p$ $p$ entier naturel non nul.
\finenonce{005690}


\finexercice
\exercice{5691, rouget, 2010/10/16}
\enonce{005691}{}
\label{ex:rou4}
Calculer les sommes des séries suivantes après avoir vérifié leur convergence.

\begin{center}
\begin{tabular}{lll}
\textbf{1) (**)} $\sum_{n=0}^{+\infty}\frac{n+1}{3^n}$&\textbf{2) (**)} $\sum_{n=3}^{+\infty}\frac{2n-1}{n^3-4n}$&\textbf{3) (***)} $\sum_{n=0}^{+\infty}\frac{1}{(3n)!}$\\
\textbf{4) (*)} $\sum_{n=2}^{+\infty}\left(\frac{1}{\sqrt{n-1}}+\frac{1}{\sqrt{n+1}}-\frac{2}{\sqrt{n}}\right)$&
\textbf{5) (**)} $\sum_{n=2}^{+\infty}\ln\left(1+\frac{(-1)^n}{n}\right)$&\textbf{6) (***)} $\sum_{n=0}^{+\infty}\ln\left(\cos\frac{a}{2^n}\right)$  $a\in\left]0,\frac{\pi}{2}\right[$\\
textbf{7)}  $\sum_{n=0}^{+\infty}\frac{\tanh\frac{a}{2^n}}{2^n}$
\end{tabular}
\end{center}
\finenonce{005691}


\finexercice
\exercice{5692, rouget, 2010/10/16}
\enonce{005692}{*** I}
Soit $(u_n)_{n\in\Nn}$ une suite décroissante de nombres réels strictement positifs telle que la série de terme général $u_n$ converge. Montrer que $u_n\underset{n\rightarrow+\infty}{=}o\left(\frac{1}{n}\right)$. Trouver un exemple de suite $(u_n)_{n\in\Nn}$ de réels strictement positifs telle que la série de terme général $u_n$ converge mais telle que la suite de terme général $nu_n$ ne tende pas vers $0$.
\finenonce{005692}


\finexercice
\exercice{5693, rouget, 2010/10/16}
\enonce{005693}{***}
Soit $\sigma$ une injection de $\Nn^*$ dans lui-même. Montrer que la série de terme général $\frac{\sigma(n)}{n^2}$ diverge.

\finenonce{005693}


\finexercice
\exercice{5694, rouget, 2010/10/16}
\enonce{005694}{**}
Soit $(u_n)_{n\in\Nn}$ une suite de réels strictement positifs. Montrer que les séries de termes généraux $u_n$, $\frac{u_n}{1+u_n}$, $\ln(1+u_n)$ et $\int_{0}^{u_n}\frac{dx}{1+x^e}$  sont de mêmes natures.
\finenonce{005694}


\finexercice
\exercice{5695, rouget, 2010/10/16}
\enonce{005695}{***}
Trouver un développement limité à l'ordre $4$ quand $n$ tend vers l'infini de $\left(e-\sum_{k=0}^{n}\frac{1}{k!}\right)\times(n+1)!$.
\finenonce{005695}


\finexercice
\exercice{5696, rouget, 2010/10/16}
\enonce{005696}{***}
Nature de la série de terme général $u_n=\sin\left(\pi(2+\sqrt{3})^n\right)$.
\finenonce{005696}


\finexercice
\exercice{5697, rouget, 2010/10/16}
\enonce{005697}{**}
Soit $(u_n)_{n\in\Nn}$ une suite positive telle que la série de terme général $u_n$ converge. Etudier la nature de la série de terme général $\frac{\sqrt{u_n}}{n}$.
\finenonce{005697}


\finexercice
\exercice{5698, rouget, 2010/10/16}
\enonce{005698}{***}
Soit $(u_n)_{n\in\Nn}$ une suite de réels positifs. Trouver la nature de la série de terme général $v_n =\frac{u_n}{(1+u_1)\ldots(1+u_n)}$, $n\geqslant1$,  connaissant la nature de la série de terme général $u_n$ puis en calculer la somme en cas de convergence.
\finenonce{005698}


\finexercice
\exercice{5699, rouget, 2010/10/16}
\enonce{005699}{****}
Soit  $(u_n)_{n\in\Nn}$ une suite de réels strictement positifs telle que la série de terme général $u_n$ diverge.

Pour $n\in\Nn$, on pose $S_n = u_0+...+u_n$. Etudier en fonction de $\alpha> 0$ la nature de la série de terme général $\frac{u_n}{(S_n)^\alpha}$.

\finenonce{005699}


\finexercice
\exercice{5700, rouget, 2010/10/16}
\enonce{005700}{**}
Soit $\alpha\in\Rr$. Nature de la série de terme général $u_n=\frac{1+(-1)^nn^\alpha}{n^{2\alpha}}$, $n\geqslant1$.  
\finenonce{005700}


\finexercice
\exercice{5701, rouget, 2010/10/16}
\enonce{005701}{****}
On sait que $1-\frac{1}{2}+\frac{1}{3}-\frac{1}{4}+\ldots=\ln2$.

A partir de la série précédente, on construit une nouvelle série en prenant $p$ termes positifs, $q$ termes négatifs, $p$ termes positifs ... (Par exemple pour $p = 3$ et $q = 2$, on s'intéresse à $1+\frac{1}{3}+\frac{1}{5}-\frac{1}{2}-\frac{1}{4}+\frac{1}{7}+\frac{1}{9}+\frac{1}{11}-\frac{1}{6}-\frac{1}{8}+\ldots$). Convergence et somme de cette série.
\finenonce{005701}


\finexercice
\exercice{5702, rouget, 2010/10/16}
\enonce{005702}{***}
Nature de la série de terme général $u_n=\sum_{k=1}^{n-1}\frac{1}{(k(n-k))^\alpha}$.  
\finenonce{005702}


\finexercice
\exercice{5703, rouget, 2010/10/16}
\enonce{005703}{}
Convergence et somme éventuelle de la série de terme général

\begin{center}
\textbf{1) (**)} $u_n =\frac{2n^3-3n^2+1}{(n+3)!}$\qquad\textbf{2) (***)} $u_n=\frac{n!}{(a+1)(a+2)\ldots(a+n)}$, $n\geqslant1$, $a\in\Rr^{+*}$ donné.
\end{center} 
\finenonce{005703}


\finexercice
\exercice{5704, rouget, 2010/10/16}
\enonce{005704}{*}
Nature de la série de terme général $u_n=\sum_{k=1}^{n}\frac{1}{(n+k)^p}$, $p\in]0,+\infty[$.
\finenonce{005704}


\finexercice
\exercice{5705, rouget, 2010/10/16}
\enonce{005705}{**}
Déterminer un équivalent simple de $\frac{n!}{(a+1)(a+2)\ldots(a+n)}$ quand $n$ tend vers l'infini ($a$ réel positif donné).
\finenonce{005705}


\finexercice
\exercice{5706, rouget, 2010/10/16}
\enonce{005706}{*}
Nature de la série de terme général $u_n=\sum_{k=1}^{n}\frac{1}{(n+k)^p}$, $p\in]0,+\infty[$.
\finenonce{005706}


\finexercice
\exercice{5707, rouget, 2010/10/16}
\enonce{005707}{*** I}
Développement limité à l'ordre $4$ de $\sum_{k=n+1}^{+\infty}\frac{1}{k^2}$ quand $n$ tend vers l'infini.
\finenonce{005707}


\finexercice
\exercice{5708, rouget, 2010/10/16}
\enonce{005708}{}
Partie principale quand $n$ tend vers $+\infty$ de

\begin{center}
\textbf{1) (***)} $\sum_{p=n+1}^{+\infty}(-1)^p\frac{\ln p}{p}$\qquad\textbf{2) (**)} $\sum_{p=1}^{n}p^p$.
\end{center}
\finenonce{005708}


\finexercice
\exercice{5709, rouget, 2010/10/16}
\enonce{005709}{***}
Soit $p\in\Nn^*$, calculer $\sum_{p\in\Nn^*}^{}\left(\sum_{n\in\Nn^*,\;n\neq p}^{}\frac{1}{n^2-p^2}\right)$ et $\sum_{n\in\Nn^*}^{}\left(\sum_{p\in\Nn^*,\;p\neq n}^{}\frac{1}{n^2-p^2}\right)$. Que peut-on en déduire ?
\finenonce{005709}


\finexercice
\exercice{5710, rouget, 2010/10/16}
\enonce{005710}{**}
 Calculer $\sum_{n=0}^{+\infty}\frac{(-1)^n}{3n+1}$.
\finenonce{005710}


\finexercice
\exercice{5711, rouget, 2010/10/16}
\enonce{005711}{****}
Soient $(u_n)_{n\geqslant1}$ une suite réelle. Pour $n\geqslant 1$, on pose $v_n=\frac{u_1+\ldots+u_n}{n}$. Montrer que si la série de terme général $(u_n)^2$ converge alors la série de terme général $(v_n)^2$ converge et que $\sum_{n=1}^{+\infty}(v_n)^2\leqslant4\sum_{n=1}^{+\infty}(u_n)^2$ (indication : majorer $v_n^2 - 2u_nv_n$).
\finenonce{005711}


\finexercice
\exercice{5712, rouget, 2010/10/16}
\enonce{005712}{***}
Convergence et somme de la série de terme général $u_n=\frac{\pi}{4}-\sum_{k=0}^{n}\frac{(-1)^k}{2k+1}$, $n\geqslant 0$.
\finenonce{005712}


\finexercice

\finfiche



 \finenonces 



 \finindications 

\noindication
\noindication
\noindication
\noindication
\noindication
\noindication
\noindication
\noindication
\noindication
\noindication
\noindication
\noindication
\noindication
\noindication
\noindication
\noindication
\noindication
\noindication
\noindication
\noindication
\noindication
\noindication
\noindication
\noindication
\noindication


\newpage

\correction{005688}
\begin{enumerate}
 \item  Pour $n\geqslant1$, on pose $u_n =\ln\left(\frac{n^2+n+1}{n^2+n-1}\right)$. $\forall n\geqslant 1$, $u_n$ existe

\begin{center} 
$u_n=\ln\left(1+\frac{1}{n}+\frac{1}{n^2}\right)-\ln\left(1+\frac{1}{n}-\frac{1}{n^2}\right)\underset{n\rightarrow+\infty}{=}\left(\frac{1}{n}+O\left(\frac{1}{n^2}\right)\right)-\left(\frac{1}{n}+O\left(\frac{1}{n^2}\right)\right)=O\left(\frac{1}{n^2}\right)$.
\end{center}

Comme la série de terme général $\frac{1}{n^2}$, $n\geqslant1$, converge (série de \textsc{Riemann} d'exposant $\alpha>1$), la série de terme général $u_n$ converge.

\item  Pour $n\geqslant2$, on pose $u_n =\frac{1}{n+(-1)^n\sqrt{n}}$. $\forall n\geqslant 2$, $u_n$ existe et de plus $u_n\underset{n\rightarrow+\infty}{\sim}\frac{1}{n}$. Comme la série de terme général $\frac{1}{n}$, $n\geqslant 2$, diverge et est positive, la série de terme général $u_n$ diverge.

\item  Pour $n\geqslant1$, on pose $u_n =\left(\frac{n+3}{2n+1}\right)^{\ln n}$. Pour $n\geqslant1$, $u_n > 0$ et 

\begin{align*}\ensuremath
\ln(u_n)&=\ln(n)\ln\left(\frac{n+3}{2n+1}\right) =\ln(n)\left(\ln\left(\frac{1}{2}\right)+\ln\left(1+\frac{3}{n}\right) -\ln\left(1+\frac{1}{2n}\right)\right)\\
 &\underset{n\rightarrow+\infty}{=}\ln(n)\left(-\ln2+O\left(\frac{1}{n}\right)\right)\underset{n\rightarrow+\infty}{=}-\ln2\ln(n)+o(1).
\end{align*}

Donc $u_n=e^{\ln(u_n)}\underset{n\rightarrow+\infty}{\sim}e^{-\ln2\ln n}=\frac{1}{n^{\ln2}}$.  Comme la série de terme général $\frac{1}{n^{\ln 2}}$, $n\geqslant1$, diverge (série de \textsc{Riemann} d'exposant $\alpha\leqslant1$) et est positive, la série de terme général $u_n$ diverge.

\item  Pour $n\geqslant2$, on pose $u_n=\frac{1}{\ln(n)\ln(\ch n)}$. $u_n$ existe pour $n\geqslant2$. $\ln(\ch n)\underset{n\rightarrow+\infty}{\sim}\ln\left(\frac{e^n}{2}\right)=n -\ln2\underset{n\rightarrow+\infty}{\sim}n$ et $un\underset{n\rightarrow+\infty}{\sim}\frac{1}{n\ln(n)}>0$.

Vérifions alors que la série de terme général $\frac{1}{n\ln n}$, $n\geqslant2$, diverge. La fonction $x\rightarrow x\ln x$ est continue, croissante et strictement positive sur $]1,+\infty[$ (produit de deux fonctions strictement positives et croissantes sur $]1,+\infty[$). Par suite, la fonction $x\rightarrow\frac{1}{x\ln x}$ est continue et décroissante sur $]1,+\infty[$ et pour tout entier $k$ supérieur ou égal à $2$, 

\begin{center}
$\frac{1}{k\ln k}\geqslant\int_{k}^{k+1}\frac{1}{x\ln x}\;dx$
\end{center}

Par suite, pour $n\geqslant2$, 

\begin{center}
$\sum_{k=2}^{n}\frac{k\ln k}\geqslant\sum_{k=2}^{n}\int_{k}^{k+1}\frac{1}{x\ln x}\;dx=\int_{2}^{n+1}\frac{1}{x\ln x}\;dx=\ln(\ln(n+1)) -\ln(\ln(2)\underset{n\rightarrow+\infty}{\rightarrow}+\infty.$
\end{center}

Donc $u_n$ est positif et équivalent au terme général d'une série divergente. La série de terme général $u_n$ diverge.

\item  Pour $n\geqslant1$, on pose $u_n=\Arccos\sqrt[3]{1-\frac{1}{n^2}}$. $u_n$ existe pour $n\geqslant 1$. De plus $u_n\underset{n\rightarrow+\infty}{\rightarrow}0$. On en déduit que 

\begin{align*}\ensuremath
u_n&\underset{n\rightarrow+\infty}{\sim}\sin(u_n)=\sin\left(\Arccos\sqrt[3]{1-\frac{1}{n^2}}\right) =\sqrt{1-\left(1-\frac{1}{n^2}\right)^{2/3}}\underset{n\rightarrow+\infty}{=}\sqrt{1-1+\frac{2}{3n^2}+o\left(\frac{1}{n^2}\right)}\\
 &\underset{n\rightarrow+\infty}{\sim}\sqrt{\frac{2}{3}}\frac{1}{n}>0
\end{align*}

terme général d'une série de \textsc{Riemann} divergente. La série de terme général un diverge.

\item  Pour $n\geqslant1$, on pose $u_n=\frac{n^2}{(n-1)!}$. $u_n$ existe  et $u_n \neq0$ pour $n\geqslant1$. De plus,

\begin{center}
$\left|\frac{u_{n+1}}{u_n}\right|=\frac{(n+1)^2}{n^2}\times\frac{(n-1)!}{n!}=\frac{(n+1)^2}{n^3}  \underset{n\rightarrow+\infty}{\sim}\frac{1}{n}\underset{n\rightarrow+\infty}{\rightarrow}0< 1$.
\end{center}

D'après la règle de d'\textsc{Alembert}, la série de terme général $u_n$ converge.

\item  Pour $n\geqslant1$, on pose $u_n=\left(\cos\frac{1}{\sqrt{n}}\right)^n-\frac{1}{\sqrt{e}}$. $u_n$ est défini pour $n\geqslant1$ car pour $n\geqslant1$, $\frac{1}{\sqrt{n}}\in\left]0,\frac{\pi}{2}\right[$ et donc $\cos\frac{1}{\sqrt{n}}>0$. Ensuite

\begin{align*}\ensuremath
\ln\left(\cos\frac{1}{\sqrt{n}}\right)&\underset{n\rightarrow+\infty}{=}\ln\left(1-\frac{1}{2n}+\frac{1}{24n^2}+o\left(\frac{1}{n^2}\right)\right)\underset{n\rightarrow+\infty}{=}-\frac{1}{2n}+\frac{1}{24n^2}-\frac{1}{8n^2}+o\left(\frac{1}{n^2}\right)\\
 &\underset{n\rightarrow+\infty}{=}-\frac{1}{2n}-\frac{1}{12n^2}+o\left(\frac{1}{n^2}\right).
\end{align*}

Puis $n\ln\left(\cos\frac{1}{\sqrt{n}}\right)\underset{n\rightarrow+\infty}{=}-\frac{1}{2}-\frac{1}{12n}+o\left(\frac{1}{n}\right)$ et donc

\begin{center}
$u_n=e^{n\ln(\cos(1/\sqrt{n})}-\frac{1}{\sqrt{e}}\underset{n\rightarrow+\infty}{=}\frac{1}{\sqrt{e}}\left(e^{-\frac{1}{12n}+o\left(\frac{1}{n}\right)}-1\right)\underset{n\rightarrow+\infty}{\sim}-\frac{1}{12n\sqrt{e}}<0$.
\end{center}

La série de terme général $-\frac{1}{12n\sqrt{e}}$ est  divergente et donc la série de terme général $u_n$ diverge.

\item 

\begin{align*}\ensuremath
\ln\left(\frac{2}{\pi}\Arctan\left(\frac{n^2+1}{n}\right)\right)&=\ln\left(1-\frac{2}{\pi}\Arctan\left(\frac{n}{n^2+1}\right)\right)\\
 &\underset{n\rightarrow+\infty}{\sim}-\frac{2}{\pi}\Arctan\left(\frac{n}{n^2+1}\right)\underset{n\rightarrow+\infty}{\sim}-\frac{2}{\pi}\frac{n}{n^2+1}\underset{n\rightarrow+\infty}{\sim}-\frac{2}{n\pi}<0.
\end{align*}

Donc, la série de terme général $u_n$ diverge.

\item  Pour $n\geqslant1$, on pose $u_n=\int_{0}^{\pi/2}\frac{\cos^2x}{n^2+\cos^2x}\;dx$.

Pour $n\geqslant1$, la fonction $x\mapsto\frac{\cos^2x}{n^2+\cos^2x}\;dx$ est continue sur $\left[0,\frac{\pi}{2}\right]$ et positive et donc, $u_n$ existe et est positif. De plus, pour $n\geqslant1$,

\begin{center}
$0\leqslant u_n\leqslant\int_{0}^{\pi/2}\frac{1}{n^2+0}\;dx=\frac{\pi}{2n^2}$.
\end{center}

La série de terme général $\frac{\pi}{2n^2}$ converge et donc la série de terme général $u_n$ converge.

\item  $-\sqrt{2}\sin\left(\frac{\pi}{4}+\frac{1}{n}\right) =-\sin\left(\frac{1}{n}\right)-\cos\left(\frac{1}{n}\right)\underset{n\rightarrow+\infty}{=}-1+O\left(\frac{1}{n}\right)$ puis

\begin{center}
$-\sqrt{2}\sin\left(\frac{\pi}{4}+\frac{1}{n}\right)\ln n\underset{n\rightarrow+\infty}{=}-\ln(n)+O\left(\frac{\ln n}{n}\right)\underset{n\rightarrow+\infty}{=}-\ln(n)+o(1)$.
\end{center}

Par suite,

\begin{center}
$0< u_n=e^{-\sqrt{2}\sin\left(\frac{\pi}{4}+\frac{1}{n}\right)\ln n}\underset{n\rightarrow+\infty}{\sim}e^{-\ln n}=\frac{1}{n}$.
\end{center}

La série de terme général $\frac{1}{n}$ diverge et la série de terme général $u_n$ diverge.

\item  $n\ln\left(1+\frac{1}{n}\right)\underset{n\rightarrow+\infty}{=}1-\frac{1}{2n}+o\left(\frac{1}{n}\right)$ et donc

\begin{center}
$u_n\underset{n\rightarrow+\infty}{=}e-e^{1-\frac{1}{2n}+o\left(\frac{1}{n}\right)}\underset{n\rightarrow+\infty}{=}e\left(1-1+\frac{1}{2n}+o\left(\frac{1}{n}\right)\right)\underset{n\rightarrow+\infty}{\sim}\frac{e}{2n}>0$.
\end{center}

La série de terme général $\frac{e}{2n}$ diverge et la série de terme général $u_n$ diverge.
\end{enumerate}
\fincorrection
\correction{005689}
\begin{enumerate}
 \item  Si $P$ n'est pas unitaire de degré $3$, $u_n$ ne tend pas vers $0$ et la série de terme général $u_n$ diverge grossièrement.

Soit $P$ un polynôme unitaire de degré $3$. Posons $P =X^3+aX^2+bX+c$.

\begin{align*}\ensuremath
u_n&=n\left(\left(1+\frac{2}{n^2}\right)^{1/4}-\left(1+\frac{a}{n}+\frac{b}{n^2}+\frac{c}{n^3}\right)^{1/3}\right)\\
 &\underset{n\rightarrow+\infty}{=}n\left(\left(1+\frac{1}{2n^2}+O\left(\frac{1}{n^3}\right)\right)-\left(1+\frac{a}{3n}+\frac{b}{3n^2}-\frac{a^2}{9n^2}+O\left(\frac{1}{n^3}\right)\right)\right)\\
 &\underset{n\rightarrow+\infty}{=}-\frac{a}{3}+\left(\frac{1}{2}-\frac{b}{3}+\frac{a^2}{9}\right)\frac{1}{n}+O\left(\frac{1}{n^2}\right).
\end{align*}

\textbullet~Si $a\neq0$, $u_n$ ne tend pas vers $0$ et la série de terme général $u_n$ diverge grossièrement.

\textbullet~Si $a=0$ et $\frac{1}{2}-\frac{b}{3}\neq0$, $u_n\underset{n\rightarrow+\infty}{\sim}\left(\frac{1}{2}-\frac{b}{3}\right)\frac{1}{n}$. $u_n$ est donc de signe constant pour $n$ grand et est équivalent au terme général d'une série divergente. Donc la série de terme général $u_n$ diverge.

\textbullet~Si $a = 0$ et $\frac{1}{2}-\frac{b}{3}= 0$, $u_n\underset{n\rightarrow+\infty}{=}O\left(\frac{1}{n^2}\right)$. Dans ce cas, la série de terme général $u_n$ converge (absolument).

En résumé, la série de terme général $u_n$ converge si et seulement si $a = 0$ et  $b=\frac{3}{2}$ ou encore la série de terme général $u_n$ converge si et seulement si $P$ est de la forme $X^3+\frac{3}{2}X+c$, $c\in\Rr$.

\item  Pour $n\geqslant2$, posons $u_n=\frac{1}{n^\alpha}S(n)$. Pour $n\geqslant2$,

\begin{center}
$0<S(n+1)=\sum_{p=2}^{+\infty}\frac{1}{p}\times\frac{1}{p^n}\leqslant\frac{1}{2}\sum_{p=2}^{+\infty}\frac{1}{p^n}=\frac{1}{2}S(n)$
\end{center}

et donc $\forall n\geqslant2$, $S(n)\leqslant\frac{S(2)}{2^{n-2}}$. Par suite,

\begin{center}
$u_n\leqslant\frac{1}{n^\alpha}\frac{S(2)}{2^{n-2}}\underset{n\rightarrow+\infty}{=}o\left(\frac{1}{n^2}\right)$.
\end{center}

Pour tout réel $\alpha$, la série de terme général $u_n$ converge.

\item  $\forall u_0\in\Rr$, $\forall n\in\Nn^*$, $u_n > 0$. Par suite, $\forall n\geqslant 2$, $0< u_n<\frac{1}{n}$.

On en déduit que $\lim_{n \rightarrow +\infty}u_n=0$ et par suite $u_n\underset{n\rightarrow+\infty}{\sim}\frac{1}{n}>0$. La série de terme général $u_n$ diverge.

\item  On sait qu'il existe une infinité de nombres premiers.

Notons $(p_n)_{n\in\Nn^*}$ la suite croissante des nombres premiers. La suite $(p_n)_{n\in\Nn^*}$ est une suite strictement croissante d'entiers et donc $\lim_{n \rightarrow +\infty}p_n= +\infty$ ou encore $\lim_{n \rightarrow +\infty}\frac{1}{p_n}=0$.

Par suite, $0<\frac{1}{p_n}\underset{n\rightarrow+\infty}{\sim}\ln\left(\left(1-\frac{1}{p_n}\right)^{-1}\right)$ et les séries de termes généraux $\frac{1}{p_n}$ et $\ln\left(\left(1-\frac{1}{p_n}\right)^{-1}\right)$ sont de même nature.

Il reste donc à étudier la nature de la série de terme général $\ln\left(\left(1-\frac{1}{p_n}\right)^{-1}\right)$.

Montrons que $\forall N\in\Nn^*$,  $\sum_{n=1}^{+\infty}\ln\left(\left(1-\frac{1}{p_n}\right)^{-1}\right)\geqslant\ln\left(\sum_{k=1}^{N}\frac{1}{k}\right)$.

Soit $n\geqslant$. Alors $\frac{1}{p_n}<1$ et la série de terme général $\frac{1}{p_n^k}$, $k\in\Nn$, est une série géométrique convergente de somme : $\sum_{k=0}^{+\infty}\frac{1}{p_n^k}= \left(1-\frac{1}{p_n}\right)^{-1}$.

Soit alors $N$ un entier naturel supérieur ou égal à $2$ et $p_1 < p_2... < p_n$ la liste des nombres premiers inférieurs ou égaux à $N$.

Tout entier entre $1$ et $N$ s'écrit de manière unique $p_1^{\beta_1}\ldots p_k^{\beta_k}$ où $\forall i\in\llbracket1,n\rrbracket$, $0\leqslant\beta_i\leqslant\alpha_i=E\left(\frac{\ln(N)}{\ln(p_i)}\right)$ et deux entiers distincts ont des décompositions distinctes. Donc

\begin{align*}\ensuremath
\sum_{k=1}^{+\infty}\ln\left(\left(1-\frac{1}{p_k}\right)^{-1}\right)&\geqslant\sum_{k=1}^{n}\ln\left(\left(1-\frac{1}{p_k}\right)^{-1}\right)\;(\text{car}\;\forall k\in\Nn^*,\;\left(1-\frac{1}{p_k}\right)^{-1}> 1)\\
 &=\sum_{k=1}^{n}\ln\left(\sum_{i=0}^{+\infty}\frac{1}{p_k^i}\right)\geqslant\sum_{k=1}^{n}\ln\left(\sum_{i=0}^{\alpha_k}\frac{1}{p_k^i}\right)\\
  &=\ln\left(\prod_{k=1}^{n}\left(\sum_{i=0}^{\alpha_k}\frac{1}{p_k^i}\right)\right)=\ln\left(\sum_{0\leqslant\beta_1\leqslant\alpha_1,\ldots,\ldots0\leqslant\beta_n\leqslant\alpha_n}^{}\frac{1}{p_1^{\beta_1}\ldots,\;p_n^{\beta_n}}\right)\\
   &\geqslant\ln\left(\sum_{k=1}^{N}\frac{1}{k}\right).
\end{align*}

Or $\lim_{N \rightarrow +\infty}\ln\left(\sum_{k=1}^{N}\frac{1}{k}\right)=+\infty$ et donc $\sum_{k=1}^{+\infty}\ln\left(\left(1-\frac{1}{p_k}\right)^{-1}\right)=+\infty$.

La série de terme général $\ln\left(1-\frac{1}{p_k}\right)^{-1}$ diverge et il en est de même de la série de terme général $\frac{1}{p_n}$.

(Ceci montre qu'il y a beaucoup de nombres premiers et en tout cas beaucoup plus de nombres premiers que de carrés parfaits par exemple).

\item  Soit $n\in\Nn^*$. Posons $n =a_p\times10^p+\ldots+a_1\times10+a_0$ où $\forall i\in\llbracket0,p\rrbracket$, $a_i\in\{0,1;...,9\}$ et $a_p\neq 0$. Alors $c(n) = p+1$.

Déterminons $p$ est  en fonction de $n$. On a $10^p\leqslant n <10^{p+1}$ et donc $p=E\left(\log(n)\right)$. Donc

\begin{center}
$\forall n\in\Nn^*$, $u_n=\frac{1}{n(E(\log n)+1)^\alpha}$.
\end{center}

Par suite, $u_n\underset{n\rightarrow+\infty}{\sim}\frac{\ln^\alpha(10)}{n\ln^\alpha(n)}$ et la série de terme général $u_n$ converge si et seulement si $\alpha> 1$ (séries de \textsc{Bertrand}). Redémontrons ce résultat qui n'est pas un résultat de cours.

La série de terme général $\frac{1}{n\ln n}$ est divergente (voir l'exercice \ref{ex:rou1ter}, 4)). Par suite, si $\alpha\leqslant1$, la série de terme général $\frac{1}{n\ln^\alpha(n)}$ est divergente car $\forall n\geqslant2$, $\frac{1}{n\ln^\alpha(n)}\geqslant\frac{1}{n\ln n}$.

Soit $\alpha>1$. Puisque la fonction $x\mapsto\frac{1}{x\ln^\alpha x}$ est continue et strictement décroissante sur $]1,+\infty[$, pour $k\geqslant3$,

\begin{center}
$\frac{1}{k\ln^\alpha k}\leqslant\int_{k-1}^{k}\frac{1}{x\ln^\alpha x}\;dx$
\end{center}

puis, pour $n\geqslant3$, en sommant pour $k\in\llbracket3,n\rrbracket$

\begin{center}
$\sum_{k=3}^{n}\frac{1}{k\ln^\alpha k}\leqslant\sum_{k=3}^{n}\int_{k-1}^{k}\frac{1}{x\ln^\alpha x}\;dx=\int_{2}^{n}\frac{1}{x\ln^\alpha x}\;dx=\frac{1}{\alpha-1}\left(\frac{1}{\ln^{\alpha-1}(2)}-\frac{1}{\ln^{\alpha-1}(n)}\right)\leqslant\frac{1}{\alpha-1}\frac{1}{\ln^{\alpha-1}(2)}$.
\end{center}

Ainsi, la suite des sommes partielles de la série à termes positifs, de terme général $\frac{1}{k\ln^\alpha k}$, est majorée et donc la série de terme général $\frac{1}{k\ln^\alpha k}$ converge.

\textbf{6} Soit $n\geqslant2$.

\begin{center}
$\left|\frac{u_{n+1}}{u_n}\right|=\frac{\ln^a(n+1)}{(n+1)^b}\underset{n\rightarrow+\infty}{\rightarrow}0 < 1$
\end{center}

et d'après la règle de d'\textsc{Alembert}, la série de terme général $u_n$ converge.

\item  $\lim_{n \rightarrow +\infty}u_n=\frac{\pi}{4}-\frac{\pi}{4}= 0$. Donc

\begin{align*}\ensuremath
u_n&\underset{n\rightarrow+\infty}{\sim}\tan(u_n)\\
 &=\frac{\left(1+\frac{1}{n}\right)^a-\left(1-\frac{1}{n}\right)^a}{1+\left(1-\frac{1}{n^2}\right)^a}\underset{n\rightarrow+\infty}{=}\frac{\frac{2a}{n}+O\left(\frac{1}{n^2}\right)}{2+O\left(\frac{1}{n^2}\right)}\underset{n\rightarrow+\infty}{=}\frac{a}{n}+O\left(\frac{1}{n^2}\right).
\end{align*}

Par suite, la série de terme général $u_n$ converge si et seulement si $a = 0$.

\item  La fonction $x\mapsto x^{3/2}$ est continue et croissante sur $\Rr^+$. Donc pour $k\geqslant 1$, $\int_{k-1}^{k}x^{3/2}\;dx\leqslant k^{3/2}\leqslant\int_{k}^{k+1}x^{3/2}\;dx$ puis pour $n\in\Nn^*$ :

\begin{center}
$\int_{0}^{n}x^{3/2}\;dx\sum_{k=1}^{n}\int_{k-1}^{k}x^{3/2}\;dx\leqslant\sum_{k=1}^{n}k^{3/2}\leqslant\sum_{k=1}^{n}\int_{k}^{k+1}x^{3/2}\;dx=\int_{1}^{n+1}x^{3/2}\;dx$
\end{center}

ce qui fournit

\begin{center}
$\frac{2}{5}n^{5/2}\leqslant\sum_{k=1}^{n}k^{3/2}\leqslant\frac{2}{5}((n+1)^{5/2}-1)$ et donc $\sum_{k=1}^{n}k^{3/2}\underset{n\rightarrow+\infty}{\sim}\frac{2n^{5/2}}{5}$.
\end{center}

Donc $u_n\underset{n\rightarrow+\infty}{\sim}\frac{2n^{\frac{5}{2}-\alpha}}{5}>0$. La série de terme général $u_n$ converge si et seulement si $\alpha>\frac{7}{2}$.

\item  Pour $n\geqslant1$,

\begin{center}
$u_n =\left(1+\frac{1}{n^\alpha}\right)\left(1+\frac{2}{n^\alpha}\right)\ldots\left(1+\frac{n}{n^\alpha}\right)-1\geqslant\frac{1}{n^\alpha}+\frac{2}{n^\alpha}+\ldots+\frac{n}{n^\alpha}=\frac{n(n+1)}{2n^\alpha}>0$.
\end{center}

Comme $\frac{n(n+1)}{2n^\alpha}\underset{n\rightarrow+\infty}{\sim}\frac{1}{2n^{\alpha-2}}$, si $\alpha\leqslant3$, on a $\alpha-2\leqslant1$ et la série de terme général $u_n$ diverge.

Si $\alpha> 3$,

\begin{align*}\ensuremath
0 < u_n&\leqslant\left(1+\frac{n}{n^\alpha}\right)^n -1= e^{n\ln\left(1+\frac{1}{n^{\alpha-1}}\right)}-1\\
 &\underset{n\rightarrow+\infty}{\sim}n\ln\left(1+\frac{1}{n^{\alpha-1}}\right)\\
 &\underset{n\rightarrow+\infty}{\sim}\frac{1}{n^{\alpha-2}}\;\text{terme général d'une série de \textsc{Riemann} convergente},
\end{align*}
			   
			   
et, puisque $\alpha-2>1$,  la série de terme général $u_n$ converge. Finalement, la série de terme général $u_n$ converge si et seulement si $\alpha > 3$.
\end{enumerate}
\fincorrection
\correction{005690}
\begin{enumerate}
 \item  Pour $n\in\Nn$,

\begin{center}
$u_n =\sin\left(\frac{\pi n^2}{n+1}\right)=\sin\left(\frac{\pi(n^2-1+1)}{n+1}\right)=\sin\left(\frac{\pi}{n+1}+(n-1)\pi\right)=(-1)^{n-1}\sin\left(\frac{\pi}{n+1}\right)$.
\end{center}

La suite $\left((-1)^{n-1}\sin\left(\frac{\pi}{n+1}\right)\right)_{n\in\Nn}$ est alternée en signe et sa valeur absolue tend vers 0 en décroissant. La série de terme général $u_n$ converge donc en vertu du critère spécial aux séries alternées.

\item  (la suite $\left(\frac{1}{n+(-1)^{n-1}}\right)_{n\in\Nn}$ n'est pas décroisante à partir d'un certain rang).

\begin{center}
$u_n=\frac{(-1)^n}{n}\frac{1}{1+\frac{(-1)^{n-1}}{n}}\underset{n\rightarrow+\infty}{=}\frac{(-1)^n}{n}\left(1+O\left(\frac{1}{n}\right)\right)\underset{n\rightarrow+\infty}{=}\frac{(-1)^n}{n}+O\left(\frac{1}{n^2}\right)$.
\end{center}

La série de terme général $\frac{(-1)^n}{n}$ converge en vertu du critère spécial aux séries alternées et la série de terme général $O\left(\frac{1}{n^2}\right)$ est absolument convergente. On en déduit que la série de terme général $u_n$ converge.

\item  $u_n=\ln\left(1+\frac{(-1)^n}{\sqrt{n}}\right)\underset{n\rightarrow+\infty}{=}\frac{(-1)^n}{\sqrt{n}}-\frac{1}{2n}+ O\left(\frac{1}{n^{3/2}}\right)$. Les séries de termes généraux respectifs $\frac{(-1)^n}{\sqrt{n}}$  et $O\left(\frac{1}{n^{3/2}}\right)$ sont convergentes et la série de terme général $-\frac{1}{2n}$  est divergente. Si la série de terme général $u_n$ convergeait alors la série de terme général $-\frac{1}{2n}=u_n-\frac{(-1)^n}{\sqrt{n}}-O\left(\frac{1}{n^{3/2}}\right)$ convergerait ce qui n'est pas. Donc la série de terme général $u_n$ diverge.

\textbf{Remarque.} La série de terme général $u_n$ diverge bien que $u_n$ soit équivalent au terme général d'une série convergente.

\item  Si $\alpha\in2\pi\Zz$, alors les deux premières séries divergent et la dernière converge.

Soit $\alpha\notin2\pi\Zz$. Pour $n\in\Nn^*$, posons $v_n=e^{in\alpha}$ et $\varepsilon_n=\frac{1}{n}$ de sorte que $u_n=\varepsilon_nv_n$. Pour $n\in\Nn^*$, posons encore $V_n =\sum_{k=1}^{n}v_k$.

Pour $(n ,p)\in(\Nn^*)^2$, posons enfin $R_n^p=\sum_{k=1}^{n+p}u_k-\sum_{k=1}^{n}u_k=\sum_{k=n+1}^{n+p}u_k$. (On effectue alors une transformation d'\textsc{Abel}).

\begin{align*}\ensuremath
R_n^p&=\sum_{k=n+1}^{n+p}\varepsilon_kv_k =\sum_{k=n+1}^{n+p}\varepsilon_k(V_k-V_{k-1}) =\sum_{k=n+1}^{n+p}\varepsilon_kV_k-\sum_{k=n+1}^{n+p}\varepsilon_kV_{k-1} =\sum_{k=n+1}^{n+p}\varepsilon_kV_k-\sum_{k=n}^{n+p-1}\varepsilon_{k+1}V_{k}\\
 &=\varepsilon_{n+p}V_{n+p}-\varepsilon_{n+1}V_n+ \sum_{k=n+1}^{n+p-1}(\varepsilon_k-\varepsilon_{k+1})V_k.
\end{align*}

Maintenant, pour $n\in\Nn^*$, $V_n=e^{i\alpha}\frac{e^{in\alpha}-1}{e^{i\alpha}-1}=e^{i\alpha}\frac{\sin(n\alpha/2)}{\sin(\alpha/2)}$  et donc $\forall n\in\Nn^*$, $|V_n|\leqslant\frac{1}{|\sin(\alpha/2)|}$. Par suite, pour $(n,p)\in(\Nn^*)^2$

\begin{align*}\ensuremath
|R_n^p|&=\left|\frac{1}{n+p}V_{n+p}-\frac{1}{n+1}V_n+ \sum_{k=n+1}^{n+p-1}\left(\frac{1}{k}-\frac{1}{k+1}\right)V_k\right|&\\
 &\leqslant\frac{1}{|\sin(\alpha/2)|}\left(\frac{1}{n+p}+\frac{1}{n+1}+ \sum_{k=n+1}^{n+p-1}\left(\frac{1}{k}-\frac{1}{k+1}\right)\right)\\
 &=\frac{1}{|\sin(\alpha/2)|}\left(\frac{1}{n+p}+\frac{1}{n+1}+\frac{1}{n+1}-\frac{1}{n+p}\right)=\frac{2}{|\sin(\alpha/2)|(n+1)}\\
  &\leqslant\frac{2}{n|\sin(\alpha/2)|}.
\end{align*}

Soit alors $\varepsilon$ un réel strictement positif. Pour $n\geqslant E\left(\frac{2}{\varepsilon|\sin(\alpha/2)|}\right)+ 1$ et p entier naturel non nul quelconque, on a $|R_n^p|<\varepsilon$.

On a montré que $\forall\varepsilon>0$, $\exists n_0\in\Nn^*/$ $\forall(n,p)\in\Nn^*$, $(n\geqslant n_0\Rightarrow\left|\sum_{k=1}^{n+p}u_k-\sum_{k=1}^{n}u_k\right|<\varepsilon$.

Ainsi, la série de terme général $u_n$ vérifie le critère de \textsc{Cauchy} et est donc convergente.
Il en est de même des séries de termes généraux respectifs  $\frac{\cos(n\alpha)}{n}=\text{Re}\left(\frac{e^{in\alpha}}{n}\right)$ et $\frac{\sin(n\alpha)}{n}=\text{Im}\left(\frac{e^{in\alpha}}{n}\right)$.

\item  Pour $x\in]0,+\infty[$, posons $f(x)=\frac{\ln x}{x}$. $f$ est dérivable sur $]0,+\infty[$ et $\forall x>e$, $f'(x)=\frac{1-\ln x}{x}< 0$.

Donc, la fonction $f$ est décroissante sur $[e,+\infty[$. On en déduit que la suite $\left(\frac{\ln n}{n}\right)_{n\geqslant3}$ est une suite décroissante. Mais alors la série de terme général $(-1)^n\frac{\ln n}{n}$ converge en vertu du critère spécial aux séries alternées.

\item  \textbullet~Si $\text{deg}P\geqslant\text{deg}Q$, $u_n$ ne tend pas vers $0$ et la série de terme général $u_n$ est grossièrement divergente.

\textbullet~Si $\text{deg}P\leqslant\text{deg}Q - 2$, $u_n=O\left(\frac{1}{n^2}\right)$ et la série de terme général $u_n$ est absolument convergente.

\textbullet~Si $\text{deg}P =\text{deg}Q - 1$, $u_n\underset{n\rightarrow+\infty}{=}(-1)^n\frac{\text{dom}P}{n\;\text{dom}Q}+O\left(\frac{1}{n^2}\right)$. $u_n$ est alors somme de deux termes généraux de séries convergentes et la série de terme général $u_n$ converge.

En résumé, la série de terme général $u_n$ converge si et seulement si $\text{deg}P <\text{deg}Q$.

\item  $e=\sum_{k=0}^{+\infty}\frac{1}{k!}$  puis pour $n\geqslant2$, $n!e=1+ n+\sum_{k=0}^{n-2}\frac{n!}{k!}+\sum_{k=n+1}^{+\infty}\frac{n!}{k!}$.

Pour $0\leqslant k\leqslant n-2$, $\frac{n!}{k!}$ est un entier divisible par $n(n-1)$ et est donc un entier pair que l'on note $2K_n$. Pour $n\geqslant2$, on obtient

\begin{center}
$\sin(n!\pi e)=\sin\left(2K_n\pi+(n+1)\pi+\pi\sum_{k=n+1}^{+\infty}\frac{n!}{k!}\right)=(-1)^{n+1}\sin\left(\pi\sum_{k=n+1}^{+\infty}\frac{n!}{k!}\right)$.
\end{center}

Déterminons un développement limité à l'ordre $2$ de $\sum_{k=n+1}^{+\infty}\frac{n!}{k!}$ quand $n$ tend vers $+\infty$.

\begin{center}
$\sum_{k=n+1}^{+\infty}\frac{n!}{k!}=\frac{1}{n+1}+\frac{1}{(n+1)(n+2)}+\sum_{k=n+3}^{+\infty}\frac{n!}{k!}$.
\end{center}

Maintenant, pour $k\geqslant n+3$, $\frac{n!}{k!}=\frac{1}{k(k-1)\ldots(n+1)}\leqslant\frac{1}{(n+1)^{k-n}}$ et donc

\begin{center} 
$\sum_{k=n+3}^{+\infty}\frac{n!}{k!}\leqslant\sum_{k=n+3}^{+\infty}\frac{1}{(n+1)^{k-n}}=\frac{1}{(n+1)^3}\times\frac{1}{1-\frac{1}{n+1}}=\frac{1}{n(n+1)^2}\leqslant\frac{1}{n^3}$.
\end{center}

On en déduit que $\sum_{k=n+3}^{+\infty}\frac{n!}{k!}\underset{n\rightarrow+\infty}{=}o\left(\frac{1}{n^2}\right)$. Il reste

\begin{center}
$\sum_{k=n+1}^{+\infty}\frac{n!}{k!}\underset{n\rightarrow+\infty}{=}
\frac{1}{n+1}+\frac{1}{(n+1)(n+2)}+o\left(\frac{1}{n^2}\right)\underset{n\rightarrow+\infty}{=}\frac{1}{n}\left(1+\frac{1}{n}\right)^{-1}+\frac{1}{n^2}+o\left(\frac{1}{n^2}\right)
\underset{n\rightarrow+\infty}{=}\frac{1}{n}+o\left(\frac{1}{n^2}\right)
$.
\end{center}

Finalement , $\sin(n!\pi e)\underset{n\rightarrow+\infty}{=}(-1)^{n+1}\sin\left(\frac{\pi}{n}+o\left(\frac{1}{n^2}\right)\right)=\frac{(-1)^{n+1}\pi}{n}+\left(\frac{1}{n^2}\right)$.

$\sin(n!\pi e)$ est somme de deux termes généraux de séries convergentes et la série de terme général $\sin(n!\pi e)$ converge.

Si $p\geqslant2$, $|\sin^p(n!\pi e)|\underset{n\rightarrow+\infty}{\sim}\frac{\pi^p}{n^p}$ et la série de terme général $\sin^p(n!\pi e)$ converge absolument.
\end{enumerate}
\fincorrection
\correction{005691}
\begin{enumerate}
 \item  $\frac{n+1}{3^n}\underset{n\rightarrow+\infty}{=}o\left(\frac{1}{n^2}\right)$. Par suite, la série de terme général $\frac{n+1}{3^n}$ converge.

\textbf{1er calcul.} Soit $S=\sum_{n=0}^{+\infty}\frac{n+1}{3^n}$. Alors

\begin{align*}\ensuremath
\frac{1}{3}S&=\sum_{n=0}^{+\infty}\frac{n+1}{3^{n+1}}=\sum_{n=1}^{+\infty}\frac{n}{3^{n}}=\sum_{n=1}^{+\infty}\frac{n+1}{3^{n}}-\sum_{n=1}^{+\infty}\frac{1}{3^{n}}\\
 &=(S-1)-\frac{1}{3}\frac{1}{1-\frac{1}{3}}=S-\frac{3}{2}.
\end{align*}

On en déduit que $S=\frac{9}{4}$.

\begin{center}
\shadowbox{
$\sum_{n=0}^{+\infty}\frac{n+1}{3^n}=\frac{9}{4}$.
}
\end{center}

\textbf{2ème calcul.} Pour $x\in\Rr$ et $n\in\Nn$, on pose $f_n(x)=\sum_{k=0}^{n}x^k$.

Soit $n\in\Nn^*$. $f_n$ est dérivable sur $\Rr$ et pour $x\in\Rr$,

\begin{center}
$f_n'(x)=\sum_{k=1}^{n}kx^{k-1}=\sum_{k=0}^{n-1}(k+1)x^k$.
\end{center}

Par suite, pour $n\in\Nn^*$ et $x\in\Rr\setminus\{1\}$

\begin{center}
$\sum_{k=0}^{n-1}(k+1)x^k=f_n'(x)=\left(\frac{x^n-1}{x-1}\right)'(x)=\frac{nx^{n-1}(x-1)-(x^n-1)}{(x-1)^2}=\frac{(n-1)x^n-nx^{n-1}+1}{(x-1)^2}$.
\end{center}

Pour $x=\frac{1}{3}$, on obtient  $\sum_{k=0}^{n-1}\frac{k+1}{3^k}=\frac{\frac{n-1}{3^n}-\frac{n}{3^{n-1}}+1}{\left(\frac{1}{3}-1\right)^2}$ et quand $n$ tend vers l'infini, on obtient de nouveau $S=\frac{9}{4}$.

\item  Pour $k\geqslant3$, $\frac{2k-1}{k^3-4k}=\frac{3}{8(k-2)}+\frac{1}{4k}-\frac{5}{8(k+2)}$. Puis

\begin{align*}\ensuremath
\sum_{k=3}^{n}\frac{2k-1}{k^3-4k}&=\frac{3}{8}\sum_{k=3}^{n}\frac{1}{k-2}+\frac{1}{4}\sum_{k=3}^{n}\frac{1}{k}-\frac{5}{8}\sum_{k=3}^{n}\frac{1}{k+2}=\frac{3}{8}\sum_{k=1}^{n-2}\frac{1}{k}+\frac{1}{4}\sum_{k=3}^{n}\frac{1}{k}-\frac{5}{8}\sum_{k=5}^{n+2}\frac{1}{k}\\
 &\underset{n\rightarrow+\infty}{=}\frac{3}{8}\left(1+\frac{1}{2}+\sum_{k=3}^{n}\frac{1}{k}\right)+\frac{1}{4}\sum_{k=3}^{n}\frac{1}{k}-\frac{5}{8}\left(-\frac{1}{3}-\frac{1}{4}\sum_{k=3}^{n}\frac{1}{k}\right)+o(1)\\
 &\underset{n\rightarrow+\infty}{=}\frac{3}{8}\times\frac{3}{2}+\frac{5}{8}\times\frac{7}{12}+o(1)\underset{n\rightarrow+\infty}{=}\frac{89}{96}+ o(1) .
\end{align*}

La série proposée est donc convergente de somme $\frac{89}{96}$.

\begin{center}
\shadowbox{
$\sum_{n=3}^{+\infty}\frac{2n-1}{n^3-4n}=\frac{89}{96}$.
}
\end{center}

\item  Pour $k\in\Nn$, on a $1^{3k}+j^{3k}+(j^2)^{3k}=3$ puis $1^{3k+1}+j^{3k+1}+(j^2)^{3k+1} =1+j+j^2= 0$ et $1^{3k+2}+j^{3k+2}+(j^2)^{3k+2}=1+j^2 + j^4 = 0$. Par suite,

\begin{center}
$e+e^j+e^{j^2}=\sum_{n=0}^{+\infty}\frac{1^n+j^n+(j^2)^n}{n!}=3\sum_{n=0}^{+\infty}\frac{1}{(3n)!}$,
\end{center}

et donc

\begin{align*}\ensuremath
\sum_{n=0}^{+\infty}\frac{1}{(3n)!}&=\frac{1}{3}(e+e^j+e^{j^2})=\frac{1}{3}\left(e+e^{-\frac{1}{2}+i\frac{\sqrt{3}}{2}}+e^{-\frac{1}{2}-i\frac{\sqrt{3}}{2}}\right) =\frac{1}{3}\left(e + 2e^{-1/2}\text{Re}(e^{-i\sqrt{3}/2})\right)\\
 &=\frac{1}{3}\left(e +2e^{-1/2}\cos\left(\frac{\sqrt{3}}{2}\right)\right).
\end{align*}

\begin{center}
\shadowbox{
$\sum_{n=0}^{+\infty}\frac{1}{(3n)!}=\frac{1}{3}\left(e +\frac{2}{\sqrt{e}}\cos\left(\frac{\sqrt{3}}{2}\right)\right)$.
}
\end{center}

\item 

\begin{align*}\ensuremath
\sum_{k=2}^{n}\left(\frac{1}{\sqrt{k-1}}+\frac{1}{\sqrt{k+1}}-\frac{2}{\sqrt{k}}\right)&=\sum_{k=2}^{n}\left(\left(\frac{1}{\sqrt{k-1}}-\frac{1}{\sqrt{k}}\right)-\left(\frac{1}{\sqrt{k}}-\frac{1}{\sqrt{k+1}}\right)\right)\\
 &=\left(1-\frac{1}{\sqrt{2}}\right)-\left(\frac{1}{\sqrt{n}}-\frac{1}{\sqrt{n+1}}\right)\;(\text{somme télescopique})\\
 &\underset{n\rightarrow+\infty}{=}1-\frac{1}{\sqrt{2}}+o(1) 
\end{align*}

\begin{center}
\shadowbox{
$\sum_{n=2}^{+\infty}\left(\frac{1}{\sqrt{n-1}}+\frac{1}{\sqrt{n+1}}-\frac{2}{\sqrt{n}}\right)=1-\frac{1}{\sqrt{2}}$.
}
\end{center}

\item  $\ln\left(1+\frac{(-1)^n}{n}\right)\underset{n\rightarrow+\infty}{=}\frac{(-1)^n}{n}+O\left(\frac{1}{n^2}\right)$. Donc la série de terme général $\ln\left(1+\frac{(-1)^n}{n}\right)$ converge.

Posons $S=\sum_{k=2}^{+\infty}\ln\left(1+\frac{(-1)^k}{k}\right)$ puis pour $n\geqslant2$, $S_n=\sum_{k=2}^{n}\ln\left(1+\frac{(-1)^k}{k}\right)$. Puisque la série converge $S=\lim_{n \rightarrow +\infty}S_n=\lim_{p \rightarrow +\infty}S_{2p+1}$ avec

\begin{align*}\ensuremath
S_{2p+1}&=\sum_{k=2}^{2p+1}\ln\left(1+\frac{(-1)^k}{k}\right)=\sum_{k=1}^{p}\left(\ln\left(1-\frac{1}{2k+1}\right)+\ln\left(1+\frac{1}{2k}\right)\right)\\
 &=\sum_{k=1}^{p}(\ln(2k)-\ln(2k+1)+\ln(2k+1)-\ln(2k))=0
\end{align*}

et quand $p$ tend vers $+\infty$, on obtient $S = 0$.

\begin{center}
\shadowbox{
$\sum_{n=2}^{+\infty}\ln\left(1+\frac{(-1)^n}{n}\right)=0$.
}
\end{center}

\item  Si $a\in\left]0,\frac{\pi}{2}\right[$ alors, pour tout entier naturel $n$, $\frac{a}{2^n}\in\left]0,\frac{\pi}{2}\right[$ et donc $\cos\left(\frac{a}{2^n}\right)>0$.

Ensuite, $\ln\left(\cos\left(\frac{a}{2^n}\right)\right)\underset{n\rightarrow+\infty}{=}\ln\left(1+O\left(\frac{1}{2^{2n}}\right)\right)\underset{n\rightarrow+\infty}{=}O\left(\frac{1}{2^{2n}}\right)$ et la série converge. Ensuite,

\begin{align*}\ensuremath
\sum_{k=0}^{n}\ln\left(\cos\left(\frac{a}{2^k}\right)\right)&=\ln\left(\prod_{k=0}^{n}\cos\left(\frac{a}{2^k}\right)\right)=\ln\left(\prod_{k=0}^{n}\frac{\sin\left(2\times\frac{a}{2^k}\right)}{2\sin\left(\frac{a}{2^k}\right)}\right)=\ln\left(\frac{1}{2^{n+1}}\prod_{k=0}^{n}\frac{\sin\left(\frac{a}{2^{k-1}}\right)}{\sin\left(\frac{a}{2^k}\right)}\right)\\
 &=\ln\left(\frac{\sin(2a)}{2^{n+1}\sin\left(\frac{a}{2^n}\right)}\right)\;(\text{produit télescopique})\\
 &\underset{n\rightarrow+\infty}{\sim}\ln\left(\frac{\sin(2a)}{2^{n+1}\times\frac{a}{2^n}}\right)=\ln\left(\frac{\sin(2a)}{2a}\right).
\end{align*}

\begin{center}
\shadowbox{
$\forall a\in\left]0,\frac{\pi}{2}\right[$, $\sum_{n=0}^{+\infty}\ln\left(\cos\left(\frac{a}{2^n}\right)\right)=\ln\left(\frac{\sin(2a)}{2a}\right)$.
}
\end{center}

\item  Vérifions que pour tout réel $x$ on a $\tanh(2x)=\frac{2\tanh x}{1+\tanh^2x}$. Soit $x\in\Rr$.

$\ch^2x+\sh^2x=\frac{1}{4}((e^x+e^{-x})^2+(e^x-e^{-x})^2)=\frac{1}{2}(e^{2x}+e^{-2x})=\ch(2x)$ et $2\sh x\ch x=\frac{1}{2}(e^{x}-e^{-x})(e^x+e^{-x})=\frac{1}{2}(e^{2x}-e^{-2x})=\sh(2x)$ puis

\begin{center}
$\frac{2\tanh x}{1+\tanh^2x}=\frac{2\sh x\ch x}{\ch^2x+\sh^2x}=\frac{\sh(2x)}{\ch(2x)}=\tanh(2x)$.
\end{center}

Par suite, pour $x\in\Rr^*$, $\tanh x=\frac{2}{\tanh(2x)}-\frac{1}{\tanh x}$. Mais alors, pour $a\in\Rr^*$ et $n\in\Nn$

\begin{align*}\ensuremath
\sum_{k=0}^{n}\frac{1}{2^k}\tanh\left(\frac{a}{2^k}\right)&=\sum_{k=0}^{n}\frac{1}{2^k}\left(\frac{2}{\tanh\frac{a}{2^{k-1}}}-\frac{1}{\tanh\frac{a}{2^{k}}}\right)=\sum_{k=0}^{n}\left(\frac{1}{2^{k-1}\tanh\frac{a}{2^{k-1}}}-\frac{1}{2^k\tanh\frac{a}{2^{k}}}\right)\\
 &=\frac{2}{\tanh(2a)}-\frac{1}{2^n\tanh\frac{a}{2^{n}}}\;(\text{somme télescopique})\\
 &\underset{n\rightarrow+\infty}{\rightarrow}\frac{2}{\tanh(2a)}-\frac{1}{a},
\end{align*}

ce qui reste vrai quand $a=0$.

\begin{center}
\shadowbox{
$\forall a\in\Rr$, $\sum_{n=0}^{+\infty}\frac{1}{2^n}\tanh\left(\frac{a}{2^n}\right)=\frac{2}{\tanh(2a)}-\frac{1}{a}$.
}
\end{center}
\end{enumerate}	
\fincorrection
\correction{005692}
Il faut vérifier que $nu_n\underset{n\rightarrow+\infty}{\rightarrow}0$. Pour $n\in\Nn$, posons $S_n =\sum_{k=0}^{n}u_k$. Pour $n\in\Nn$, on a

\begin{align*}\ensuremath
0<(2n)u_{2n}&=2(\underbrace{u_{2n}+\ldots+u_{2n}}_{n})\leqslant2\sum_{k=n+1}^{2n}u_k\;(\text{car la suite}\;u\;\text{est décroissante})\\
 &= 2(S_{2n} - S_n).
\end{align*}

Puisque la série de terme général $u_n$ converge, $\lim_{n \rightarrow +\infty}2(S_{2n} - S_n)=0$ et donc $\lim_{n \rightarrow +\infty}(2n)u_{2n}=0$.

Ensuite, $0 < (2n+1)u_{2n+1}\leqslant(2n+1)u_{2n}=(2n)u_{2n}+u_{2n}\underset{n\rightarrow+\infty}{\rightarrow}0$. Donc les suites des termes de rangs pairs et impairs extraites de la suite $(nu_n)_{n\in\Nn}$ convergent et ont même limite à savoir $0$. On en déduit que $\lim_{n \rightarrow +\infty}nu_n=0$ ou encore que $u_n\underset{n\rightarrow+\infty}{=}o\left(\frac{1}{n}\right)$.

Contre exemple avec $u$ non monotone. Pour $n\in\Nn$, on pose $u_n=\left\{
\begin{array}{l}
0\;\text{si}\;n=0\\
\rule[-4mm]{0mm}{10mm}\frac{1}{n}\;\text{si}\;n\;\text{est un carré parfait non nul}\\
0\;\text{sinon}
\end{array}
\right.$. La suite $u$ est positive et $\sum_{n=0}^{+\infty}u_n=\sum_{p=1}^{+\infty}\frac{1}{p^2}<+\infty$. Pourtant, $p^2u_{p^2}=1\underset{p\rightarrow+\infty}{\rightarrow}1$ et la suite $(nu_n)$ admet une suite extraite convergeant vers $1$. On a donc pas $\lim_{n \rightarrow +\infty}nu_n=0$.
\fincorrection
\correction{005693}
Soit $\sigma$ une permutation de $\llbracket1,n\rrbracket$. Montrons que la suite $S_n=\sum_{k=1}^{n}\frac{\sigma(k)}{k^2}$, $n\geqslant1$, ne vérifie pas le critère de \textsc{Cauchy}. Soit $n\in\Nn^*$.

\begin{align*}\ensuremath
S_{2n}-S_n&=\sum_{k=n+1}^{2n}\frac{\sigma(k)}{k^2}\geqslant\frac{1}{(2n)^2}\sum_{k=n+1}^{2n}\sigma(k)\\
 &\geqslant\frac{1}{4n^2}(1+2+...+n)\;(\text{car les}\;n\;\text{entiers}\;\sigma(k),\;1\leqslant k\leqslant n,\;\text{sont strictement positifs et deux à deux distincts})\\
 &=\frac{n(n+1)}{8n^2}\geqslant\frac{n^2}{8n^2}=\frac{1}{8}.
\end{align*}

Si la suite $(S_n)$ converge, on doit avoir $\lim_{n \rightarrow +\infty}(S_{2n}-S_n)=0$ ce qui contredit l'inégalité précédente. Donc la série de terme général $\frac{\sigma(n)}{n^2}$, $n\geqslant1$, diverge.
\fincorrection
\correction{005694}
Pour $n\in\Nn$, posons $v_n=\ln(1+u_n)$, $w_n=\frac{u_n}{1+u_n}$ et $t_n=\int_{0}^{u_n}\frac{dx}{1+x^e}$.

\textbullet~Si $u_n\underset{n\rightarrow+\infty}{\rightarrow}0$, alors $0\leqslant u_n\underset{n\rightarrow+\infty}{\sim}v_n\underset{n\rightarrow+\infty}{\sim}w_n$. Dans ce cas, les séries de termes généraux $u_n$, $v_n$ et $w_n$ sont de même nature.

D'autre part, pour $n\in\Nn$, $\frac{u_n}{1+u_n^e}\leqslant t_n\leqslant u_n$ puis $\frac{1}{1+u_n^e}\leqslant\frac{t_n}{u_n}\leqslant1$et donc $t_n\underset{n\rightarrow+\infty}{\sim}u_n$. Les séries de termes généraux $u_n$ et $t_n$ sont aussi de même nature.

\textbullet~Si $u_n$ ne tend pas vers $0$, la série de terme général $u_n$ est grossièrement divergente. Puisque $u_n =e^{v_n}-1$, $v_n$ ne tend pas vers $0$ et la série de terme général $v_n$ est grossièrement divergente. Dans ce cas aussi, les séries de termes généraux sont de même nature.

De même, puisque $w_n=\frac{u_n}{1+u_n}<1$, on a $u_n =\frac{w_n}{1-w_n}$ et $w_n$ ne peut tendre vers $0$.

Enfin, puisque $u_n$ ne tend pas vers $0$, il existe $\varepsilon> 0$ tel que pour tout entier naturel $N$, il existe $n = n(N)\geqslant N$ tel que $u_n\geqslant\varepsilon$. Pour cet $\epsilon$ et ces $n$, on a $t_n\geqslant\int_{0}^{\varepsilon}\frac{dx}{1+x^e}>0$ (fonction continue, positive et non nulle) et la suite $t_n$ ne tend pas vers $0$. Dans le cas où $u_n$ ne tend pas vers $0$, les quatre séries sont grossièrement divergentes.

\fincorrection
\correction{005695}
Pour $n\in\Nn$, posons $u_n= (n+1)!\left(e-\sum_{k=0}^{n}\frac{1}{k!}\right)$. Soit $n\in\Nn$.

\begin{align*}\ensuremath
u_n&=\sum_{k=n+1}^{+\infty}\frac{(n+1)!}{k!}\\
 &=1+\frac{1}{n+2}+\frac{1}{(n+2)(n+3)}+\frac{1}{(n+2)(n+3)(n+4)}+\frac{1}{(n+2)(n+3)(n+4)(n+5)}+\sum_{k=n+6}^{+\infty}\frac{1}{(n+2)(n+3)\ldots k}
\end{align*}
	

On a $0 <\sum_{k=n+6}^{+\infty}\frac{1}{(n+2)(n+3)\ldots k}=\sum_{k=n+1}^{+\infty}\frac{1}{(n+2)^{k-(n+1)}}=\frac{1}{(n+2)^5}\frac{1}{1-\frac{1}{n+2}}=\frac{1}{(n+2)^4(n+1)}\leqslant\frac{1}{n^5}$. On en déduit que $\sum_{k=n+6}^{+\infty}\frac{1}{(n+2)(n+3)\ldots k}\underset{n\rightarrow+\infty}{=}o\left(\frac{1}{n^4}\right)$. Donc

\begin{align*}\ensuremath
u_n&\underset{n\rightarrow+\infty}{=}1+\frac{1}{n+2}+\frac{1}{(n+2)(n+3)}+\frac{1}{(n+2)(n+3)(n+4)}+\frac{1}{(n+2)(n+3)(n+4)(n+5)}+o\left(\frac{1}{n^4}\right)\\
 &\underset{n\rightarrow+\infty}{=}1+\frac{1}{n}\left(1+\frac{2}{n}\right)^{-1}+\frac{1}{n^2}\left(1+\frac{2}{n}\right)^{-1}\left(1+\frac{3}{n}\right)^{-1}+\frac{1}{n^3}\left(1+\frac{2}{n}\right)^{-1}\left(1+\frac{3}{n}\right)^{-1}\left(1+\frac{4}{n}\right)^{-1}+\frac{1}{n^4}+o\left(\frac{1}{n^4}\right)\\
  &\underset{n\rightarrow+\infty}{=}1+\frac{1}{n}\left(1-\frac{2}{n}+\frac{4}{n^2}-\frac{8}{n^3}\right)+\frac{1}{n^2}\left(1-\frac{2}{n}+\frac{4}{n^2}\right)\left(1-\frac{3}{n}+\frac{9}{n^2}\right)+\frac{1}{n^3}\left(1-\frac{2}{n}\right)\left(1-\frac{3}{n}\right)\left(1-\frac{4}{n}\right)\\
  &\;+\frac{1}{n^4}+o\left(\frac{1}{n^4}\right)\\
  &\underset{n\rightarrow+\infty}{=}1+\frac{1}{n}\left(1-\frac{2}{n}+\frac{4}{n^2}-\frac{8}{n^3}\right)+\frac{1}{n^2}\left(1-\frac{5}{n}+\frac{19}{n^2}\right)+\frac{1}{n^3}\left(1-\frac{9}{n}\right)+\frac{1}{n^4}+o\left(\frac{1}{n^4}\right)\\
  &\underset{n\rightarrow+\infty}{=}1+\frac{1}{n}-\frac{1}{n^2}+\frac{3}{n^4}+o\left(\frac{1}{n^4}\right).
\end{align*}

Finalement

\begin{center}
\shadowbox{
$(n+1)!\left(e-\sum_{k=0}^{n}\frac{1}{k!}\right)\underset{n\rightarrow+\infty}{=}1+\frac{1}{n}-\frac{1}{n^2}+\frac{3}{n^4}+o\left(\frac{1}{n^4}\right)$.
}
\end{center}
\fincorrection
\correction{005696}
 Pour $n\in\Nn$, posons $u_n=\sin\left(\pi(2+\sqrt{3})^n\right)$. D'après la formule du binôme de \textsc{Newton}, $(2+\sqrt{3})^n=A_n +B_n\sqrt{3}$ où $A_n$ et $B_n$ sont des entiers naturels. Un calcul conjugué fournit aussi $(2-\sqrt{3})^n=A_n-B_n\sqrt{3}$. Par suite, $(2+\sqrt{3})^n+(2-\sqrt{3})^n= 2A_n$ est un entier pair. Par suite, pour $n\in\Nn$,

\begin{center}
$u_n =\sin\left(2A_n\pi-\pi(2-\sqrt{3})^n\right)=-\sin\left(\pi(2-\sqrt{3})^n\right)$.
\end{center}

Mais $0< 2-\sqrt{3}< 1$ et donc $(2-\sqrt{3})^n\underset{n\rightarrow+\infty}{\rightarrow}0$. On en déduit que $|u_n|\underset{n\rightarrow+\infty}{\sim}\pi(2-\sqrt{3})^n$ terme général d'une série géométrique convergente. Donc la série de terme général $u_n$ converge.
\fincorrection
\correction{005697}
Pour $n\in\Nn^*$, on a $\left(\sqrt{u_n}-\frac{1}{n}\right)^2$ et donc $0\leqslant\frac{\sqrt{u_n}}{n}\leqslant\frac{1}{2}\left(u_n+\frac{1}{n^2}\right)$. Comme la série terme général $\frac{1}{2}\left(u_n+\frac{1}{n^2}\right)$ converge, la série de terme général $\frac{\sqrt{u_n}}{n}$  converge.
\fincorrection
\correction{005698}
 Pour $n\geqslant2$, $v_n=\frac{u_n+1-1}{(1+u_1)\ldots(1+u_n)}=\frac{1}{(1+u_1)\ldots(1+u_{n-1})}-\frac{1}{(1+u_1)\ldots(1+u_n)}$  et d'autre part $v_1=1-\frac{1}{1+u_1}$. Donc, pour $n\geqslant2$

\begin{center}
$\sum_{k=1}^{n}v_k =1-\frac{1}{(1+u_1)\ldots(1+u_n)}$ (somme télescopique).
\end{center}

Si la série de terme général $u_n$ converge alors $\lim_{n \rightarrow +\infty}u_n=0$ et donc $0<u_n\underset{n\rightarrow+\infty}{\sim}\ln(1+u_n)$. Donc la série de terme général $\ln(1+u_n)$ converge ou encore la suite  $\left(\ln\left(\prod_{k=1}^{n}(1+u_k)\right)\right)_{n\geqslant1}$
converge vers un certain réel $\ell$. Mais alors la suite $\left(\prod_{k=1}^{n}(1+u_k)\right)_{n\geqslant1}$ converge vers le réel strictement positif $P=e^{\ell}$.
Dans ce cas, la suite $\left(\sum_{k=1}^{n}v_k\right)_{n\geqslant1}$ converge vers $1-\frac{1}{P}$.

Si la série de terme général $u_n$ diverge alors la série de terme général $\ln(1+u_n)$ diverge vers $+\infty$ et il en est de même que la suite $\left(\prod_{k=1}^{n}(1+u_k)\right)_{n\geqslant1}$. Dans ce cas, la suite $\left(\sum_{k=1}^{n}v_k\right)_{n\geqslant1}$ converge vers $1$.
\fincorrection
\correction{005699}
Etudions tout d'abord la convergence de la série de terme général $\frac{u_n}{S_n}$.

Si  $\frac{u_n}{S_n}$ tend vers $0$ alors

\begin{center}
$0<\frac{u_n}{S_n}\underset{n\rightarrow+\infty}{\sim}-\ln\left(1-\frac{u_n}{S_n}\right)=\ln\left(\frac{S_{n-1}}{S_n}\right)=\ln(S_n) -\ln(S_{n-1})$.
\end{center}

Par hypothèse, $\lim_{n \rightarrow +\infty}S_n=+\infty$. On en déduit que la série de terme général $\ln(S_n) - \ln(S_{n-1})$ est divergente car  $\sum_{k=1}^{n}\ln(S_k) - \ln(S_{k-1}) =\ln(S_n)-\ln(S_0)\underset{n\rightarrow+\infty}{\rightarrow}+\infty$. Dans ce cas, la série de terme général $\frac{u_n}{S_n}$  diverge ce qui est aussi le cas si $\frac{u_n}{S_n}$ ne tend pas vers $0$.

Donc, dans tous les cas, la série de terme général $\frac{u_n}{S_n}$  diverge.

Si $\alpha\leqslant1$, puisque $S_n$ tend vers $+\infty$, à partir d'un certain rang on a $S_n^\alpha\leqslant S_n$ et donc  $\frac{u_n}{S_n^\alpha}\geqslant\frac{u_n}{S_n}$. Donc, si $\alpha\leqslant1$, la série de terme général  $\frac{u_n}{S_n^\alpha}$ diverge.

Si $\alpha> 1$, puisque la suite $(S_n)$ est croissante,

\begin{center}
$0<\frac{u_n}{S_n^\alpha}=\frac{S_n-S_{n-1}}{S_n^\alpha}=\int_{S_{n-1}}^{S_n}\frac{dx}{S_n^\alpha}\leqslant\frac{dx}{x^\alpha}=\frac{1}{\alpha-1}\left(\frac{1}{S_{n-1}^{\alpha-1}}-\frac{1}{S_n^{\alpha-1}}\right)$,
\end{center}

qui est le terme général d'une série télescopique convergente puisque $\frac{1}{S_n^{\alpha-1}}$ tend vers $0$ quand $n$ tend vers l'infini. Dans ce cas, la série de terme général $\frac{u_n}{S_n^\alpha}$  converge.

\begin{center}
\shadowbox{
La série de terme général $\frac{u_n}{S_n^\alpha}$ converge si et seulement si $\alpha>1$.
}
\end{center}
\fincorrection
\correction{005700}
Si $\alpha<0$, $u_n\underset{n\rightarrow+\infty}{\sim}n^{-2\alpha}$  et si $\alpha=0$, $u_n=1+(-1)^n$. Donc si $\alpha\leqslant0$, $u_n$ ne tend pas vers $0$. La série de terme général $u_n$ diverge grossièrement dans ce cas.

On suppose dorénavant que $\alpha> 0$. Pour tout entier naturel non nul $n$, $|u_n|\underset{n\rightarrow+\infty}{\sim}\frac{1}{n^\alpha}$ et donc la série de terme général $u_n$ converge absolument si et seulement si $\alpha> 1$.

Il reste à étudier le cas où $0 <\alpha\leqslant1$. On a $u_n=\frac{(-1)^n}{n^\alpha}+\frac{1}{n^{2\alpha}}$. La suite $\left(\frac{1}{n^\alpha}\right)_{n\geqslant1}$ tend vers $0$ en décroissant et donc la série de terme général $\frac{(-1)^n}{n^\alpha}$ converge en vertu du critère spécial aux séries alternées. On en déduit que la série de terme général $u_n$ converge si et seulement si la série de terme général $\frac{1}{n^{2\alpha}}$ converge ou encore si et seulement si $\alpha>\frac{1}{2}$.

En résumé

\begin{center}
\shadowbox{
\begin{tabular}{l}
Si $\alpha\leqslant0$, la série de terme général $\frac{1+(-1)^nn^\alpha}{n^{2\alpha}}$ diverge grossièrement,\\
\rule[-4mm]{0mm}{10mm}si $0 <\alpha\leqslant\frac{1}{2}$, la série de terme général $\frac{1+(-1)^nn^\alpha}{n^{2\alpha}}$ diverge,\\
\rule[-4mm]{0mm}{10mm}si  $\frac{1}{2}<\alpha\leqslant1$, la série de terme général $\frac{1+(-1)^nn^\alpha}{n^{2\alpha}}$ est semi convergente,\\
si $\alpha> 1$, la série de terme général $\frac{1+(-1)^nn^\alpha}{n^{2\alpha}}$ converge absolument.
\end{tabular}
}
\end{center}
\fincorrection
\correction{005701}
Pour $n\in\Nn^*$, on note $S_n$ la somme des $n$ premiers termes de la série considérée et on pose $H_n=\sum_{k=1}^{n}\frac{1}{k}$. Il est connu que $H_n\underset{n\rightarrow+\infty}{=}\ln n+\gamma+o(1)$.

Soit $m\in\Nn^*$.

\begin{align*}\ensuremath
S_{m(p+q)}&=\left(1+\frac{1}{3}+\ldots+\frac{1}{2p-1}\right) -\left(\frac{1}{2}+\frac{1}{4}+\ldots+\frac{1}{2q}\right)+\left(\frac{1}{2p+1}+...+\frac{1}{4p-1}\right) -\left(\frac{1}{2q+2}+...+\frac{1}{4q}\right) +...\\
 &\;+\left(\frac{1}{2(m-1)p+1}+...+\frac{1}{2mp-1}\right) -\left(\frac{1}{2(m-1)q+2}+...+\frac{1}{2mq}\right)\\
 &=\sum_{k=1}^{mp}\frac{1}{2k-1}-\sum_{k=1}^{mq}\frac{1}{2k}=\sum_{k=1}^{2mp}\frac{1}{k}-\sum_{k=1}^{mp}\frac{1}{2k}-\sum_{k=1}^{mq}\frac{1}{2k}= H_{2mp} -\frac{1}{2}(H_{mp}+H_{mq})\\
  &\underset{m\rightarrow+\infty}{=}(\ln(2mp)+\gamma) -\frac{1}{2}(\ln(mp) +\gamma+\ln(mq) +\gamma)+ o(1)=\ln2 +\frac{1}{2}\ln\left(\frac{p}{q}\right)+o(1).
\end{align*}

Ainsi, la suite extraite $(S_{m(p+q)})_{m\in\Nn^*}$ converge vers $\ln2 +\frac{1}{2}\ln\left(\frac{p}{q}\right)$.

Montrons alors que la suite $(S_n)_{n\in\Nn^*}$ converge. Soit $n\in\Nn^*$. Il existe un unique entier naturel non nul $m_n$ tel que $m_n(p+q)\leqslant n <(m_n+1)(p+q)$ à savoir $m_n=E\left(\frac{n}{p+q}\right)$.

\begin{align*}\ensuremath
|S_n-S_{m_n(p+q)}|&\leqslant\frac{1}{2m_np+1}+\ldots+\frac{1}{2(m_n+1)p-1}+\frac{1}{2m_nq+2}+\frac{1}{2(m_n+1)q}\\
 &\leqslant\frac{p}{2m_np+1}+\frac{q}{2m_nq+2}\leqslant\frac{1}{2m_n}+\frac{1}{2m_n}=\frac{1}{m_n}.
\end{align*}

Soit alors $\varepsilon>0$.

Puisque $\lim_{n \rightarrow +\infty}m_n=+\infty$, il existe $n_0\in\Nn^*$ tel que pour $n\geqslant n_0$, $\frac{1}{m_n}<\frac{\varepsilon}{2}$ et aussi $\left|S_{m_n(p+q)}-\ln2-\frac{1}{2}\ln\left(\frac{p}{q}\right)\right|<\frac{\varepsilon}{2}$. Pour $n\geqslant n_0$, on a alors

\begin{align*}\ensuremath
\left|S_{n}-\ln2 -\frac{1}{2}\ln\left(\frac{p}{q}\right)\right|&\leqslant|S_n-S_{m_n(p+q)}|+\left|S_{m_n(p+q)}-\ln2-\frac{1}{2}\ln\left(\frac{p}{q}\right)\right|\leqslant\frac{1}{m_n}+\left|S_{m_n(p+q)}-\ln2-\frac{1}{2}\ln\left(\frac{p}{q}\right)\right|\\
 &<\frac{\varepsilon}{2}+\frac{\varepsilon}{2}=\varepsilon.
\end{align*}

On a montré que $\forall \varepsilon>0,\;\exists n_0\in\Nn^*/\;\forall n\in\Nn,\;(n\geqslant n_0\Rightarrow\left|S_{n}-\left(\ln2 +\frac{1}{2}\ln\left(\frac{p}{q}\right)\right)\right|<\varepsilon)$ et donc, la série proposée converge et a pour somme $\ln2 +\frac{1}{2}\ln\left(\frac{p}{q}\right)$.
\fincorrection
\correction{005702}
La série proposée est le produit de \textsc{Cauchy} de la série de terme général   $\frac{1}{n^\alpha}$, $n\geqslant1$, par elle même.

\textbullet~Si $\alpha>1$, on sait que la série de terme général $\frac{1}{n^\alpha}$ converge absolument et donc que la série proposée converge.

\textbullet~Si $0\leqslant\alpha\leqslant1$, pour $0 < k < n$ on a $0<k(n-k)\leqslant\frac{n}{2}\left(n-\frac{n}{2}\right)=\frac{n^2}{4}$. Donc $u_n\geqslant\frac{n-1}{\left(\frac{n^2}{4}\right)^\alpha}$ avec $\frac{n-1}{\left(\frac{n^2}{4}\right)^\alpha}\underset{n\rightarrow+\infty}{\sim}\frac{4^\alpha}{n^{2\alpha-1}}$. Comme $2\alpha-1\leqslant1$, la série proposée diverge.

\textbullet~Si $\alpha< 0$, $u_n\geqslant\frac{1}{(n-1)^\alpha}$  et donc $u_n$ ne tend pas vers $0$. Dans ce cas, la série proposée diverge grossièrement.

\fincorrection
\correction{005703}
\begin{enumerate}
 \item  Soit $n\in\Nn$.

\begin{align*}\ensuremath
2n^3-3n^2+1&=2(n+3)(n+2)(n+1)-15n^2 - 22n -11= 2(n+3)(n+2)(n+1)-15(n+3)(n+2) +53n+79\\
 &= 2(n+3)(n+2)(n+1)-15(n+3)(n+2)+53(n+3)-80
\end{align*}

Donc

\begin{align*}\ensuremath
\sum_{n=0}^{+\infty}\frac{2n^3-3n^2+1}{(n+3)!}&=\sum_{n=0}^{+\infty}\left(\frac{2}{n!}-\frac{15}{(n+1)!}+\frac{53}{(n+2)!}-\frac{80}{(n+3)!}\right)=2e-15(e-1)+53(e-2)-80\left(e-\frac{5}{2}\right)\\
 &=-40e + 111.
\end{align*}

\begin{center}
\shadowbox{
$\sum_{n=0}^{+\infty}\frac{2n^3-3n^2+1}{(n+3)!}=-40e + 111$.
}
\end{center}

\item  Pour $n\in\Nn$, on a $u_{n+1}=\frac{n+1}{a+n+1}u_n$. Par suite $(n+a+1)u_{n+1}= (n+1)u_n = (n+a)u_n + (1-a)u_n$ puis

\begin{center}
$(1-a)\sum_{k=1}^{n}u_k=\sum_{k=1}^{n}(k+a+1)u_{k+1}-\sum_{k=1}^{n}(k+a)u_k =(n+a+1)u_{n+1}-(a+1)u_1 =(n+a+1)u_{n+1}-1.$
\end{center}

Si $a = 1$, $\forall n\in\Nn^*$, $u_n=\frac{1}{n+1}$. Dans ce cas, la série diverge.

Si $a\neq 1$, $\forall n\in\Nn^*$, $\sum_{k=1}^{n}u_k=\frac{1}{1-a}((n+a+1)u_{n+1} - 1)=\frac{1}{a-1}- \frac{1}{a-1}(a+n+1)u_{n+1}$.

Si $a > 1$, la suite $u$ est strictement positive et la suite des sommes partielles $(S_n)$ est majorée par $\frac{1}{a-1}$. Donc la série de terme général $u_n$ converge. Il en est de même de la suite $((a+n+1)u_{n+1})$. Soit $\ell=\lim_{n \rightarrow +\infty}(a+n+1)u_{n+1}$.

Si $\ell\neq0$, $u_{n+1}\underset{n\rightarrow+\infty}{\sim}\frac{\ell}{n+a+1}$ contredisant la convergence de la série de terme général $u_n$. Donc $\ell= 0$ et

\begin{center}
si $a > 1$,  $\sum_{n=1}^{+\infty}u_n=\frac{1}{a-1}$.
\end{center}

Si $0<a<1$, pour tout $n\in\Nn^*$, $u_n\geqslant\frac{1\times2\times\ldots\times n}{2\times3\ldots\times(n+1)}=\frac{1}{n+1}$. Dans ce cas, la série diverge.
\end{enumerate}
\fincorrection
\correction{005704}
Pour tout entier naturel non nul $n$, $0<\frac{1}{2^pn^{p-1}}=\sum_{k=1}^{n}\frac{1}{(2n)^p}\leqslant\sum_{k=1}^{n}\frac{1}{(n+k)^p}\leqslant\sum_{k=1}^{n}\frac{1}{n^p}=\frac{1}{n^{p-1}}$ et la série de terme général $u_n$ converge si et seulement si $p > 2$.
\fincorrection
\correction{005705}
(On applique la règle de \textsc{Raabe}-\textsc{Duhamel} qui n'est pas un résultat de cours.) Pour $n\in\Nn$, posons $u_n=\frac{n!}{(a+1)(a+2)\ldots(a+n)}$.

\begin{center}
$\frac{u_{n+1}}{u_n}=\frac{n+1}{a+n+1}=\left(1+\frac{1}{n}\right)\left(1+\frac{a+1}{n}\right)^{-1}\underset{n\rightarrow+\infty}{=}\left(1+\frac{1}{n}\right)\left(1-\frac{a+1}{n}+O\left(\frac{1}{n^2}\right)\right)\underset{n\rightarrow+\infty}{=}1-\frac{a}{n}+O\left(\frac{1}{n^2}\right)$,
\end{center}

et \og on sait \fg~qu'il existe un réel strictement positif $K$ tel que $u_n\underset{n\rightarrow+\infty}{\sim}\frac{K}{n^a}$.  

\fincorrection
\correction{005706}
Pour tout entier naturel non nul $n$, $0<\frac{1}{2^pn^{p-1}}=\sum_{k=1}^{n}\frac{1}{(2n)^p}\leqslant\sum_{k=1}^{n}\frac{1}{(n+k)^p}\leqslant\sum_{k=1}^{n}\frac{1}{n^p}=\frac{1}{n^{p-1}}$ et la série de terme général $u_n$ converge si et seulement si $p > 2$.
\fincorrection
\correction{005707}
Pour $n\in\Nn^*$, posons $R_n=\sum_{k=n+1}^{+\infty}\frac{1}{k^2}$. Puisque la série de terme général $\frac{1}{k^2}$, $k\geqslant 1$, converge, la suite $(R_n)$ est définie et tend vers $0$ quand $n$ tend vers $+\infty$.

$0<\frac{1}{k^2}\underset{k\rightarrow+\infty}{\sim}\frac{1}{k(k-1)}=\frac{1}{k-1}-\frac{1}{k}$  et puisque la série de terme général $\frac{1}{k^2}$ converge, la règle de l'équivalence des restes de séries à termes positifs convergentes permet d'affirmer que

\begin{align*}\ensuremath
R_n&= \sum_{k=n+1}^{+\infty}\frac{1}{k^2}\underset{n\rightarrow+\infty}{\sim} \sum_{k=n+1}^{+\infty}\left(\frac{1}{k-1}-\frac{1}{k}\right)\\
 &=\lim_{N \rightarrow +\infty}\sum_{k=n+1}^{N}\left(\frac{1}{k-1}-\frac{1}{k}\right)\;(\text{surtout ne pas décomposer en deux sommes})\\
 &=\lim_{N \rightarrow +\infty}\left(\frac{1}{n}-\frac{1}{N}\right)\;(\text{somme télescopique})\\
 &=\frac{1}{n} 
\end{align*}

ou encore $R_n\underset{n\rightarrow+\infty}{=}\frac{1}{n}+o\left(\frac{1}{n}\right)$.

Plus précisément, pour $n\in\Nn^*$, $R_n-\frac{1}{n}=\sum_{k=n+1}^{+\infty}\frac{1}{k^2}-\sum_{k=n+1}^{+\infty}\frac{1}{k(k-1)}=-\sum_{k=n+1}^{+\infty}\frac{1}{k^2(k-1)}$.

Or $-\frac{1}{k^2(k-1)}+\frac{1}{k(k-1)(k-2)}=\frac{2}{k^2(k-1)(k-2)}$ puis

$\frac{2}{k^2(k-1)(k-2)}-\frac{2}{k(k-1)(k-2)(k-3)}=-\frac{6}{k^2(k-1)(k-2)(k-3)}$ et donc

\begin{align*}\ensuremath
R_n&=\frac{1}{n}-\sum_{k=n+1}^{+\infty}\frac{1}{k^2(k-1)}=\frac{1}{n}-\sum_{k=n+1}^{+\infty}\frac{1}{k(k-1)(k-2)}+\sum_{k=n+1}^{+\infty}\frac{2}{k^2(k-1)(k-2)}\\
 &=\frac{1}{n}-\sum_{k=n+1}^{+\infty}\frac{1}{k(k-1)(k-2)}+\sum_{k=n+1}^{+\infty}\frac{2}{k(k-1)(k-2)(k-3)}-\sum_{k=n+1}^{+\infty}\frac{6}{k^2(k-1)(k-2)(k-3)}
\end{align*}

Ensuite  $\sum_{k=n+1}^{+\infty}\frac{1}{k^2(k-1)(k-2)(k-3)}\underset{n\rightarrow+\infty}{\sim}\sum_{k=n+1}^{+\infty}\frac{1}{k^5}\underset{n\rightarrow+\infty}{\sim}\frac{1}{4n^4}$ ou encore $-\sum_{k=n+1}^{+\infty}\frac{6}{k^2(k-1)(k-2)(k-3)}\underset{n\rightarrow+\infty}{=}-\frac{3}{2n^4}+o\left(\frac{1}{n^4}\right)$. Puis

\begin{align*}\ensuremath
\sum_{k=n+1}^{+\infty}\frac{1}{k(k-1)(k-2)}&=\lim_{N \rightarrow +\infty}\frac{1}{2}\sum_{k=n+1}^{N}\left(\frac{1}{(k-1)(k-2)}-\frac{1}{k(k-1)}\right)=\lim_{N \rightarrow +\infty}\frac{1}{2}\left(\frac{1}{n(n-1)}-\frac{1}{N(N-1)}\right)=\frac{1}{2n(n-1)}\\
 &=\frac{1}{2n^2}\left(1-\frac{1}{n}\right)^{-1}\underset{n\rightarrow+\infty}{=}\frac{1}{2n^2}+\frac{1}{2n^3}+\frac{1}{2n^4}+o\left(\frac{1}{n^4}\right)
\end{align*}

et

\begin{align*}\ensuremath
\sum_{k=n+1}^{+\infty}\frac{2}{k(k-1)(k-2)(k-3)}&=\lim_{N \rightarrow +\infty}\frac{2}{3}\sum_{k=n+1}^{N}\left(\frac{1}{(k-1)(k-2)(k-3)}-\frac{1}{k(k-1)(k-2)}\right)\\
 &=\lim_{N \rightarrow +\infty}\frac{2}{3}\left(\frac{1}{n(n-1)(n-2)}-\frac{1}{N(N-1)(N-2)}\right)=\frac{2}{3n(n-1)(n-2)}\\
 &=\frac{2}{3n^3}\left(1-\frac{1}{n}\right)^{-1}\left(1-\frac{2}{n}\right)^{-1}\underset{n\rightarrow+\infty}{=}\frac{2}{3n^3}\left(1+\frac{1}{n}+o\left(\frac{1}{n}\right)\right)\left(1+\frac{2}{n}+o\left(\frac{1}{n}\right)\right)\\
  &\underset{n\rightarrow+\infty}{=}\frac{2}{3n^3}+\frac{2}{n^4}+o\left(\frac{1}{n^4}\right)
\end{align*}

et finalement

\begin{center}
$R_n\underset{n\rightarrow+\infty}{=}\frac{1}{n}-\left(\frac{1}{2n^2}+\frac{1}{2n^3}+\frac{1}{2n^4}\right)+\left(\frac{2}{3n^3}+\frac{2}{n^4}\right)-\frac{3}{2n^4}+o\left(\frac{1}{n^4}\right)\underset{n\rightarrow+\infty}{=}\frac{1}{n}-\frac{1}{2n^2}+\frac{1}{6n^3}+o\left(\frac{1}{n^4}\right)$.
\end{center}

\begin{center}
\shadowbox{
$\sum_{k=n+1}^{+\infty}\frac{1}{k^2}\underset{n\rightarrow+\infty}{=}\frac{1}{n}-\frac{1}{2n^2}+\frac{1}{6n^3}+o\left(\frac{1}{n^4}\right)$.
}
\end{center}
\fincorrection
\correction{005708}
\begin{enumerate}
 \item  La suite $\left(\frac{\ln n}{n}\right)_{n\in\Nn^*}$ tend vers $0$, en décroissant à partir du rang $3$ (fourni par l'étude de la fonction $x\mapsto\frac{\ln x}{x}$ sur $[e,+\infty[$) et donc la série de terme général $(-1)^n\frac{\ln n}{n}$, $n\geqslant1$, converge en vertu du critère spécial aux séries alternées. Pour $n\in\Nn^*$, on pose $R_n=\sum_{p=n+1}^{+\infty}(-1)^p\frac{\ln p}{p}$.

$(-1)^k\frac{\ln k}{k}$  n'est pas de signe constant à partir d'un certain rang et on ne peut donc lui appliquer la règle de l'équivalence des restes.

Par contre, puisque la série de terme général $(-1)^k\frac{\ln k}{k}$ converge, on sait que l'on peut associer les termes à volonté et pour $k\in\Nn^*$, on a

\begin{center}
$R_{2k-1}=\sum_{p=2k}^{+\infty}(-1)^p\frac{\ln p}{p}=\sum_{p=k}^{+\infty}\left(\frac{\ln(2p)}{2p}-\frac{\ln(2p+1)}{2p+1}\right)$.
\end{center}

Puisque la fonction  $x\mapsto\frac{\ln x}{x}$ est décroissante sur $[e,+\infty[$ et donc sur $[3,+\infty[$, pour $p\geqslant 2$, $\frac{\ln(2p)}{2p}-\frac{\ln(2p+1)}{2p+1}\geqslant0$ et on peut utiliser la règle de l'équivalence des restes de séries à termes positifs convergentes.

Cherchons déjà un équivalent plus simple de $\frac{\ln(2p)}{2p}-\frac{\ln(2p+1)}{2p+1}$ quand $p$ tend vers $+\infty$.

\begin{align*}\ensuremath
\frac{\ln(2p)}{2p}-\frac{\ln(2p+1)}{2p+1}&=\frac{\ln(2p)}{2p}-\frac{1}{2p}\left(\ln(2p)+\ln\left(1+\frac{1}{2p}\right)\right)\left(1+\frac{1}{2p}\right)^{-1}\\
 &\underset{p\rightarrow+\infty}{=}\frac{\ln(2p)}{2p}-\frac{1}{2p}\left(\ln(2p)+\frac{1}{2p}+o\left(\frac{1}{p}\right)\right)\left(1-\frac{1}{2p}+o\left(\frac{1}{p}\right)\right)\\
 &\underset{p\rightarrow+\infty}{=}\frac{\ln(2p)}{4p^2}+o\left(\frac{\ln p}{p^2}\right)\underset{p\rightarrow+\infty}{=}\frac{\ln p+\ln2}{4p^2}+o\left(\frac{\ln p}{p^2}\right)\\
 &\underset{p\rightarrow+\infty}{\sim}\frac{\ln p}{4p^2}.
\end{align*}   

et donc $R_{2k-1}\underset{k\rightarrow+\infty}{\sim}\frac{1}{4}\sum_{p=k}^{+\infty}\frac{\ln p}{p^2}$.

Cherchons maintenant un équivalent simple de $\frac{\ln p}{p^2}$  de la forme $v_p - v_{p+1}$. 

Soit $v_p=\frac{\ln p}{p}-\frac{\ln(p+1)}{p+1}$ (suggéré par $\left(\frac{\ln x}{x}\right)'=\frac{1-\ln x}{x^2}\underset{x\rightarrow+\infty}{\sim}-\frac{\ln x}{x^2}$). Alors 

\begin{align*}\ensuremath
v_p- v_{p+1}&=\frac{\ln p}{p}-\frac{1}{p}\left(\ln p+\ln\left(1+\frac{1}{p}\right)\right)\left(1+\frac{1}{p}\right)^{-1}\underset{p\rightarrow+\infty}{=}\frac{\ln p}{p}-\frac{1}{p}\left(\ln p+\frac{1}{p}+o\left(\frac{1}{p}\right)\right)\left(1-\frac{1}{p}+o\left(\frac{1}{p}\right)\right)\\
 &\underset{p\rightarrow+\infty}{\sim}\frac{\ln p}{p^2}.
\end{align*}

D'après la règle de l'équivalence des restes de séries à termes positifs convergentes, $R_{2k-1}\underset{k\rightarrow+\infty}{\sim}\frac{1}{4}\sum_{p=k}^{+\infty}\left(\frac{\ln p}{p}-\frac{\ln(p+1)}{p+1}\right)=\frac{\ln k}{4k}$ (série télescopique). 

Puis, $R_{2k}=R_{2k-1}-\frac{\ln(2k)}{2k}\underset{k\rightarrow+\infty}{\sim}\frac{\ln k}{4k}-\frac{\ln(2k)}{2k}+o\left(\frac{\ln k}{k}\right)\underset{k\rightarrow+\infty}{\sim}\frac{\ln k}{4k}-\frac{\ln k}{2k}+o\left(\frac{\ln k}{k}\right)\underset{k\rightarrow+\infty}{\sim}-\frac{\ln k}{4k}+o\left(\frac{\ln k}{k}\right)$.

En résumé, $R_{2k-1}\underset{k\rightarrow+\infty}{\sim}\frac{\ln k}{4k}$ et $R_{2k}\underset{k\rightarrow+\infty}{\sim}-\frac{\ln k}{4k}$.

On peut unifier : $R_{2k-1}\underset{k\rightarrow+\infty}{\sim}\frac{\ln k}{4k}\underset{k\rightarrow+\infty}{\sim}\frac{\ln(2k-1)}{2(2k-1)}$   et $R_{2k}\underset{k\rightarrow+\infty}{\sim}-\frac{\ln k}{4k}\underset{k\rightarrow+\infty}{\sim}-\frac{\ln(2k)}{2(2k)}$. Finalement,

\begin{center}
\shadowbox{
$\sum_{p=n+1}^{+\infty}(-1)^p\frac{\ln p}{p}\underset{n\rightarrow+\infty}{\sim}(-1)^{n-1}\frac{\ln n}{2n}$.
}
\end{center}

\item  $\sum_{}^{}n^n$ est une série à termes positifs grossièrement divergente.

\textbf{1 ère solution.}

$0< n^n\underset{n\rightarrow+\infty}{\sim} n^n - (n-1)^{n-1}$ car $\frac{n^n - (n-1)^{n-1}}{n^n}=  1-\frac{1}{n-1}\left(1-\frac{1}{n}\right)^{n}\underset{n\rightarrow+\infty}{=}1-\frac{1}{ne}+o\left(\frac{1}{n}\right)\underset{n\rightarrow+\infty}{\rightarrow}1$.

D'après la règle de l'équivalence des sommes partielles de séries à termes positifs divergentes,

\begin{center}
$\sum_{p=1}^{n}p^p\underset{n\rightarrow+\infty}{\sim}\sum_{p=2}^{n}p^p\underset{n\rightarrow+\infty}{\sim}\sum_{p=2}^{n}(p^p-(p-1)^{p-1}) = n^n-1\underset{n\rightarrow+\infty}{\sim}n^n$.
\end{center}

(La somme est équivalente à son dermier terme.)

\textbf{2 ème solution.} Pour $n\geqslant3$, $0\leqslant\frac{1}{n^n}\sum_{p=1}^{n-2}p^p\leqslant\frac{1}{n^n}\times(n-2)(n-2)^{n-2}\leqslant \frac{n^{n-1}}{n^n}=\frac{1}{n}$. Donc $\frac{1}{n^n}\sum_{p=1}^{n-2}p^p$. On en déduit que $\frac{1}{n^n}\sum_{p=1}^{n}p^p=1+\frac{(n-1)^{n-1}}{n^n}+\frac{1}{n^n}\sum_{p=1}^{n-2}p^p\underset{n\rightarrow+\infty}{=}1+o(1)+o(1)=1+o(1)$.

\begin{center}
\shadowbox{
$\sum_{p=1}^{n}p^p\underset{n\rightarrow+\infty}{\sim}n^n$.
}
\end{center}
\end{enumerate}
\fincorrection
\correction{005709}
Soit $p\in\Nn^*$. Pour $n\in\Nn^*\setminus\{p\}$, $\frac{1}{n^2-p^2}=\frac{1}{2p}\left(\frac{1}{n-p}-\frac{1}{n+p}\right)$. Donc pour $N > p$,

\begin{align*}
\sum_{1\leqslant n\leqslant N,\;n\neq p}^{}\frac{1}{n^2-p^2}&=\frac{1}{2p}\sum_{1\leqslant n\leqslant N,\;n\neq p}^{}\left(\frac{1}{n-p}-\frac{1}{n+p}\right)=\frac{1}{2p}\left(\sum_{1-p\leqslant k\leqslant N-p,\;k\neq 0}^{}\frac{1}{k}-\sum_{p+1\leqslant k\leqslant N+p,\;k\neq 2p}^{}\frac{1}{k}\right)\\
 &=\frac{1}{2p}\left(-\sum_{k=1}^{p-1}\frac{1}{k}+\sum_{k=1}^{N-p}\frac{1}{k}-\sum_{k=1}^{N+p}\frac{1}{k}+\frac{1}{2p}+\sum_{k=1}^{p}\frac{1}{k}\right)=\frac{1}{2p}\left(\frac{3}{2p}-\sum_{k=N-p+1}^{N+p}\frac{1}{k}\right)
\end{align*}

Maintenant,  $\sum_{k=N-p+1}^{N+p}\frac{1}{k}=\frac{1}{N-p+1}+\ldots+\frac{1}{N+p}$ est une somme de $2p-1$ termes tendant vers $0$ quand $N$ tend vers $+\infty$. Puisque $2p-1$ est constant quand $N$ varie, $\lim_{N \rightarrow +\infty}\sum_{k=N-p+1}^{N+p}\frac{1}{k}=0$ et donc

\begin{center}
$\sum_{n\in\Nn^*,\;n\neq p}^{}\frac{1}{n^2-p^2}=\frac{1}{2p}\times\frac{3}{2p}=\frac{3}{4p^2}$ puis $\sum_{p\in\Nn^*}^{}\left(\sum_{n\in\Nn^*,\;n\neq p}^{}\frac{1}{n^2-p^2}\right)=\sum_{p=1}^{+\infty}\frac{3}{4p^2}=\frac{\pi^2}{8}$.
\end{center}

Pour $n\in\Nn^*$ donné, on a aussi $\sum_{p\in\Nn^*,\;p\neq n}^{}\frac{1}{n^2-p^2}=-\sum_{p\in\Nn^*,\;p\neq n}^{}\frac{1}{p^2-n^2}=-\frac{3}{4n^2}$ et donc

\begin{center}
$\sum_{n\in\Nn^*}^{}\left(\sum_{p\in\Nn^*,\;p\neq n}^{}\frac{1}{n^2-p^2}\right)=-\frac{\pi^2}{8}$.
\end{center}

On en déduit que la suite double $\left(\frac{1}{n^2-p^2}\right)_{(n,p)\in(\Nn^*)^2,\;n\neq p}$ n'est pas sommable.
\fincorrection
\correction{005710}
La suite $\left((-1)^n\frac{1}{3n+1}\right)_{n\in\Nn}$ est alternée en signe et sa valeur absolue tend vers $0$ en décroissant. Donc la série de terme général $(-1)^n\frac{1}{3n+1}$, $n\geqslant 1$, converge en vertu du critère spécial aux séries alternées.

Soit $n\in\Nn$.

\begin{center}
$\sum_{k=0}^{n}\frac{(-1)^k}{3k+1}=\sum_{k=0}^{n}(-1)^k\int_{0}^{1}t^{3k}\;dt=\int_{0}^{1}\frac{1-(-t^3)^{n+1}}{1-(-t^3)}\;dt=\int_{0}^{1}\frac{1}{1+t^3}\;dt+(-1)^n\int_{0}^{1}\frac{t^{3n+3}}{1+t^3}\;dt$.
\end{center}

Mais $\left|(-1)^n\int_{0}^{1}\frac{t^{3n+3}}{1+t^3}\;dt\right|=\int_{0}^{1}\frac{t^{3n+3}}{1+t^3}\;dt\leqslant\int_{0}^{1}t^{3n+3}\;dt=\frac{1}{3n+4}$. On en déduit que $(-1)^n\int_{0}^{1}\frac{t^{3n+3}}{1+t^3}\;dt$ tend vers $0$ quand $n$ tend vers $+\infty$ et donc que

\begin{center}
$\sum_{n=0}^{+\infty}\frac{(-1)^n}{3n+1}=\int_{0}^{1}\frac{1}{1+t^3}\;dt$.
\end{center}

Calculons cette dernière intégrale.

\begin{align*}\ensuremath
\frac{1}{X^3+1}&=\frac{1}{(X+1)(X+j)(X+j^2)}=\frac{1}{3}\left(\frac{1}{X+1}+\frac{j}{X+j}+\frac{j^2}{X+j^2}\right)=\frac{1}{3}\left(\frac{1}{X+1}+\frac{-X+2}{X^2-X+1}\right)\\
 &\frac{1}{3}\left(\frac{1}{X+1}-\frac{1}{2}\frac{2X-1}{X^2-X+1}+
 \frac{3}{2}\frac{1}{\left(X-\frac{1}{2}\right)^2+\left(\frac{\sqrt{3}}{2}\right)^2}
 \right).
\end{align*}

 
Donc, 

\begin{center}
$\sum_{n=0}^{+\infty}\frac{(-1)^n}{3n+1}=\frac{1}{3}\left[\ln(t+1)-\frac{1}{2}\ln(t^2-t+1)+\sqrt{3}\Arctan\left(\frac{2t-1}{\sqrt{3}}\right)\right]_0^1=\frac{1}{3}\left(\ln2+\sqrt{3}\left(\frac{\pi}{6}-\left(-\frac{\pi}{6}\right)\right)\right)=\frac{3\ln2+\pi\sqrt{3}}{9}$.
\end{center}

\begin{center}
\shadowbox{
$\sum_{n=0}^{+\infty}\frac{(-1)^n}{3n+1}=\frac{3\ln2+\pi\sqrt{3}}{9}$.
}
\end{center}
\fincorrection
\correction{005711}
Pour tout entier $n\geqslant 2$, on a $nv_n-(n-1)v_{n-1}=u_n$ ce qui reste vrai pour $n = 1$ si on pose de plus $v_0 = 0$. Par suite, pour $n\in\Nn^*$

\begin{align*}\ensuremath
v_n^2 -2u_nv_n&=v_n^2 - 2(nv_n - (n-1)v_{n-1})v_n = -(2n-1) v_n^2 + 2(n-1)v_{n-1}v_n\\
 &\leqslant -(2n-1) v_n^2 +(n-1)(v_{n-1}2+v_n^2) =(n-1)v_{n-1}^2 - n v_n^2.
\end{align*}
	
        
Mais alors, pour $N\in\Nn^*$,

\begin{center}
$\sum_{n=1}^{N}(v_n^2 - 2u_nv_n)\leqslant\sum_{n=1}^{N}((n-1)v_{n-1}^2 - n v_n^2)= - nv_n^2\leqslant0$.
\end{center}

Par suite, 

\begin{center}
$\sum_{n=1}^{N}v_n^2\leqslant\sum_{n=1}^{N}2u_nv_n\leqslant2\left(\sum_{n=1}^{N}u_n^2\right)^{1/2}\left(\sum_{n=1}^{N}v_n^2\right)^{1/2}\;$ (inégalité de \textsc{Cauchy}-\textsc{Schwarz}).
\end{center}

Si $\left(\sum_{n=1}^{N}v_n^2\right)^{1/2}>0$, on obtient après simplification par $\left(\sum_{n=1}^{N}v_n^2\right)^{1/2}$ puis élévation au carré

\begin{center}
$\sum_{n=1}^{N}v_n^2\leqslant4\sum_{n=1}^{N}u_n^2$,
\end{center}

cette inégalité restant claire si $\left(\sum_{n=1}^{N}v_n^2\right)^{1/2}=0$. Finalement,

\begin{center}
$\sum_{n=1}^{N}v_n^2\leqslant4\sum_{n=1}^{N}u_n^2\leqslant4\sum_{n=1}^{+\infty}u_n^2$.
\end{center}

La suite des sommes partielles de la série de terme général $v_n^2(\geqslant0)$ est majorée. Donc la série de terme général $v_n^2$ converge et de plus, quand $N$ tend vers l'infini, on obtient

\begin{center}
$\sum_{n=1}^{+\infty}v_n^2\leqslant 4\sum_{n=1}^{+\infty}u_n^2$.
\end{center} 
\fincorrection
\correction{005712}
Soit $n\in\Nn$.

\begin{align*}\ensuremath
u_n&=\frac{\pi}{4}-\sum_{k=0}^{n}\frac{(-1)^k}{2k+1}=\int_{0}^{1}\frac{1}{1+t^2}\;dt-\sum_{k=0}^{n}(-1)^k\int_{0}^{1}t^{2k}\;dt=\int_{0}^{1}\frac{1}{1+t^2}\;dt-\int_{0}^{1}\frac{1-(-t^2)^{n+1}}{1+t^2}\;dt\\
 &= (-1)^{n+1}\int_{0}^{1}\frac{t^{2n+2}}{1+t^2}\;dt.
\end{align*}

Par suite, pour $N\in\Nn$,

\begin{align*}\ensuremath
\sum_{n=0}^{N}u_n=\int_{0}^{1}\sum_{n=0}^{N}\frac{(-t^2)^{n+1}}{1+t^2}\;dt  =\int_{0}^{1}(-t^2)\frac{1-(-t^2)^{N+1}}{(1+t^2)^2}\;dt=-\int_{0}^{1}\frac{t^2}{(1+t^2)^2}\;dt+(-1)^{N+1}\int_{0}^{1}\frac{t^{2N+2}}{(1+t^2)^2}\;dt.
\end{align*}

Or $\left|(-1)^{N+1}\int_{0}^{1}\frac{t^{2N+2}}{(1+t^2)^2}\;dt\right|=\int_{0}^{1}\frac{t^{2N+2}}{(1+t^2)^2}\;dt\leqslant\int_{0}^{1}t^{2N+2}\;dt=\frac{1}{2N+3}$. Comme $\frac{1}{2N+3}$ tend vers $0$ quand $N$ tend vers $+\infty$, il en est de même de $(-1)^{N+1}\int_{0}^{1}\frac{t^{2N+2}}{(1+t^2)^2}\;dt$. On en déduit que la série de terme général $u_n$, $n\in\Nn$, converge et de plus

\begin{align*}\ensuremath
\sum_{n=0}^{+\infty}u_n&=-\int_{0}^{1}\frac{t^2}{(1+t^2)^2}\;dt=\int_{0}^{1}\frac{t}{2}\times\frac{-2t}{(1+t^2)^2}\;dt\\
 &=\left[\frac{t}{2}\times\frac{1}{1+t^2}\right]_0^1-\int_{0}^{1}\frac{1}{2}\times\frac{1}{1+t^2}\;dt=\frac{1}{4}-\frac{\pi}{8}.
\end{align*}

\begin{center}
\shadowbox{
$\sum_{n=0}^{+\infty}\left(\frac{\pi}{4}-\sum_{k=0}^{n}\frac{(-1)^k}{2k+1}\right)=\frac{1}{4}-\frac{\pi}{8}$.
}
\end{center}
\fincorrection


\end{document}

