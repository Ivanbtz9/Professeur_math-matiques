\documentclass[a4paper,10pt]{article}



\usepackage{fancyhdr} % pour personnaliser les en-têtes
\usepackage[utf8]{inputenc}
\usepackage[T1]{fontenc}
\usepackage{lastpage}
\usepackage[frenchb]{babel}
\usepackage{amsfonts,amssymb}
\usepackage{amsmath,amsthm,mathtools}
\usepackage{paralist}
\usepackage{xspace}
\usepackage{xcolor,multicol}
\usepackage{variations}
\usepackage{xypic}
\usepackage{eurosym}
\usepackage{graphicx}
\usepackage[np]{numprint}
\usepackage{hyperref} 
\usepackage{listings} % pour écrire des codes avec coloration syntaxique  

\usepackage{tikz}
\usetikzlibrary{calc, arrows, plotmarks,decorations.pathreplacing}
\usepackage{colortbl}
\usepackage{multirow}
\usepackage[top=2cm,bottom=1.5cm,right=2cm,left=1.5cm]{geometry}

\newtheorem{thm}{Théorème}
\newtheorem*{pro}{Propriété}
\newtheorem*{exemple}{Exemple}

\theoremstyle{definition}
\newtheorem*{remarque}{Remarque}
\theoremstyle{definition}
\newtheorem{exo}{Exercice}
\newtheorem{definition}{Définition}


\newcommand{\vtab}{\rule[-0.4em]{0pt}{1.2em}}
\newcommand{\V}{\overrightarrow}
\renewcommand{\thesection}{\Roman{section} }
\renewcommand{\thesubsection}{\arabic{subsection} }
\renewcommand{\thesubsubsection}{\alph{subsubsection} }
\newcommand*{\transp}[2][-3mu]{\ensuremath{\mskip1mu\prescript{\smash{\mathrm t\mkern#1}}{}{\mathstrut#2}}}%

\newcommand{\C}{\mathbb{C}}
\newcommand{\R}{\mathbb{R}}
\newcommand{\Q}{\mathbb{Q}}
\newcommand{\Z}{\mathbb{Z}}
\newcommand{\N}{\mathbb{N}}



\definecolor{vert}{RGB}{11,160,78}
\definecolor{rouge}{RGB}{255,120,120}
% Set the beginning of a LaTeX document
\pagestyle{fancy}
\lhead{Optimal Sup Spé, groupe IPESUP}\chead{Année~2021-2022}\rhead{Niveau: Première année de PCSI }\lfoot{M. Botcazou}\cfoot{\thepage/2}\rfoot{mail: ibotca52@gmail.com }\renewcommand{\headrulewidth}{0.4pt}\renewcommand{\footrulewidth}{0.4pt}

\begin{document}
	
	
	\begin{center}
		\Large \sc colle 9 = Polynômes, fractions rationnelles et Développements limités
	\end{center}
	
\section *{Questions de cours:}

\begin{enumerate} 


\item Démontrer la propriété suivante:
\begin{pro}\hfil\\
Soient $P\in K[X]$ et $a\in K$.\\
$a$ est racine de $P$ si et seulement si $(X-a)$ divise $P$
\end{pro}
\item Soit $P$ le polynôme $X^4-2X^3-X-1$. Donner la division euclidienne de $P$ par $P'$. Soit $\alpha$ une racine réelle de $P'$ en déduire le signe de $P(\alpha)$.
\item Décomposer les fractions suivantes en éléments simples sur $\R$.
\begin{multicols}{2}
\begin{enumerate}[$\square$]
\item $A = \dfrac{X}{X^2-4}$
\item$B = \dfrac{2X^3+X^2-X+1}{X^2-2X+1}$
\end{enumerate}
\end{multicols}
\item Soit $f$ une fonction de $\mathcal{C}^n\left(I,\R\right)$ et $a\in I$. Donner la formule de Taylor-Young correspondant au développement limité d'ordre $n$ de $f$  au voisinage de $a$. Calculer les développements limités suivants à l’ordre 2 :
\begin{multicols}{2}
\begin{enumerate}[$\square$]
\item $\dfrac{\ln(x)}{x}$ en 3.
\item $xe^x$ en 1.
\end{enumerate}
\end{multicols}
\end{enumerate}

\section*{Polynômes et fractions rationnelles:}
\begin{minipage}{1\linewidth}
\begin{minipage}[t]{0.48\linewidth}
\raggedright



\begin{exo}\quad\\
On pose $\omega_k=e^{2ik\pi/n}$ et $Q=1+2X+...+nX^{n-1}$. Calculer $\prod_{k=0}^{n-1}Q(\omega_k)$.

\centering
\rule{1\linewidth}{0.6pt}
\end{exo}

\begin{exo}\quad\\
Soit $P$ un polynôme à coefficients réels tel que $\forall x\in\R,\;P(x)\geq 0$. Montrer qu'il existe deux polynômes $R$ et $S$ à coefficients réels tels que $P=R^2+S^2$.

\centering
\rule{1\linewidth}{0.6pt}
\end{exo}


\begin{exo}\quad\\
Montrer que pour tout $n\in\N$:
$$\sum_{k=0}^{n}\begin{pmatrix}
n\\k
\end{pmatrix}^2  =  \begin{pmatrix}
2n\\n
\end{pmatrix}$$
(\textit{Indication: étudier le polynôme } $(X+1)^{2n} $ )

\centering\rule{1\linewidth}{0.6pt}
\end{exo}

\begin{exo}\quad\\
Trouver les polynômes $P$ de $\R[X]$ vérifiant $P(X^2)=P(X)P(X+1)$ (penser aux racines de $P$).

\centering
\rule{1\linewidth}{0.6pt}
\end{exo}

\begin{exo}\quad\\
Décomposer les fractions suivantes en éléments simples sur $\R$.
\begin{enumerate}
\item \`A l'aide de divisions euclidiennes successives :
% De 447, cousquer
$$F=\frac{4X^6-2X^5+11X^4-X^3+11X^2+2X+3}{X(X^2+1)^3}$$
\end{enumerate}

\end{exo}


\end{minipage}	
\hfill\vrule\hfill
\begin{minipage}[t]{0.48\linewidth}
\raggedright

\begin{enumerate}
\item[2.] A l'aide du changement d'indéterminée $X=Y+1$ :
% De 444, cousquer
$$K=\frac{X^5+X^4+1}{X(X-1)^4}$$
\end{enumerate}
\centering
\rule{1\linewidth}{0.6pt}
\begin{exo}\quad\\
Montrer que la suite $u$ définie par:
$$u_n = \sum_{k=1}^{n}\dfrac{1}{k(k+1)(k+2)}$$ 
converge et déterminer sa limite.

\centering\rule{1\linewidth}{0.6pt}
\end{exo}

\begin{exo}\quad\\
Calculer une primitive de: 
$$f: x \longmapsto  \dfrac{1}{x^2+x-6}$$
sur chaque intervalle de son domaine de définition.

\centering\rule{1\linewidth}{0.6pt}
\end{exo}

\begin{exo}\quad\\
Soit $P$ un polynôme à coefficients entiers relatifs de degré supérieur ou égal à $1$. Soit $n$ un entier relatif 
et $m=P(n)$.
\begin{enumerate}
\item  Montrer que $\forall k\in\Z,\;P(n+km)$ est un entier divisible par $m$.
\item  Montrer qu'il n'existe pas de polynômes non constants à coefficients entiers tels que $P(n)$ soit premier pour tout entier $n$.
\end{enumerate}
\centering\rule{1\linewidth}{0.6pt}
\end{exo}
\end{minipage}
\end{minipage}

\newpage

\section*{Études locales et asymptotiques}
\begin{minipage}{1\linewidth}
\begin{minipage}[t]{0.48\linewidth}
\raggedright



\begin{exo}\quad\\
"Nettoyer" les expressions suivantes:
\begin{enumerate}
\item $... \underset{x\rightarrow 0}{=} \dfrac{1}{x^2} + o\left(\dfrac{1}{x^2}\right) + 3 + \dfrac{1}{x} + o\left(\dfrac{1}{x}\right) $
\item $... \underset{n\rightarrow +\infty}{=} \dfrac{1}{\sqrt{n}} +\dfrac{1}{2^n}+  o\left(\dfrac{1}{2^n}\right) + \dfrac{1}{n} + o\left(\dfrac{1}{n}\right) + \dfrac{1}{n^2}  $
\item $... \underset{x\rightarrow 0}{=} x + o\left(x\right) + x\ln(x) +  o\left(x^2\ln(x)\right) + x^2 + o(x^2)$
\end{enumerate}
\centering
\rule{1\linewidth}{0.6pt}
\end{exo}

\begin{exo}\quad\\

\begin{enumerate}
\item Développement limité en $1$ à l'ordre $3$ de $f(x)=\sqrt{x}$.

\item Développement limité en $1$ à l'ordre $3$ de $g(x)= e^{\sqrt{x}}$.

\item Développement limité à l'ordre $3$ en $\frac\pi 3$ de $h(x)=\ln (\sin x)$.
\end{enumerate}
\centering
\rule{1\linewidth}{0.6pt}
\end{exo}


\end{minipage}	
\hfill\vrule\hfill
\begin{minipage}[t]{0.48\linewidth}
\raggedright


\begin{exo}\quad\\
Calculer les limites suivantes
$$\lim_{x\rightarrow 0}\frac{\ln (1+x)-\sin x}{x}
\quad\quad \lim_{x\rightarrow 0}\frac{\cos x-\sqrt{1-x^2}}{x^4}$$
$$\lim_{x\rightarrow 0}\frac{e^{x^2}-\cos x}{x^2}
\quad\quad$$ 
\centering\rule{1\linewidth}{0.6pt}
\end{exo}

\begin{exo}\quad\\
\begin{enumerate}
  \item  Montrer que l'équation $\tan x = x$ possède une unique solution
   $x_n$ dans
   $\left]n\pi-\frac \pi 2, n\pi+\frac \pi 2\right[$ $(n\in \N)$.
  \item  Quelle relation lie $x_n$ et $\arctan(x_n)$ ? \label{relation}
  \item  Donner un DL de $x_n$ en fonction de $n$ à l'ordre $0$ pour $n\to\infty$.
  \item  En reportant dans la relation trouvée en \ref{relation},
     obtenir un DL de $x_n$ à l'ordre 2.
\end{enumerate}
\centering\rule{1\linewidth}{0.6pt}
\end{exo}






\end{minipage}
\end{minipage}
\end{document}