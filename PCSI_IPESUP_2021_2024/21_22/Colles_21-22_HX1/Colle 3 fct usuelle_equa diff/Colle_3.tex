\documentclass[10pt,a4paper]{article}


\usepackage[T1]{fontenc}
\usepackage[francais]{babel}
\usepackage{times}
\usepackage[utf8]{inputenc}
\usepackage{enumitem}
\usepackage{multicol}
\usepackage{fancyhdr}
\usepackage{tcolorbox}
\usepackage{tablists}
\usepackage[a4paper,bottom = 50pt]{geometry}
\usepackage{color}
\usepackage{amsmath,amssymb,amsthm, mathrsfs,pifont}
\usepackage{pgf,tikz,pgfplots,tkz-tab}
\usepackage{hyperref}
\usepackage{cancel}
\usepackage{array}
\usepackage{xcolor} 
\usepackage{amssymb}
\usepackage{amsthm}
\usepackage{graphicx}
\usepackage{pstricks}

\usepackage{geometry}


\setcellgapes{1pt}
\makegapedcells
\newcolumntype{R}[1]{>{\raggedleft\arraybackslash }b{#1}}
\newcolumntype{L}[1]{>{\raggedright\arraybackslash }b{#1}}
\newcolumntype{C}[1]{>{\centering\arraybackslash }b{#1}}

\geometry{top=2cm, bottom=2cm, left=2cm, right=2cm}

\newcommand{\R}{\mathbb{R}}
\newcommand{\Z}{\mathbb{Z}}
\newcommand{\N}{\mathbb{N}}
\newcommand{\notni}{\not\owns}

\newtheorem{thm}{Théorème}
\newtheorem*{pro}{Propriété}
\newtheorem*{exemple}{Exemple}

\theoremstyle{definition}
\newtheorem*{remarque}{Remarque}
\theoremstyle{definition}
\newtheorem{exo}{Exercice}
\newtheorem{definition}{Définition}



\begin{document}
	
	\leftline{\bfseries Optimal Sup Spé, groupe IPESUP \hfill Année~2021-2022}
	\leftline{\bfseries Niveau: Première année de PCSI  }
	\leftline{\bfseries M.Botcazou \hfill mail: ibotca52@gmail.com  }
	\rule[0.5ex]{\textwidth}{0.1mm}	
	
	\begin{center}
		\Large \sc colle 3 = équations différentielles	et suites numériques
	\end{center}
	


\section*{ Équations différentielles :}
\begin{center}
\begin{minipage}[t]{0.47\linewidth}
\raggedright

\begin{exo}\quad\\
Déterminer les fonctions $f\in\mathcal{C}(\R,\R)$ telles que pour tout $x\in\R$:

$$f(x)  + \int_{0}^{x}tf(t)dt \ = \ 1$$

\end{exo}
\begin{center}
\rule{1\linewidth}{0.6pt}
\end{center} 

\begin{exo}\quad\\ 
On dira qu'une fonction à valeurs réelles dérivable sur $\R$ est solution de \eqref{equ1} si pour tous $x,y\in\R$:
\begin{equation}
f(x+y) \ = \ f(x) + f(y)
\label{equ1}
\end{equation}

\begin{enumerate}
\item Montrer que toute solution de \eqref{equ1} est solution d'une équation différentielle du premier ordre à préciser.
\item En déduire toutes les solutions de \eqref{equ1}

\end{enumerate}
\end{exo}
\begin{center}
\rule{1\linewidth}{0.6pt}
\end{center}

\begin{exo}\quad\\
Résoudre les équations différentielles suivantes :
\begin{enumerate}
\item $y' + y\tanh(x) = \tanh(x)  \ \  \text{sur} \ \R;$
\item$\sqrt{1-x^2}y'  + xy + 3(x-x^3)  = 0 \  \ \text{sur} \  ]-1;1[; $
\end{enumerate}
\end{exo} 
\begin{center}
\rule{1\linewidth}{0.6pt}
\end{center}


\begin{exo}\quad\\
Donner une équation différentielle dont les solutions sont les fonctions de la forme 
$$x\longmapsto \dfrac{C + x}{1+ x^2}, \  C \in\R$$
\end{exo}
\begin{center}
\rule{1\linewidth}{0.6pt}
\end{center}

\end{minipage}	
\hfill\vrule\hfill
\begin{minipage}[t]{0.47\linewidth}
\raggedright

\begin{exo}\quad\\
Donner l'ensemble solution des équations différentielles suivantes :
\begin{enumerate}
\item$y'' + y'-2y \ = \ 0 \ , \text{ avec }  \  y(0)=0 \ \text{ et } \  y'(0) = 1; $
\item$y'' \ = \  4y -4y'\ , \text{ avec }  \   y(0)=y'(0)= 1; $
\end{enumerate}
\end{exo}
\begin{center}
\rule{1\linewidth}{0.6pt}
\end{center}

\begin{exo}\quad\\
\begin{enumerate}
\item
\begin{enumerate}
\item Monter que l'équation : $y'' + y = 3x^2$ a une solution de la forme: $ \ x \mapsto ax^2 +bx +c$ avec $a,b,c\in\R$.
\item En déduire une expression explicite de l'unique solution sur $\R$ de l'équation différentielle:  
$$y'' + y = 3x^2 , \text{ avec }  \  y(0)=1 \ \text{ et } \  y'(0) = 2 $$
\end{enumerate}
\item
\begin{enumerate}
\item Monter que l'équation : $2y'' -3y'+  y = xe^x$ a une solution de la forme:  $ \ x \mapsto \left(ax^2+bx\right)e^x$ \quad avec $a,b\in\R$.
\item En déduire toutes les solutions réelles sur $\R$ de l'équation différentielle:  
$$2y'' -3y'+  y = xe^x $$
\end{enumerate}
\end{enumerate}
\end{exo}
\begin{center}
\rule{1\linewidth}{0.6pt}
\end{center}

\begin{exo}\quad\\
Déterminer une équation différentielle vérifiée par la famille de fonctions
$$y(x) \ = \ C_1e^{3x} + C_2e^{-2x} + x\sinh(x) ~,~~ C_1,C_2\in\R$$
\end{exo}

\begin{center}
\rule{1\linewidth}{0.6pt}
\end{center}

	\end{minipage}
\end{center}

\newpage

\section*{Suites numériques :}

\begin{center}
\begin{minipage}[t]{0.47\linewidth}
\raggedright

\begin{exo}\quad\\
On dira qu'une suite réelle $\left(u_n\right)_{n\in\N} $ est solution de \eqref{equ2} si pour tous $n\in\N$:
\begin{equation}
u_{n+2} = 2u_{n+1} +8u_n+9n^2
\label{equ2}
\end{equation}

\begin{enumerate}
\item Montrer que \eqref{equ2} possède une solution de la forme $\left(an^2+bn+c\right)_{n\in\N} $ avec $a,b,c\in\R$.

\item En déduire une expression explicite de l'unique solution de \eqref{equ2} pour laquelle:
$$u_{0} = 0 \ \text{ et } u_1 = 1$$

\end{enumerate}
\end{exo}
\begin{center}
\rule{1\linewidth}{0.6pt}
\end{center}

\begin{exo}\quad\\
Pour tout vecteur du plan fixé $\left(\begin{array}{c}
x_0\\
y_0
\end{array}\right) \ \in\R^2$, on considère la suite définie par récurrence :
$$\left(\begin{array}{c}
x_{n+1}\\
y_{n+1}
\end{array}\right) \ = \ \left(\begin{array}{cc}
0 & -1\\
1 & 0
\end{array}\right) \left(\begin{array}{c}
x_n\\
y_n
\end{array}\right)$$
\begin{enumerate}
\item En partant de $\left(\begin{array}{c}
x_0\\
y_0
\end{array}\right)  =  \left(\begin{array}{c}
1\\
0
\end{array}\right) $, représenter les $8$ premiers termes de la suite.
\item Cette suite est elle convergente ? 
\end{enumerate}
\end{exo}

\begin{center}
\rule{1\linewidth}{0.6pt}
\end{center}

\end{minipage}	
\hfill\vrule\hfill
\begin{minipage}[t]{0.47\linewidth}
\raggedright

\begin{exo}\quad\\
Soient $a$ et $b$ deux réels, $(u_n)_{n\geq 0}$ une suite telle que :
$$\forall n\in \N,\quad  u_{n+1} =au_n +b.$$
\begin{enumerate}
\item Traiter le cas $a = 1$.\\
On suppose désormais $a\neq 1$
\item Résoudre l'équation $x = ax + b$. On note $l$ la solution. \\
 On pose, pour $n$ dans $\N$:
$$v_n = u_n - l$$
\item Montrer que $(v_n)_{n\in\N}$ est une suite géométrique. Conclure.
\item La suite $(u_n)_{n\in\N}$ est-elle convergente ? 
\item Calculer en fonction de $n$ le terme général de la suite $\left(u_n\right)_{n\in\N}$ définie par :
$$\left\{\begin{array}{l}
u_{n+1} = 5u_n - 3\\
u_0 = 1
\end{array}\right.$$
\end{enumerate}
\end{exo}
\begin{center}
\rule{1\linewidth}{0.6pt}
\end{center}

\begin{exo}\quad\\
Calculer en fonction de $n$ le terme général de la suite $\left(u_n\right)_{n\in\N}$ définie par :
\begin{enumerate}
\item $u_0 = 1$ et pour tout $n\in\N$:  %\mathcal{U}
$$u_{n+1} = 2u_n^2$$
\item $u_0 = 1$, $u_1 = 2$ et pour tout $n\in\N$:
$$u_{n+2} = \dfrac{u_{n+1}^6}{u_{n}^5}$$
\end{enumerate}
\end{exo}
\begin{center}
\rule{1\linewidth}{0.6pt}
\end{center}
	\end{minipage}
\end{center}
\quad \\



\quad\\
\section*{Exercices supplémentaires :}

\begin{center}
\begin{minipage}[t]{0.47\linewidth}
\raggedright


\begin{exo}\quad\\
\begin{enumerate}
\item Pour tous $\alpha\in\R$ et $x>-1$ comparer $\left(1+x\right)^{\alpha}$ et $1+\alpha x$.\quad (\textbf{indication}: dérivées successives)

\begin{remarque}
Si $\alpha\in\N^*$ on parle de l'inégalité de Bernouilli.
\end{remarque}
\item En déduire que pour tous $\alpha\in[0;1]$ et $n\in\N^{*}:$

$$\prod_{k=1}^{n}\left(1+\dfrac{\alpha}{k}\right) \geq \left(n+1\right)^{\alpha}$$
\end{enumerate}

\end{exo}
\begin{center}
\rule{1\linewidth}{0.6pt}
\end{center}

\begin{exo}\quad\\
Montrer que la fonction 
$x \longmapsto \dfrac{1}{\cosh(x)}$ possède un unique point fixe.
\end{exo}

\begin{center}
\rule{1\linewidth}{0.6pt}
\end{center}

\end{minipage}	
\hfill\vrule\hfill
\begin{minipage}[t]{0.47\linewidth}
\raggedright

\begin{exo}\quad\\
\begin{enumerate}
\item Montrer que la fonction $\sinh$ est bijective de $\R$ sur $\R$ et déterminer une expression explicite de sa réciproque en résolvant l'équation : $y = \sinh(x)$ d'inconnue $x\in\R$ pour tout $y\in\R$.
\item Même question avec la fonction $\tanh$, bijective de $\R$ sur $]-1;1[$. 
\item Même question avec la fonction $\cosh$, bijective de $\R^+$ sur $[1;+\infty[$.
\end{enumerate}
\end{exo}
\begin{center}
\rule{1\linewidth}{0.6pt}
\end{center}
\begin{exo}\quad\\
Montrer que pour tous $x,y\in\R$ tels que $0<x<y$\\
 on a:
$$\dfrac{y-x}{\ln(y)-\ln(x)} < \dfrac{x+y}{2}$$
\textbf{indication:} on utilisera   $t=\dfrac{y}{x}$. 
\end{exo}

\begin{center}
\rule{1\linewidth}{0.6pt}
\end{center}
	\end{minipage}
\end{center}


\end{document}