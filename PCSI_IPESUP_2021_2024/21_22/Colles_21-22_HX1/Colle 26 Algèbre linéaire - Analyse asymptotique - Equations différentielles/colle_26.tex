\documentclass[a4paper,10pt]{article}



\usepackage{fancyhdr} % pour personnaliser les en-têtes
\usepackage[utf8]{inputenc}
\usepackage[T1]{fontenc}
\usepackage{lastpage}
\usepackage[frenchb]{babel}
\usepackage{amsfonts,amssymb}
\usepackage{amsmath,amsthm,mathtools}
\usepackage{paralist}
\usepackage{xspace,xypic}
\usepackage{xcolor,multicol,tabularx}
\usepackage{variations}
\usepackage{xypic}
\usepackage{eurosym,multicol}
\usepackage{graphicx}
\usepackage{mathdots}%faire des points suspendus en diagonale
\usepackage[np]{numprint}
\usepackage{hyperref} 
\usepackage{relsize,exscale}
\usepackage{listings} % pour écrire des codes avec coloration syntaxique  

\usepackage{tikz}
\usetikzlibrary{calc, arrows, plotmarks,decorations.pathreplacing}
\usepackage{colortbl}
\usepackage{multirow}
\usepackage[top=2cm,bottom=1.5cm,right=2cm,left=1.5cm]{geometry}

\newtheorem{thm}{Théorème}
\newtheorem*{pro}{Propriété}
\newtheorem*{exemple}{Exemple}

\theoremstyle{definition}
\newtheorem*{remarque}{Remarque}
\theoremstyle{definition}
\newtheorem{exo}{Exercice}
\newtheorem{definition}{Définition}


\newcommand{\vtab}{\rule[-0.4em]{0pt}{1.2em}}
\newcommand{\V}{\overrightarrow}
\renewcommand{\thesection}{\Roman{section} }
\renewcommand{\thesubsection}{\arabic{subsection} }
\renewcommand{\thesubsubsection}{\alph{subsubsection} }
\newcommand*{\transp}[2][-3mu]{\ensuremath{\mskip1mu\prescript{\smash{\mathrm t\mkern#1}}{}{\mathstrut#2}}}%

\newcommand{\K}{\mathbb{K}}
\newcommand{\C}{\mathbb{C}}
\newcommand{\R}{\mathbb{R}}
\newcommand{\Q}{\mathbb{Q}}
\newcommand{\Z}{\mathbb{Z}}
\newcommand{\N}{\mathbb{N}}
\newcommand{\p}{\mathbb{P}}
\newcommand{\M}{\mathcal{M}}

\renewcommand{\Im}{\mathop{\mathrm{Im}}\nolimits}



\definecolor{vert}{RGB}{11,160,78}
\definecolor{rouge}{RGB}{255,120,120}
% Set the beginning of a LaTeX document
\pagestyle{fancy}
\lhead{Optimal Sup Spé, groupe IPESUP}\chead{Année~2021-2022}\rhead{Niveau: Première année de PCSI }\lfoot{M. Botcazou}\cfoot{\thepage}\rfoot{mail: ibotca52@gmail.com }\renewcommand{\headrulewidth}{0.4pt}\renewcommand{\footrulewidth}{0.4pt}

\begin{document}
 	

\begin{center}
\Large \sc colle 26 = Algèbre linéaire, analyse asymptotique et équations différentielles
\end{center}

\section*{Exercices mixtes:}%[-0.25cm]

\raggedright

\begin{exo}\textbf{}\quad\\[0.25cm]
Soit $n$ un entier naturel tel que $n \geq 2$.
Soit $E$ l’espace vectoriel des polynômes à coefficients dans $\K$ ($\K = \R$ ou $\K = \C$) de degré inférieur ou égal à $n$.
On pose : \ $ \forall P \in E, \  f (P) = P - P'$
\begin{enumerate}
	\item Justifier que $f$ est un automorphisme sans utiliser de matrice de f.
	\item Donner la matrice $A$ de $f$ dans la base canonique de $E$.
	 \item Soit $Q \in E$. Trouver $P$ tel que $f (P) = Q$.\\
	(\textit{Indication : si $P \in E$, quel est le polynôme $P^{(n+1)}$ ?})
	\item Existe-t-il une base $\beta$ de $E$ telle que la matrice de $f$ dans cette base soit la matrice $I_{n+1}$. 
	
\end{enumerate}

\centering
\rule{1\linewidth}{0.6pt}
\end{exo}

\begin{exo}\textbf{}\quad\\[0.25cm]
Soit $f$ un endomorphisme d’un espace vectoriel $E$ de dimension finie $n$.
\begin{enumerate}
	\item Démontrer que : $E = \Im f \oplus Ker f \Rightarrow \Im f = \Im f^2$.
	\item \begin{enumerate}
		\item Démontrer que : $\Im f = \Im f^2 \Longleftrightarrow Ker f = Kerf^2$ .
		\item Démontrer que : $\Im f = \Im f^2 \Rightarrow E = \Im f \oplus Kerf$ .
		\item Que pouvez-vous conclure? 
	\end{enumerate}
\end{enumerate}

\centering
\rule{1\linewidth}{0.6pt}
\end{exo}

\begin{exo}\textbf{}\quad\\[0.25cm]
Soit la matrice $A = \begin{pmatrix}
1 & 2\\2&4\end{pmatrix}$ et $f$ l’endomorphisme de $\M_2(\R)$ défini par : $f (M) = AM$.
 
	\begin{enumerate}
		\item Déterminer une base de $Kerf$ .
		\item f est-il surjectif ?
		\item Déterminer une base de $\Im f$.
		\item A-t-on $\M_2 (\R) = Kerf \oplus \Im f$ ? 
	\end{enumerate}
	
	\centering
	\rule{1\linewidth}{0.6pt}
\end{exo}


\begin{exo}\quad\\[0.25cm]
Soit $n\in\N $, on note $(u_n)$  la suite définie par
$u_0 = 0$ et pour tout $n  \in\N$ : $u_{n+1}= \sqrt{u_n + n^2}$ .\\
Montrer que: $$ \large u_n \underset{n\rightarrow+\infty}{=} n -\dfrac{1}{2} -\dfrac{3}{8n} + o\left(\dfrac{1}{n}\right)$$

\centering
\rule{1\linewidth}{0.6pt}
\end{exo}			


\begin{exo}\quad\\[0.25cm]
	Soit $n\in\N $, on note $(u_n)$  la suite définie par
	$u_0 = 1$ et pour tout $n  \in\N$ : $u_{n+1}= u_n + \dfrac{1}{u_n}$ .\\
	Montrer que: $$ \large u_n \underset{n\rightarrow+\infty}{=} \sqrt{2n} + O\left(\dfrac{\ln(n)}{\sqrt{n}}\right)$$
	
	
	\centering
	\rule{1\linewidth}{0.6pt}
\end{exo}

\begin{exo}\quad\\[0.25cm]
	\begin{enumerate}
		\item Justifier que pour tout $\varepsilon > 0$, l’équation $e^{-\varepsilon x}= x$ d’inconnue $x$ possède une et une seule solution \\$x_\varepsilon$ dans $R^+.$
		\item Montrer que: $$ \large x_\varepsilon \underset{\varepsilon\rightarrow0}{=} 1 - \varepsilon + \dfrac{3\varepsilon^2}{2} + o(\varepsilon^2) $$
	\end{enumerate}

\newpage	
	
	
	\centering
	\rule{1\linewidth}{0.6pt}
\end{exo}

\begin{exo}\quad\\[0.25cm]
	\begin{enumerate}
		\item Déterminer une primitive de $x\longmapsto\cos^4(x)$
		\item Résoudre sur $\R$ l’équation différentielle : $y'' + y = \cos^3(x)$  en utilisant la méthode de variation des
		constantes.
	\end{enumerate}
	
	\centering
	\rule{1\linewidth}{0.6pt}
\end{exo}




\end{document}