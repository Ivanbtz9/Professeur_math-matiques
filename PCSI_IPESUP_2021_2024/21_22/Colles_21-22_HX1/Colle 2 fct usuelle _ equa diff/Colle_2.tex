\documentclass[10pt,a4paper]{article}


\usepackage[T1]{fontenc}
\usepackage[francais]{babel}
\usepackage{times}
\usepackage[utf8]{inputenc}
\usepackage{enumitem}
\usepackage{multicol}
\usepackage{fancyhdr}
\usepackage{tcolorbox}
\usepackage{tablists}
\usepackage[a4paper,bottom = 50pt]{geometry}
\usepackage{color}
\usepackage{amsmath,amssymb,amsthm, mathrsfs,pifont}
\usepackage{pgf,tikz,pgfplots,tkz-tab}
\usepackage{hyperref}
\usepackage{cancel}
\usepackage{array}
\usepackage{xcolor} 
\usepackage{amssymb}
\usepackage{amsthm}
\usepackage{graphicx}
\usepackage{pstricks}

\usepackage{geometry}


\setcellgapes{1pt}
\makegapedcells
\newcolumntype{R}[1]{>{\raggedleft\arraybackslash }b{#1}}
\newcolumntype{L}[1]{>{\raggedright\arraybackslash }b{#1}}
\newcolumntype{C}[1]{>{\centering\arraybackslash }b{#1}}

\geometry{top=2cm, bottom=2cm, left=2cm, right=2cm}

\newcommand{\R}{\mathbb{R}}
\newcommand{\Z}{\mathbb{Z}}
\newcommand{\N}{\mathbb{N}}
\newcommand{\notni}{\not\owns}

\newtheorem{thm}{Théorème}
\newtheorem*{pro}{Propriété}
\newtheorem*{exemple}{Exemple}

\theoremstyle{definition}
\newtheorem*{remarque}{Remarque}
\theoremstyle{definition}
\newtheorem{exo}{Exercice}
\newtheorem{definition}{Définition}



\begin{document}
	
	\leftline{\bfseries Optimal Sup Spé, groupe IPESUP \hfill Année~2021-2022}
	\leftline{\bfseries Niveau: Première année de PCSI  }
	\leftline{\bfseries M.Botcazou \hfill mail: ibotca52@gmail.com  }
	\rule[0.5ex]{\textwidth}{0.1mm}	
	
	\begin{center}
		\Large \sc colle 2 = fonctions usuelles, sommes, produits et équations différentielles	
	\end{center}
	
\section*{Fonctions usuelles :}

\begin{center}
\begin{minipage}[t]{0.47\linewidth}
\raggedright

%\begin{exo}\quad\\ 
%Soit f une fonction continue sur un intervalle $I$ de $\R$ telle que:
%$$\forall x \in I ,~f(x)^2 \ = \ 1$$
%Montrer que la fonction $f$ est constante sur $I$
%\end{exo}
\begin{exo}
Résoudre l'équation $\cosh(x)=2$.
\end{exo}
\begin{center}
\rule{1\linewidth}{0.6pt}
\end{center}

\begin{exo}
Montrer que pour tout $x\neq 0$,
$$\sum\limits_{k=0}^{n} \cosh(kx) \ = \ \dfrac{\cosh\left(\dfrac{nx}{2}\right)\sinh\left(\dfrac{(n+1)x}{2}\right)}{\sinh\left(\dfrac{x}{2}\right)}$$
\end{exo}
\begin{center}
\rule{1\linewidth}{0.6pt}
\end{center}

\begin{exo}
Montrer que la fonction 
$x \longmapsto \dfrac{1}{\cosh(x)}$ possède un unique point fixe.
\end{exo}

\begin{center}
\rule{1\linewidth}{0.6pt}
\end{center}




\end{minipage}	
\hfill\vrule\hfill
\begin{minipage}[t]{0.47\linewidth}
\raggedright

\begin{exo}
Montrer que pour tout $n\geq 2$:
$$ \left(1+\dfrac{1}{n}\right)^n \ \leq \ e \ \leq \ \left(1-\dfrac{1}{n}\right)^n$$
\end{exo}

\begin{center}
\rule{1\linewidth}{0.6pt}
\end{center}

\begin{exo}
Démontrer que, pour tout $x\in\R$ et tout $n\geq 1$, on a
$$\left(\dfrac{1+ \tanh(x)}{1 - \tanh(x)}\right)^n \ = \ \dfrac{1+ \tanh(nx)}{1- \tanh(nx)}$$ 
\end{exo}

\begin{center}
\rule{1\linewidth}{0.6pt}
\end{center}

\begin{exo}
Résoudre l'équation: $$2x\ln(x) + 3(x-1) \ = \ 0 $$
\end{exo}
\begin{center}
\rule{1\linewidth}{0.6pt}
\end{center}


	\end{minipage}
\end{center}
\quad \\

\section*{Sommes et Produits :}
\begin{center}
\begin{minipage}[t]{0.47\linewidth}
\raggedright

\begin{exo}\quad\\
Pour $n\in\N$ montrer que : 
$$\sum\limits_{k=0}^{n}\sum\limits_{l=0}^{n} \min(k,l) \ = \ \dfrac{n}{6}\left(2n^2+3n+1\right)$$
\end{exo}
\begin{center}
\rule{1\linewidth}{0.6pt}
\end{center} 

\begin{exo}\quad\\ 
Calculer $\sum\limits_{k=1}^{n} k \ln\left(1+\dfrac{1}{k}\right)$ pour tout     $n\in\N^*$ en faisant apparaître un téléscopage.
\end{exo}
\begin{center}
\rule{1\linewidth}{0.6pt}
\end{center}

\begin{exo}\quad\\
On pose pour tout $n\in\N^*$ 
$$ H_n  \ = \ \sum_{k=1}^{n} \dfrac{1}{k}$$
Montrer que:  
$$\forall n\in\N^* , \  \ \sum_{i=1}^{n} H_i  \ = \ (n+1)H_n - n$$
\end{exo}
\begin{center}
\rule{1\linewidth}{0.6pt}
\end{center}

\end{minipage}	
\hfill\vrule\hfill
\begin{minipage}[t]{0.47\linewidth}
\raggedright


\begin{exo}\quad\\
\begin{enumerate}
\item Factoriser $(k^3-1)$ par $(k-1)$ et $(k^3+1)$ par $(k+1)$ pour tout $k\geq 2$
\item En déduire une simplification du produit 
$$\prod\limits_{k=2}^{n}~\dfrac{k^3-1}{k^3+1}$$
\item~En déduire l'existence et la valeur de $$\lim\limits_{n \rightarrow +\infty} \prod\limits_{k=2}^{n}~\dfrac{k^3-1}{k^3+1} $$ que l'on notera aussi~~~ $\prod\limits_{k=2}^{+\infty}~\dfrac{k^3-1}{k^3+1}$
\end{enumerate}
\end{exo}
\begin{center}
\rule{1\linewidth}{0.6pt}
\end{center}

\begin{exo}\quad\\
Montrer que pour tout $n\geq 2, n\in\N$ on a :

$$\prod\limits_{k=1}^{n}\prod\limits_{l=1}^{n} \min(k,l) \ = \ n!\prod\limits_{k=1}^{n-1}k!(n-k)^{k}$$



\end{exo}
\begin{center}
\rule{1\linewidth}{0.6pt}
\end{center}




	\end{minipage}
\end{center}
\quad\\
\section*{Équations différentielles :}
\begin{center}
\begin{minipage}[t]{0.47\linewidth}
\raggedright

\begin{exo}
Résoudre les équations différentielles suivantes :
\begin{enumerate}
\item $y' + 2y = x^2 -2x +3  \ \  \text{sur} \ \R;$
\item$y' + y =  \dfrac{1}{1+e^x}  \ \  \text{sur} \ \R; $
\item$y' - 2xy  =  -(2x-1)e^x \  \ \text{sur} \  \R; $


\end{enumerate}
\end{exo} 
\begin{center}
\rule{1\linewidth}{0.6pt}
\end{center}

\begin{exo}
Donner une équation différentielle dont les solutions sont les fonctions de la forme 
$$x\longmapsto \dfrac{C + x}{1+ x^2}, \  C \in\R$$
\end{exo}
\begin{center}
\rule{1\linewidth}{0.6pt}
\end{center} 

\end{minipage}	
\hfill\vrule\hfill
\begin{minipage}[t]{0.47\linewidth}
\raggedright

\begin{exo}\quad\\
Donner l'ensemble solution des équations différentielles suivantes :
\begin{enumerate}
\item$y'' -2y'+y \ = \ 0 \ , \  y(0)= y'(0) = 1; $
\item$y'' +9y \ = \ 0 \ , \  y(0)= 0; $
\item $y'' +  y'  - y\ = \ 0 $
\end{enumerate}
\end{exo}
\begin{center}
\rule{1\linewidth}{0.6pt}
\end{center}

\begin{exo}\quad\\
Déterminer une équation différentielle vérifiée par la famille de fonctions

$$y(x) \ = \ C_1e^{2x} + C_2e^{-x} + x\cosh(x) ~,~~ C_1,C_2\in\R$$
\end{exo}
\begin{center}
\rule{1\linewidth}{0.6pt}
\end{center}

 
	\end{minipage}
\end{center}
\quad\\
\section*{Exercice supplémentaire :}

\noindent Soient $n\in\N^*$ et $a_1,...a_n,b_1,...,b_n$ des nombres réels. \\
On définit la fonction $f$ par:

$$\forall x \in \R ~, ~f(x) \ = \ \sum\limits_{i=1}^{n}(a_ix+b_i)^2$$

Montrer l'inégalité de Cauchy-Schwarz :

 $$\left|\sum\limits_{i=1}^{n}a_ib_i\right| \ \leq \ \sqrt{\sum\limits_{i=1}^{n}a_i^2} \times \sqrt{\sum\limits_{i=1}^{n}b_i^2}$$
\quad \\

$\left(\textit{\underline{Indication}: remarquer que la fonction f est à valeur dans $\R^+$}\right)$


\end{document}