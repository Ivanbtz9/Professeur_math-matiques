
%%%%%%%%%%%%%%%%%% PREAMBULE %%%%%%%%%%%%%%%%%%

\documentclass[11pt,a4paper]{article}

\usepackage{amsfonts,amsmath,amssymb,amsthm}
\usepackage[utf8]{inputenc}
\usepackage[T1]{fontenc}
\usepackage[francais]{babel}
\usepackage{mathptmx}
\usepackage{fancybox}
\usepackage{graphicx}
\usepackage{ifthen}

\usepackage{tikz}   

\usepackage{hyperref}
\hypersetup{colorlinks=true, linkcolor=blue, urlcolor=blue,
pdftitle={Exo7 - Exercices de mathématiques}, pdfauthor={Exo7}}

\usepackage{geometry}
\geometry{top=2cm, bottom=2cm, left=2cm, right=2cm}

%----- Ensembles : entiers, reels, complexes -----
\newcommand{\Nn}{\mathbb{N}} \newcommand{\N}{\mathbb{N}}
\newcommand{\Zz}{\mathbb{Z}} \newcommand{\Z}{\mathbb{Z}}
\newcommand{\Qq}{\mathbb{Q}} \newcommand{\Q}{\mathbb{Q}}
\newcommand{\Rr}{\mathbb{R}} \newcommand{\R}{\mathbb{R}}
\newcommand{\Cc}{\mathbb{C}} \newcommand{\C}{\mathbb{C}}
\newcommand{\Kk}{\mathbb{K}} \newcommand{\K}{\mathbb{K}}

%----- Modifications de symboles -----
\renewcommand{\epsilon}{\varepsilon}
\renewcommand{\Re}{\mathop{\mathrm{Re}}\nolimits}
\renewcommand{\Im}{\mathop{\mathrm{Im}}\nolimits}
\newcommand{\llbracket}{\left[\kern-0.15em\left[}
\newcommand{\rrbracket}{\right]\kern-0.15em\right]}
\renewcommand{\ge}{\geqslant} \renewcommand{\geq}{\geqslant}
\renewcommand{\le}{\leqslant} \renewcommand{\leq}{\leqslant}

%----- Fonctions usuelles -----
\newcommand{\ch}{\mathop{\mathrm{ch}}\nolimits}
\newcommand{\sh}{\mathop{\mathrm{sh}}\nolimits}
\renewcommand{\tanh}{\mathop{\mathrm{th}}\nolimits}
\newcommand{\cotan}{\mathop{\mathrm{cotan}}\nolimits}
\newcommand{\Arcsin}{\mathop{\mathrm{arcsin}}\nolimits}
\newcommand{\Arccos}{\mathop{\mathrm{arccos}}\nolimits}
\newcommand{\Arctan}{\mathop{\mathrm{arctan}}\nolimits}
\newcommand{\Argsh}{\mathop{\mathrm{argsh}}\nolimits}
\newcommand{\Argch}{\mathop{\mathrm{argch}}\nolimits}
\newcommand{\Argth}{\mathop{\mathrm{argth}}\nolimits}
\newcommand{\pgcd}{\mathop{\mathrm{pgcd}}\nolimits} 

%----- Structure des exercices ------

\newcommand{\exercice}[1]{\video{0}}
\newcommand{\finexercice}{}
\newcommand{\noindication}{}
\newcommand{\nocorrection}{}

\newcounter{exo}
\newcommand{\enonce}[2]{\refstepcounter{exo}\hypertarget{exo7:#1}{}\label{exo7:#1}{\bf Exercice \arabic{exo}}\ \  #2\vspace{1mm}\hrule\vspace{1mm}}

\newcommand{\finenonce}[1]{
\ifthenelse{\equal{\ref{ind7:#1}}{\ref{bidon}}\and\equal{\ref{cor7:#1}}{\ref{bidon}}}{}{\par{\footnotesize
\ifthenelse{\equal{\ref{ind7:#1}}{\ref{bidon}}}{}{\hyperlink{ind7:#1}{\texttt{Indication} $\blacktriangledown$}\qquad}
\ifthenelse{\equal{\ref{cor7:#1}}{\ref{bidon}}}{}{\hyperlink{cor7:#1}{\texttt{Correction} $\blacktriangledown$}}}}
\ifthenelse{\equal{\myvideo}{0}}{}{{\footnotesize\qquad\texttt{\href{http://www.youtube.com/watch?v=\myvideo}{Vidéo $\blacksquare$}}}}
\hfill{\scriptsize\texttt{[#1]}}\vspace{1mm}\hrule\vspace*{7mm}}

\newcommand{\indication}[1]{\hypertarget{ind7:#1}{}\label{ind7:#1}{\bf Indication pour \hyperlink{exo7:#1}{l'exercice \ref{exo7:#1} $\blacktriangle$}}\vspace{1mm}\hrule\vspace{1mm}}
\newcommand{\finindication}{\vspace{1mm}\hrule\vspace*{7mm}}
\newcommand{\correction}[1]{\hypertarget{cor7:#1}{}\label{cor7:#1}{\bf Correction de \hyperlink{exo7:#1}{l'exercice \ref{exo7:#1} $\blacktriangle$}}\vspace{1mm}\hrule\vspace{1mm}}
\newcommand{\fincorrection}{\vspace{1mm}\hrule\vspace*{7mm}}

\newcommand{\finenonces}{\newpage}
\newcommand{\finindications}{\newpage}


\newcommand{\fiche}[1]{} \newcommand{\finfiche}{}
%\newcommand{\titre}[1]{\centerline{\large \bf #1}}
\newcommand{\addcommand}[1]{}

% variable myvideo : 0 no video, otherwise youtube reference
\newcommand{\video}[1]{\def\myvideo{#1}}

%----- Presentation ------

\setlength{\parindent}{0cm}

\definecolor{myred}{rgb}{0.93,0.26,0}
\definecolor{myorange}{rgb}{0.97,0.58,0}
\definecolor{myyellow}{rgb}{1,0.86,0}

\newcommand{\LogoExoSept}[1]{  % input : echelle       %% NEW
{\usefont{U}{cmss}{bx}{n}
\begin{tikzpicture}[scale=0.1*#1,transform shape]
  \fill[color=myorange] (0,0)--(4,0)--(4,-4)--(0,-4)--cycle;
  \fill[color=myred] (0,0)--(0,3)--(-3,3)--(-3,0)--cycle;
  \fill[color=myyellow] (4,0)--(7,4)--(3,7)--(0,3)--cycle;
  \node[scale=5] at (3.5,3.5) {Exo7};
\end{tikzpicture}}
}


% titre
\newcommand{\titre}[1]{%
\vspace*{-4ex} \hfill \hspace*{1.5cm} \hypersetup{linkcolor=black, urlcolor=black} 
\href{http://exo7.emath.fr}{\LogoExoSept{3}} 
 \vspace*{-5.7ex}\newline 
\hypersetup{linkcolor=blue, urlcolor=blue}  {\Large \bf #1} \newline 
 \rule{12cm}{1mm} \vspace*{3ex}}

%----- Commandes supplementaires ------



\begin{document}

%%%%%%%%%%%%%%%%%% EXERCICES %%%%%%%%%%%%%%%%%%
\fiche{f00095, rouget, 2010/07/11}

\titre{Dénombrements} 

Exercices de Jean-Louis Rouget.
Retrouver aussi cette fiche sur \texttt{\href{http://www.maths-france.fr}{www.maths-france.fr}}

\begin{center}
* très facile\quad** facile\quad*** difficulté moyenne\quad**** difficile\quad***** très difficile\\
I~:~Incontournable\quad T~:~pour travailler et mémoriser le cours
\end{center}


\exercice{5278, rouget, 2010/07/04}
\enonce{005278}{IT}
\begin{enumerate}
\item (***) Trouver une démonstration combinatoire de l'identité $\sum_{}^{}C_n^{2k}=\sum_{}^{}C_n^{2k+1}$ ou encore démontrer directement qu'un ensemble à $n$ éléments contient autant de parties de cardinal pair que de parties de cardinal impair.
\item (****) Trouver une démonstration combinatoire de l'identité $kC_n^k=nC_{n-1}^{k-1}$.
\item (****) Trouver une démonstration combinatoire de l'identité $C_{2n}^n=\sum_{k=0}^{n}(C_{n}^k)^2$.
\end{enumerate}
\finenonce{005278}


\finexercice
\exercice{5279, rouget, 2010/07/04}
\enonce{005279}{***}
Combien y a-t-il de partitions d'un ensemble à $pq$ éléments en $p$ classes ayant chacune $q$ éléments~? (Si $E$ est un ensemble à $pq$ éléments et si $A_1$,..., $A_p$ sont $p$ parties de $E$, $A_1$,..., $A_p$ forment une partition de $E$ si et seulement si tout élément de $E$ est dans une et une seule des parties $A_i$. Il revient au même de dire que la réunion des $A_i$ est $E$ et que les $A_i$ sont deux à deux disjoints.)
\finenonce{005279}


\finexercice\exercice{5280, rouget, 2010/07/04}
\enonce{005280}{***}
Combinaisons avec répétitions. Montrer que le nombre de solutions en nombres entiers $x_i\geq0$ de l'équation $x_1+x_2+...+x_n=k$ ($k$ entier naturel donné) est $C_{n+k-1}^k$. (Noter $a_{n,k}$ le nombre de solutions et procéder par récurrence.) 
\finenonce{005280}


\finexercice
\exercice{5281, rouget, 2010/07/04}
\enonce{005281}{*}
Combien y a-t-il de nombres de $5$ chiffres où $0$ figure une fois et une seule~?
\finenonce{005281}


\finexercice
\exercice{5282, rouget, 2010/07/04}
\enonce{005282}{***I}
Quelle est la probabilité $p_n$ pour que dans un groupe de $n$ personnes choisies au hasard, deux personnes au moins aient le même anniversaire (on considèrera que l'année a toujours $365$ jours, tous équiprobables). Montrer que pour $n\geq23$, on a $p_n\geq\frac{1}{2}$.
\finenonce{005282}


\finexercice
\exercice{5283, rouget, 2010/07/04}
\enonce{005283}{***}
Montrer que le premier de l'an tombe plus souvent un dimanche qu'un samedi.
\finenonce{005283}


\finexercice
\exercice{5284, rouget, 2010/07/04}
\video{4EJcG9wA3cI}
\enonce{005284}{**I}

On part du point de coordonnées $(0,0)$ pour rejoindre le point de coordonnées $(p,q)$ ($p$ et $q$ entiers naturels donnés) en se déplaçant à chaque étape d'une unité vers la droite ou vers le haut. Combien y a-t-il de chemins possibles~?
\finenonce{005284}





\finexercice

\exercice{5285, rouget, 2010/07/04}
\enonce{005285}{***I}
De combien de façons peut-on payer $100$ euros avec des pièces de $10$, $20$ et $50$ centimes~?
\finenonce{005285}


\finexercice
\exercice{5286, rouget, 2010/07/04}
\enonce{005286}{****}
\begin{enumerate}
\item  Soit $E$ un ensemble fini et non vide. Soient $n$ un entier naturel non nul et $A_1$,..., $A_n$, $n$ parties de $E$. Montrer la \og~formule du crible~\fg~:

\begin{align*}\ensuremath
\mbox{card}(A_1\cup...\cup A_n)&=\sum_{i=1}^{n}\mbox{card}(A_i)-\sum_{1\leq i_1< i_2\leq n}^{}\mbox{card}(A_{i_1}\cap A_{i_2})\\
 &+...+(-1)^{k-1}\sum_{1\leq i_1<i_2<...<i_k\leq n}^{}\mbox{card}(A_{i_1}\cap A_{i_2}\cap...\cap A_{i_k})\\
 &+...+(-1)^{n-1}\mbox{card}(A_1\cap...\cap A_n).
\end{align*}

\item  Combien y a-t-il de permutations $\sigma$ de $\{1,...,n\}$ vérifiant $\forall i\in\{1,...,n\},\;\sigma(i)\neq i$~?~(Ces permutations sont appelées dérangements (permutations sans point fixe)). Indication~:~noter $A_i$ l'ensemble des permutations qui fixent $i$ et utiliser 1).

On peut alors résoudre un célèbre problème de probabilité, le problème des chapeaux. $n$ personnes laissent leur chapeau à un vestiaire. En repartant, chaque personne reprend un chapeau au hasard. Montrer que la probabilité qu'aucune de ces personnes n'ait repris son propre chapeau est environ $\frac{1}{e}$ quand $n$ est grand.
\end{enumerate}
\finenonce{005286}


\finexercice
\exercice{5287, rouget, 2010/07/04}
\enonce{005287}{**}
Combien y a-t-il de surjections de $\{1,...,n+1\}$ sur $\{1,...,n\}$~?
\finenonce{005287}


\finexercice
\exercice{5288, rouget, 2010/07/04}
\enonce{005288}{***}
Soit $(P)$ un polygone convexe à $n$ sommets. Combien ce polygone a-t-il de diagonales~?~En combien de points distincts des sommets se coupent-elles au maximum~?
\finenonce{005288}


\finexercice
\exercice{5289, rouget, 2010/07/04}
\enonce{005289}{***}
\begin{enumerate}
\item  On donne $n$ droites du plan. On suppose qu'il n'en existe pas deux qui soient parallèles, ni trois qui soient concourantes. Déterminer le nombre $P(n)$ de régions délimitées par ces droites.
\item  On donne $n$ plans de l'espace. On suppose qu'il n'en existe pas deux qui soient parallèles, ni trois qui soient concourants en une droite, ni quatre qui soient concourants en un point. Déterminer le nombre $Q(n)$ de régions délimitées par ces plans.
\end{enumerate}
\finenonce{005289}


\finexercice
\exercice{5290, rouget, 2010/07/04}
\enonce{005290}{***}
Soit $P_n^k$ le nombre de partitions d'un ensemble à $n$ éléments en $k$ classes.

Montrer que $P_n^k=P_{n-1}^{k-1}+kP_{n-1}^k$ pour $2\leq k\leq n-1$.

Dresser un tableau pour $1\leq k,n\leq 5$.

Calculer en fonction de $P_n^k$ le nombre de surjections d'un ensemble à $n$ éléments sur un ensemble à $p$ éléments.
\finenonce{005290}


\finexercice

\finfiche


 \finenonces 



 \finindications 

\noindication
\noindication
\noindication
\noindication
\noindication
\noindication
\indication{005284}
Coder un chemin par un mot : $D$ pour droite, $H$ pour haut.
\finindication
\noindication
\noindication
\noindication
\noindication
\noindication
\noindication


\newpage

\correction{005278}
\begin{enumerate}
\item  Soit $E$ un ensemble à $n$ éléments, $n\geq 1$, et $a$ un élément fixé de $E$. Soit $\begin{array}[t]{cccc}
f~:&\mathcal{P}(E)&\rightarrow&\mathcal{P}(E)\\
 &A&\mapsto&\left\{
 \begin{array}{l}
 A\setminus\{a\}\;\mbox{si}\;a\in A\\
 A\cup\{a\}\;\mbox{si}\;a\notin A
 \end{array}
 \right.
\end{array}$.

Montrons que $f$ est involutive (et donc bijective). Soit $A$ un élément de $\mathcal{P}(E)$.

Si $a\notin A$, $f(A)=A\cup\{a\}$ et donc, puisque $a\in A\cup\{a\}$, $f(f(A))=(A\cup\{a\})\setminus\{a\}=A$.

Si $a\in A$, $f(A)=A\setminus\{a\}$ et $f(f(A))=(A\setminus\{a\})\cup\{a\}=A$.

Ainsi, $\forall A\in\mathcal{P}(E),\;f\circ f(A)=A$ ou encore, $f\circ f=Id_{\mathcal{P}(E)}$.

Maintenant clairement, en notant $\mathcal{P}_p(E)$ (resp. $\mathcal{P}_i(E)$) l'ensemble des parties de $E$ de cardinal pair (resp. impair), $f(\mathcal{P}_p(E))\subset\mathcal{P}_i(E)$ et $f(\mathcal{P}_i(E))\subset\mathcal{P}_p(E)$. Donc, puisque $f$ est bijective 
$$\mbox{card}(\mathcal{P}_p(E))=\mbox{card}(f(P_p(E))\leq\mbox{card}\mathcal{P}_i(E)$$

et de même $\mbox{card}(\mathcal{P}_i(E))\leq\mbox{card}\mathcal{P}_p(E)$. Finalement, $\mbox{card}(\mathcal{P}_i(E))=\mbox{card}\mathcal{P}_p(E)$.

\item  
Soient $E=\{a_1,...,a_n\}$ un ensemble à $n$ éléments et $a$ un élément fixé de $E$. Soit $k\in\{1,...,n-1\}$.

Il y a  $C_{n-1}^{k-1}$ parties à $k$ éléments qui contiennent $a$. Donc, $nC_{n-1}^{k-1}(=C_{n-1}^{k-1}+...+C_{n-1}^{k-1})$ est donc la somme du nombre de parties à $k$ éléments qui contiennent $a_1$ et du nombre de parties à $k$ éléments qui contiennent $a_2$ ... et du nombre de parties à $k$ éléments qui contiennent $a_n$.

Dans cette dernière somme, chaque partie à $k$ éléments de $E$ a été comptée plusieurs fois et toutes les parties à $k$ éléments (en nombre égal à $C_n^k$) ont été comptés un même nombre de fois. Combien de fois a été comptée $\{a_1,a_2...a_k\}$~?~Cette partie a été comptée une fois en tant que partie contenant $a_1$, une fois en tant que partie contenant $a_2$... et une fois comme partie contenant $a_k$ et donc a été comptée $k$ fois.

Conclusion~:~$kC_n^k=nC_{n-1}^{k-1}$. 

\item  Soit $E=\{a_1,...,a_n,b_1,...,b_n\}$ un ensemble à $2n$ éléments. Il y a $C_{2n}^n$ parties à $n$ éléments de $E$. Une telle partie a $k$ éléments dans $\{a_1,...,a_n\}$ et $n-k$ dans $\{b_1,...,b_n\}$ pour un certain $k$ de $\{0,...,n\}$. Il y a $C_n^k$ choix possibles de $k$ éléments dans $\{a_1,...,a_n\}$ et $C_n^{n-k}$ choix possibles de $n-k$ éléments dans $\{b_1,...,b_n\}$ pour $k$ donné dans $\{0,...,n\}$ et quand $k$ varie de $0$ à $n$, on obtient~:

$$C_{2n}^n=\sum_{k=0}^{n}C_n^kC_n^{n-k}=\sum_{k=0}^{n}(C_n^k)^2.$$

\end{enumerate}
\fincorrection
\correction{005279}
Il y a $C_{pq}^q$ choix possibles d'une première classe. Cette première classe étant choisie, il y a $C_{pq-q}^{q}=C_{(p-1)q}^q$ choix possibles de la deuxième classe... et $C_{q}^q$ choix possibles de la $p$-ième classe. Au total, il y a $C_{pq}^qC_{(p-1)q}^q...C_{q}^q$ choix possibles d'une première classe, puis d'une deuxième ...puis d'une $p$-ième.

Maintenant dans le nombre $C_{pq}^qC_{(p-1)q}^q...C_{q}^q$, on a compté plusieurs fois chaque partition, chacune ayant été compté un nombre égal de fois.

On a compté chaque partition autant de fois qu'il y a de permutations des $p$ classes à savoir $p!$. Le nombre cherché est donc~:  

$$\frac{1}{p!}C_{pq}^qC_{(p-1)q}^q...C_{q}^q=\frac{1}{p!}\frac{(pq)!}{q!((p-1)q)!}\frac{((p-1)q)!}{q!((p-2)q)!}...
\frac{(2q)!}{q!q!}\frac{q!}{q!0!}=\frac{(pq)!}{p!(q!)^p}.$$
\fincorrection
\correction{005280}
Clairement, $\forall n\in\Nn^*,\;a_{n,0}=1$ (unique solution~:~$0+0+...+0=0$) et $\forall k\in\Nn,\;a_{1,k}=1$ (unique solution~:~$k=k$).

Soient $n\geq 1$ et $k\geq 0$ fixés. $a_{n+1,k}$ est le nombre de solutions en nombre entiers positifs $x_i$ de l'équation $x_1+...+x_n+x_{n+1}=k$.
Il y a $a_{n,k}$ solutions telles que $x_{n+1}=0$ puis $a_{n,k-1}$ solutions telles que $x_{n+1}=1$ ... puis $a_{n,0}$ solutions telles que $x_{n+1}=k$.

Donc, $\forall n\in\Nn^*,\;\forall k\in\Nn,\;a_{n+1,k}=a_{n,k}+a_{n,k-1}+...+a_{n,0}$ (et on rappelle $a_{n,0}=a_{1,k}=1$).

Montrons alors par récurrence sur $n$, entier naturel non nul, que~:~$\forall n\in\Nn^*,\;\forall k\in\Nn,\;a_{n,k}=C_{n+k-1}^k$.
 
Pour $n=1$, on a pour tout naturel $k$, $a_{1,k}=1=C_{1+k-1}^k$.

Soit $n\geq1$, supposons que $\forall k\in\Nn,\;a_{n,k}=C_{n+k-1}^k$. Soit $k\geq1$.

$$a_{n+1,k}=\sum_{i=0}^{k}a_{n,i}=\sum_{i=0}^{k}C_{n+i-1}^i=1+\sum_{i=1}^{k}(C_{n+i}^{i+1}-C_{n+i}^i)=1+C_{n+k}^{k+1}-1=
C_{n+k}^{k+1},$$

ce qui reste clair pour $k=0$.
 
On a montré par récurrence que $\forall n\in\Nn^*,\;\forall k\in\Nn,\;a_{n,k}=C_{n+k-1}^k$.
\fincorrection
\correction{005281}
On place le $0$ soit au chiffre des unités, soit au chiffre des dizaines, soit au chiffre des centaines, soit au chiffre des milliers (mais pas au chiffre des dizaines de milliers) et le $0$ étant placé, on n'y a plus droit.

Réponse~:~$4.9.9.9.9=4.9^4=4.(80+1)^2=4.6561=26244$.

\fincorrection
\correction{005282}
Si $n\geq366$, on a clairement $p_n=1$ (Principe des tiroirs~:~si $366$ personnes sont à associer à $365$ dates d'anniversaire, alors $2$ personnes au moins sont à associer à la même date d'anniversaire).

Si $2\leq n\leq 365$, on a $p_n=1-q_n$ où $q_n$ est la probabilité que les dates d'anniversaire soient deux à deux distinctes. Il y a $(365)^n$ répartitions possibles des dates d'anniversaires (cas possibles) et parmi ces répartitions, il y en a $365.364.363....(365-n+1)$ telles que les dates d'anniversaire soient deux à deux distinctes. Finalement 

$$p_n=1-\frac{1}{(365)^n}365.364.363....(365-n+1)=1-\prod_{k=1}^{n-1}\frac{365-k}{365}=1-\prod_{k=1}^{n-1}(1-\frac{k}{365}).$$

Ensuite,

$$p_n\geq\frac{1}{2}\Leftrightarrow\prod_{k=1}^{n-1}(1-\frac{k}{365})\leq\frac{1}{2}\Leftrightarrow\sum_{k=1}^{n-1}\ln(1-\frac{k}{365}) \leq\ln\frac{1}{2}\Leftrightarrow\sum_{k=1}^{n-1}-\ln(1-\frac{k}{365})\geq\ln2.$$

Maintenant, soit $x\in[0,1[$. On a

$$-\ln(1-x)=\int_{0}^{x}\frac{1}{1-t}dt\geq\int_{0}^{x}\frac{1}{1-0}dt=x.$$

Pour $k$ élément de $\{1,...,n-1\}(\subset\{1,...,364\})$, $\frac{k}{365}$ est un réel élément de $[0,1[$.

En appliquant l'inégalité précédente, on obtient 

$$\sum_{k=1}^{n-1}-\ln(1-\frac{k}{365})\geq\sum_{k=1}^{n-1}\frac{k}{365}=\frac{n(n-1)}{730}.$$

Ainsi,

$$p_n\geq\frac{1}{2}\Leftarrow\frac{n(n-1)}{730}\geq\ln 2\Leftrightarrow n^2-n-730\ln2\geq0\Leftrightarrow n\geq\frac{1+\sqrt{1+2920\ln2}}{2}=22,99...\Leftrightarrow n\geq23.$$

Finalement, dans un groupe d'au moins $23$ personnes, il y a plus d'une chance sur deux que deux personnes au moins aient la même date d'anniversaire.

\fincorrection
\correction{005283}

\begin{enumerate}
\item  Notre calendrier est $400$ ans périodique (et presque $4.7=28$ ans périodique).
En effet,
\begin{enumerate}
\item la répartition des années bissextiles est $400$ ans périodique ($1600$ et $2000$ sont bissextiles mais $1700$, $1800$ et $1900$ ne le sont pas (entre autre pour regagner $3$ jours tous les $400$ ans et coller le plus possible au rythme du soleil))
\item il y a un nombre entier de semaines dans une période de $400$ ans. En effet, sur $400$ ans, le quart des années, soit $100$ ans, moins $3$ années sont bissextiles et donc sur toute période de $400$ ans il y a $97$ années bissextiles et $303$ années non bissextiles.

Une année non bissextile de $365$ jours est constituée de $52.7+1$ jours ou encore d'un nombre entier de semaines plus un jour et une année bissextile est constituée d'un nombre entier de semaine plus deux jours.

Une période de $400$ ans est donc constituée d'un nombre entier de semaines plus~:~$97.2+303.1=194+303=497=7.71$ jours qui fournit encore un nombre entier de semaines.
\end{enumerate}

\item  Deux périodes consécutives de $28$ ans ne contenant pas d'exception (siècles non bissextiles) reproduisent le même calendrier. En effet, les $7$ années bissextiles fournissent un nombre entiers de semaines plus $2.7$ jours $=2$ semaines et les $21$ années non bissextiles fournissent un nombre entier de semaines plus $21.1$ jours $=3$ semaines.

\item  D'après ce qui précède, il suffit de compter les 1ers de l'an qui tombe un dimanche ou un samedi sur une période de $400$ ans donnée, par exemple de $1900$ à $2299$ (inclus).

On décompose cette période comme suit~:

$$\begin{array}{c}
1900,\;1901\rightarrow1928,\;1929\rightarrow1956,\;1957\rightarrow1984,\;1985\rightarrow2012,\;2013\rightarrow2040,\;
2041\rightarrow2068,\;2069\rightarrow2096,\\
2097\rightarrow2100,\;2101\rightarrow2128,\;2129\rightarrow2156,\;2157\rightarrow2184,\;2185\rightarrow2200,\;2201\rightarrow2228\;2229\rightarrow2256,\\
2257\rightarrow2284,\;2285\rightarrow2299.
\end{array}$$

\item  On montre ensuite que sur toute période de $28$ ans sans siècle non bissextile, le premier de l'an tombe un même nombre de fois chaque jour de la semaine (Lundi, mardi,..). (La connaissance des congruences modulo $4$ et $7$ seraient bien utile). Quand on passe d'une année non bissextile à l'année suivante, comme une telle année contient un nombre entier de semaines plus un jour, le 1er de l'an tombe un jour plus tard l'année qui suit et deux jours plus tard si l'année est bissextile. Par exemple,

1er janvier 1998~:~jeudi  1999~:~vendredi  2000~:~samedi  2001~:~Lundi  2002~:~Mardi 2003~:~Mercredi 2004~:~Jeudi 2005~:~samedi...

Notons A,B,C,D,E,F,G les jours de la semaine. Sur une période de 28 ans sans siècle non bissextile finissant par exemple une année bissextile, on trouve la séquence suivante~:

ABCD FGAB DEFG BCDE GABC EFGA CDEF (puis çà redémarre ABCD...) soit 4A, 4B, 4C, 4D, 4E, 4F, et 4G.

\item  Il reste à étudier les périodes à exception (soulignées dans le 3)).

Détermination du 1er janvier 1900.
Le 1er janvier 1998 était un jeudi . Il en est donc de même du 1er janvier 1998-28 = 1970 et des premiers janvier 1942 et 1914 puis on remonte~:

1914 Jeudi 1913 Mercredi 1912 Lundi 1911 Dimanche 1910 Samedi 1909 Vendredi 1908 Mercredi 1907 Mardi
1906 Lundi 1905 Dimanche 1904 Vendredi 1903 Jeudi 1902 Mercredi 1901 Mardi 1900 Lundi (1900 n'est pas bissextile)

Les premiers de l'an 2000, 2028 , 2056 et 2084 sont des samedis, 2088 un jeudi, 2092 un mardi, 2096 un dimanche et donc 2097 mardi 2098 mercredi 2099 jeudi 2100 vendredi.

2101 est un samedi de même que 2129, 2157, 2185 ce qui donne de 2185 à 2200 inclus la séquence :

S D L Ma J V S D Ma Me J V D L Ma Me

2201 est un jeudi de même que 2285 ce qui donne de 2285 à 2299 inclus la séquence :

J V S D Ma Me J V D L Ma Me V S D
\end{enumerate}

Le décompte des Lundis , mardis ... soulignés est : 6D 4L 6Ma 5Me 5J 6V 4S. Dans toute période de 400 ans, le 1er de l'an tombe $2$ fois de plus le dimanche que le samedi et donc plus souvent le dimanche que le samedi.
\fincorrection
\correction{005284}

On pose $H=$ "vers le haut"  et $D=$ "vers la droite". 
Un exemple de chemin de $(0,0)$ à $(p,q)$ est le mot $DD...DHH...H$ 
où $D$ est écrit $p$ fois et $H$ est écrit $q$ fois. 
Le nombre de chemins cherché est clairement le nombre d'anagrammes 
du mot précédent. 


Le nombre de choix de l'emplacement du $H$ est $C_{p+q}^q$. Une fois que les lettres $H$
sont placées il n'y a plus de choix pour les lettres $D$.
Il y a donc $C_{p+q}^q$ chemins possibles.

Remarque : si on place d'abord les lettres $D$ alors on a $C_{p+q}^p$ choix possibles.
Mais on trouve bien sûr le même nombre de chemins car $C_{p+q}^p = C_{p+q}^{(p+q) - p} = C_{p+q}^q$.

\fincorrection
\correction{005285}
On note respectivement $x$, $y$ et $z$ le nombre de pièces de $10$, $20$ et $50$ centimes. Il s'agit de résoudre dans $\Nn^3$ l'équation $10x+20y+50z=10000$ ou encore $x+2y+5z=1000$.

Soit $k\in\Nn$. $x+2y=k\Leftrightarrow x=k-2y$ et le nombre de solutions de cette équation est~:

$$\sum_{k=0}^{E(k/2)}1=E(\frac{k}{2})+1.$$

Pour $0\leq z\leq 200$ donné, le nombre de solutions de l'équation $x+2y=1000-5z$ est donc $E(\frac{1000-5z}{2})+1$. Le nombre de solutions en nombres entiers de l'équation $x+2y+5z=1000$ est donc 

$$\sum_{z=0}^{200}(E(\frac{1000-5z}{2})+1)=\sum_{z=0}^{200}(E(\frac{-5z}{2})+ 501)=201.501+\sum_{z=0}^{200}E(\frac{-5z}{2})=100701+\sum_{z=0}^{200}E(\frac{-5z}{2}).$$
Maintenant  

$$\sum_{z=0}^{200}E(\frac{-5z}{2})=\sum_{k=1}^{100}(E(\frac{-5(2k-1)}{2})+E(\frac{-5(2k)}{2}))=\sum_{k=1}^{100}(E(-5k+\frac{5}{2})-5k)=\sum_{k=1}^{100}(-10k+2)=200-10\frac{100.101}{2}.$$

Le nombre de solutions cherchés est donc $100701-50300=50401$. Il y a $50401$ façons de payer $100$ euros avec des pièces de $10$, $20$ et $50$ centimes.
\fincorrection
\correction{005286}
\begin{enumerate}
\item  
\begin{align*}\ensuremath
\chi_{A_1\cup...\cup A_n}&=1-\chi_{\overline{A_1\cup...\cup A_n}}
=1-\chi_{\overline{A_1}\cap...\cap\overline{A_n}}=1-\chi_{\overline{A_1}}\times...\times\chi_{\overline{A_n}}\\
 &=1-(1-\chi_{A_1})...(1-\chi_{A_n})=1-(\sum_{k=1}^{n}(-1)^{k-1}(\sum_{1\leq i_1<...<i_k\leq n}^{}\chi_{A_{i_1}}...\chi_{A_{i_k}}))
\end{align*}

et en sommant sur l'ensemble des $x$ de $E$, on obtient le résultat.

\item  Pour $1\leq k\leq n$, posons $A_k=\{\sigma\in S_n/\;\sigma(k)=k\}$. L'ensemble des permutations ayant au moins un point fixe est $A_1\cup A_2\cup...\cup A_n$. L'ensemble des permutations sans points fixes est le complémentaire dans $S_n$ de $A_1\cup A_2\cup...\cup A_n$.

D'après 1), leur nombre est donc~:

\begin{align*}\ensuremath
\mbox{card}(S_n)-\mbox{card}(A_1\cup A_2...\cup A_n)
&=\mbox{card}(S_n)-\sum_{i=1}^{n}\mbox{card}(A_i)+\sum_{i<j}^{}\mbox{card}(A_i\cap A_j)\\
 &-...+(-1)^{k}\sum_{i_1<i_2<...<i_k}^{}\mbox{card}(A_{i_1}\cap...A_{i_k})+...\\
 &+(-1)^{n}\mbox{card}(A_1\cap...\cap A_n).
\end{align*} 

$A_i$ est l'ensemble des permutations qui fixent $i$. Il y en a $(n-1)!$ ( nombre de permutations de $\{1,...,n\}\setminus\{i\}$). $A_i\cap A_j$ est l'ensemble des permutations qui fixent $i$ et $j$. Il y en a $(n-2)!$. Plus généralement, $\mbox{card}(A_{i_1}\cap...\cap A_{i_k})=(n-k)!$.

D'autre part, il y a $n=C_n^1$ entiers $i$ dans $\{1,n\}$ puis $C_n^2$ couples $(i,j)$ tels que $i<j$ et plus généralement, il y a $C_n^k$ $k$-uplets $(i_1,...,i_k)$ tels que $i_1<i_2< ...<i_k$. Le nombre de dérangements est  

$$n!+\sum_{k=1}^{n}(-1)^k\frac{n!}{k!(n-k)!}(n-k)!=n!\sum_{k=0}^{n}\frac{(-1)^k}{k!}.$$
 
Ainsi le \og~problème des chapeaux~\fg admet pour réponse

$$p_n=\sum_{k=0}^{n}\frac{(-1)^k}{k!}.$$ 

Montrons que cette suite tend très rapidement vers $\frac{1}{e}=0,36...$ quand $n$ tend vers l'infini.

(On adapte un  calcul déjà mené pour le nombre $e$.)

Montrons que $\forall n\in\Nn,\;e^{-1}=\sum_{k=0}^{n}\frac{(-1)^k}{k!}+(-1)^{n+1}\int_{0}^{1}\frac{(1-t)^n}{n!}e^{-t}\;dt$.

Pour $n=0$, $(-1)^{n+1}\int_{O}^{1}\frac{(1-t)^n}{n!}e^{-t}\;dt=-\int_{0}^{1}e^{-t}\;dt=-1+e^{-1}$ et donc, on a bien $e^{-1}=1-\int_{0}^{1}e^{-t}\;dt$.

Soit $n\geq0$. Supposons que $e^{-1}=\sum_{k=0}^{n}\frac{(-1)^k}{k!}+(-1)^{n+1}\int_{0}^{1}\frac{(1-t)^n}{n!}e^{-t}\;dt$.

Une intégration par parties fournit

$$\int_{0}^{1}\frac{(1-t)^n}{n!}e^{-t}\;dt=\left[-\frac{(1-t)^{n+1}}{(n+1)!}e^{-t}\right]_0^1-\int_{0}^{1}\frac{(1-t)^{n+1}}{(n+1)!}e^{-t}\;dt=\frac{1}{(n+1)!}-\int_{0}^{1}\frac{(1-t)^{n+1}}{(n+1)!}e^{-t}\;dt.$$

Mais alors,

$$e^{-1}=\sum_{k=0}^{n+1}\frac{(-1)^k}{k!}+(-1)^{n+2}\int_{0}^{1}\frac{(1-t)^{n+1}}{(n+1)!}e^{-t}\;dt.$$

Le résultat est démontré par récurrence.

On en déduit que

$$|p_n-\frac{1}{e}|=\left|(-1)^{n+1}\int_{0}^{1}\frac{(1-t)^n}{n!}e^{-t}\;dt\right|
=\int_{0}^{1}\frac{(1-t)^n}{n!}e^{-t}\;dt\leq\int_{0}^{1}\frac{(1-t)^n}{n!}\;dt=\frac{1}{(n+1)!}.$$

Ceci montre que $p_n$ tend très rapidement vers $\frac{1}{e}$.

\end{enumerate}
\fincorrection
\correction{005287}
Soit $n$ un naturel non nul. Dire que $f$ est une surjection de $\{1,...,n+1\}$ sur $\{1,...,n\}$ équivaut à dire que deux des entiers de $\{1,...,n+1\}$ ont même image $k$ par $f$ et que les autres ont des images deux à deux distinctes et distinctes de $k$. On choisit ces deux entiers : $C_{n+1}^2$ choix et leur image commune : $n$ images possibles ce qui fournit $nC_{n+1}^2$ choix d'une paire de $\{1,...,n+1\}$ et de leur image commune. Puis il y a $(n-1)!$ choix des images des $n-1$ éléments restants. Au total, il y a $n!\frac{n(n+1)}{2}=\frac{n.(n+1)!}{2}$ surjections de $\{1,...,n+1\}$ sur $\{1,...,n\}$.
\fincorrection
\correction{005288}
Soit $n\geq5$. De chaque sommet part $n-1$ droites (vers les $n-1$ autres sommets) dont $2$ sont des cotés et $n-3$ des diagonales. Comme chaque diagonale passe par 2 sommets , il y a $\frac{n(n-3)}{2}$ diagonales.

Ces diagonales se recoupent en $C_{n(n-3)/2}^2$ points distincts ou confondus. Dans ce décompte, chaque sommet a été compté autant de fois que l'on a choisi une paire de deux diagonales passant par ce sommet à savoir $C_{n-3}^2$. Maintenant, il y a $n$ sommets.

Réponse : 
\begin{align*}\ensuremath 
C_{n(n-3)/2}^2-nC_{n-3}^2&=\frac{1}{2}\frac{n(n-3)}{2}(\frac{n(n-3)}{2}-1)-n\frac{(n-3)(n-4)}{2}=\frac{n(n-3)}{8}(n(n-3)-2-4(n-4))\\
 &=\frac{n(n-3)}{8}(n^2-7n+14)
\end{align*}

Les diagonales se recoupent en $\frac{n(n-3)(n^2-7n+14)}{8}$ points distincts ou confondus et distincts des sommets (ou encore en $\frac{n(n-3)(n^2-7n+14)}{8}$ points au maximum).
\fincorrection
\correction{005289}
\begin{enumerate}
\item  On a bien sûr $P(1)=2$. Soit $n\geq1$. On trace $n$ droites vérifiant les conditions de l'énoncé. Elles partagent le plan en $P(n)$ régions. On trace ensuite $D_{n+1}$, une $(n+1)$ème droite. Par hypothèse, elle coupe chacune des $n$ premières droites en $n$ points deux à deux distincts. Ces $n$ points définissent $(n+1)$ intervalles sur la droite $D_{n+1}$. Chacun de ces $(n+1)$ intervalles partage une des $P(n)$ régions déjà existantes en deux régions et rajoute donc une nouvelle région. Ainsi, $P(n+1)=P(n)+(n+1)$.

Soit $n\geq2$.

\begin{align*}\ensuremath
P(n)&=P(1)+\sum_{k=1}^{n-1}(P(k+1)-P(k))=2+\sum_{k=1}^{n-1}(k+1)=1+\sum_{k=1}^{n}k=1+\frac{n(n+1)}{2}\\
 &=\frac{n^2+n+2}{2}
\end{align*}

ce qui reste vrai pour $n=1$.

\item  On a bien sûr $Q(1)=2$.Soit $n\geq1$. On trace $n$ plans vérifiant les conditions de l'énoncé. Ils partagent l'espace en $Q(n)$ régions. On trace ensuite $P_{n+1}$, un $(n+1)$ème plan. Par hypothèse, il recoupe chacun des $n$ premiers plans en $n$ droites vérifiant les conditions du 1). Ces $n$ droites délimitent $P(n)=1+\frac{n(n+1)}{2}$ régions sur le plan $P_{n+1}$. Chacune de ces régions partage une des $Q(n)$ régions déjà existantes en deux régions et rajoute donc une nouvelle région. Ainsi, $Q(n+1)=Q(n)+P(n)=Q(n)+\frac{n^2+n+2}{2}$.

Soit $n\geq2$.

\begin{align*}\ensuremath
Q(n)&=P(1)+\sum_{k=1}^{n-1}(Q(k+1)-Q(k))=2+\sum_{k=1}^{n-1}\frac{k^2+k+2}{2})=2+(n-1)+\frac{1}{2}\sum_{k=1}^{n-1}k^2+\frac{1}{2}\sum_{k=1}^{n-1}k\\
 &=(n+1)+\frac{(n-1)n(2n-1)}{12}+\frac{n(n-1)}{4}=\frac{n^3+5n+6}{6}
\end{align*}
\end{enumerate}
\fincorrection
\correction{005290}
Soient $n$ et $k$ des entiers naturels tels que $2\leq k\leq n-1$.

Soit $E$ un ensemble à $n$ éléments et $a$ un élément fixé de $E$.

Il y a $P_n^k$ partitions de $E$ en $k$ classes. Parmi ces partitions, il y a celles dans lesquelles $a$ est dans un singleton. Elles s'identifient aux partitions en $k-1$ classes de $E\setminus\{a\}$ et sont au nombre de $P_{n-1}^{k-1}$. Il y a ensuite les partitions dans lesquelles $a$ est élément d'une partie de cardinal au moins $2$. Une telle partition est obtenue en partitionnant $E\setminus\{a\}$ en $k$ classes puis en adjoignant à l'une de ces $k$ classes au choix l'élément $a$. Il y a $kP_{n-1}{k}$ telles partitions. Au total, $P_n^k=P_{n-1}^{k-1}+kP_{n-1}^k$.

Valeurs de $P_n^k$ pour $1\leq k,n\leq 5$.

$$\begin{array}{|c|c|c|c|c|c|}
\hline
n\;/\;k&1&2&3&4&5\\
\hline
1&1&0&0&0&0\\
\hline
2&1&1&0&0&0\\
\hline
3&1&3&1&0&0\\
\hline
4&1&7&6&1&0\\
\hline
5&1&15&25&10&1\\
\hline
\end{array}$$
Exprimons maintenant en fonction des $P_n^k$, le nombre de surjections d'un ensemble à $n$ éléments dans un ensemble à $p$ éléments.

Si $p>n$, il n'y a pas de surjections de $E_n$ dans $E_p$ (où $E_n$ et $E_p$ désignent des ensembles à $n$ et $p$ éléments respectivement).

On suppose dorénavant $p\leq n$. La donnée d'une surjection $f$ de $E_n$ sur $E_p$ équivaut à la donnée d'une partition de l'ensemble $E_n$ en $p$ classes (chaque élément d'une même classe ayant même image par $f$) puis d'une bijection de l'ensemble des parties de la partition vers $E_p$.

Au total, il y a donc $p!P_n^k$ surjections d'un ensemble à $n$ éléments dans un ensemble à $p$ éléments pour $1\leq p\leq n$.
\fincorrection


\end{document}

