\documentclass[10pt,a4paper]{article}


\usepackage[T1]{fontenc}
\usepackage[francais]{babel}
\usepackage{times}
\usepackage[utf8]{inputenc}
\usepackage{enumitem}
\usepackage{multicol}
\usepackage{fancyhdr}
\usepackage{tcolorbox}
\usepackage{tablists}
\usepackage[a4paper,bottom = 50pt]{geometry}
\usepackage{color}
\usepackage{amsmath,amssymb,amsthm,mathrsfs,pifont,amsfonts}
\usepackage{pgf,tikz,pgfplots,tkz-tab}
\usepackage{hyperref}
\usepackage{cancel}
\usepackage{array}
\usepackage{xcolor} 
\usepackage{amssymb}
\usepackage{amsthm}
\usepackage{graphicx}
\usepackage{pstricks}
\usepackage{geometry}


\setcellgapes{1pt}
\makegapedcells
\newcolumntype{R}[1]{>{\raggedleft\arraybackslash }b{#1}}
\newcolumntype{L}[1]{>{\raggedright\arraybackslash }b{#1}}
\newcolumntype{C}[1]{>{\centering\arraybackslash }b{#1}}

\geometry{top=2cm, bottom=1.5cm, left=1.5cm, right=2cm}

\newcommand{\R}{\mathbb{R}}
\newcommand{\Z}{\mathbb{Z}}
\newcommand{\N}{\mathbb{N}}
\newcommand{\notni}{\not\owns}

\newtheorem{thm}{Théorème}
\newtheorem*{pro}{Propriété}
\newtheorem*{exemple}{Exemple}

\theoremstyle{definition}
\newtheorem*{remarque}{Remarque}
\theoremstyle{definition}
\newtheorem{exo}{Exercice}
\newtheorem{definition}{Définition}



\begin{document}
	
	\leftline{\bfseries Optimal Sup Spé, groupe IPESUP \hfill Année~2021-2022}
	\leftline{\bfseries Niveau: Première année de PCSI  }
	\leftline{\bfseries M.Botcazou \hfill mail: ibotca52@gmail.com  }
	\rule[0.5ex]{\textwidth}{0.1mm}	
	
	\begin{center}
		\Large \sc colle 4 = suites numériques
	\end{center}
	
\section*{Questions de cours:}
\noindent Soient $(U_n)_{n\in\N}$ et $(V_n)_{n\in\N}$ deux suites réelles et $l_1,l_2 \in \R$. 
\begin{enumerate} %[$\square$]
\item  \begin{enumerate}
\item Montrer que si les suites $(U_n)_{n\in\N}$ et $(V_n)_{n\in\N}$ sont bornées alors les suites $(U_n+ V_N)_{n\in\N}$ et $(U_n \times V_N)_{n\in\N}$ sont bornées.
\item Donner deux suites $(U_n)_{n\in\N}$ et $(V_n)_{n\in\N}$ bornées telle que la suite $\left(\dfrac{U_n}{V_n}\right)_{n\in\N}$ soit définie et non bornée. 
\end{enumerate}
%\item Montrer que toute suite convergente est bornée.
\item Montrer que si une suite est convergente alors sa limite est unique. 
\item Montrer que toute suite convergente à valeur dans $\Z$ est stationnaire.
\item On suppose que $\lim\limits_{n\rightarrow+\infty}(U_n) = l_1$ et $\lim\limits_{n\rightarrow+\infty}(V_n) = l_2$
\begin{enumerate}
\item Montrer que $\lim\limits_{n\rightarrow+\infty}(U_n + V_n) = l_1 + l_2$
\item Montrer que $\lim\limits_{n\rightarrow+\infty}(U_n \times V_n) = l_1 \times l_2$
\end{enumerate}
\item Démontrer que pour tout $\alpha>0$ on a: $\lim\limits_{\small n\rightarrow+\infty}\left(n^{-\alpha}\right) = 0$
\item Rappeler le théorème de Cesàro  
\end{enumerate}



\section*{Suites numériques :}

\begin{minipage}{1.0\linewidth}
\begin{minipage}[t]{0.47\linewidth}
\raggedright
\begin{exo}\quad\\
Pour tout $n\in\N$, on pose : 
$$a_n = \int_{0}^{\dfrac{\pi}{2}}\cos^n(t)dt$$\hfill(\textit{Intégrale de Wallis})
\begin{enumerate}
\item Calculer $a_0$ et $a_1$ puis montrer que pour tout $n\in\N$: $a_n > 0$
\item Montrer que pour tout $n\in\N$: $a_{n+2} = \dfrac{n+1}{n+2}a_n$
\item En déduire que pour tout $n\in\N$
$$a_{2n} = \dfrac{(2n)!}{(2^{n}n!)^2}\dfrac{\pi}{2} $$
$$a_{2n+1} = \dfrac{(2^{n}n!)^2}{(2n+1)!}$$
\end{enumerate}
\centering
\rule{1\linewidth}{0.6pt}
\end{exo}

\begin{exo}\quad\\
Pour tout vecteur du plan fixé $\left(\begin{array}{c}
x_0\\
y_0
\end{array}\right) \ \in\R^2$, on considère la suite définie par récurrence :
$$\left(\begin{array}{c}
x_{n+1}\\
y_{n+1}
\end{array}\right) \ = \ \left(\begin{array}{cc}
0 & -1\\
1 & 0
\end{array}\right) \left(\begin{array}{c}
x_n\\
y_n
\end{array}\right)$$
\begin{enumerate}
\item En partant de $\left(\begin{array}{c}
x_0\\
y_0
\end{array}\right)  =  \left(\begin{array}{c}
1\\
0
\end{array}\right) $, représenter les $8$ premiers termes de la suite.
\item Cette suite est elle convergente ? 
\end{enumerate}
\centering
\rule{1\linewidth}{0.6pt}
\end{exo}

\end{minipage}	
\hfill\vrule\hfill
\begin{minipage}[t]{0.47\linewidth}
\raggedright

\begin{exo}\quad\\
Soit $\left(u_n\right)_{n\geq 1}$ une suite réelle. On pose $ S_n = \dfrac{\sum_{k=1}^{n}u_k}{n}$
\begin{enumerate}
\item On suppose que  $\left(u_n\right)_{n\geq 1}$ converge vers $0$. Soient $\epsilon>0$ et $n_0\in\N$  tel que, pour tout $n\geq n_0$, on a $|u_n|\leq\epsilon$.
\begin{enumerate}
\item Montrer qu'il existe une constante $M$ telle que, pour $n\geq n_0$ , on a 
$$|S_n| \leq \dfrac{M(n_0-1)}{n} + \epsilon$$
\item En déduire que $(S_n)$ converge vers $0$. 
\end{enumerate}
\item  On suppose que $u_n=(-1)^n$. Que dire de $(Sn)$? Qu'en déduisez-vous? 
\item On suppose que $(u_n)$ converge vers $l$. Montrer que $(S_n)$ converge vers $l$.
\item On suppose que $(u_n)$ tend vers $+\infty$. Montrer que $(S_n)$ tend vers $+\infty$  
\end{enumerate}
\centering
\rule{1\linewidth}{0.6pt}
\end{exo}

\begin{exo}\quad\\
Calculer en fonction de $n$ le terme général des la suites $\left(u_n\right)_{n\in\N}$ et $\left(v_n\right)_{n\in\N}$définies par :
\begin{enumerate}
\item $u_0 = 1$ et pour tout $n\in\N$:  %\mathcal{U}
$$u_{n+1} = 3u_n^2$$
\item $v_0 = 1$, $v_1 = 2$ et pour tout $n\in\N$:
$$v_{n+2} = \dfrac{v_{n+1}^3}{v_{n}^4}$$
\end{enumerate}
\centering
\rule{1\linewidth}{0.6pt}
\end{exo}


	\end{minipage}
\end{minipage}

\newpage


\begin{minipage}{1.0\linewidth}
\begin{minipage}[t]{0.47\linewidth}
\raggedright
\begin{exo}\quad\\
Soit $(u_n)_{n\in\N}$ une suite à termes réels strictement positifs telle que $\left(\dfrac{u_{n+1}}{u_n}\right)_{n\in\N}$ converge vers un réel $l\in\R^+$.
\begin{enumerate}
\item On suppose $l<1$ et on fixe $\epsilon>0$ tel que $l+\epsilon<1$. 
\begin{enumerate}
\item Démontrer qu'il existe un entier $n_0\in\N$ tel que, pour $n\geq n_0$, on a 
$$u_n\leq\left(l+\epsilon\right)^{n-n_0}u_{n_0}$$
\item En déduire que la suite $(u_n)_{n\in\N}$ est convergente et donner sa limite. 


\end{enumerate}
\item  On suppose $l>1$. Démontrer que $(u_n)$ diverge vers $+\infty$.
\item Étudier le cas $l=1$\\
(\textit{Indication: étudier les suites } $(n^\alpha)_{n\in\N^*}$)
\end{enumerate} 
\centering
\rule{1\linewidth}{0.6pt}
\end{exo}

\begin{exo}\quad\\

Soient $(u_n)$ et $(v_n)$ deux suites réelles convergeant respectivement vers $u$ et $v$. Montrer que la suite $w_n=\dfrac{u_0v_n+...+u_nv_0}{n+1}$ converge vers $uv$.\\\hfill\\
(\textit{Indication: on coupera ici la somme en 3 en isolant les bords})

\centering\rule{1\linewidth}{0.6pt}
\end{exo}

\end{minipage}	
\hfill\vrule\hfill
\begin{minipage}[t]{0.47\linewidth}
\raggedright

\begin{exo}\quad\\
Soit $(u_n)$ une suite de réels positifs vérifiant $$u_n\leq \dfrac{1}{k}+\dfrac{k}{n}$$ pour tous $(k,n)\in(\N^*)^2$.\\ Démontrer que $(u_n)$ tend vers $0$.

\centering
\rule{1\linewidth}{0.6pt}
\end{exo}

\begin{exo}\quad\\
Démontrer que  
\begin{enumerate}
\item$ \ln(n+ e^n) \sim_{+\infty} n$
\item $b^n-a^n   \sim_{+\infty} a^n+b^n$, \ $0<a<b$
\item $ 4\ln(1+\sqrt{n}) \sim_{+\infty} \ln(1+n^2)$
\end{enumerate}
\centering
\rule{1\linewidth}{0.6pt}
\end{exo}
\begin{exo}\quad\\
Montrer que  

$$\sum_{k=1}^{n-1}k! \ =_{+\infty} \  o\left(n!\right)$$

En déduire que
$$\sum_{k=1}^{n}k!  \sim_{+\infty} n!$$
\centering
\rule{1\linewidth}{0.6pt}
\end{exo}
\end{minipage}
\end{minipage}




\end{document}