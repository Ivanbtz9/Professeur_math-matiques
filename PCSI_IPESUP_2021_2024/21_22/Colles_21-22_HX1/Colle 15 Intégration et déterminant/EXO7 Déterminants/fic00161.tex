
%%%%%%%%%%%%%%%%%% PREAMBULE %%%%%%%%%%%%%%%%%%

\documentclass[11pt,a4paper]{article}

\usepackage{amsfonts,amsmath,amssymb,amsthm}
\usepackage[utf8]{inputenc}
\usepackage[T1]{fontenc}
\usepackage[francais]{babel}
\usepackage{mathptmx}
\usepackage{fancybox}
\usepackage{graphicx}
\usepackage{ifthen}

\usepackage{tikz}   

\usepackage{hyperref}
\hypersetup{colorlinks=true, linkcolor=blue, urlcolor=blue,
pdftitle={Exo7 - Exercices de mathématiques}, pdfauthor={Exo7}}

\usepackage{geometry}
\geometry{top=2cm, bottom=2cm, left=2cm, right=2cm}

%----- Ensembles : entiers, reels, complexes -----
\newcommand{\Nn}{\mathbb{N}} \newcommand{\N}{\mathbb{N}}
\newcommand{\Zz}{\mathbb{Z}} \newcommand{\Z}{\mathbb{Z}}
\newcommand{\Qq}{\mathbb{Q}} \newcommand{\Q}{\mathbb{Q}}
\newcommand{\Rr}{\mathbb{R}} \newcommand{\R}{\mathbb{R}}
\newcommand{\Cc}{\mathbb{C}} \newcommand{\C}{\mathbb{C}}
\newcommand{\Kk}{\mathbb{K}} \newcommand{\K}{\mathbb{K}}

%----- Modifications de symboles -----
\renewcommand{\epsilon}{\varepsilon}
\renewcommand{\Re}{\mathop{\mathrm{Re}}\nolimits}
\renewcommand{\Im}{\mathop{\mathrm{Im}}\nolimits}
\newcommand{\llbracket}{\left[\kern-0.15em\left[}
\newcommand{\rrbracket}{\right]\kern-0.15em\right]}
\renewcommand{\ge}{\geqslant} \renewcommand{\geq}{\geqslant}
\renewcommand{\le}{\leqslant} \renewcommand{\leq}{\leqslant}

%----- Fonctions usuelles -----
\newcommand{\ch}{\mathop{\mathrm{ch}}\nolimits}
\newcommand{\sh}{\mathop{\mathrm{sh}}\nolimits}
\renewcommand{\tanh}{\mathop{\mathrm{th}}\nolimits}
\newcommand{\cotan}{\mathop{\mathrm{cotan}}\nolimits}
\newcommand{\Arcsin}{\mathop{\mathrm{arcsin}}\nolimits}
\newcommand{\Arccos}{\mathop{\mathrm{arccos}}\nolimits}
\newcommand{\Arctan}{\mathop{\mathrm{arctan}}\nolimits}
\newcommand{\Argsh}{\mathop{\mathrm{argsh}}\nolimits}
\newcommand{\Argch}{\mathop{\mathrm{argch}}\nolimits}
\newcommand{\Argth}{\mathop{\mathrm{argth}}\nolimits}
\newcommand{\pgcd}{\mathop{\mathrm{pgcd}}\nolimits} 

%----- Structure des exercices ------

\newcommand{\exercice}[1]{\video{0}}
\newcommand{\finexercice}{}
\newcommand{\noindication}{}
\newcommand{\nocorrection}{}

\newcounter{exo}
\newcommand{\enonce}[2]{\refstepcounter{exo}\hypertarget{exo7:#1}{}\label{exo7:#1}{\bf Exercice \arabic{exo}}\ \  #2\vspace{1mm}\hrule\vspace{1mm}}

\newcommand{\finenonce}[1]{
\ifthenelse{\equal{\ref{ind7:#1}}{\ref{bidon}}\and\equal{\ref{cor7:#1}}{\ref{bidon}}}{}{\par{\footnotesize
\ifthenelse{\equal{\ref{ind7:#1}}{\ref{bidon}}}{}{\hyperlink{ind7:#1}{\texttt{Indication} $\blacktriangledown$}\qquad}
\ifthenelse{\equal{\ref{cor7:#1}}{\ref{bidon}}}{}{\hyperlink{cor7:#1}{\texttt{Correction} $\blacktriangledown$}}}}
\ifthenelse{\equal{\myvideo}{0}}{}{{\footnotesize\qquad\texttt{\href{http://www.youtube.com/watch?v=\myvideo}{Vidéo $\blacksquare$}}}}
\hfill{\scriptsize\texttt{[#1]}}\vspace{1mm}\hrule\vspace*{7mm}}

\newcommand{\indication}[1]{\hypertarget{ind7:#1}{}\label{ind7:#1}{\bf Indication pour \hyperlink{exo7:#1}{l'exercice \ref{exo7:#1} $\blacktriangle$}}\vspace{1mm}\hrule\vspace{1mm}}
\newcommand{\finindication}{\vspace{1mm}\hrule\vspace*{7mm}}
\newcommand{\correction}[1]{\hypertarget{cor7:#1}{}\label{cor7:#1}{\bf Correction de \hyperlink{exo7:#1}{l'exercice \ref{exo7:#1} $\blacktriangle$}}\vspace{1mm}\hrule\vspace{1mm}}
\newcommand{\fincorrection}{\vspace{1mm}\hrule\vspace*{7mm}}

\newcommand{\finenonces}{\newpage}
\newcommand{\finindications}{\newpage}


\newcommand{\fiche}[1]{} \newcommand{\finfiche}{}
%\newcommand{\titre}[1]{\centerline{\large \bf #1}}
\newcommand{\addcommand}[1]{}

% variable myvideo : 0 no video, otherwise youtube reference
\newcommand{\video}[1]{\def\myvideo{#1}}

%----- Presentation ------

\setlength{\parindent}{0cm}

\definecolor{myred}{rgb}{0.93,0.26,0}
\definecolor{myorange}{rgb}{0.97,0.58,0}
\definecolor{myyellow}{rgb}{1,0.86,0}

\newcommand{\LogoExoSept}[1]{  % input : echelle       %% NEW
{\usefont{U}{cmss}{bx}{n}
\begin{tikzpicture}[scale=0.1*#1,transform shape]
  \fill[color=myorange] (0,0)--(4,0)--(4,-4)--(0,-4)--cycle;
  \fill[color=myred] (0,0)--(0,3)--(-3,3)--(-3,0)--cycle;
  \fill[color=myyellow] (4,0)--(7,4)--(3,7)--(0,3)--cycle;
  \node[scale=5] at (3.5,3.5) {Exo7};
\end{tikzpicture}}
}


% titre
\newcommand{\titre}[1]{%
\vspace*{-4ex} \hfill \hspace*{1.5cm} \hypersetup{linkcolor=black, urlcolor=black} 
\href{http://exo7.emath.fr}{\LogoExoSept{3}} 
 \vspace*{-5.7ex}\newline 
\hypersetup{linkcolor=blue, urlcolor=blue}  {\Large \bf #1} \newline 
 \rule{12cm}{1mm} \vspace*{3ex}}

%----- Commandes supplementaires ------



\begin{document}

%%%%%%%%%%%%%%%%%% EXERCICES %%%%%%%%%%%%%%%%%%

\fiche{f00161, bodin, 2012/09/01} 

\titre{Calculs de déterminants}

Fiche corrigée par Arnaud Bodin

\bigskip

\exercice{6885, exo7, 2012/09/05}
\video{ifWv3IZcAKU}
% D'après #1134 barraud et #2755 tumpach
\enonce{006885}{}
  Calculer les déterminants des matrices suivantes :
$$
  \begin{pmatrix}
    7 & 11 \\
    -8 & 4
  \end{pmatrix}
\quad
  \begin{pmatrix}
    1 & 0 & 6 \\
    3 & 4 & 15\\
    5 & 6 & 21
  \end{pmatrix}
\quad
  \begin{pmatrix}
    1 & 0 & 2 \\
    3 & 4 & 5  \\
    5 & 6  & 7
  \end{pmatrix}
\quad
  \begin{pmatrix}
    1 & 0 & -1 \\
    2 & 3 & 5  \\
    4 & 1 & 3
  \end{pmatrix}
$$

$$
  \begin{pmatrix} 0 & 1 & 2 & 3\\ 1 & 2 & 3 & 0\\ 2 & 3 & 0 & 1\\3 & 0 & 1 & 2\end{pmatrix}
  \begin{pmatrix} 0 & 1 & 1 & 0\\ 1 & 0 & 0 & 1\\ 1 & 1 & 0 & 1\\  1 & 1 & 1 &0\end{pmatrix}
  \begin{pmatrix} 1 & 2 & 1 & 2\\ 1 & 3 & 1 & 3\\ 2 & 1 & 0& 6\\ 1 & 1& 1&7\end{pmatrix}
$$


\finenonce{006885}



\finexercice
\exercice{2753, tumpach, 2009/10/25}
\video{d4ZZDotSUcw}
\enonce{002753}{}
\begin{enumerate}
\item Calculer l'aire du parall\'elogramme construit sur les vecteurs 
$\vec{u} = \left(\begin{array}{c}2\\3\end{array}\right)$ et 
$\vec{v} = \left(\begin{array}{c}1\\4\end{array}\right)$.
\item Calculer le volume du  parall\'el\'epip\`ede construit sur les vecteurs \\
$\vec{u} = \left(\begin{array}{c}1\\2\\0\end{array}\right)$, 
$\vec{v} = \left(\begin{array}{c}0\\1\\3\end{array}\right)$ et 
$\vec{w} = \left(\begin{array}{c}1\\1\\1\end{array}\right)$.
\item Montrer que le volume d'un parall\'el\'epip\`ede dont les sommets sont des points de 
$\Rr^3$ \`a coefficients entiers est un nombre entier.
\end{enumerate}
\finenonce{002753}


\finexercice
\exercice{6886, exo7, 2012/09/05}
\video{ibKDAQnNURg}

% D'après #1121 ridde, #2756 tumpach, #1130 barraud; #1124 liousse
\enonce{006886}{}  

Calculer les déterminants des matrices suivantes :

$$
\begin{pmatrix}
a&b&c\\c&a&b\\b&c&a
\end{pmatrix}
\begin{pmatrix}
1&0&0&1 \\ 0&1&0&0 \\ 1&0&1&1 \\ 2&3&1&1
\end{pmatrix}
\begin{pmatrix}
-1 & 1 & 1 & 1\\ 1 & -1 & 1 & 1\\ 1 & 1 & -1& 1\\ 1 & 1& 1&-1
\end{pmatrix}
\begin{pmatrix}
10 & 0 & -5 & 15 \\ -2 & 7 & 3 & 0 \\ 8 & 14 & 0 & 2 \\ 0 & -21 & 1 & -1
\end{pmatrix}
$$
 
$$
\begin{pmatrix}
a&a&b&0 \\  a&a&0&b \\  c&0&a&a \\ 0&c&a&a
\end{pmatrix}
\begin{pmatrix}
1&0&3&0&0 \\ 0&1&0&3&0 \\ a&0&a&0&3 \\ b&a&0&a&0 \\ 0&b&0&0&a  
\end{pmatrix}
\begin{pmatrix}
1&0&0&1&0 \\ 0&-4&3&0&0 \\ -3&0&0&-3&-2 \\ 0&1&7&0&0 \\ 4&0&0&7&1  
\end{pmatrix}
$$
\finenonce{006886}




\finexercice
\exercice{6887, barraud, 2012/09/05}
\video{D-wMy5vRJo4}
% D'apres #1139 barraud
\enonce{006887}{}
  Calculer les déterminants suivant~:
$$
 \left\vert
   \begin{matrix}
     a_1   &a_2   &\cdots&a_n    \\
     a_1   &a_1   &\ddots&\vdots \\
     \vdots&\ddots&\ddots&a_2    \\
     a_1   &\cdots&a_1   &a_1
   \end{matrix}
 \right\vert
\qquad
\left\vert
\begin{matrix}
  1    &      &        &1 \\
  1    &1      & (0)   & \\
       &\ddots &\ddots & \\  
  (0)  &       &1      &1
\end{matrix}
\right\vert
\qquad
\left\vert
\begin{matrix}
    a+b  &    a   & \cdots &  a       \\
     a   &   a+b  & \ddots & \vdots   \\
  \vdots & \ddots & \ddots &  a       \\
     a   & \cdots &    a   & a+b 
\end{matrix}
\right\vert
$$
\finenonce{006887}


\finexercice\exercice{1143, barraud, 2003/09/01}
\video{64911kcKNmg}
\enonce{001143}{}
  Soit $(a_{0},...,a_{n-1})\in\C^{n}$, $x\in\C$. Calculer
 $$
 \Delta_{n}=
 \left|
   \begin{matrix}
   x &  0    &        & a_{0}   \\
    -1 &\ddots &\ddots  &\vdots  \\
      &\ddots &x      & a_{n-2} \\
    0 &       & -1      & x+a_{n-1}
   \end{matrix}
\right|
 $$
\finenonce{001143}


\finexercice
\exercice{1145, barraud, 2003/09/01}
\video{YveAcpVpK2g}
\enonce{001145}{}
Soit $a$ un réel.
On note $\Delta_n$ le déterminant suivant : 
$$
\Delta_n = 
\left\vert
\begin{matrix}
       a   &    0   & \cdots & 0      & n-1 \\
       0   &    a   & \ddots & \vdots & \vdots \\
    \vdots & \ddots & \ddots & 0      & 2 \\
       0   & \cdots &   0    & a      & 1 \\
      n-1  & \cdots &   2    & 1      & a
\end{matrix}
\right\vert
$$
\begin{enumerate}
  \item Calculer $\Delta_n$ en fonction de $\Delta_{n-1}$.
  \item Démontrer que : $\displaystyle \forall n\geq2\quad
\Delta_n=a^n-a^{n-2}\sum_{i=1}^{n-1}{i^2}$.
\end{enumerate}
 
\finenonce{001145}


\finexercice

\exercice{2453, matexo1, 2002/02/01}
\video{NwFtb_SYHA8}
\enonce{002453}{Déterminant de Vandermonde}
Montrer que
$$\left|
\begin{array}{ccccc}
1 & t_1 & t_1^2 & \ldots & t_1^{n-1} \\
1 & t_2 & t_2^2 & \ldots & t_2^{n-1} \\\
\ldots&\ldots&\ldots& \ldots & \ldots \\
1 & t_n & t_n^2 & \ldots & t_n^{n-1}
\end{array}\right|
 = \prod_{1 \le i < j \le n} (t_j - t_i) $$
\finenonce{002453}


\finexercice


\finfiche


 \finenonces 



 \finindications 

\noindication
\noindication
\indication{006886}
\begin{enumerate}
  \item Règle de Sarrus.
  \item Développer par rapport à la deuxième ligne.
  \item Faire apparaître des $0$ sur la première colonne.
  \item Utiliser la linéarité par rapports à chaque ligne et chaque colonne
pour simplifier les coefficients.
  \item Faire apparaître des $0$...
  \item Faire apparaître des $0$...
  \item Permuter les lignes et les colonnes pour faire apparaître une matrice triangulaire
par blocs.
\end{enumerate}
\finindication
\noindication
\indication{001143}
Développer par rapport à la dernière colonne.
\finindication
\indication{001145}
Développer par rapport à la première colonne
pour obtenir $\Delta_{n-1}$ et un autre déterminant facile à calculer
en développant par rapport à sa première ligne.
\finindication
\indication{002453}
Faire les opérations suivantes sur les colonnes
$C_n \leftarrow C_n-t_n C_{n-1}$,
puis $C_{n-1} \leftarrow C_{n-1}-t_n C_{n-2}$,...,
$C_2 \leftarrow  C_2-t_nC_1$.
Développer par rapport a la bonne ligne et reconnaître
que l'on obtient le déterminant recherché mais au rang $n-1$.
\finindication


\newpage

\correction{006885}
\begin{enumerate}
  \item  Le déterminant de la matrice $\begin{pmatrix} a & b \\ c & d \end{pmatrix}$
est $\begin{vmatrix} a & b \\ c & d \end{vmatrix} =  ad-bc$.
Donc $\begin{vmatrix} 7 & 11 \\ -8 & 4 \end{vmatrix} = 7 \times 4 - 11 \times (-8) = 116$.

  \item Nous allons voir différentes méthodes pour calculer les déterminants.

\textbf{Première méthode.} \emph{Règle de Sarrus.} 
Pour le matrice $3\times 3$ il existe une formule qui permet de calculer directement le déterminant.
$$\begin{vmatrix} 
a_{11} & a_{12} & a_{13} \\   
a_{21} & a_{22} & a_{23} \\  
a_{31} & a_{32} & a_{33} \\ 
  \end{vmatrix}
= a_{11}a_{22}a_{33} + a_{12}a_{23}a_{31} + a_{21}a_{32}a_{13}
- a_{13}a_{22}a_{31} - a_{11} a_{32}a_{23}  - a_{12}a_{21}a_{33}$$
Donc 
$$\begin{vmatrix}
    1 & 0 & 6 \\
    3 & 4 & 15\\
    5 & 6 & 21
  \end{vmatrix}
= 1\times 4 \times 21 + 0 \times 15 \times 5 + 3\times 6 \times 6
- 5\times 4\times 6 -6\times 15 \times 1 -3 \times 0 \times 21 = -18$$


Attention ! La règle de Sarrus ne s'applique qu'aux matrices $3\times 3$.

  \item \textbf{Deuxième méthode.} \emph{Se ramener à une matrice diagonale ou triangulaire.}


Si dans une matrice on change un ligne $L_i$ en $L_i-\lambda L_j$ alors le déterminant reste le même.
Même chose avec les colonnes. 

$$ \begin{array}{l|ccc|}
    _{L_1} & 1 & 0 & 2 \\
    _{L_2} & 3 & 4 & 5\\
    _{L_3} & 5 & 6  & 7
  \end{array}
= \begin{array}{l|ccc|}
    & 1 & 0 & 2  \\
    _{L_2 \leftarrow L_2-3L_1} & 0 & 4 & -1 \\
    _{L_3 \leftarrow L_3-5L_1} & 0 & 6  & -3 \\
  \end{array}
= \begin{array}{l|ccc|}
    & 1 & 0 & 2 \\
    & 0 & 4 & -1 \\
    _{L_3 \leftarrow L_3-\frac32 L_2} & 0 & 0  & -\frac32 \\
  \end{array} = 1\times 4 \times (-\tfrac32) = -6
$$
On a utilisé le fait  que le déterminant d'une matrice diagonale (ou triangulaire) est le produit
des coefficients sur la diagonale.


  \item \textbf{Troisième méthode.} \emph{Développement par rapport à une ligne ou une colonne.}
Nous allons développer par rapport à la deuxième colonne. 
$$\begin{vmatrix}
    1 & 0 & -1 \\
    2 & 3 & 5  \\
    4 & 1 & 3
  \end{vmatrix}
= (-0) \times \begin{vmatrix}
    2 & 5  \\
    4 & 3
  \end{vmatrix}
+ (+3)\times \begin{vmatrix}
    1 & -1 \\
    4 & 3
  \end{vmatrix}
+ (-1) \times \begin{vmatrix}
    1 & -1 \\
    2 & 5  \\
  \end{vmatrix}
= 0 + 3 \times 7 - 1  \times 7 = 14$$


Bien souvent on commence par simplifier la matrice en faisant apparaître un maximum de $0$
par les opérations élémentaires sur les lignes et les colonnes. Puis on développe en choisissant 
la ligne ou la colonne qui a le plus de $0$.


  \item 
On fait apparaître des $0$ sur la première colonne puis on développe par rapport à cette colonne.
$$\Delta = \begin{array}{l|cccc|} 
_{L_1} & 0 & 1 & 2 & 3 \\ _{L_2} & 1 & 2 & 3 & 0 \\ _{L_3} & 2 & 3 & 0 & 1 \\ _{L_4} & 3 & 0 & 1 & 2 \end{array}
=  \begin{array}{l|cccc|}& 0 & 1 & 2 & 3\\  & 1 & 2 & 3 & 0\\ _{L_3 \leftarrow L_3 -2 L_2} & 0 & -1 & -6 & 1\\ _{L_4\leftarrow L_4 -3 L_2} & 0 & -6 & -8 & 2\end{array}
= - \begin{array}{|ccc|}  1 & 2 & 3\\  -1 & -6 & 1\\  -6 & -8 & 2\end{array}
$$
Pour calculer le déterminant $3\times 3$ on fait apparaître des $0$ sur la première colonne, puis on la développe.
$$-\Delta = \begin{array}{l|ccc|} _{L_1} & 1 & 2 & 3\\ _{L_2} & -1 & -6 & 1\\ _{L_3} & -6 & -8 & 2\end{array}
= \begin{array}{l|ccc|}  & 1 & 2 & 3\\ _{L_2\leftarrow L_2+L_1} & 0 & -4 & 4\\ _{L_3\leftarrow L_3+6L_1} & 0 & 4 & 20\end{array}
=1\begin{vmatrix}-4 & 4 \\ 4 & 20\end{vmatrix} = -96
$$
Donc $\Delta=96$.


  \item La matrice a déjà beaucoup de $0$ mais on peut en faire apparaître davantage sur la dernière colonne, puis on développe par rapport à la dernière colonne.
$$
\Delta'= \begin{array}{l|cccc|} 
_{L_1} & 0 & 1 & 1 & 0\\ _{L_2} & 1 & 0 & 0 & 1\\ _{L_3} & 1 & 1 & 0 & 1\\  _{L_4} & 1 & 1 & 1 &0  
\end{array}
= \begin{array}{l|cccc|}  & 0 & 1 & 1 & 0\\  & 1 & 0 & 0 & 1\\ _{L_3 \leftarrow L_3-L_2} & 0 & 1 & 0 & 0 \\  & 1 & 1 & 1 &0  
\end{array}
= \begin{array}{|ccc|}  0 & 1 & 1 \\  0 & 1 & 0  \\   1 & 1 & 1  
\end{array}
$$
On développe ce dernier déterminant par rapport à la première colonne :
$$ \Delta'=\begin{array}{|ccc|}  0 & 1 & 1 \\  0 & 1 & 0  \\   1 & 1 & 1  
\end{array} = 1 \times \begin{vmatrix} 1 & 1 \\ 1 & 0 \end{vmatrix} = -1$$

  \item Toujours la même méthode, on fait apparaître des $0$ sur la première colonne, puis on développe par rapport à cette colonne.
$$
\Delta''= \begin{array}{l|cccc|} 
_{L_1} & 1 & 2 & 1 & 2\\  _{L_2} &1 & 3 & 1 & 3\\  _{L_3} &2 & 1 & 0& 6\\  _{L_4} &1 & 1& 1&7\end{array}
=\begin{array}{l|cccc|}  & 1 & 2 & 1 & 2\\  _{L_2\leftarrow L_2-L_1} &0 & 1 & 0 & 1\\  _{L_3\leftarrow L_3-2L_1} & 0 & -3 & -2 & 2 \\ 
 _{L_4\leftarrow L_4-L_1} & 0 & -1 & 0 & 5 \end{array}
= \begin{array}{|ccc|} 1 & 0 & 1\\-3 & -2 & 2 \\ -1 & 0 & 5 \end{array}
$$
On développe par rapport à la deuxième colonne :
$$\Delta''= -2 \times \begin{vmatrix}  1 & 1 \\ -1 & 5 \end{vmatrix} = -12$$
\end{enumerate}
\fincorrection
\correction{002753}
\begin{enumerate}
  \item L'aire $\mathcal{A}$ du parallélogramme construit sur les vecteurs $\vec{u} = \left(\begin{array}{c}a\\c\end{array}\right)$ et 
$\vec{v} = \left(\begin{array}{c}b\\d\end{array}\right)$ est la valeur absolue du déterminant $\begin{vmatrix} a & b \\ c & d \end{vmatrix}$
donc $\mathcal{A} =  |ad-bc|$.
Ici on trouve 
$\mathcal{A} = \text{abs} \begin{vmatrix} 2 & 1 \\ 3 & 4 \end{vmatrix} = + 5 $
où $\text{abs}$ désigne la fonction valeur absolue.

  \item Le volume  du  parall\'el\'epip\`ede construit sur trois vecteurs de $\Rr^3$ est la valeur absolue du déterminant
de la matrice formée des trois vecteurs.
Ici $$\mathcal{V} = \text{abs} \begin{vmatrix} 1 & 0 & 1 \\ 2 & 1 & 1 \\ 0 & 3 & 1\end{vmatrix} 
= \text{abs} \Big(+1 \begin{vmatrix} 1 & 1 \\ 3 & 1\end{vmatrix}  + 1  \begin{vmatrix} 2 & 1 \\ 0 & 3 \end{vmatrix}
\Big)= 4$$
où l'on a développé par rapport à la première ligne.

  \item Si un parall\'el\'epip\`ede est construit sur trois vecteurs de $\Rr^3$
dont les coefficients sont des entiers alors le volume correspond au déterminant d'une matrice à coefficients
entiers. C'est donc un entier.

\end{enumerate}
\fincorrection
\correction{006886}
\begin{enumerate}
  \item Par la règle de Sarrus :
$$\Delta_1 = \begin{vmatrix}
a&b&c\\c&a&b\\b&c&a
\end{vmatrix} 
= a^3+b^3+c^3-3abc.$$

  \item On développe par rapport à la seconde ligne qui ne contient qu'un coefficient non nul et on calcule le déterminant
$3\times 3$ par la règle de Sarrus :
$$\Delta_2 = \begin{vmatrix}
1&0&0&1 \\ 0&1&0&0 \\ 1&0&1&1 \\ 2&3&1&1
\end{vmatrix}
= +1 \begin{vmatrix}
1&0&1 \\ 1&1&1 \\ 2&1&1
\end{vmatrix}
= -1.$$

  \item
$$\Delta_3 = 
\begin{array}{l|cccc|} 
_{L_1} & -1 & 1 & 1 & 1 \\ _{L_2} & 1 & -1 & 1 & 1\\ _{L_3} & 1 & 1 & -1& 1\\ _{L_4} & 1 & 1& 1&-1
\end{array}
= \begin{array}{l|cccc|} 
 & -1 & 1 & 1 & 1 \\ _{L_2\leftarrow L_2+L_1} & 0 & 0 & 2 & 2\\
 _{L_3\leftarrow L_3+L_1} & 0 & 2 & 0 & 2 \\  _{L_4\leftarrow L_4+L_1} &0 & 2 & 2 & 0
\end{array}
$$
On développe par rapport à la première colonne :
$$\Delta_3 = (-1) \times \begin{array}{|ccc|} 
  0 & 2 & 2\\  2 & 0 & 2 \\  2 & 2 & 0
\end{array} = -16$$

  \item Le déterminant est linéaire par rapport à chacune de ses lignes
et aussi chacune de ses colonnes. 
Par exemple les coefficients de la première ligne sont tous des multiples
de $5$ donc 
$$\Delta_4 = 
\begin{vmatrix}
10 & 0 & -5 & 15 \\ -2 & 7 & 3 & 0 \\ 8 & 14 & 0 & 2 \\ 0 & -21 & 1 & -1
\end{vmatrix}
= 5 \times \begin{vmatrix}
2 & 0 & -1 & 3 \\ -2 & 7 & 3 & 0 \\ 8 & 14 & 0 & 2 \\ 0 & -21 & 1 & -1
\end{vmatrix}
$$
On fait la même chose avec la troisième ligne :
$$\Delta_4 = 5 \times 2 \times \begin{vmatrix}
2 & 0 & -1 & 3 \\ -2 & 7 & 3 & 0 \\ 4 & 7 & 0 & 1 \\ 0 & -21 & 1 & -1
\end{vmatrix}
$$
Et enfin les coefficients la première colonne sont des multiples de $2$ et 
ceux de la troisième colonne sont des multiples de $7$ donc :
$$\Delta_4 = 5 \times 2 \times 2 \times\begin{vmatrix}
1 & 0 & -1 & 3 \\ -1 & 7 & 3 & 0 \\ 2 & 7 & 0 & 1 \\ 0 & -21 & 1 & -1
\end{vmatrix}
= 5 \times 2 \times 2 \times 7  \times
\begin{vmatrix}
1 & 0 & -1 & 3 \\ -1 & 1 & 3 & 0 \\ 2 & 1 & 0 & 1 \\ 0 & -3 & 1 & -1
\end{vmatrix}
$$

Les coefficients sont plus raisonnables !
On fait $L_2\leftarrow L_2+L_1$ et $L_3\leftarrow L_3-2L_1$
pour obtenir :
$$\Delta_4 = 140 \times 
\begin{vmatrix}
1 & 0 & -1 & 3 \\ 0 & 1 & 2 & 3 \\ 0 & 1 & 2 & -5 \\ 0 & -3 & 1 & -1
\end{vmatrix}
=140 \times \begin{vmatrix}
 1 & 2 & 3 \\  1 & 2 & -5 \\ -3 & 1 & -1
\end{vmatrix}
= 140 \times 56 = 7840$$


  \item 
$$\Delta_5 =\begin{array}{l|cccc|} 
_{L_1} &a&a&b&0 \\ _{L_2} & a&a&0&b \\ _{L_3} & c&0&a&a \\ _{L_4} &0&c&a&a
\end{array}
= \begin{array}{l|cccc|} 
 &a&a&b&0 \\ _{L_2\leftarrow L_2-L_1} & 0&0&-b&b \\  & c&0&a&a \\ _{L_4\leftarrow L_4-L_3} & -c&c&0&0
\end{array}
$$
On fait ensuite les opérations suivantes sur les colonnes :
$C_2 \leftarrow C_2+C_1$ et $C_3 \leftarrow C_3-C_4$ pour obtenir 
une dernière ligne facile à développer :
$$\Delta_5 
= \begin{array}{|cccc|} 
 a&2a&b&0 \\ 0&0&-2b&b \\  c&c&0&a \\ -c&0&0&0
\end{array}
= +c \times \begin{array}{|ccc|} 
 2a&b&0 \\ 0&-2b&b \\  c&0&a \\
 \end{array} = bc(bc-4a^2)
$$
  \item
On fait d'abord les opérations $C_1 \leftarrow C_1-C_3$ et $C_2 \leftarrow C_2-C_4$
et on développe par rapport à la première ligne :
$$\Delta_6 = \begin{vmatrix}
1&0&3&0&0 \\ 0&1&0&3&0 \\ a&0&a&0&3 \\ b&a&0&a&0 \\ 0&b&0&0&a  
\end{vmatrix}
=
\begin{vmatrix}
-2&0&3&0&0 \\ 0&-2&0&3&0 \\ 0&0&a&0&3 \\ b&0&0&a&0 \\ 0&b&0&0&a  
\end{vmatrix}
= (-2) \times \begin{vmatrix}
-2&0&3&0 \\ 0&a&0&3 \\ 0&0&a&0 \\ b&0&0&a  
\end{vmatrix}
+ 3  \times \begin{vmatrix}
0&-2&3&0 \\ 0&0&0&3 \\ b&0&a&0 \\ 0&b&0&a  
\end{vmatrix}
$$
Le premier déterminant à calculer se développe par rapport à la deuxième colonne et le second déterminant
par rapport à la première colonne :
$$\Delta_6 = (-2)\times a \times 
\begin{vmatrix}
-2&3&0 \\  0&a&0 \\ b&0&a  
\end{vmatrix}
+ 3  \times b \times 
\begin{vmatrix}
-2&3&0 \\ 0&0&3 \\ b&0&a  
\end{vmatrix}=4a^3+27b^2$$

  \item Nous allons permuter des lignes et des colonnes pour se ramener à une matrice diagonale par blocs.
Souvenons-nous que lorsque l'on échange deux lignes (ou deux colonnes) alors le déterminant change de signe.
Nous allons rassembler les zéros.
On commence par échanger les colonnes $C_1$ et $C_3$ : $C_1\leftrightarrow C_3$ :
$$\Delta_7=
\begin{vmatrix}
1&0&0&1&0 \\ 0&-4&3&0&0 \\ -3&0&0&-3&-2 \\ 0&1&7&0&0 \\ 4&0&0&7&1  
\end{vmatrix}=
- \begin{vmatrix}
0&0&1&1&0 \\ 3&-4&0&0&0 \\ 0&0&-3&-3&-2 \\ 7&1&0&0&0 \\ 0&0&4&7&1  
\end{vmatrix}$$
Puis on échange les lignes $L_1$ et $L_4$ : $L_1\leftrightarrow L_4$ :
$$\Delta_7=+\begin{array}{|ccccc|} 
7&1&0&0&0 \\ 3&-4&0&0&0 \\  0&0&-3&-3&-2 \\ 0&0&1&1&0 \\ 0&0&4&7&1  
\end{array}$$
Notre matrice est sous la forme d'une matrice diagonale par blocs
 et son déterminant est le produit des déterminants.
$$\Delta_7=\begin{array}{|cc|ccc|} 
7&1&0&0&0 \\ 3&-4&0&0&0 \\ \hline 0&0&-3&-3&-2 \\ 0&0&1&1&0 \\ 0&0&4&7&1  
\end{array}
= \begin{vmatrix}
 7&1 \\  3&-4 \\  
  \end{vmatrix} \times
\begin{vmatrix}
 -3&-3&-2 \\  1&1&0 \\  4&7&1  
  \end{vmatrix}
= (-31)\times (-6) = 186
$$

\end{enumerate}
\fincorrection
\correction{006887}
\begin{enumerate}
  \item On retire la première colonne à toutes les autres colonnes 
$$
\Delta_1 =   \begin{vmatrix}
     a_1   &a_2   &\cdots&a_n    \\
     a_1   &a_1   &\ddots&\vdots \\
     \vdots&\ddots&\ddots&a_2    \\
     a_1   &\cdots&a_1   &a_1
   \end{vmatrix}
 =  \begin{vmatrix}
     a_1   &a_2-a_1  & a_3-a_1 &\cdots&a_n -a_1   \\
     a_1   &0  & &\ddots&\vdots \\
     \vdots&\vdots &\ddots&\ddots&a_2-a_1    \\
     a_1   &0&\cdots&0   & 0
   \end{vmatrix}$$
On développe par rapport à la dernière ligne :
$$\Delta_1 = (-1)^{n-1}a_1 \begin{vmatrix}
     a_2-a_1  &  &\cdots&a_n -a_1   \\
     0  &\ddots &&\vdots \\
     \vdots &\ddots&\ddots&    \\
     0&\cdots&0   & a_2-a_1
   \end{vmatrix}
= (-1)^{n-1}a_1(a_2-a_1)^{n-1}
$$
Où l'on a reconnu le déterminant d'un matrice triangulaire supérieure.
Donc 
$$\Delta_1 = a_1(a_1-a_2)^{n-1}.$$


  \item 
On va transformer la matrice correspondante en une matrice triangulaire supérieure,
on commence par remplacer la ligne $L_2$ par $L_2-L_1$ (on ne note que les coefficients non nuls) : 
$$\Delta_2 =   \begin{vmatrix}
  1 &      &    &   &+1 \\
  1 &1      && &  \\
   &1      &1  & &\\
   & &\ddots &\ddots & \\
  &  &       &1      &1
\end{vmatrix}=
 \begin{vmatrix}
  1 &      &    &   &+1 \\
  0 &1      && &  -1\\
   &1      &1  & &\\
   & &\ddots &\ddots & \\
  &  &       &1      &1
\end{vmatrix}
$$
Puis on remplace la ligne $L_3$ par $L_3-L_2$ (attention il s'agit de la nouvelle ligne $L_2$) et on 
continue ainsi de suite jusqu'à $L_{n-1} \leftarrow L_{n-1}-L_{n-2}$ ($n$ est la taille de la matrice sous-jacente) :
$$\Delta_2 =   \begin{vmatrix}
  1 &      &    &   &&+1 \\
  0 &1      && &  &-1\\
    & 0      &1  & & &+1\\
    & & 1      &1  & && \\
   & &&\ddots &\ddots & \\
  &  &       &&1      &1                 
               \end{vmatrix}
= \cdots = 
\begin{vmatrix}
  1 &      &    &   &&+1 \\
  0 &1      && &  &-1\\
    & 0      &1  & & &+1\\
    & & \ddots      &\ddots  & &\vdots& \\
   & &&0 &1 & (-1)^{n} \\
  &  &       &&1      &1                 
               \end{vmatrix}
$$
On fait attention pour le dernier remplacement $L_n \leftarrow L_n-L_{n-1}$ légèrement différent et qui conduit au déterminant d'une matrice triangulaire : :
$$\Delta_2 = \begin{vmatrix}
  1 &      &    &   &&+1 \\
  0 &1      && &  &-1\\
    & 0      &1  & & &+1\\
    & & \ddots      &\ddots  & &\vdots& \\
   & && &1 & (-1)^{n}\\
  &  &       &&0      &1 - (-1)^{n}                 
               \end{vmatrix}
= 1 - (-1)^{n}.
$$

En conclusion $\Delta_2 = \begin{cases}
                           0 & \text{ si $n$ est pair} \\
                           2 & \text{ si $n$ est impair} \\                           
                          \end{cases}$

  \item
On retire la colonne $C_1$ aux autres colonnes $C_i$ pour faire apparaître des $0$ :
$$\Delta_3 = \begin{vmatrix}
    a+b  &    a   & \cdots &  a       \\
     a   &   a+b  & \ddots & \vdots   \\
  \vdots & \ddots & \ddots &  a       \\
     a   & \cdots &    a   & a+b 
\end{vmatrix}
= \begin{vmatrix}
    a+b  &    -b   & & \cdots &  -b       \\
     a   &   b     &   0 & \cdots        &  0   \\
     a   &   0 &  \ddots & \ddots & \vdots   \\
  \vdots & \ddots & \ddots &  b & 0       \\
     a   &  0 & \cdots &    0   & b 
\end{vmatrix}$$
On remplace ensuite $L_1$ par $L_1+L_2+L_3+\cdots +L_n$
(ou ce qui revient au même : faites les opérations 
$L_1 \leftarrow L_1+L_2$ puis $L_1 \leftarrow L_1+L_3$,\ldots
chacune de ces opérations fait apparaître un $0$ sur la première ligne)
pour obtenir une matrice triangulaire inférieure :
$$\Delta_3 = \begin{vmatrix}
   na+b  &    0   & & \cdots &  0       \\
     a   &   b     &   0 & \cdots        &  0   \\
     a   &   0 &  \ddots & \ddots & \vdots   \\
  \vdots & \ddots & \ddots &  b & 0       \\
     a   &  0 & \cdots &    0   & b 
\end{vmatrix}=(na+b)b^{n-1}.$$
\end{enumerate}
\fincorrection
\correction{001143}


Commençons par un travail préparatoire : le calcul du déterminant de taille $(n-1)\times (n-1)$ :
{\footnotesize$$\Gamma_k = 
\begin{array}{|cccc|cccc|} 
x  &        &        &   &&&&\\
-1 & x      &        &   &&&&\\
   & \ddots & \ddots &   &&&&\\
   &        & -1     & x &&&&\\ 
   \hline
&&&&  -1 & x      &       &   \\
&&&&     &\ddots & \ddots      &   \\
&&&&     & &\ddots & x  \\
&&&&     &       &     & -1 \\
\end{array}
$$}
où le bloc en haut à gauche est de taille $k\times k$.

On développe, en commençant par la première ligne, puis encore une fois par la première ligne,...
pour trouver que 
$$\Gamma_k = x^k\times (-1)^{n-1-k}$$

Autre méthode : on retrouve le même résultat en utilisant les déterminant par blocs :
$$\begin{array}{|c|c|} 
A & B \\
\hline
(0) & C \\ 
\end{array}
= \det A \times \det C$$

\bigskip


Revenons à l'exercice !

Contrairement à l'habitude on développe par rapport à la colonne qui a le moins de $0$.
En développant par rapport à la dernière colonne on obtient :
{\footnotesize
\begin{align*}
 \Delta_{n}
& =
   \begin{vmatrix}
   x &  0    &        & a_{0}   \\
    -1 &\ddots &\ddots  &\vdots  \\
      &\ddots &x      & a_{n-2} \\
    0 &       & -1      & x+a_{n-1}
   \end{vmatrix} \\
 & = (-1)^{n-1} a_0 
   \begin{vmatrix}
   -1 &  x    &        &    \\
     & -1 & x  &  \\
      & & \ddots & \ddots       \\
     &       &       & -1
   \end{vmatrix}
+ (-1)^{n} a_1   
\begin{vmatrix}
   x & &&&\\
   & -1 &  x    &        &    \\
   &  & -1 & x  &  \\
    &  & & \ddots & \ddots       \\
    & &       &       & -1
   \end{vmatrix} 
\\
& \quad + \cdots +
(-1)^{2n-3} a_{n-2}   
\begin{vmatrix}
  x &      &    & &      \\
    -1 &\ddots &  & & \\
     &\ddots &\ddots & &  \\
      && -1 & x    &   \\
    &&&& -1 \\
   \end{vmatrix}
+ (-1)^{2n-2}(x+a_{n-1})
\begin{vmatrix}
  x &      &    &       \\
    -1 &\ddots &  &  \\
     &\ddots &\ddots &   \\
    &  & -1 & x       \\
   \end{vmatrix} \\
 & = \sum_{k=0}^{n-2} (-1)^{n-1+k} a_ k \times \Gamma_k \quad  + \quad (-1)^{2n-2}(x+a_{n-1})\Gamma_{n-1} \\
 & = \sum_{k=0}^{n-2} (-1)^{n-1+k} a_ k \times x^k\times (-1)^{n-1-k} \quad + \quad (x+a_{n-1})x^{n-1}\\
 & = a_0+a_1x+a_2x^2+\cdots + a_{n-1}x^{n-1} + x^n\\
\end{align*}
}
\fincorrection
\correction{001145}
\begin{enumerate}
  \item En développant par rapport à la première colonne on trouve la relation suivante :

$$\Delta_n = a \Delta_{n-1} + (-1)^{n-1}(n-1) 
\left\vert
\begin{matrix}
       0   &    0   & \cdots & 0      & n-1 \\
       a   &    0   & \ddots & \vdots & \vdots \\
    \vdots & \ddots & \ddots & 0      & 3 \\
       0   & \cdots &   a    & 0      & 2 \\
       0  & \cdots &   0    & a      & 1
\end{matrix}
\right\vert
$$
Notons $\delta$ ce dernier déterminant (dont la matrice est de taille $n-1\times n-1$). 
On le calcule en développant par rapport à la première ligne 
$$\delta  = (-1)^{n-2}(n-1)
\left\vert
\begin{matrix}
       a   &    0   & \cdots      & 0 \\
       0   &    a   & \ddots      & \vdots \\
    \vdots & \ddots & \ddots      & 0  \\
       0   & \cdots &   0         & a
\end{matrix}\right\vert
= (-1)^{n-2}(n-1) a^{n-2}.
$$
Donc 
$$\Delta_n = a \Delta_{n-1} - a^{n-2}(n-1)^2.$$

  \item Prouvons la formule 
$$\Delta_n=a^n-a^{n-2}\sum_{i=1}^{n-1}{i^2}$$
par récurrence sur $n\ge 2$.

\begin{itemize}
  \item \textbf{Initialisation.} Pour $n=2$,
$\Delta_2=\begin{vmatrix}a&1\\1&a\end{vmatrix}=a^2-1$
donc la formule est vraie.

  \item \textbf{Hérédité.} Supposons la formule vraie vraie au rang $n-1$,
c'est-à-dire $\Delta_{n-1}=a^{n-1}-a^{n-3}\sum_{i=1}^{n-2}{i^2}$.
Calculons $\Delta_n$ :
\begin{align*}
\Delta_n 
  & = a \Delta_{n-1} - a^{n-2}(n-1)^2  \quad \text{ par la première question } \\
  & = a\Big(a^{n-1}-a^{n-3}\sum_{i=1}^{n-2}{i^2} \Big) - a^{n-2}(n-1)^2 \quad \text{ par l'hypothèse de récurrence} \\
  & = a^n -  a^{n-2}\sum_{i=1}^{n-2}{i^2} - a^{n-2}(n-1)^2 \\
  & = a^n-a^{n-2}\sum_{i=1}^{n-1}{i^2}
\end{align*}
La formule est donc vraie au rang $n$.

  \item \textbf{Conclusion.} Par le principe de récurrence la formule est vraie
pour tout entier $n\ge 2$.
\end{itemize}

\end{enumerate}
\fincorrection
\correction{002453}
Notons $V_n$ le déterminant à calculer
et $C_1,\ldots,C_n$ les colonnes de la matrice correspondante.

Nous allons faire les opérations suivantes sur les colonnes
en partant de la dernière colonne.
$C_n$ est remplacée par $C_n-t_n C_{n-1}$,
puis $C_{n-1}$ est remplacée par $C_{n-1}-t_n C_{n-2}$,...
jusqu'à $C_2$ qui est remplacée par $C_2-t_nC_1$.
On obtient donc

$$V_n=\begin{vmatrix}
1 & t_1 & t_1^2 & \ldots & t_1^{n-1} \\
1 & t_2 & t_2^2 & \ldots & t_2^{n-1} \\\
\ldots&\ldots&\ldots& \ldots & \ldots \\
1 & t_n & t_n^2 & \ldots & t_n^{n-1}
\end{vmatrix} 
= 
\begin{vmatrix}
1 & t_1-t_n & t_1^2-t_1t_n & \ldots & t_1^{n-1}-t_1^{n-2}t_n \\
1 & t_2-t_n & t_2^2-t_2t_n & \ldots & t_2^{n-1}-t_2^{n-2}t_n \\\
\ldots&\ldots&\ldots& \ldots & \ldots \\
1 & 0 & 0 & \ldots & 0
\end{vmatrix}
$$

On développe par rapport à la dernière ligne et on écrit $t_i^k-t_i^{k-1}t_n=t_i^{k-1}(t_i-t_n)$ pour obtenir :
$$V_n = (-1)^{n-1}\begin{vmatrix}
 t_1-t_n & t_1(t_1-t_n) & \ldots & t_1^{n-2}(t_1-t_n) \\
 t_2-t_n & t_2(t_2-t_n) & \ldots & t_2^{n-2}(t_2-t_n) \\\
\ldots&\ldots& \ldots & \ldots \\
 t_{n-1}-t_n & \ldots & \ldots & \ldots
\end{vmatrix}$$

Nous utilisons maintenant la linéarité du déterminant par rapport à chacune des lignes :
on factorise la première ligne par $t_1-t_n$ ; la second par $t_2-t_n$,...
On obtient 
$$V_n = (-1)^{n-1}(t_1-t_n)(t_2-t_n)\cdots(t_{n-1}-t_n)
\begin{vmatrix}
1 & t_1 & t_1^2 & \ldots & t_1^{n-2} \\
1 & t_2 & t_2^2 & \ldots & t_2^{n-2} \\\
\ldots&\ldots&\ldots& \ldots & \ldots \\
1 & t_{n-1} & t_{n-1}^2 & \ldots & t_{n-1}^{n-2}
\end{vmatrix} 
$$
Donc $$V_n = V_{n-1}\prod_{j=1}^{n-1}(t_n-t_j).$$

Si maintenant on suppose la formule connue pour $V_{n-1}$
c'est-à-dire $V_{n-1}(t_1,\ldots,t_{n-1})
= \prod_{1 \le i < j \le n-1} (t_j - t_i)$

Alors on obtient par récurrence que
$$V_n(t_1,\ldots,t_{n-1},t_n)= V_{n-1}(t_1,\ldots,t_{n-1})\prod_{j=1}^{n-1}(t_n-t_j) =  \prod_{1 \le i < j \le n} (t_j - t_i).$$

\fincorrection


\end{document}

