
%%%%%%%%%%%%%%%%%% PREAMBULE %%%%%%%%%%%%%%%%%%

\documentclass[11pt,a4paper]{article}

\usepackage{amsfonts,amsmath,amssymb,amsthm}
\usepackage[utf8]{inputenc}
\usepackage[T1]{fontenc}
\usepackage[francais]{babel}
\usepackage{mathptmx}
\usepackage{fancybox}
\usepackage{graphicx}
\usepackage{ifthen}

\usepackage{tikz}   

\usepackage{hyperref}
\hypersetup{colorlinks=true, linkcolor=blue, urlcolor=blue,
pdftitle={Exo7 - Exercices de mathématiques}, pdfauthor={Exo7}}

\usepackage{geometry}
\geometry{top=2cm, bottom=2cm, left=2cm, right=2cm}

%----- Ensembles : entiers, reels, complexes -----
\newcommand{\Nn}{\mathbb{N}} \newcommand{\N}{\mathbb{N}}
\newcommand{\Zz}{\mathbb{Z}} \newcommand{\Z}{\mathbb{Z}}
\newcommand{\Qq}{\mathbb{Q}} \newcommand{\Q}{\mathbb{Q}}
\newcommand{\Rr}{\mathbb{R}} \newcommand{\R}{\mathbb{R}}
\newcommand{\Cc}{\mathbb{C}} \newcommand{\C}{\mathbb{C}}
\newcommand{\Kk}{\mathbb{K}} \newcommand{\K}{\mathbb{K}}

%----- Modifications de symboles -----
\renewcommand{\epsilon}{\varepsilon}
\renewcommand{\Re}{\mathop{\mathrm{Re}}\nolimits}
\renewcommand{\Im}{\mathop{\mathrm{Im}}\nolimits}
\newcommand{\llbracket}{\left[\kern-0.15em\left[}
\newcommand{\rrbracket}{\right]\kern-0.15em\right]}
\renewcommand{\ge}{\geqslant} \renewcommand{\geq}{\geqslant}
\renewcommand{\le}{\leqslant} \renewcommand{\leq}{\leqslant}

%----- Fonctions usuelles -----
\newcommand{\ch}{\mathop{\mathrm{ch}}\nolimits}
\newcommand{\sh}{\mathop{\mathrm{sh}}\nolimits}
\renewcommand{\tanh}{\mathop{\mathrm{th}}\nolimits}
\newcommand{\cotan}{\mathop{\mathrm{cotan}}\nolimits}
\newcommand{\Arcsin}{\mathop{\mathrm{arcsin}}\nolimits}
\newcommand{\Arccos}{\mathop{\mathrm{arccos}}\nolimits}
\newcommand{\Arctan}{\mathop{\mathrm{arctan}}\nolimits}
\newcommand{\Argsh}{\mathop{\mathrm{argsh}}\nolimits}
\newcommand{\Argch}{\mathop{\mathrm{argch}}\nolimits}
\newcommand{\Argth}{\mathop{\mathrm{argth}}\nolimits}
\newcommand{\pgcd}{\mathop{\mathrm{pgcd}}\nolimits} 

%----- Structure des exercices ------

\newcommand{\exercice}[1]{\video{0}}
\newcommand{\finexercice}{}
\newcommand{\noindication}{}
\newcommand{\nocorrection}{}

\newcounter{exo}
\newcommand{\enonce}[2]{\refstepcounter{exo}\hypertarget{exo7:#1}{}\label{exo7:#1}{\bf Exercice \arabic{exo}}\ \  #2\vspace{1mm}\hrule\vspace{1mm}}

\newcommand{\finenonce}[1]{
\ifthenelse{\equal{\ref{ind7:#1}}{\ref{bidon}}\and\equal{\ref{cor7:#1}}{\ref{bidon}}}{}{\par{\footnotesize
\ifthenelse{\equal{\ref{ind7:#1}}{\ref{bidon}}}{}{\hyperlink{ind7:#1}{\texttt{Indication} $\blacktriangledown$}\qquad}
\ifthenelse{\equal{\ref{cor7:#1}}{\ref{bidon}}}{}{\hyperlink{cor7:#1}{\texttt{Correction} $\blacktriangledown$}}}}
\ifthenelse{\equal{\myvideo}{0}}{}{{\footnotesize\qquad\texttt{\href{http://www.youtube.com/watch?v=\myvideo}{Vidéo $\blacksquare$}}}}
\hfill{\scriptsize\texttt{[#1]}}\vspace{1mm}\hrule\vspace*{7mm}}

\newcommand{\indication}[1]{\hypertarget{ind7:#1}{}\label{ind7:#1}{\bf Indication pour \hyperlink{exo7:#1}{l'exercice \ref{exo7:#1} $\blacktriangle$}}\vspace{1mm}\hrule\vspace{1mm}}
\newcommand{\finindication}{\vspace{1mm}\hrule\vspace*{7mm}}
\newcommand{\correction}[1]{\hypertarget{cor7:#1}{}\label{cor7:#1}{\bf Correction de \hyperlink{exo7:#1}{l'exercice \ref{exo7:#1} $\blacktriangle$}}\vspace{1mm}\hrule\vspace{1mm}}
\newcommand{\fincorrection}{\vspace{1mm}\hrule\vspace*{7mm}}

\newcommand{\finenonces}{\newpage}
\newcommand{\finindications}{\newpage}


\newcommand{\fiche}[1]{} \newcommand{\finfiche}{}
%\newcommand{\titre}[1]{\centerline{\large \bf #1}}
\newcommand{\addcommand}[1]{}

% variable myvideo : 0 no video, otherwise youtube reference
\newcommand{\video}[1]{\def\myvideo{#1}}

%----- Presentation ------

\setlength{\parindent}{0cm}

\definecolor{myred}{rgb}{0.93,0.26,0}
\definecolor{myorange}{rgb}{0.97,0.58,0}
\definecolor{myyellow}{rgb}{1,0.86,0}

\newcommand{\LogoExoSept}[1]{  % input : echelle       %% NEW
{\usefont{U}{cmss}{bx}{n}
\begin{tikzpicture}[scale=0.1*#1,transform shape]
  \fill[color=myorange] (0,0)--(4,0)--(4,-4)--(0,-4)--cycle;
  \fill[color=myred] (0,0)--(0,3)--(-3,3)--(-3,0)--cycle;
  \fill[color=myyellow] (4,0)--(7,4)--(3,7)--(0,3)--cycle;
  \node[scale=5] at (3.5,3.5) {Exo7};
\end{tikzpicture}}
}


% titre
\newcommand{\titre}[1]{%
\vspace*{-4ex} \hfill \hspace*{1.5cm} \hypersetup{linkcolor=black, urlcolor=black} 
\href{http://exo7.emath.fr}{\LogoExoSept{3}} 
 \vspace*{-5.7ex}\newline 
\hypersetup{linkcolor=blue, urlcolor=blue}  {\Large \bf #1} \newline 
 \rule{12cm}{1mm} \vspace*{3ex}}

%----- Commandes supplementaires ------



\begin{document}

%%%%%%%%%%%%%%%%%% EXERCICES %%%%%%%%%%%%%%%%%%

\fiche{f00162, bodin, 2012/09/01} 

\newcommand{\Ker}{\mathop{\mathrm{Ker}}\nolimits}

\titre{Matrice d'une application linéaire}

Corrections d'Arnaud Bodin.

\bigskip

\exercice{1087, ridde, 1999/11/01}
\video{SRrzcCzYHo0}
\enonce{001087}{}
Soit $\Rr^2$ muni de la base canonique $\mathcal{B}=(\vec{i}, \vec{j})$.
Soit $f : \Rr^2 \to \Rr^2$ la projection sur l'axe des abscisses $\Rr \vec{i}$ 
parall\`element à $\Rr (\vec{i} + \vec{j})$.
Déterminer $\textrm{Mat}_{\mathcal{B},\mathcal{B}}(f)$, la matrice de $f$ dans la base $(\vec{i}, \vec{j})$.

Même question avec $\textrm{Mat}_{\mathcal{B}',\mathcal{B}}(f)$ où $\mathcal{B'}$ est la base 
$(\vec{i} - \vec{j}, -2\vec{i}+3\vec{j})$ de $\Rr^2$.
Même question avec $\textrm{Mat}_{\mathcal{B}',\mathcal{B}'}(f)$.
\finenonce{001087}



\finexercice\exercice{1097, cousquer, 2003/10/01}
\video{jQcF6rnWgyI}
\enonce{001097}{}
Soient trois vecteurs $e_1,e_2,e_3$ formant une base de $\Rr^3$.
On note $\phi$ l'application linéaire définie par
$\phi(e_1)=e_3$, $\phi(e_2)=-e_1+e_2+e_3$ et $\phi(e_3)=e_3$.

\begin{enumerate}
\item Écrire la matrice $A$ de $\phi$ dans la base $(e_1,e_2,e_3)$.
Déterminer le noyau de cette application. 

\item On pose $f_1=e_1-e_3$, $f_2=e_1-e_2$,  $f_3=-e_1+e_2+e_3$.
Calculer $e_1,e_2,e_3$ en fonction de $f_1,f_2,f_3$.
Les vecteurs $f_1,f_2,f_3$ forment-ils une base de $\Rr^3$ ?

\item Calculer $\phi(f_1), \phi(f_2), \phi(f_3)$ en fonction de $f_1,f_2,f_3$.
Écrire la matrice $B$ de $\phi$ dans la base $(f_1,f_2,f_3)$ et trouver la nature
de l'application $\phi$.

\item On pose $P=\begin{pmatrix}1&1&-1\cr 0&-1&1\cr-1&0&1\cr\end{pmatrix}$. Vérifier que $P$ est
inversible et calculer $P^{-1}$. Quelle relation lie $A$, $B$, $P$ et $P^{-1}$ ?
\end{enumerate}
\finenonce{001097}



\finexercice\exercice{2433, matexo1, 2002/02/01}
\video{F1PmEC1haKY}
\enonce{002433}{}
Soit $f$ l'endomorphisme de $\R^3$ dont la matrice par
rapport \`a la base canonique $(e_1, e_2, e_3)$ est
$$A= \left( 
       \begin{array}{ccc}
        15 & -11 & 5 \\
        20 & -15 & 8 \\
        8 &  -7 & 6
       \end{array}
       \right).$$
Montrer que les vecteurs
$$ e'_1 = 2e_1+3e_2+e_3,\quad e'_2 = 3e_1+4e_2+e_3,\quad e'_3 =
e_1+2e_2+2e_3$$
forment une base de $\R^3$ et calculer la matrice de $f$ par
rapport \`a cette base.
\finenonce{002433}


\finexercice


\exercice{1101, ridde, 1999/11/01}
\video{WqbDAp8ZsBE}
\enonce{001101}{}

% anti diagonal dots
\def\Ddots{\mathinner{\mkern2mu\raise1pt\hbox{.}\mkern2mu
\newline \raise4pt\hbox{.}\mkern2mu\raise7pt\hbox{.}\mkern1mu}}

Soit $A = \begin{pmatrix} 
0&&\dots&0&1 \\ 
\vdots&&&1&0 \\
 & & \Ddots && \\
0&1& & &\vdots \\ 
1&0&&\dots&0
\end{pmatrix}$. 
En utilisant l'application linéaire associée de 
$\mathcal{L} (\Rr^n,\Rr^n)$, calculer $A^p$ pour $p \in \Zz$.
\finenonce{001101}



\finexercice

\exercice{2444, matexo1, 2002/02/01}
\video{GsaYq45QDFE}
\enonce{002444}{}
Soient $A, B$ deux matrices semblables (i.e. il existe $P$
inversible telle que $B = P^{-1} A P$). Montrer que si l'une est
inversible, l'autre aussi\,; que si l'une est idempotente, l'autre
aussi\,; que si l'une est nilpotente, l'autre aussi\,; que si $A =
\lambda I$, alors $A = B$.
\finenonce{002444}



\finexercice

\exercice{1104, ridde, 1999/11/01}
\video{VTTwQGImicw}
\enonce{001104}{}
Soit $f$ l'endomorphisme de $\Rr^2$ de matrice $A=\begin{pmatrix} 2&\frac 23\\
-\frac 52&-\frac 23 \end{pmatrix}$ dans la base canonique. Soient 
$e_1 = \begin{pmatrix} -2 \\ 3\end{pmatrix}$
et $e_2 = \begin{pmatrix} -2 \\ 5 \end{pmatrix}$.
\begin{enumerate}
\item Montrer que $\mathcal{B}'= (e_1, e_2)$ est une base de $\Rr^2$ et déterminer 
$\text{Mat}_{\mathcal{B}'}(f)$.
\item Calculer $A^n$ pour $n \in \Nn$.
\item Déterminer l'ensemble des suites réelles qui vérifient $\forall n \in \Nn$
$\begin{cases} x_{n + 1} = 2x_n + \dfrac 23 y_n \\ y_{n + 1} = -\dfrac 52 x_n -
\dfrac 23 y_n \end{cases}$.
\end{enumerate}
\finenonce{001104}


\finexercice
\exercice{2774, tumpach, 2009/10/25}
\video{JhulB3S-Wf4}
\enonce{002774}{} 
Soit $a $ et $b$ deux réels et $A$ la matrice
$$
A = \left(
\begin{array}{cccc}
a & 2 & -1 & b\\
3 & 0 & 1 & -4\\ 
5 & 4 & -1 & 2
\end{array}
\right)
$$
Montrer que $\textrm{rg}(A) \geq 2$. 
Pour quelles valeurs de $a$ et $b$ a-t-on $\textrm{rg}(A) = 2$~?
\finenonce{002774}



\finexercice

\exercice{1099, legall, 1998/09/01}
\video{96dAjfnOIhI}
\enonce{001099}{}
Soient  
$A=\begin{pmatrix} 
1 & 2 & 1 \cr
3 & 4 & 1 \cr
5 & 6 & 1 \cr
7 & 8 & 1 \cr
\end{pmatrix},\ 
B=\begin{pmatrix} 
2 & 2 & -1 & 7  \cr
4 & 3 & -1 & 11 \cr
0 & -1 & 2 & -4 \cr
3 & 3 & -2 & 11 \cr 
\end{pmatrix} $.
Calculer $\textrm{rg}(A)$ et $\textrm{rg}(B)$. Déterminer une base du
noyau et une base de l'image pour chacune des applications linéaires associées $f_A$ et $f_B$.
\finenonce{001099}



\finexercice
\exercice{1093, legall, 1998/09/01}
\video{A2_9r3WOfTM}
\enonce{001093}{}
\label{exo1093}
Soit  $E$  un espace vectoriel et  $f$  une application linéaire de
$E$  dans  lui-m\^eme telle que  $f^2=f$.
\begin{enumerate}
    \item Montrer que  $E= \Ker f \oplus \Im f$.

    \item Supposons que  $E$ soit de dimension finie  $n$. 
Posons  $r= \dim \Im f$. 
Montrer qu'il existe une base 
$\mathcal{B}= ( e_1, \ldots ,e_n)$ de  $E$  telle que : 
 $f(e_i)=e_i$ si $i\le r$ et $f(e_i)=0$ si $i>r$. 
Déterminer la matrice de  $f$ dans cette base $\mathcal{B}$.
\end{enumerate}
\finenonce{001093}



\finexercice

\exercice{2475, matexo1, 2002/02/01}
\video{zf0ETNslBnc}
\enonce{002475}{}
\label{exo2475}
Trouver toutes les matrices de $\mathcal{M}_3(\Rr)$ qui vérifient
\begin{enumerate}
\item $M^2 = 0$ ;
\item $M^2 = M$ ; 
\item $M^2 = I$. 
\end{enumerate}
\finenonce{002475}



\finexercice\exercice{1094, legall, 1998/09/01}
\video{10aMCsQRMG4}
\enonce{001094}{}
Soit $f$ l'application de $\Rr_n[X]$  dans  $\Rr[X]$
définie en posant pour tout  $P(X)\in \Rr_n[X]$ : $f(P(X))=P(X+1)+P(X-1)-2P(X).$
\begin{enumerate}
    \item Montrer que  $f$  est linéaire et que son image est
incluse dans  $\Rr_n[X]$.
    \item Dans le cas o\`u  $n=3$, donner la matrice de  $f$  dans
la base  $1,X, X^2, X^3$. Déterminer ensuite, pour une valeur de  $n$
quelconque, la matrice de  $f$  dans la base  $1,X,\ldots,X^n$.
    \item Déterminer le noyau et l'image de  $f$. Calculer
leur dimension respective.
    \item Soit  $Q$  un élément de l'image de  $f$.
Montrer qu'il existe un unique  $P\in \Rr_n[X]$
tel que : $f(P)=Q$  et  $P(0)=P'(0)=0$.
\end{enumerate}
\finenonce{001094}



\finexercice
\exercice{2442, matexo1, 2002/02/01}
\video{ATfy0zSe_04}
\enonce{002442}{}
Pour toute matrice carrée $A$ de dimension $n$, 
on appelle trace de $A$, et l'on note $\textrm{tr}\, A$, la
somme des éléments diagonaux de $A$\,:
$$\textrm{tr}\, A = \sum_{i = 1}^n a_{i,i}$$
\begin{enumerate}
\item Montrer que si $A, B$ sont deux matrices carrées
d'ordre $n$, alors $\textrm{tr}(AB)=\textrm{tr} (BA)$.

\item Montrer que si $f$ est un endomorphisme d'un espace
vectoriel $E$ de dimension $n$, $M$ sa matrice par rapport à
une base $e$, $M'$ sa matrice par rapport à une base $e'$,
alors $\textrm{tr}\, M = \textrm{tr}\, M'$. 
On note $\textrm{tr}\, f$ la valeur commune de ces quantités.

\item Montrer que si $g$ est un autre endomorphisme de $E$,
$\textrm{tr}(f\circ g - g\circ f) = 0$.
\end{enumerate}
\finenonce{002442}




\finexercice

\finfiche


 \finenonces 



 \finindications 

\indication{001087}
$f$ est l'application qui à $\begin{pmatrix}x\\y\end{pmatrix}$ associe
 $\begin{pmatrix}x-y\\0\end{pmatrix}$.
\finindication
\noindication
\noindication
\noindication
\indication{002444}
$A$ est \emph{idempotente} s'il existe un $n$ tel que $A^n=I$ (la matrice identité).

$A$ est \emph{nilpotente} s'il existe un $n$ tel que $A^n=(0)$ (la matrice nulle).
\finindication
\noindication
\noindication
\noindication
\noindication
\indication{002475}
Il faut trouver les propriétés de l'application linéaire $f$ associée à chacune de ces matrices.
Les résultats s'expriment en explicitant une (ou plusieurs) matrice $M'$ qui est la matrice de $f$ dans une base bien choisie
et ensuite en montrant que toutes les autres matrices sont de la forme
$M=P^{-1}M'P$.

Plus en détails pour chacun des cas :
\begin{enumerate}
  \item $\Im f \subset \Ker f$ et discuter suivant la dimension du noyau.
  \item Utiliser l'exercice \ref{exo1093} : $\Ker f \oplus \Im f$ et il existe une base telle que 
$f(e_i)=0$ ou $f(e_i)=e_i$.
  \item Poser $N= \frac{I+M}{2}$ (et donc $M=\cdots$) chercher à quelle condition $M^2=I$.
\end{enumerate}
\finindication
\noindication
\noindication


\newpage

\correction{001087}
L'expression de $f$ dans la base $\mathcal{B}$ est la suivante $f(x,y)=(x-y,0)$.
Autrement dit à un vecteur $\begin{pmatrix}x\\y\end{pmatrix}$ on associe
le vecteur  $\begin{pmatrix}x-y\\0\end{pmatrix}$.
On note que $f$ est bien une application linéaire.
Cette expression nous permet de calculer les matrices demandées.


Remarque : comme $\mathcal{B}$ est la base canonique on note
$\begin{pmatrix}x\\y\end{pmatrix}$ pour $\begin{pmatrix}x\\y\end{pmatrix}_\mathcal{B}$
qui est le vecteur $x \vec{i}+y\vec{j}$.



\begin{enumerate}
  \item Calcul de $\textrm{Mat}(f,\mathcal{B},\mathcal{B})$.
Comme $\mathcal{B}=(\vec{i}, \vec{j})$, la matrice s'obtient en calculant $f(\vec{i})$ et $f(\vec{j})$ :
$$f(\vec{i})=f\begin{pmatrix}1\\0\end{pmatrix} = \begin{pmatrix}1\\0\end{pmatrix} = \vec{i}
\quad 
f(\vec{j})=f\begin{pmatrix}0\\1\end{pmatrix} = \begin{pmatrix}-1\\0\end{pmatrix} = -\vec{i}$$
donc
$$\textrm{Mat}(f,\mathcal{B},\mathcal{B}) = \begin{pmatrix} 1 & -1 \\ 0 & 0 \end{pmatrix}$$

  \item On garde la même application linéaire mais la base de départ change (la base d'arrivée reste $\mathcal{B}$).
Si on note $\vec{u} = \vec{i}-\vec{j}$ et $\vec{v} = -2\vec{i}+3\vec{j}$, on a 
$\mathcal{B'}=(\vec{i} - \vec{j}, -2\vec{i}+3\vec{j}) = (\vec{u},\vec{v})$. On exprime 
$f(\vec{u})$ et $f(\vec{v})$ dans la base d'arrivée $\mathcal{B}$.
$$f(\vec{u})=f(\vec{i}- \vec{j})=f\begin{pmatrix}1\\-1\end{pmatrix} = \begin{pmatrix}2\\0\end{pmatrix}
\quad 
f(\vec{v})=f(-2\vec{i}+3\vec{j})=f\begin{pmatrix}-2\\3\end{pmatrix} = \begin{pmatrix}-5\\0\end{pmatrix}$$
donc
$$\textrm{Mat}(f,\mathcal{B}',\mathcal{B}) = \begin{pmatrix} 2 & -5 \\ 0 & 0 \end{pmatrix}$$
  
  \item Toujours avec le même $f$ on prend $\mathcal{B}'$ comme base de départ et d'arrivée,
il s'agit donc d'exprimer $f(\vec{u})$ et $f(\vec{v})$ dans la base $\mathcal{B}'=(\vec{u},\vec{v})$.
Nous venons de calculer que 
$$f(\vec{u})=f(\vec{i}- \vec{j})=f\begin{pmatrix}1\\-1\end{pmatrix} = \begin{pmatrix}2\\0\end{pmatrix}=2\vec{i}
\quad 
f(\vec{v})=f(2\vec{i}+3\vec{j})=f\begin{pmatrix}-2\\3\end{pmatrix} = \begin{pmatrix}-5\\0\end{pmatrix}=-5\vec{i}$$
Mais il  nous faut obtenir une expression en fonction de la base $\mathcal{B}'$.
Remarquons que 
$$\left\{\begin{array}{lcr}
\vec{u} &=& \vec{i}-\vec{j} \\
\vec{v} &=& -2\vec{i}+3\vec{j} \\           
         \end{array}\right.
\implies
\left\{\begin{array}{lcr}
\vec{i} &=& 3\vec{u}+\vec{v} \\
\vec{j} &=& 2\vec{u}+\vec{v} \\           
         \end{array}\right.$$
Donc 
$$f(\vec{u})=f(\vec{i}- \vec{j})=2\vec{i}=6\vec{u}+2\vec{v} = \begin{pmatrix}6\\2\end{pmatrix}_{\mathcal{B}'}
\quad
f(\vec{v})=f(-2\vec{i}+3\vec{j})=-5\vec{i}=-15\vec{u}-5\vec{v} = \begin{pmatrix}-15\\-5\end{pmatrix}_{\mathcal{B}'}$$
Donc
$$\textrm{Mat}(f,\mathcal{B}',\mathcal{B}') = \begin{pmatrix} 6 & -15 \\ 2 & -5 \end{pmatrix}$$

Remarque :
$\begin{pmatrix}x\\y\end{pmatrix}_{\mathcal{B}'}$
désigne le vecteur $x \vec{u}+y\vec{v}$.
\end{enumerate}

\fincorrection
\correction{001097}
\begin{enumerate}
  \item 
  On note la base $\mathcal{B}=(e_1,e_2,e_3)$
  et $X=\begin{pmatrix}x\\y\\z\end{pmatrix}_{\mathcal{B}}= x e_1+y e_2+z e_3$.
  La matrice $A=\textrm{Mat}_{\mathcal{B}}(f)$ est composée des vecteurs colonnes $\phi(e_i)$,
on sait 
$$\phi(e_1)=e_3 = \begin{pmatrix}0\\0\\1\end{pmatrix}_{\mathcal{B}} \quad
\phi(e_2)=-e_1+e_2+e_3 = \begin{pmatrix}-1\\1\\1\end{pmatrix}_{\mathcal{B}} \quad 
\phi(e_3)=e_3 = \begin{pmatrix}0\\0\\1\end{pmatrix}_{\mathcal{B}} \quad
$$

$$\text{donc } \quad A=\begin{pmatrix}
0 & -1 & 0 \\
0 & 1  & 0 \\
1 & 1  & 1 \\    
\end{pmatrix}$$

Le noyau de $\phi$ (ou celui de $A$) est l'ensemble de $X=\begin{pmatrix}x\\y\\z\end{pmatrix}$
tel que $AX=0$.

$$AX=0 \iff \begin{pmatrix}
0 & -1 & 0 \\
0 & 1  & 0 \\
1 & 1  & 1 \\  
\end{pmatrix} \times \begin{pmatrix}x\\y\\z\end{pmatrix}
=\begin{pmatrix}0\\0\\0\end{pmatrix}
\iff \left\{
\begin{array}{rcl}
-y&=&0\\
y&=&0\\
x+y+z&=&0\\
\end{array}\right.
$$
Donc $\Ker \phi = \big\{ \begin{pmatrix}x \\ 0 \\-x\end{pmatrix}_{\mathcal{B}}  \in \Rr^3 \mid x\in \Rr \big\}= 
\textrm{Vect} \begin{pmatrix}1\\0\\-1\end{pmatrix}_{\mathcal{B}} = \textrm{Vect} (e_1-e_3)$.
Le noyau est donc de dimension $1$.


  \item On applique le pivot de Gauss comme si c'était un système linéaire :
$$\left\{
\begin{array}{cccccclr}
e_1  & &     &-& e_3  &=& f_1 &_{L_1}\\
e_1  &-& e_2 & &      &=& f_2 &_{L_2}\\
-e_1 &+& e_2 &+& e_3  &=& f_3 &_{L_3}\\
\end{array}\right.
\iff  \left\{
\begin{array}{cccccclr}
e_1  & &     &-& e_3  &=& f_1 &\\
     &-& e_2 &+& e_3  &=& f_2-f_1 &_{L_2-L_1}\\
     & & e_2 & &      &=& f_3+f_1 &_{L_3+L_1}\\
\end{array}\right.
$$
On en déduit
$$\left\{
\begin{array}{rcl}
 e_1 &=& f_1+f_2+f_3 \\
 e_2 &=& f_1+f_3\\
 e_3 &=& f_2+f_3 \\
\end{array}\right.
$$

Donc tous les vecteurs de la base $\mathcal{B}=(e_1,e_2,e_3)$ s'expriment en fonction
de $(f_1,f_2,f_3)$, ainsi la famille $(f_1,f_2,f_3)$ est génératrice.
Comme elle a exactement $3$ éléments dans l'espace vectoriel $\Rr^3$ de dimension $3$ alors
$\mathcal{B}'=(f_1,f_2,f_3)$ est une base.


  \item
$$\phi(f_1)=\phi(e_1-e_3)=\phi(e_1)-\phi(e_3)=e_3-e_3=0$$

$$\phi(f_2)=\phi(e_1-e_2)= \phi(e_1)-\phi(e_2)=e_3 - (-e_1+e_2+e_3) = e_1-e_2 = f_2$$

$$\phi(f_3)=\phi(-e_1+e_2+e_3)=-\phi(e_1)+\phi(e_2)+\phi(e_3)=-e_1+e_2+e_3=f_3$$

Donc, dans la base $\mathcal{B}'=(f_1,f_2,f_3)$, nous avons
$$\phi(f_1)=0=\begin{pmatrix}0\\0\\0\end{pmatrix}_{\mathcal{B}'}\quad
\phi(f_2)=f_2=\begin{pmatrix}0\\1\\0\end{pmatrix}_{\mathcal{B}'}
\phi(f_3)=f_3=\begin{pmatrix}0\\0\\1\end{pmatrix}_{\mathcal{B}'}$$

Donc la matrice de $\phi$ dans la base $\mathcal{B}'$ est
$$B=\begin{pmatrix}
0 & 0 & 0 \\
0 & 1 & 0 \\
0 & 0 & 1 \\    
\end{pmatrix}$$

$\phi$ est la projection sur  $\textrm{Vect} (f_2,f_3)$ parallèlement à $\textrm{Vect} (f_1)$ (autrement dit
c'est la projection sur le plan d'équation $(x'=0)$, parallèlement à l'axe des $x'$, ceci dans la base $\mathcal{B}'$).

  \item $P$ est la matrice de passage de $\mathcal{B}$ vers $\mathcal{B}'$.
En effet la matrice de passage contient -en colonnes- les coordonnées des vecteurs
de la nouvelle base $\mathcal{B}'$ exprimés dans l'ancienne base $\mathcal{B}$.


Si un vecteur a pour coordonnées $X$ dans la base $\mathcal{B}$ et $X'$ dans la base $\mathcal{B}'$
alors $PX'=X$ (attention à l'ordre).
Et si $A$ est la matrice de $\phi$ dans la base $\mathcal{B}$ et $B$ est la matrice de $\phi$ dans la base
$\mathcal{B}'$ alors
$$B=P^{-1}AP$$
(Une matrice de passage entre deux bases est inversible.)

Ici on calcule l'inverse de $P$ :
$$P^{-1} = \begin{pmatrix}
1 & 1 & 0 \\
1 & 0 & 1 \\
1 & 1 & 1 \\    
\end{pmatrix}
\quad \text{ donc } \quad 
B=P^{-1}AP=\begin{pmatrix}
0 & 0 & 0 \\
0 & 1 & 0 \\
0 & 0 & 1 \\   
\end{pmatrix}
$$

On retrouve donc bien les mêmes résultats que précédemment.

\end{enumerate}

\fincorrection
\correction{002433}
Notons l'ancienne base $\mathcal{B}=(e_1,e_2,e_3)$
et ce qui sera la nouvelle base $\mathcal{B}'=(e'_1,e'_2,e'_3)$.
Soit $P$ la matrice de passage qui contient -en colonnes- les coordonnées des vecteurs
de la nouvelle base $\mathcal{B}'$ exprimés dans l'ancienne base $\mathcal{B}$
$$P=\begin{pmatrix}
2 & 3 & 1 \\
3 & 4 & 2 \\
1 & 1 & 2 \\      
\end{pmatrix}$$

On vérifie que $P$ est inversible (on va même calculer son inverse) donc
$\mathcal{B}'$ est bien une base.
De plus 
$$P^{-1} = \begin{pmatrix}
-6 & 5 & -2 \\
4 & -3 & 1 \\
1 & -1 & 1 \\             
           \end{pmatrix}
\text{ et on calcule  } B=P^{-1} A P = 
\begin{pmatrix}
1 & 0 & 0 \\
0 & 2 & 0 \\
0 & 0 & 3 \\  
\end{pmatrix}$$

$B$ est la matrice de $f$ dans la base $\mathcal{B}'$.
\fincorrection
\correction{001101}
Nous associons à la matrice $A$ son application linéaire naturelle $f$.
Si $\mathcal{B}=(e_1,e_2,\ldots,e_n)$ est la base canonique de $\Rr^n$
alors $f(e_1)$ est donné par le premier vecteur colonne, $f(e_2)$ par le deuxième, etc.
Donc ici 
$$f(e_1)=\begin{pmatrix}0\\ \vdots \\ 0 \\ 0 \\ 1 \end{pmatrix}=e_n, \
f(e_2)= \begin{pmatrix}0\\ \vdots \\ 0 \\ 1 \\0 \end{pmatrix}=e_{n-1},...  \quad \text{ et en général }
f(e_i) = e_{n+1-i}$$ 
Calculons ce que vaut la composition $f\circ f$.
Comme une application linéaire est déterminée par
 les images des éléments d'une base alors
on calcule $f\circ f(e_i)$, $i=1,\ldots,n$ en appliquant deux fois la formule précédente :
$$f\circ f(e_i) = f\big( f(e_i) \big) =  f(e_{n+1-i})=e_{n+1-(n+1-i)}=e_i$$
Comme $f\circ f$ laisse invariant tous les vecteurs de la base alors
$f\circ f (x)=x$ pour tout $x\in \Rr^n$. Donc $f\circ f=\mathrm{id}$.

On en déduit $f^{-1}=f$ et que la composition itérée vérifie $f^{p}=\mathrm{id}$ si $p$ est pair
et $f^{p}=f$ si $p$ est impair.
Conclusion : $A^p=I$ si $p$ est pair et $A^p=A$ si $p$ est impair.


\fincorrection
\correction{002444}
Soit $A,B$ tel que $B = P^{-1} A P$.
\begin{enumerate}
  \item Supposons $A$ inversible, alors il existe $A'$ tel que $A\times A'=I$ et $A'\times A=I$.
Notons alors $B'= P^{-1} A'P$. On a
$$B \times B' = \big(P^{-1}A P \big)\times \big(P^{-1} A' P\big)=P^{-1}A \big(P P^{-1}\big)A' P 
= P^{-1}A A' P=P^{-1} I P=I$$
De même $B' \times B=I$. Donc $B$ est inversible d'inverse $B'$.

  \item Supposons que $A^n=I$. Alors 
$$\begin{array}{rcl}
B^n 
&=&\big(P^{-1} A P\big)^n= \big(P^{-1} A P \big)\big(P^{-1} A P \big)\cdots \big(P^{-1} A P \big) \\
&=&P^{-1} A (P P^{-1}) A (P P^{-1}) \cdots  A P \\
&=& P^{-1} A^n P  \\
&=& P^{-1} I P = I \\
\end{array}$$
Donc $B$ est idempotente.

  \item Si $A^n=(0)$ alors le même calcul qu'au-dessus conduit à $B^n=(0)$.

  \item Si $A = \lambda I$ alors $B=P^{-1} (\lambda I) P = \lambda I \times P^{-1}P= \lambda I$
(car la matrice $\lambda I$ commute avec toutes les matrices).
\end{enumerate}

\fincorrection
\correction{001104}
\begin{enumerate}
  \item Notons $P$ la matrice de passage de la base canonique 
$\mathcal{B}=\big((1,0),(0,1)\big)$ vers (ce qui va être) la base
$\mathcal{B}' = (e_1, e_2)$. C'est la matrice composée des vecteurs colonnes
$e_1$ et $e_2$ :
$$P = \begin{pmatrix} -2 & -2 \\ 3 & 5 \\  \end{pmatrix}$$
$\det P=-4 \neq 0$ donc $P$ est inversible et ainsi $\mathcal{B}'$ est bien une base.

Alors la matrice de $f$ dans la base $\mathcal{B}'$ est :
$$B= P^{-1}AP = -\frac14\begin{pmatrix} 5 & 2 \\ -3 & -2 \\  \end{pmatrix} \begin{pmatrix} 2&\frac 23\\
-\frac 52&-\frac 23 \end{pmatrix}\begin{pmatrix} -2 & -2 \\ 3 & 5 \\  \end{pmatrix}=
\begin{pmatrix} 1 & 0 \\ 0 & \frac13 \\  \end{pmatrix}$$

  \item Il est très facile de calculer la puissance d'une matrice diagonale :
$$B^n=\begin{pmatrix} 1 & 0 \\ 0 & \big(\frac13\big)^n \\  \end{pmatrix}$$

Comme $A=PBP^{-1}$ on va en déduire $A^n$ : 

$$A^n = \big( PBP^{-1} \big)^n = P B^n P^{-1} = 
\frac14\begin{pmatrix} 
10- \frac{6} {3^n} & 4- \frac{4} {3^n}\\
-15 + \frac{15} {3^n}& -6 + \frac{10} {3^n}\\                    
       \end{pmatrix}$$

  \item Si l'on note $X_n = \begin{pmatrix}x_n \\ y_n \end{pmatrix}$
alors les équations que vérifient les suites s'écrivent en terme matriciel :
$$X_{n+1}=AX_n.$$

Si l'on note les conditions initiales $X_0 = \begin{pmatrix}x_0 \\ y_0 \end{pmatrix} \in \Rr^2$ alors
$X_n = A^n X_0$.
On en déduit 
$$
\left\{\begin{array}{rcl}
x_n &=& \frac14 \Big(( 10- \frac{6} {3^n}) x_0 + (4- \frac{4} {3^n})  y_0 \Big)          \\
y_n &=& \frac14 \Big( (-15 + \frac{15} {3^n})x_0 + (-6 + \frac{10} {3^n}) y_0 \Big)   
\end{array}\right.
$$
\end{enumerate}


\fincorrection
\correction{002774}
Avant toute, un coup d'\oe il sur la matrice nous informe de deux choses :
(a) $A$  n'est pas la matrice nulle donc $\textrm{rg}(A)\ge 1$ ;
(b) il y a $3$ lignes donc $\textrm{rg}(A)\le 3$ (le rang est plus petit que le nombre 
de colonnes et que le nombre de lignes).

\begin{enumerate}
  \item Montrons de différentes façons que $\textrm{rg}(A)\ge 2$.
  \begin{itemize}
     \item \textbf{Première méthode : sous-déterminant non nul.}
On trouve une sous-matrice $2\times 2$ dont le déterminant est non nul.
Par exemple la sous-matrice extraite du coin en bas à gauche vérifie
$\begin{vmatrix}3 & 0\\ 5 & 4\end{vmatrix}= 12 \neq 0$ donc $\textrm{rg}(A)\ge 2$.

     \item \textbf{Deuxième méthode : espace vectoriel engendré par les colonnes.}
On sait que l'image de l'application linéaire associée à la matrice $A$
est engendrée par les vecteurs colonnes. Et le rang est la dimension de cette image.
On trouve facilement deux colonnes linéairement indépendantes : 
la deuxième $\begin{pmatrix}2\\0\\4\end{pmatrix}$
et la troisième $\begin{pmatrix}-1\\1\\-1\end{pmatrix}$ colonne.
Donc $\textrm{rg}(A)\ge 2$.

     \item \textbf{Troisième méthode : espaces vectoriel engendré par les lignes.}
Il se trouve que la dimension de l'espace vectoriel engendré par les lignes
égal la dimension de l'espace vectoriel engendré par les colonnes (car $\textrm{rg}(A)=\textrm{rg}({}^tA)$).
Comme les deuxième et troisième lignes sont linéairement indépendantes alors
$\textrm{rg}(A)\ge 2$. 

Attention : les dimensions des espaces vectoriels engendrés sont égales mais les espaces sont différents !
  \end{itemize}

  \item En utilisant la dernière méthode : le rang est exactement $2$ si la première ligne est dans le sous-espace engendré
par les deux autres.
Donc 
\begin{align*}
\textrm{rg}(A) = 2
 & \iff (a,2,-1,b) \in \textrm{Vect} \big\{ (3,0,1,-4), (5,4,-1,2) \big\} \\
 & \iff \exists \lambda,\mu \in \Rr \quad (a,2,-1,b) = \lambda (3,0,1,-4) + \mu(5,4,-1,2) \\
 & \iff \exists \lambda,\mu \in \Rr \quad 
\left\{
\begin{array}{rcl}
3\lambda+5\mu &=& a \\
4\mu &=& 2 \\
\lambda-\mu &=& -1 \\
-4\lambda+2\mu &=& b \\
\end{array}
\right. 
 \iff  \left\{
\begin{array}{rcl}
\lambda &=& -\frac12 \\
\mu &=& \frac12 \\
a &=& 1 \\
b &=& 3 \\
\end{array} 
\right.\\  
\end{align*}
Conclusion la rang de $A$ est $2$ si $(a,b)=(1,3)$. Sinon le rang de $A$ est $3$.
\end{enumerate}

\fincorrection
\correction{001099}
\begin{enumerate}
  \item 
  \begin{enumerate}
     \item Commençons par des remarques élémentaires : la matrice est non nulle donc $\textrm{rg}(A) \ge 1$
et comme il y a $p=4$ lignes et $n=3$ colonnes alors $\textrm{rg}(A) \le \min(n,p)=3$.

     \item Ensuite on va montrer $\textrm{rg}(A) \ge 2$ en effet le sous-déterminant $2\times 2$ 
(extrait du coin en haut à gauche) :
$\begin{vmatrix} 
1 & 2 \cr
3 & 4 \cr
\end{vmatrix}= -2$ est non nul.

     \item Montrons que $\textrm{rg}(A)=2$. Avec les déterminants il faudrait vérifier que pour toutes
les sous-matrices $3\times 3$ les déterminants sont nuls. Pour éviter de nombreux calculs on remarque ici
que les colonnes sont liées par la relation $v_2=v_1+v_3$. Donc $\textrm{rg}(A)=2$.

     \item L'application linéaire associée à la matrice $A$ est l'application 
$f_A : \Rr^3 \to \Rr^4$. Et le théorème du rang 
$\dim \Ker f_A+ \dim \Im f_A = \dim \Rr^3$ donne ici
$\dim \Ker f_A = 3 - \textrm{rg}(A)=1$.

Mais la relation $v_2=v_1+v_3$ donne immédiatement un élément du noyau :
en écrivant $v_1-v_2+v_3=0$ alors $A\begin{pmatrix}1\\-1\\1\end{pmatrix}=\begin{pmatrix}0\\0\\0\end{pmatrix}$
Donc $\begin{pmatrix}1\\-1\\1\end{pmatrix} \in \Ker f_A$. Et comme le noyau est de dimension $1$ alors
$$\Ker f_A = \textrm{Vect} \begin{pmatrix}1\\-1\\1\end{pmatrix}$$

     \item Pour un base de l'image, qui est de dimension $2$, 
     il suffit par exemple de prendre les deux premiers vecteurs colonnes de la matrice $A$ (ils sont clairement non colinéaires) :
$$\Im f_A = \textrm{Vect} \left\{ v_1, v_2 \right\} = 
\textrm{Vect} \left\{  \begin{pmatrix}1\\3\\5\\7\end{pmatrix},  \begin{pmatrix}1\\1\\1\\1\end{pmatrix} \right\}
$$

  \end{enumerate}


  \item On fait le même travail avec $B$ et $f_B$.
  \begin{enumerate}
     \item Matrice non nulle avec $4$ lignes et $4$ colonnes donc $1 \le \textrm{rg}(B) \le 4$.

     \item Comme le sous-déterminant (du coin supérieur gauche)
$\begin{vmatrix} 
2 & 2 \cr
4 & 3 \cr
\end{vmatrix}= -2$ est non nul alors $\textrm{rg}(B) \ge 2$.

     \item Et pareil avec le sous-déterminant $3\times 3$ :
$$\begin{vmatrix} 
2 & 2 & -1 \cr
4 & 3 & -1 \cr
0 & -1 & 2 \cr
\end{vmatrix} = -2$$
qui est non nul donc $\textrm{rg}(B) \ge 3$.

     \item Maintenant on calcule le déterminant de la matrice $B$ et 
on trouve $\det B = 0$, donc $\textrm{rg}(B) < 4$. Conclusion $\textrm{rg}(B) = 3$.
Par le théorème du rang alors $\dim \Ker f_B=1$.



     \item Cela signifie que les colonnes (et aussi les lignes) sont liées, comme il n'est pas clair
de trouver la relation à la main on résout le système $B X = 0$ pour trouver cette relation ; autrement dit :
$$\begin{pmatrix} 
2 & 2 & -1 & 7  \cr
4 & 3 & -1 & 11 \cr
0 & -1 & 2 & -4 \cr
3 & 3 & -2 & 11 \cr 
\end{pmatrix}
\cdot 
\begin{pmatrix} x\\y\\z\\t \end{pmatrix} = 
\begin{pmatrix} 0\\0\\0\\0 \end{pmatrix}
\text{ ou encore }
\left\{ 
\begin{array}{rcl}
2x + 2y -z + 7t  &=& 0  \cr
4x + 3y -z + 11t &=& 0 \cr
     -y +2z -4t  &=& 0  \cr
3x + 3y -2z+ 11t &=& 0  \cr   
\end{array}
\right.$$
Après résolution de ce système on trouve que 
les solutions s'écrivent $(x,y,z,t)= (-\lambda,-2\lambda,\lambda,\lambda)$.
Et ainsi 
$$\Ker f_B = \textrm{Vect} \begin{pmatrix}-1\\-2\\1\\1\end{pmatrix}$$
Et pour une base de l'image il suffit, par exemple, de prendre les $3$ premiers vecteurs colonnes $v_1,v_2,v_3$ 
de la matrice $B$, car ils sont linéairement indépendants :
$$\Im f_B = \textrm{Vect} \left\{ v_1, v_2, v_3 \right\} = 
\textrm{Vect} \left\{  
\begin{pmatrix}2\\4\\0\\3\end{pmatrix},  
\begin{pmatrix}2\\3\\-1\\3\end{pmatrix},
\begin{pmatrix}-1\\-1\\2\\-2\end{pmatrix} 
\right\}
$$

  \end{enumerate}
\end{enumerate}


\fincorrection
\correction{001093}
\begin{enumerate}
  \item Nous devons montrer $\Ker f \cap \Im f = \{0\}$ et 
$\Ker f + \Im f = E$.
  \begin{enumerate}
     \item Si $x \in \Ker f \cap \Im f$ alors d'une part $f(x)=0$ et d'autre part il existe $x'\in E$ tel que 
$x=f(x')$. Donc $0=f(x)=f\big(f(x')\big)= f(x')=x$ donc $x=0$ (on a utilisé $f\circ f=f$). 
Donc $\Ker f \cap \Im f = \{0\}$.

     \item Pour $x\in E$ on le réécrit $x=x-f(x) + f(x)$. Alors $x-f(x) \in \Ker f$ 
(car $f\big(x-f(x) \big)=f(x)-f\circ f (x)=0$)
et $f(x)\in \Im f$. Donc $x \in \Ker f + \Im f$. Donc $\Ker f + \Im f = E$.

     \item Conclusion  : $E= \Ker f \oplus \Im f$.
  \end{enumerate}

  \item Notons $r$ le rang de $f$ : $r=\dim \Im f$.
Soit $\{e_1,\ldots,e_r\}$ une base de $\Im f$ et 
soit $\{e_{r+1},\ldots, e_n\}$ une base de $\Ker f$.
Comme 
$E= \Ker f \oplus \Im f$ alors
$(e_1,\ldots ,e_n)$ est une base de $E$.
Pour $i > r$ alors $e_i \in \Ker f$ donc $f(e_i)=0$.

Comme $f\circ f=f$ alors pour n'importe quel $x\in \Im f$ on a $f(x)=x$ :
en effet comme $x\in \Im f$, il existe $x'\in E$ tel que $x=f(x')$ 
ainsi $f(x)=f\big(f(x')\big)=f(x')=x$. En particulier si $i\le r$ alors
$f(e_i)=e_i$.

  \item La matrice de $f$ dans la base $(e_1,\ldots,e_n)$ est donc :
$$\begin{pmatrix}
  I & (0) \\
  (0) & (0) \\  
  \end{pmatrix}$$
où $I$ désigne la matrice identité de taille $r\times r$
et les $(0)$ désignent des matrices nulles.

\end{enumerate}

\fincorrection
\correction{002475}
\begin{enumerate}
  \item Soit $M$ une matrice telle que $M^2=0$ et soit $f$ l'application linéaire associée à $M$.
Comme $M^2=0$ alors $f\circ f = 0$. Cela entraîne $\Im f \subset \Ker f$. Discutons suivant la dimension
du noyau :
   \begin{enumerate}
      \item Si $\dim \Ker f=3$ alors $f=0$ donc $M=0$ (la matrice nulle).

      \item Si $\dim \Ker f=2$ alors prenons une base de $\Rr^3$ formée de deux vecteurs du noyau et d'un troisième vecteur.
Dans cette base la matrice de $f$ est
$M'=\begin{pmatrix}0&0&a\\0& 0 & b\\0&0&c\end{pmatrix}$ mais comme $f\circ f=0$ alors $M'^2=0$ ;
un petit calcul implique $c=0$. Donc $M$ et $M'$ sont les matrices de la même application linéaire $f$ mais exprimées dans des bases différentes,
donc $M$ et $M'$ sont semblables.

      \item Si $\dim \Ker f=1$ alors comme $\Im f \subset \Ker f$ on a $\dim \Im f \le 1$ mais alors cela contredit le théorème
du rang : $\dim \Ker f + \dim \Im f = \dim \Rr^3$. Ce cas n'est pas possible.

      \item Conclusion : $M$ est une matrice qui vérifie $M^2=0$ si et seulement si 
il existe une matrice inversible $P$ et des réels $a,b$ tels que 
$$M=P^{-1} \begin{pmatrix}0&0&a\\0& 0 & b\\0&0&0\end{pmatrix}P$$

   \end{enumerate}

  \item On va s'aider de l'exercice \ref{exo1093}. Si $M^2=M$ et $f$ est l'application linéaire associée 
alors $f\circ f= f$. On a vu dans l'exercice \ref{exo1093} qu'alors $\Ker f \oplus \Im f$
et que l'on peut choisir une base $(e_1,e_2,e_3)$ telle que $f(e_i)=e_i$ 
puis $f(e_i)=0$. Suivant la dimension du noyau cela donne que la matrice $M'$ de $f$ dans cette base est
$$A_0=\begin{pmatrix}0&0&0\\0&0&0\\0&0&0\end{pmatrix}\quad
A_1=\begin{pmatrix}1&0&0\\0&0&0\\0&0&0\end{pmatrix} \quad 
A_2=\begin{pmatrix}1&0&0\\0&1&0\\0&0&0\end{pmatrix} \quad 
A_3=\begin{pmatrix}1&0&0\\0&1&0\\0&0&1\end{pmatrix}.$$
Maintenant $M$ est semblable à l'une de ces matrices : il existe $P$ inversible telle que $M=P^{-1}M'P$
où $M'$ est l'une des quatre matrices $A_i$ ci-dessus.

Géométriquement notre application est une projection (projection sur une droite pour la seconde matrice 
et sur un plan pour la troisième).

  \item Posons $N= \frac{I+M}{2}$ et donc $M=2N-I$.
  Alors $M^2=I \iff (2N-I)^2=I \iff 4N^2-4N-I=I \iff N^2=N$.
  Donc par la deuxième question $N$ est semblable à l'une des matrice $A_i$ :
  $N= P^{-1}A_iP$.
  Donc $M=2P^{-1}A_iP-I = P^{-1}(2A_i-I)P$.
  Ainsi $M$ est semblable à l'une des matrices $2A_i-I$ suivantes :
$$\begin{pmatrix}-1&0&0\\0&-1&0\\0&0&-1\end{pmatrix}\quad
\begin{pmatrix}1&0&0\\0&-1&0\\0&0&-1\end{pmatrix} \quad 
\begin{pmatrix}1&0&0\\0&1&0\\0&0&-1\end{pmatrix} \quad 
\begin{pmatrix}1&0&0\\0&1&0\\0&0&1\end{pmatrix}.$$  
Ce sont des matrices de symétrie (par rapport à l'origine pour la première matrice, par rapport à une droite pour la seconde matrice 
et par rapport à un plan pour la troisième).

L'idée de poser $N= \frac{I+M}{2}$ est la suivante : si $M^2=I$ alors géométriquement l'application
linéaire $s$ associée à $M$ est une \emph{symétrie}, alors que si $N^2=N$ alors l'application
linéaire $p$ associée est une \emph{projection}. Et projection et symétrie sont liées par $p(x) =\frac{x+s(x)}{2}$ 
(faites un dessin !) c'est-à-dire $p =\frac{\text{id}+s}{2}$ ou encore $N= \frac{I+M}{2}$.


\end{enumerate}

\fincorrection
\correction{001094}
\begin{enumerate}
  \item Il est facile de voir que $f(\lambda P + \mu Q) = \lambda f(P)+\mu f(Q)$ donc $f$ est linéaire,
de plus, $P$ étant un polynôme de degré $\le n$ alors $f(P)$ aussi.

  \item Pour $n=3$ on calcule l'image de chacun des éléments de la base :
$$f(1)=1+1-2=0,\quad f(X)=(X+1)+(X-1)-2X=0,$$
$$f(X^2)=(X+1)^2+(X-1)^2-2X^2=2,
\quad f(X^3)=(X+1)^3+(X-1)^3-2X^3=6X.$$
Donc la matrice de $f$ dans la base $(1, X, X^2, X^3)$ est
$$\begin{pmatrix}
0 & 0 & 2 & 0 \\
0 & 0 & 0 & 6 \\
0 & 0 & 0 & 0 \\
0 & 0 & 0 & 0 \\     
  \end{pmatrix}$$

Pour le cas général on calcule 
\begin{align*}
f(X^p)
 &=(X+1)^p+(X-1)^p-2X^p \\
 &= \sum_{k=1}^p \binom{p}{k}X^k + \sum_{k=1}^p \binom{p}{k}X^k(-1)^{p-k} -2X^p\\
 &= \sum_{p-k \text{ pair et } k<p} 2\binom{p}{k}X^k  
\end{align*}

Donc la matrice est 
$$\begin{pmatrix} 
0 & 0 & 2\binom{2}{0} & 0             & \cdots & 2\binom{p}{0} & 0               &        \\
  & 0 & 0             & 2\binom{3}{1} &        & 0             & 2\binom{p+1}{1} &        \\
  &   & 0             & 0             & \cdots & 2\binom{p}{2} & 0               &        \\
  &   &               & 0             &        & 0             & 2\binom{p+1}{3} & \vdots \\  
  &   &               &               & \ddots & \vdots        & 0               &        \\
  &   &               &               &        & 0             & \vdots          &        \\
  &   &               &               &        &               & 0               &        \\
  &   &               &               &        &               &                 & 0      \\
  \end{pmatrix}$$
Dans cet exemple de matrice, $p$ est pair.
Chaque colonne commence en alternant une valeur nulle/une valeur non-nulle 
jusqu'à l'élément diagonal (qui est nul).


  \item Nous savons que $f(1)=0$ et $f(X)=0$ donc $1$ et $X$ sont dans le noyau $\Ker f$.
Il est aussi clair que les colonnes de la matrices $f(X^2),\cdots, f(X^n)$ sont linéairement indépendantes
(car la matrice est échelonnée). Donc $\Im f = \textrm{Vect}\{f(X^2),f(X^3),\ldots,f(X^n)\}$ 
et $\dim \Im f = n-1$.

Par la formule du rang $\dim \Ker f + \dim \Im f = \dim \Rr_n[X]$ donc
$\dim \Ker f = 2$. Comme nous avons déjà deux vecteurs du noyau alors 
$\Ker f =  \textrm{Vect}\{1,X\}$.

   \item 
   \begin{enumerate}
      \item Soit $Q \in \Im f$. Il existe donc $R\in\Rr_n[X]$ tel que $f(R)=Q$.
On pose ensuite $P(X)=R(X)-R(0)-R'(0)X$.
On a tout fait pour que $P(0)=0$ et $P'(0)=0$.
De plus par la linéarité de $f$ et son noyau alors
$$f(P)= f\big( R(X)-R(0)-R'(0)X\big) = f\big( R(X)\big)-R(0)f(1) -R'(0)f(X)=f(R)=Q.$$
Donc notre polynôme $P$ convient.

      \item Montrons l'unicité. Soient $P$ et $\tilde P$ tels que $f(P)=f(\tilde P)=Q$
avec $P(0)=P'(0)=0 = \tilde P(0)=\tilde P'(0)$.
Alors  $f(P-\tilde P) = Q-Q=0$ donc $P-\tilde P \in \Ker f = \textrm{Vect}\{1,X\}$.
Ainsi $P-\tilde P$ s'écrit $P-\tilde P = aX+b$. Mais comme $(P-\tilde P)(0)=0$ alors 
$b=0$, et comme  $(P-\tilde P)'(0)=0$ alors $a=0$. Ce qui prouve $P = \tilde P$.

    \end{enumerate}
\end{enumerate}
\fincorrection
\correction{002442}
\begin{enumerate}
  \item Notons $C=AB$ et $D=BA$.
Alors par la définition du produit de matrice :
$$c_{ij}=\sum_{1\le k \le n} a_{ik}b_{kj} \quad \text{ donc } c_{ii}=\sum_{1\le k \le n} a_{ik}b_{ki}$$
Ainsi 
$$\textrm{tr}(AB) = \textrm{tr}\, C = \sum_{1\le i \le n} c_{ii} = \sum_{1\le i \le n} \sum_{1\le k \le n} a_{ik}b_{ki}$$

De même $$\textrm{tr}(BA) = \textrm{tr}\, D = \sum_{1\le i \le n} \sum_{1\le k \le n} b_{ik}a_{ki}$$
Si dans cette dernière formule on renomme l'indice $i$ en $k$ et l'indice $k$ en $i$ (ce sont des variables muettes 
donc on leur donne le nom qu'on veut) alors on obtient :
$$\textrm{tr}(BA) =\sum_{1\le k \le n} \sum_{1\le i \le n} b_{ki}a_{ik} =  \sum_{1\le i \le n}\sum_{1\le k \le n} a_{ik}b_{ki} =\textrm{tr}(AB)$$

  \item $M$ et $M'$ sont semblables donc il existe une matrice de passage $P$ telle que $M'=P^{-1}MP$ donc
$$\textrm{tr}\, M' = \textrm{tr}\big( P^{-1}(MP) \big) = \textrm{tr}\big( (MP)P^{-1} \big) = \textrm{tr} ( M I ) = \textrm{tr}\, M$$

  \item La trace a aussi la propriété évidente que 
$$\textrm{tr}(A+B)=\textrm{tr}\,A+\textrm{tr}\,B.$$

Fixons une base de $E$. Notons $A$ la matrice de $f$ dans cette base et $B$ la matrice de $g$ dans cette même base.
Alors $AB$ est la matrice de $f\circ g$ et $BA$ est la matrice de $g\circ f$.
Ainsi la matrice de $f\circ g - g\circ f$ est $AB-BA$
Donc
$$\textrm{tr}(f\circ g - g\circ f)= \textrm{tr}(AB-BA) = \textrm{tr}(AB)-\textrm{tr}(BA)=0.$$
\end{enumerate}


\fincorrection


\end{document}

