\documentclass[a4paper,10pt]{article}



\usepackage{fancyhdr} % pour personnaliser les en-têtes
\usepackage[utf8]{inputenc}
\usepackage[T1]{fontenc}
\usepackage{lastpage}
\usepackage[frenchb]{babel}
\usepackage{amsfonts,amssymb}
\usepackage{amsmath,amsthm,mathtools}
\usepackage{paralist}
\usepackage{xspace,xypic}
\usepackage{xcolor,multicol,tabularx}
\usepackage{variations}
\usepackage{xypic}
\usepackage{eurosym,multicol}
\usepackage{graphicx}
\usepackage{mathdots}%faire des points suspendus en diagonale
\usepackage[np]{numprint}
\usepackage{hyperref} 
\usepackage{listings} % pour écrire des codes avec coloration syntaxique  

\usepackage{tikz}
\usetikzlibrary{calc, arrows, plotmarks,decorations.pathreplacing}
\usepackage{colortbl}
\usepackage{multirow}
\usepackage[top=2cm,bottom=1.5cm,right=2cm,left=1.5cm]{geometry}

\newtheorem{thm}{Théorème}
\newtheorem*{pro}{Propriété}
\newtheorem*{exemple}{Exemple}

\theoremstyle{definition}
\newtheorem*{remarque}{Remarque}
\theoremstyle{definition}
\newtheorem{exo}{Exercice}
\newtheorem{definition}{Définition}


\newcommand{\vtab}{\rule[-0.4em]{0pt}{1.2em}}
\newcommand{\V}{\overrightarrow}
\renewcommand{\thesection}{\Roman{section} }
\renewcommand{\thesubsection}{\arabic{subsection} }
\renewcommand{\thesubsubsection}{\alph{subsubsection} }
\newcommand*{\transp}[2][-3mu]{\ensuremath{\mskip1mu\prescript{\smash{\mathrm t\mkern#1}}{}{\mathstrut#2}}}%

\newcommand{\K}{\mathbb{K}}
\newcommand{\C}{\mathbb{C}}
\newcommand{\R}{\mathbb{R}}
\newcommand{\Q}{\mathbb{Q}}
\newcommand{\Z}{\mathbb{Z}}
\newcommand{\N}{\mathbb{N}}
\newcommand{\p}{\mathbb{P}}

\renewcommand{\Im}{\mathop{\mathrm{Im}}\nolimits}



\definecolor{vert}{RGB}{11,160,78}
\definecolor{rouge}{RGB}{255,120,120}
% Set the beginning of a LaTeX document
\pagestyle{fancy}
\lhead{Optimal Sup Spé, groupe IPESUP}\chead{Année~2021-2022}\rhead{Niveau: Première année de PCSI }\lfoot{M. Botcazou}\cfoot{\thepage}\rfoot{mail: ibotca52@gmail.com }\renewcommand{\headrulewidth}{0.4pt}\renewcommand{\footrulewidth}{0.4pt}

\begin{document}
 	
	
	\begin{center}
		\Large \sc colle 21 = Séries numériques et Variables aléatoires
	\end{center}


\section*{Séries numériques:}\hfill\\%[-0.25cm]
\begin{minipage}{1\linewidth}
	\begin{minipage}[t]{0.48\linewidth}
		\raggedright
		
		
		\begin{exo}\quad\\[0.2cm]
			Donner la nature de la série de terme général:
			\begin{multicols}{2}
				\begin{enumerate}
					\item $\ln\left(\frac{n^2+n+1}{n^2+n-1}\right)$
					\item $\frac{1}{n+(-1)^n\sqrt{n}}$
					\item $\left(\cos\frac{1}{\sqrt{n}}\right)^n-\frac{1}{\sqrt{e}}$
					\item $\frac{n^2}{(n-1)!}$
					\item $\frac{2n^3-3n^2+1}{(n+3)!}$
					\item $\frac{1}{\sqrt{n-1}}+\frac{1}{\sqrt{n+1}}-\frac{2}{\sqrt{n}}$
				\end{enumerate}
			\end{multicols}
			 
			
			\centering
			\rule{1\linewidth}{0.6pt}
		\end{exo}
	
				\begin{exo}\quad\\[0.2cm]
		Déterminer un équivalent simple de $\sum_{k=1}^{n}\sqrt{k}$ quand $n$ tend vers l'infini.
		
		\centering
		\rule{1\linewidth}{0.6pt}
	\end{exo}

	\begin{exo}\quad\\[0.2cm]
	Soit $\alpha\in\R$. Donner la nature de la série de terme général $u_n=\frac{1+(-1)^nn^\alpha}{n^{2\alpha}}$, $n\geqslant1$. 
	
	\centering
	\rule{1\linewidth}{0.6pt}
\end{exo}	
	
		
					\begin{exo}\quad\\[0.2cm]
						Soit $(u_n)_{n\in\N}$ une suite de réels strictement positifs. Montrer que les séries de termes généraux $u_n$, $\frac{u_n}{1+u_n}$, $\ln(1+u_n)$ et $\int_{0}^{u_n}\frac{dx}{1+x^e}$  sont de mêmes natures.
			
			\centering
			\rule{1\linewidth}{0.6pt}
		\end{exo}
	
			
	\begin{exo}\quad\\[0.2cm]
		
		Trouver un développement limité à l'ordre $3$ quand $n$ tend vers l'infini de $\left(e-\sum_{k=0}^{n}\frac{1}{k!}\right)\times(n+1)!$.
		
		\centering
		\rule{1\linewidth}{0.6pt}
	\end{exo}
		
	\end{minipage}	
	\hfill\vrule\hfill
	\begin{minipage}[t]{0.48\linewidth}
		\raggedright
		
		
		\begin{exo}\quad\\[0.2cm]
			Soit $(u_n)_{n\in\N}$ une suite décroissante de nombres réels strictement positifs telle que la série de terme général $u_n$ converge.\\[0.2cm] Montrer que $u_n\underset{n\rightarrow+\infty}{=}o\left(\frac{1}{n}\right)$\\[0.2cm]Trouver un exemple de suite $(u_n)_{n\in\N}$ de réels strictement positifs telle que la série de terme général $u_n$ converge mais telle que la suite de terme général $nu_n$ ne tende pas vers $0$.
			
			\centering
			\rule{1\linewidth}{0.6pt}
		\end{exo}	
		
		\begin{exo}\quad\\[0.2cm]
		Calculer les sommes des séries suivantes après avoir vérifié leur convergence.
		\begin{center}
			\begin{tabular}{ll}
				\textbf{1)} $\sum_{n=0}^{+\infty}\frac{n+1}{3^n}$ &\textbf{2)} $\sum_{n=3}^{+\infty}\frac{2n-1}{n^3-4n}$\\[0.3cm] \textbf{3)} $\sum_{n=2}^{+\infty}\ln\left(1+\frac{(-1)^n}{n}\right)$ & \\
				
			\end{tabular}
		\end{center}
			
			\centering
			\rule{1\linewidth}{0.6pt}
		\end{exo}
	
	
		\begin{exo}\quad\\[0.2cm]
		 Soit $(u_n)_{n\in\N}$ une suite positive telle que la série de terme général $u_n$ converge. Etudier la nature de la série de terme général $\frac{\sqrt{u_n}}{n}$.
		
		\centering
		\rule{1\linewidth}{0.6pt}
		\end{exo}
	
	\begin{exo}\quad\\[0.2cm]
		Déterminer un équivalent simple de $\sum_{k=n+1}^{+\infty}\frac{1}{k^2}$ quand $n$ tend vers l'infini.
		
		\centering
		\rule{1\linewidth}{0.6pt}
	\end{exo}	
		
		
	\end{minipage}
\end{minipage}

\section*{Variables aléatoires:}\hfill\\%[-0.25cm]
\begin{minipage}{1\linewidth}
	\begin{minipage}[t]{0.48\linewidth}
		\raggedright
		
		
		\begin{exo}\quad\\[0.2cm]
			Une entreprise souhaite recrute un cadre. $n$ personnes se présentent pour le poste. Chacun d'entre eux passe à tour de rôle un test, et le premier qui réussit le test est engagé. La probabilité de réussir le test est $p\in ]0,1[$. On pose également $q=1-p$. On définit la variable aléatoire $X$ par $X=k$ si le k-ième candidat qui réussit le test est engagé, et $X=n+1$ si personne n'est engagé.
			\begin{enumerate}
				\item Déterminer la loi de $X$.
				\item En dérivant la fonction  $x\mapsto \sum_{k=0}^{n}x^k$\\[0.1cm]
				En déduire la valeur de $\sum_{k=1}^{n}kx^{k-1}$ pour $x\neq 1$.
				\item En déduire l'espérance de $X$. 
				\item Quelle est la valeur minimale de $p$ pour avoir plus d'une chance sur deux de recruter l'un des candidats? 
			\end{enumerate}
			\centering
			\rule{1\linewidth}{0.6pt}
		\end{exo}

		
		
		
	\end{minipage}	
	\hfill\vrule\hfill
	\begin{minipage}[t]{0.48\linewidth}
		\raggedright
		
				\begin{exo}\quad\\[0.2cm]
			On lance $n$ fois une pièce parfaitement équilibrée. Quelle est la probabilité d'obtenir strictement plus de piles que de faces.
			
			\centering
			\rule{1\linewidth}{0.6pt}
		\end{exo}	
	
		\begin{exo}\quad\\[0.2cm]
		Soit $X,Y$ deux variables aléatoires indépendantes suivant la loi uniforme sur $\{1,...,n\}$.
		\begin{enumerate}
			\item Déterminer $P(X=Y)$.
			\item Déterminer $P(X\geq Y)$.
			\item Déterminer la loi de $X+Y$.
		\end{enumerate}
		
		\centering
		\rule{1\linewidth}{0.6pt}
	\end{exo}	

	\begin{exo}\quad\\[0.2cm]
	On jette $3600$ fois un dé équilibré. Minorer la probabilité que le nombre d'apparitions du numéro $1$ soit compris entre $480$ et $720$.
	
	\centering
	\rule{1\linewidth}{0.6pt}
\end{exo}
		
			
		
		
	\end{minipage}
\end{minipage}	



\end{document}