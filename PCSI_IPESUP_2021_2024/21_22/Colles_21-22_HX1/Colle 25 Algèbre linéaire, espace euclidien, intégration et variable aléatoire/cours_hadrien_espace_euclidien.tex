\documentclass[a4paper,10pt]{article}



\usepackage{fancyhdr} % pour personnaliser les en-têtes
\usepackage[utf8]{inputenc}
\usepackage[T1]{fontenc}
\usepackage{lastpage}
\usepackage[frenchb]{babel}
\usepackage{amsfonts,amssymb}
\usepackage{amsmath,amsthm,mathtools}
\usepackage{paralist}
\usepackage{xspace,xypic}
\usepackage{xcolor,multicol,tabularx}
\usepackage{variations}
\usepackage{xypic}
\usepackage{eurosym,multicol}
\usepackage{graphicx}
\usepackage{mathdots}%faire des points suspendus en diagonale
\usepackage[np]{numprint}
\usepackage{hyperref} 
\usepackage{relsize,exscale}
\usepackage{listings} % pour écrire des codes avec coloration syntaxique  

\usepackage{tikz}
\usetikzlibrary{calc, arrows, plotmarks,decorations.pathreplacing}
\usepackage{colortbl}
\usepackage{multirow}
\usepackage[top=2cm,bottom=1.5cm,right=2cm,left=1.5cm]{geometry}

\newtheorem{thm}{Théorème}
\newtheorem*{pro}{Propriété}
\newtheorem*{exemple}{Exemple}

\theoremstyle{definition}
\newtheorem*{remarque}{Remarque}
\theoremstyle{definition}
\newtheorem{exo}{Exercice}
\newtheorem{definition}{Définition}


\newcommand{\vtab}{\rule[-0.4em]{0pt}{1.2em}}
\newcommand{\V}{\overrightarrow}
\renewcommand{\thesection}{\Roman{section} }
\renewcommand{\thesubsection}{\arabic{subsection} }
\renewcommand{\thesubsubsection}{\alph{subsubsection} }
\newcommand*{\transp}[2][-3mu]{\ensuremath{\mskip1mu\prescript{\smash{\mathrm t\mkern#1}}{}{\mathstrut#2}}}%

\newcommand{\K}{\mathbb{K}}
\newcommand{\C}{\mathbb{C}}
\newcommand{\R}{\mathbb{R}}
\newcommand{\Q}{\mathbb{Q}}
\newcommand{\Z}{\mathbb{Z}}
\newcommand{\N}{\mathbb{N}}
\newcommand{\p}{\mathbb{P}}

\renewcommand{\Im}{\mathop{\mathrm{Im}}\nolimits}



\definecolor{vert}{RGB}{11,160,78}
\definecolor{rouge}{RGB}{255,120,120}
% Set the beginning of a LaTeX document
\pagestyle{fancy}
\lhead{Cours Hadrien}\chead{Année~2021-2022}\rhead{Niveau: MP Sup}\lfoot{M. Botcazou}\cfoot{\thepage}\rfoot{mail: ibotca52@gmail.com }\renewcommand{\headrulewidth}{0.4pt}\renewcommand{\footrulewidth}{0.4pt}

\begin{document}
 	

\begin{center}
\Large \sc Préparer sa colle = Algèbre linéaire, espaces euclidiens
\end{center}

\section*{Exercices mixtes:}%[-0.25cm]

\raggedright

\begin{exo}\quad\\[0.25cm]
	Soit $E$ l'ensemble des fonctions continues strictement positives sur $[a,b]$.
	
	Soit $\begin{array}[t]{cccc}
	\varphi~:&E&\rightarrow&\R\\
	&f&\mapsto&\left(\int_{a}^{b}f(t)\;dt\right)\left(\int_{a}^{b}\frac{1}{f(t)}\;dt\right)
	\end{array}$.
	
	\begin{enumerate}
		\item  Montrer que $\varphi(E)$ n'est pas majoré.
		\item  Montrer que $\varphi(E)$ est minoré. Trouver $m=\mbox{Inf}\{\varphi(f),\;f\in E\}$. Montrer que cette borne infèrieure est atteinte et trouver toutes les $f$ de $E$ telles que $\varphi(f)=m$.
	\end{enumerate}	
	
	
	\centering
	\rule{1\linewidth}{0.6pt}
\end{exo}


\begin{exo}\textbf{(Concours communs Polytechniques 2018)}\quad\\[0.25cm]
	On note $E$ l’espace vectoriel des applications continues sur le segment $[-1,1]$ et à valeurs réelles.
\begin{enumerate}
	\item Démontrer que l’on définit un produit scalaire sur $E$ en posant pour $f$ et $g$ éléments de $E$ :
	$$\left(f|g\right) = \int_{-1}^{1}f(t)g(t)dt$$
	\item On note $u : t \mapsto1$, $v : t \mapsto t$ et $F = vect \{ u , v \}$ , déterminer une base orthonormée de $F$.
	\item Déterminer le projeté orthogonal de la fonction $w : t \mapsto e^t$ sur le sous-espace $F$ et en déduire \\la valeur du réel:
	$$\underset{(a,b)\in\R^2}{inf}\left[\int_{-1}^{1}\left(e^t-(a+bt)\right)^2dt\right]$$
\end{enumerate}
\centering
\rule{1\linewidth}{0.6pt}
\end{exo}

\begin{exo}\quad\\[0.25cm]
Soit $E$ un $\R$ espace vectoriel de dimension finie. Soit $||\;||$ une norme sur $E$ vérifiant l'identité du parallèlogramme, c'est-à-dire~:~$\forall(x,y)\in E^2,\;||x+y||^2+||x-y||^2=2(||x||^2+||y||^2)$. On se propose de démontrer que $||\;||$ est associée à un produit scalaire.
On définit sur $E^2$ une application $f$ par~:~$\forall(x,y)\in E^2,\;f(x,y)=\frac{1}{4}(||x+y||^2-||x-y||^2)$.
\begin{enumerate}
	\item  Montrer que pour tout $(x,y,z)$ de $E^3$, on a~:~$f(x+z,y)+f(x-z,y)=2f(x,y)$.
	\item  Montrer que pour tout $(x,y)$ de $E^2$, on a~:~$f(2x,y)=2f(x,y)$.
	\item  Montrer que pour tout $(x,y)$ de $E^2$ et tout rationnel $r$, on a~:~$f(rx,y)=rf(x,y)$.
	
	On admettra que pour tout réel $\lambda$ et tout $(x,y)$ de $E^2$ on a~:~$f(\lambda x,y)=\lambda f(x,y)$ ( ce résultat provient de la continuité de $f$).
	\item  Montrer que pour tout $(u,v,w)$ de $E^3$, $f(u,w)+f(v,w)=f(u+v,w)$.
	\item  Montrer que $f$ est bilinéaire.
	\item  Montrer que $||\;||$ est une norme euclidienne.
\end{enumerate}

\centering
\rule{1\linewidth}{0.6pt}
\end{exo}			








\end{document}